\documentclass[10pt,a4paper]{article}

% Packages
\usepackage{fancyhdr}           % For header and footer
\usepackage{multicol}           % Allows multicols in tables
\usepackage{tabularx}           % Intelligent column widths
\usepackage{tabulary}           % Used in header and footer
\usepackage{hhline}             % Border under tables
\usepackage{graphicx}           % For images
\usepackage{xcolor}             % For hex colours
%\usepackage[utf8x]{inputenc}    % For unicode character support
\usepackage[T1]{fontenc}        % Without this we get weird character replacements
\usepackage{colortbl}           % For coloured tables
\usepackage{setspace}           % For line height
\usepackage{lastpage}           % Needed for total page number
\usepackage{seqsplit}           % Splits long words.
%\usepackage{opensans}          % Can't make this work so far. Shame. Would be lovely.
\usepackage[normalem]{ulem}     % For underlining links
% Most of the following are not required for the majority
% of cheat sheets but are needed for some symbol support.
\usepackage{amsmath}            % Symbols
\usepackage{MnSymbol}           % Symbols
\usepackage{wasysym}            % Symbols
%\usepackage[english,german,french,spanish,italian]{babel}              % Languages

% Document Info
\author{}
\pdfinfo{
  /Title (python-turtle-module-cheatsheet.pdf)
  /Subject (Python turtle module cheatsheet Cheat Sheet)
}

% Lengths and widths
\addtolength{\textwidth}{6cm}
\addtolength{\textheight}{-1cm}
\addtolength{\hoffset}{-3cm}
\addtolength{\voffset}{-2cm}
\setlength{\tabcolsep}{0.2cm} % Space between columns
\setlength{\headsep}{-12pt} % Reduce space between header and content
\setlength{\headheight}{85pt} % If less, LaTeX automatically increases it
\renewcommand{\footrulewidth}{0pt} % Remove footer line
\renewcommand{\headrulewidth}{0pt} % Remove header line
\renewcommand{\seqinsert}{\ifmmode\allowbreak\else\-\fi} % Hyphens in seqsplit
% This two commands together give roughly
% the right line height in the tables
\renewcommand{\arraystretch}{1.3}
\onehalfspacing

% Commands
\newcommand{\SetRowColor}[1]{\noalign{\gdef\RowColorName{#1}}\rowcolor{\RowColorName}} % Shortcut for row colour
\newcommand{\mymulticolumn}[3]{\multicolumn{#1}{>{\columncolor{\RowColorName}}#2}{#3}} % For coloured multi-cols
\newcolumntype{x}[1]{>{\raggedright}p{#1}} % New column types for ragged-right paragraph columns
\newcommand{\tn}{\tabularnewline} % Required as custom column type in use

% Font and Colours
\definecolor{HeadBackground}{HTML}{333333}
\definecolor{FootBackground}{HTML}{666666}
\definecolor{TextColor}{HTML}{333333}
\definecolor{DarkBackground}{HTML}{47A325}
\definecolor{LightBackground}{HTML}{F3F9F1}
\renewcommand{\familydefault}{\sfdefault}
\color{TextColor}

% Header and Footer
\pagestyle{fancy}


\begin{document}
\raggedright
\raggedcolumns

% Set font size to small. Switch to any value
% from this page to resize cheat sheet text:
% www.emerson.emory.edu/services/latex/latex_169.html
\footnotesize % Small font.

\begin{multicols*}{2}

\begin{tabularx}{8.4cm}{x{2.56 cm} x{5.44 cm} }
\SetRowColor{DarkBackground}
\mymulticolumn{2}{x{8.4cm}}{\bf\textcolor{white}{Turtle Pen}}  \tn
% Row 0
\SetRowColor{LightBackground}
up() & Sets the pen state to be up (not drawing). \tn 
% Row Count 2 (+ 2)
% Row 1
\SetRowColor{white}
\seqsplit{down()} & Sets the pen state to be down (drawing). \tn 
% Row Count 4 (+ 2)
% Row 2
\SetRowColor{LightBackground}
\seqsplit{color(r},g,b) & See below \tn 
% Row Count 6 (+ 2)
% Row 3
\SetRowColor{white}
\seqsplit{color(s)} & Sets the color that the pen will draw until the color is changed. It takes either \tn 
% Row Count 9 (+ 3)
% Row 4
\SetRowColor{LightBackground}
 & 1. three arguments, each a floating point number between 0.0 — 1.0, where the first the amount of red, the second is the amount of green, and the third is the amount of blue \tn 
% Row Count 16 (+ 7)
% Row 5
\SetRowColor{white}
 & 2. a "color string" the name of a TK color (e.g.,  "black", "red", "blue", ...) \tn 
% Row Count 20 (+ 4)
% Row 6
\SetRowColor{LightBackground}
\seqsplit{begin\_fill()} & See below \tn 
% Row Count 22 (+ 2)
% Row 7
\SetRowColor{white}
\seqsplit{end\_fill()} & To fill a figure, use begin\_fill() before you start drawing the figure. Draw the figure.  Then execute end\_fill().  The figure drawn between the two fill commands will be filled with the present color setting. \tn 
% Row Count 31 (+ 9)
\SetRowColor{LightBackground}
\seqsplit{hideturtle()} & See below \tn 
% Row Count 2 (+ 2)
% Row 9
\SetRowColor{white}
\seqsplit{showturtle()} & Sets the state to hide / show the   When shown, you see it as a small arrowhead pointed in the direction of the heading. \tn 
% Row Count 7 (+ 5)
\hhline{>{\arrayrulecolor{DarkBackground}}--}
\SetRowColor{LightBackground}
\mymulticolumn{2}{x{8.4cm}}{The default pen color is "black".}  \tn 
\hhline{>{\arrayrulecolor{DarkBackground}}--}
\end{tabularx}
\par\addvspace{1.3em}

\begin{tabularx}{8.4cm}{x{3.28 cm} x{4.72 cm} }
\SetRowColor{DarkBackground}
\mymulticolumn{2}{x{8.4cm}}{\bf\textcolor{white}{Turtle other}}  \tn
% Row 0
\SetRowColor{LightBackground}
xcor(), ycor() & Returns the x - coordinate / y - coordinate of the current pen position. \tn 
% Row Count 4 (+ 4)
% Row 1
\SetRowColor{white}
bye() & Close the turtle drawing window \tn 
% Row Count 6 (+ 2)
\hhline{>{\arrayrulecolor{DarkBackground}}--}
\end{tabularx}
\par\addvspace{1.3em}

\begin{tabularx}{8.4cm}{x{3.04 cm} x{4.96 cm} }
\SetRowColor{DarkBackground}
\mymulticolumn{2}{x{8.4cm}}{\bf\textcolor{white}{Turtle Draw}}  \tn
% Row 0
\SetRowColor{LightBackground}
\seqsplit{right(degrees)} & Turns the direction that the turtle is facing right (clockwise) by the amount indicated (in degrees). \tn 
% Row Count 5 (+ 5)
% Row 1
\SetRowColor{white}
\seqsplit{left(degrees)} & Turns the direction that the turtle is facing left (counter clockwise) by the amount indicated (in degrees). \tn 
% Row Count 10 (+ 5)
% Row 2
\SetRowColor{LightBackground}
\seqsplit{forward(distance)} & Moves the turtle forward (in the direction the turtle is facing) the distance indicated (in pixels). Draws a line if the pen is down, not if the pen is up. \tn 
% Row Count 17 (+ 7)
% Row 3
\SetRowColor{white}
\seqsplit{backward(distance)} & Moves the turtle  backward (in the direction opposite to how the turtle is facing) the distance indicated (in pixels).  Draws a line if the pen is down, not if the pen is up. \tn 
% Row Count 25 (+ 8)
% Row 4
\SetRowColor{LightBackground}
\seqsplit{setheading(angle)} & Sets the orientation of the turtle to angle. Here are some common directions in degrees: \tn 
% Row Count 29 (+ 4)
% Row 5
\SetRowColor{white}
 & 0 (east) \tn 
% Row Count 30 (+ 1)
\SetRowColor{LightBackground}
 & 90 (north) \tn 
% Row Count 1 (+ 1)
% Row 7
\SetRowColor{white}
 & 180 (west) \tn 
% Row Count 2 (+ 1)
% Row 8
\SetRowColor{LightBackground}
 & 270 (south) \tn 
% Row Count 3 (+ 1)
% Row 9
\SetRowColor{white}
goto(x,y) & Moves the turtle to the specified coordinates, drawing a straight line to the destination (x,y) if the pen is down, and not drawing if the pen is up. \tn 
% Row Count 10 (+ 7)
% Row 10
\SetRowColor{LightBackground}
\seqsplit{circle(radius)} & Draws a circle of the indicated radius. The turtle draws the circle tangent to the direction the turtle is facing. \tn 
% Row Count 15 (+ 5)
\hhline{>{\arrayrulecolor{DarkBackground}}--}
\end{tabularx}
\par\addvspace{1.3em}

\end{multicols*}

\end{document}