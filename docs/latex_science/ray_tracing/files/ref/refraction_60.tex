%--------------------------------------------------
% Refraction Diagram — TikZ
% Author: GMC
% Date: Nov 2025
% Description: Illustrates refraction of light from air into acrylic
%--------------------------------------------------

\documentclass[a4paper, landscape]{article}
\usepackage{amsmath}
\usepackage[margin=0cm]{geometry}
\usepackage{tikz}
\usetikzlibrary{
    arrows.meta,
    calc,
    shapes.geometric,
    decorations.markings,
    angles,
    quotes,
    intersections
}
\pagestyle{empty}

% Macro to draw partial segments along a line
\newcommand{\drawSegmentedRay}[4]{
  % #1 = start coordinate
  % #2 = end coordinate
  % #3 = TikZ style (e.g., incNoArrow)
  % #4 = TikZ style (e.g., incEndArrow)
  % Intermediate points
  \path let \p1 = (#1), \p2 = (#2) in
    coordinate (segStart) at ($(#1)!0.35!(#2)$)
    coordinate (segEnd) at ($(#1)!0.65!(#2)$);
  % Draw segments
  \draw[#3] (#1) -- (segStart);
  \draw[#4] (segEnd) -- (#2);
}


\begin{document}
\begin{tikzpicture}[remember picture, overlay]

    %--------------------------------------
    % arrow styles
    %--------------------------------------

    \tikzset{
        incNoArrow/.style={gray},
        incEndArrow/.style={gray, {Straight Barb[scale=2.5]}-},
        incArrow/.style={gray, postaction={decorate}, decoration={markings, mark=at position 0.75 with {\arrowreversed[scale=2.5]{Straight Barb}}}},
        refNoArrow/.style={gray},
        refEndArrow/.style={gray, -{Straight Barb[scale=2.5]}},
        refArrow/.style={gray, postaction={decorate}, decoration={markings, mark=at position 0.48 with {\arrow[scale=2.5]{Straight Barb}}}},
        exitNoArrow/.style={gray},
        exitEndArrow/.style={gray, -{Straight Barb[scale=2.5]}},
        exitArrow/.style={gray, postaction={decorate}, decoration={markings, mark=at position 1.00 with {\arrow[scale=2.5]{Straight Barb}}}},
    }
    %--------------------------------------
    % Numeric variables
    %--------------------------------------
    \pgfmathsetmacro{\angInc}{60}
    \pgfmathsetmacro{\angRef}{36}
    \pgfmathsetmacro{\incAngleFromVertical}{90 + \angInc}
    \pgfmathsetmacro{\refractAngleFromVertical}{\angRef - 90}
    \pgfmathsetmacro{\exitAngleFromVertical}{270+ \angInc}

    % Angle labels
    \newcommand{\incLabel}{$i=\angInc^{\circ}$}
    \newcommand{\refLabel}{$r=\angRef^{\circ}$}
    % Full version -------------------------------------
    \newcommand{\exitIncLabel}{$i=\angInc^{\circ}$}
    \newcommand{\exitRefLabel}{$r=\angRef^{\circ}$}
    % Student version--------------------------------------
    % \newcommand{\exitIncLabel}{\raisebox{-2.5ex}{$\displaystyle i=\raisebox{-1.0ex}{\text{\Large$\cdot\ \cdot\ \cdot$}}^\circ$}}
    % \newcommand{\exitRefLabel}{\raisebox{-2.5ex}{$\displaystyle r=\raisebox{-1.0ex}{\text{\Large$\cdot\ \cdot\ \cdot$}}^\circ$}}

    % Description variable
    % Full version -------------------------------------
    \newcommand{\refractionDescription}{Light travelling \\from air into an acrylic block\\ bends towards the normal.}
    \newcommand{\refractionDescriptionExit}{Light travelling \\from an acrylic block into air\\ bends away from the normal.}
    % Student version--------------------------------------
    % \newcommand{\refractionDescription}{Light travelling \\from air into an acrylic block\\ bends \raisebox{-1.0ex}{\Large\hbox to 4cm{\leaders\hbox{$\cdot$\hskip 0.6em}\hfill}} the normal.}
    % \newcommand{\refractionDescriptionExit}{Light travelling \\from an acrylic block into air\\ bends \raisebox{-1.0ex}{\Large\hbox to 4cm{\leaders\hbox{$\cdot$\hskip 0.6em}\hfill}} the normal.}


    %--------------------------------------
    % Core geometry setup
    %--------------------------------------
    % ray length
    \newlength{\rayLength}
    \setlength{\rayLength}{4cm}

    % block length and width
    \newlength{\blockFullWidth}
    \setlength{\blockFullWidth}{77mm}
    \newlength{\blockFullHeight}
    \setlength{\blockFullHeight}{42mm}

    \newlength{\halfBlockWidth}
    \pgfmathsetlength{\halfBlockWidth}{0.5*\blockFullWidth}
    \newlength{\halfBlockHeight}
    \pgfmathsetlength{\halfBlockHeight}{0.5*\blockFullHeight}

    \coordinate (center) at ([xshift=0.5\paperwidth, yshift=0.5\paperheight]current page.south west);
    \coordinate (blockSW) at ($(center) - (\halfBlockWidth,\halfBlockHeight)$);
    \coordinate (blockNE) at ($(center) + (\halfBlockWidth,\halfBlockHeight)$);
    \coordinate (blockE) at ($(center) + (\halfBlockWidth,0)$);
    \coordinate (topCenter) at ($(center) + (0,\halfBlockHeight)$);
    \coordinate (bottomCenter) at ($(center) - (0,\halfBlockHeight)$);

    % Draw block
    \draw[gray, fill=blue!3!white, thin] (blockSW) rectangle (blockNE);

    % Draw crosshairs
    \draw[gray!50, line width=0.3pt]
        ([xshift=0.5\paperwidth]current page.south west) --
        ([xshift=0.5\paperwidth]current page.north west);
    \draw[gray!50, line width=0.3pt]
        ([yshift=0.5\paperheight]current page.south west) --
        ([yshift=0.5\paperheight]current page.south east);

	%--------------------------------------
	% Normals and Rays
	%--------------------------------------
    % Normal at top
    \draw[dashed, black, thick] ($(topCenter)+(0,-1.6cm)$) -- ($(topCenter)+(0,1.6cm)$);

    % inc ray
    \coordinate (incEnd) at ($(topCenter)+({\incAngleFromVertical}:\rayLength)$);
    % Full version--------------------------------------
    \draw[incArrow]{ (topCenter) -- (incEnd)};
    % Student version--------------------------------------
    % \drawSegmentedRay{topCenter}{incEnd}{incNoArrow}{incEndArrow}

    % ref ray inside block
    \path[name path=refractLine] (topCenter) -- ++({\refractAngleFromVertical}:10cm);
    \path[name path=bottomEdge] (blockSW) -- ++(\blockFullWidth,0);
    \path[name intersections={of=refractLine and bottomEdge, by=refractEnd}];

    % Full version--------------------------------------
    \draw[refArrow] (topCenter) -- (refractEnd);
    % Student version--------------------------------------
    % \drawSegmentedRay{topCenter}{refractEnd}{refEndArrow}{refNoArrow}

    % Exit ray
    \coordinate (exitPoint) at (refractEnd);
    \coordinate (exitRayEnd) at ($(exitPoint)+({\exitAngleFromVertical}:\rayLength)$);

    % Full version--------------------------------------
    \draw[exitArrow] (exitPoint) -- (exitRayEnd);
    % Student version--------------------------------------
    % \drawSegmentedRay{exitPoint}{exitRayEnd}{exitNoArrow}{exitEndArrow}

    % Normal at bottom
    \draw[dashed, black, thick] ($(exitPoint)+(0,-1.6cm)$) -- ($(exitPoint)+(0,1.6cm)$);

    %--------------------------------------
    % Angle labels
    %--------------------------------------
    % Top angles
    \coordinate (normalUp) at ($(topCenter)+(0,1)$);
    \coordinate (normalDown) at ($(topCenter)+(0,-1)$);
    \coordinate (refractDir) at ($(topCenter)+({\refractAngleFromVertical}:1.5cm)$);

    \pic [draw, -, angle radius=10mm, pic text = $i^\circ$, angle eccentricity = 1.2]
    {angle = normalUp--topCenter--incEnd};
    \node[anchor=west] at ($(topCenter)+(10mm, 6mm)$) {\LARGE \incLabel};

    \pic [draw, -, angle radius=10mm, pic text = $r^\circ$, angle eccentricity = 1.2]
    {angle = normalDown--topCenter--refractDir};
    \node[anchor=west] at ($(topCenter)+(10mm,-6mm)$) {\LARGE \refLabel};

    % Bottom angles
    \coordinate (exitNormalUp) at ($(exitPoint)+(0,1)$);
    \coordinate (exitNormalDown) at ($(exitPoint)+(0,-1)$);
    \coordinate (internalRefractDir) at ($(refractDir)$);
    \coordinate (exitRayDir) at ($(exitPoint)+({\exitAngleFromVertical}:1.5cm)$);

    \pic [draw, -, angle radius=10mm, pic text = $i^\circ$, angle eccentricity = 1.2] {angle = exitNormalDown--exitPoint--exitRayDir};
    \node[anchor=west] at ($(exitPoint)+(-35mm, -6mm)$) {\LARGE \exitIncLabel};

    \pic [draw, -, angle radius=10mm, pic text = $r^\circ$, angle eccentricity = 1.2] {angle = exitNormalUp--exitPoint--refractDir};
    \node[anchor=west] at ($(exitPoint)+(-35mm,+6mm)$) {\LARGE \exitRefLabel};

    %--------------------------------------
    % Caption
    %--------------------------------------
    \node[anchor=west, text width=9cm, align=flush left, font=\LARGE, inner sep=0pt] at ($(blockNE)+(0.8cm,0cm)$) {\Large \refractionDescription};

    \node[anchor=north west, text width=9cm, align=flush left, font=\LARGE,inner sep=0pt] at ($(blockE)+(0.8cm,-0.1cm)$) {\Large \refractionDescriptionExit};

    %--------------------------------------
    % Heading
    %--------------------------------------
    \node[blue!5!black, above right] at ($(topCenter)+(0,3cm)$) {\Huge Refraction \LARGE \incLabel};

\end{tikzpicture}
\end{document}
