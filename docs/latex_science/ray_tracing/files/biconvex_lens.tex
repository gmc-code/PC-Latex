\documentclass[a4paper,landscape]{article}
\usepackage[margin=0cm]{geometry}
\usepackage{tikz}
\usetikzlibrary{arrows.meta,calc,shapes.geometric}
\pagestyle{empty}

\begin{document}
\begin{tikzpicture}[remember picture,overlay] % keep options simple here

    % Draw page crosshairs
    % Light grey, very thin crosshairs
    \draw[gray!50, line width=0.3pt]
      ([xshift=0.5\paperwidth]current page.south west) --
      ([xshift=0.5\paperwidth]current page.north west);
    \draw[gray!50, line width=0.3pt]
      ([yshift=0.5\paperheight]current page.south west) --
      ([yshift=0.5\paperheight]current page.south east);


  % SHIFT origin to page center and set 1cm = 1 unit
  \begin{scope}[shift={(current page.center)}, x=1cm, y=1cm, >=Stealth]

    % PARAMETERS
    \def\f{7.5}        % focal length (cm)
    \def\rayEnd{12.5}  % ray end x (cm) roughly 5 cm past focus
    \def\dy{0.95}      % spacing between rays (cm)
    \def\rdy{0.85}      % starting height of refracted rays (cm)
    \def\lensYR{2.5}   % lens half-height (cm)
    \def\R{7.35}       % radius of curvature from lensmaker formula
    \pgfmathsetmacro{\lensXR}{0.2 + \R - sqrt(\R^2 - \lensYR^2)} % half-sagitta bulge at center  -- by testing added 0.2 to match curvature of circle
    %https://www.edmundoptics.com.au/knowledge-center/tech-tools/sag/?srsltid=AfmBOoriU3DzOuHtTMaNQiDByYJ3AmudXaeIipH2bTqMLz8YtFUxiIPJ
    \def\lensXC{0.25}  % lens half-thickness at ends
    % \def\lensYS{0.8}   % control height for benzier curve for lens
    \pgfmathsetmacro{\lensYS}{0.4*\lensYR}

    % lens
    \draw[fill=blue!5!white, opacity=0.2, thick]
        (-\lensXC, -\lensYR)
          .. controls (-\lensXR-\lensXC, -\lensYS) and (-\lensXR-\lensXC, \lensYS) ..
          (-\lensXC, \lensYR)
        -- (\lensXC, \lensYR)
          .. controls (\lensXR+\lensXC, \lensYS) and (\lensXR+\lensXC, -\lensYS) ..
          (\lensXC, -\lensYR)
        -- cycle;

    \node[blue!5!black, above right] at (0,\lensYR+1.2) {\Huge Biconvex lens};

    % Focal point on right
    \coordinate (F) at (\f,0);
    \draw[fill=black] (F) circle (0.07) node[below=12pt] {\Large $Focus$};

    % Incoming parallel rays (with arrowhead halfway before lens)
    \foreach \y in {- \dy, 0, \dy} {
      % Dashed parallel rays with arrow
      % \draw[gray, thick, dash pattern=on 1cm off 3cm] (-12,\y) -- (-6,\y);
      \draw[gray, thick, ->] (-12,\y) -- (-6,\y);
      % Continue to lens plane (solid line)
      \draw[gray, thick] (-6,\y) -- (0.2-\lensXR-\lensXC,\y);
    }


    % Refracted rays: line through (0,\y) and (f,0), extend to rayEnd
    \foreach \y in {- \rdy, 0, \rdy} {
      % compute numeric y coordinate at x=\rayEnd using y_end = y0*(1 - rayEnd/f)
      \pgfmathsetmacro{\yend}{\y*(1 - \rayEnd/\f)}
      % draw from lens plane to focus, then from focus to end (arrow)
      \draw[gray, thick] (-0.2+\lensXR+\lensXC,\y) -- (F);
      \draw[gray, thick, ->] (F) -- (\rayEnd,\yend);
    }

    % focal length indicator
    \draw[<->] (0,-\lensYR-0.6) -- (\f,-\lensYR-0.6) node[midway,below] {\Large $f=7.5$ cm};

    % circle to check curvature
    % \coordinate (F2) at (\f - \lensXR -\lensXC,0);
    % \draw[red, loosely dashed] (F2) circle (\R);

  \end{scope}

\end{tikzpicture}
\end{document}
