\documentclass[a4paper,landscape]{article}
\usepackage{amsmath}
\usepackage[margin=0cm]{geometry}
\usepackage{tikz}
\usetikzlibrary{
  arrows.meta,
  calc,
  shapes.geometric,
  decorations.markings,
  angles,
  quotes
}
\pagestyle{empty}

% Define angle variables
\def\angleIncidence{30}
\def\angleRefraction{20}
% Angle labels
\newcommand{\incLabel}{$i=\angleIncidence^\circ$}
\newcommand{\refLabel}{$r=\angleRefraction^\circ$}
% Description variable
\def\refractionDescription{Light travelling from air to acrylic bends towards the normal.}

% Declare layers
\pgfdeclarelayer{crosslayer}
\pgfdeclarelayer{main}
\pgfdeclarelayer{toplayer}
\pgfsetlayers{main,crosslayer,toplayer}


\begin{document}
\begin{tikzpicture}[remember picture,overlay]

  % Layer 1: Crosshairs
  \begin{pgfonlayer}{crosslayer}
    \draw[gray!50, line width=0.3pt]
      ([xshift=0.5\paperwidth]current page.south west) --
      ([xshift=0.5\paperwidth]current page.north west);
    \draw[gray!50, line width=0.3pt]
      ([yshift=0.5\paperheight]current page.south west) --
      ([yshift=0.5\paperheight]current page.south east);
  \end{pgfonlayer}

 % Layer 2: Rectangle and rays
    \begin{pgfonlayer}{main}
    % Center and block coordinates
    \coordinate (center) at ([xshift=0.5\paperwidth, yshift=0.5\paperheight]current page.south west);
    \path let \p1 = (center) in
      coordinate (blockSW) at (\x1-38.5mm, \y1-21mm)
      coordinate (blockNE) at (\x1+38.5mm, \y1+21mm)
      coordinate (topCenter) at (\x1, \y1+21mm);

     % Draw block
    \draw[gray, fill=blue!3!white, thin] (blockSW) rectangle (blockNE);

    % Draw Heading
    \node[blue!5!black, above right] at ($(topCenter)+(0,3cm)$) {\Huge Refraction \incLabel};

    % Normal line (4 cm dashed)
    \draw[black, dashed, line width=1pt]
      ($(topCenter)+(0,-1.6cm)$) -- ($(topCenter)+(0,1.6cm)$);

    % Incident ray
    \coordinate (incidentEnd) at ($(topCenter)+( {90 + \angleIncidence} :4cm)$);
    \draw[gray, postaction={decorate},
      decoration={markings, mark=at position 0.8 with {\arrowreversed[scale=1.5]{Latex}}}]
      (topCenter) -- (incidentEnd);

    % Refracted ray
    \coordinate (refractEnd) at ($(topCenter)+( { \angleRefraction - 90 } :4cm)$);
    \draw[-{Latex[length=3mm]}, gray, thick]
      (topCenter) -- (refractEnd);

    % Reference points for angle markings
    \coordinate (normalUp) at ($(topCenter)+(0,1)$);
    \coordinate (normalDown) at ($(topCenter)+(0,-1)$);

    % Angle of incidence
    \pic [draw, -, angle radius=10mm]
      {angle = normalUp--topCenter--incidentEnd};
      \node at ($(topCenter)+ (18mm, 6mm)$) {\LARGE \incLabel};

    % Angle of refraction
    \pic [draw, -, angle radius=10mm]
      {angle = normalDown--topCenter--refractEnd};
    \node at ($(topCenter)+ (18mm, -6mm)$) {\LARGE \refLabel};

   \end{pgfonlayer}

    \begin{pgfonlayer}{toplayer}
        % Description below block
        \node[fill=white, text width=12cm, align=center, font=\LARGE]
            at ($(center)+(0,-35mm)$) {\refractionDescription};
    \end{pgfonlayer}

\end{tikzpicture}
\end{document}
