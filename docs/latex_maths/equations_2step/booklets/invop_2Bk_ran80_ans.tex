\documentclass[12pt]{article}
\usepackage{tikz}
\usepackage{amsmath}
% Underlining package
\usepackage[normalem]{ulem} % [normalem] prevents the package from changing the default behavior of `\\emph` to underline.
\usepackage[a4paper, portrait, margin=1cm]{geometry}
\usepackage{multicol}
\usepackage{fancyhdr}

\def \HeadingAnswers {\section*{\Large Name: \underline{\hspace{8cm}} \hfill Date: \underline{\hspace{3cm}}} \vspace{-3mm}
{Inverse operations: Answers} \vspace{1pt}\hrule}

% raise footer with page number; no header
\fancypagestyle{myfancypagestyle}{
  \fancyhf{}% clear all header and footer fields
  \renewcommand{\headrulewidth}{0pt} % no rule under header
  \fancyfoot[C] {\thepage} \setlength{\footskip}{6pt} % raise page number 6pt
}
\pagestyle{myfancypagestyle}  % apply myfancypagestyle
\newcounter{minipagecount}
\begin{document}
\HeadingAnswers
\vspace{1mm}
\begin{multicols}{2}
  \refstepcounter{minipagecount} % increments the counter minipagecount by one.
\noindent{(\theminipagecount)}\hspace{0.1mm} % By default, LaTeX indents the first line of a new paragraph, but \noindent overrides this
% and inserts the current value of the minipagecount counter, enclosed in parentheses
\begin{minipage}[t]{0.45\textwidth} % The [t] option aligns the top of the minipage with the baseline of the surrounding text.
    \vspace{-26pt}  % moves the content of the minipage up, reducing the space between the minipage content and the preceding text.
    \raggedright %  the text lines up on the left side, but the right side will be ragged.
    \begin{align*} % The * align environment does not number the equations- Each line is aligned at the & symbol
        4(x + 9) &= 68\\
        \frac{4(x+9)}{4} &= \frac{68}{4}\\
        x + 9 &= 17\\
        x + 9 - 9 &= 17 - 9\\
        x &= 8\\
    \end{align*}
\end{minipage}\refstepcounter{minipagecount} % increments the counter minipagecount by one.
\noindent{(\theminipagecount)}\hspace{0.1mm} % By default, LaTeX indents the first line of a new paragraph, but \noindent overrides this
% and inserts the current value of the minipagecount counter, enclosed in parentheses
\begin{minipage}[t]{0.45\textwidth} % The [t] option aligns the top of the minipage with the baseline of the surrounding text.
    \vspace{-26pt}  % moves the content of the minipage up, reducing the space between the minipage content and the preceding text.
    \raggedright %  the text lines up on the left side, but the right side will be ragged.
    \begin{align*} % The * align environment does not number the equations- Each line is aligned at the & symbol
        \frac{x}{3} + 6 &= 10\\
        \frac{x}{3} + 6 - 6 &= 10 - 6\\
        \frac{x}{3} &= 4\\
        \frac{x}{3} \times3 &= 4 \times3\\
        x &= 12\\
    \end{align*}
\end{minipage}\refstepcounter{minipagecount} % increments the counter minipagecount by one.
\noindent{(\theminipagecount)}\hspace{0.1mm} % By default, LaTeX indents the first line of a new paragraph, but \noindent overrides this
% and inserts the current value of the minipagecount counter, enclosed in parentheses
\begin{minipage}[t]{0.45\textwidth} % The [t] option aligns the top of the minipage with the baseline of the surrounding text.
    \vspace{-26pt}  % moves the content of the minipage up, reducing the space between the minipage content and the preceding text.
    \raggedright %  the text lines up on the left side, but the right side will be ragged.
    \begin{align*} % The * align environment does not number the equations- Each line is aligned at the & symbol
        \frac{x + 5}{10} &= 8\\
        \frac{x + 5}{10} \times10 &= 8 \times10\\
        x + 5 &= 80\\
        x + 5 - 5 &= 80 - 5\\
        x &= 75\\
    \end{align*}
\end{minipage}\refstepcounter{minipagecount} % increments the counter minipagecount by one.
\noindent{(\theminipagecount)}\hspace{0.1mm} % By default, LaTeX indents the first line of a new paragraph, but \noindent overrides this
% and inserts the current value of the minipagecount counter, enclosed in parentheses
\begin{minipage}[t]{0.45\textwidth} % The [t] option aligns the top of the minipage with the baseline of the surrounding text.
    \vspace{-26pt}  % moves the content of the minipage up, reducing the space between the minipage content and the preceding text.
    \raggedright %  the text lines up on the left side, but the right side will be ragged.
    \begin{align*} % The * align environment does not number the equations- Each line is aligned at the & symbol
        \frac{x + 7}{6} &= 9\\
        \frac{x + 7}{6} \times6 &= 9 \times6\\
        x + 7 &= 54\\
        x + 7 - 7 &= 54 - 7\\
        x &= 47\\
    \end{align*}
\end{minipage}\refstepcounter{minipagecount} % increments the counter minipagecount by one.
\noindent{(\theminipagecount)}\hspace{0.1mm} % By default, LaTeX indents the first line of a new paragraph, but \noindent overrides this
% and inserts the current value of the minipagecount counter, enclosed in parentheses
\begin{minipage}[t]{0.45\textwidth} % The [t] option aligns the top of the minipage with the baseline of the surrounding text.
    \vspace{-26pt}  % moves the content of the minipage up, reducing the space between the minipage content and the preceding text.
    \raggedright %  the text lines up on the left side, but the right side will be ragged.
    \begin{align*} % The * align environment does not number the equations- Each line is aligned at the & symbol
        4x - 7 &= 33\\
        4x - 7 + 7 &= 33 + 7\\
        4x &= 40\\
        \frac{4x}{4} &= \frac{40}{4}\\
        x &= 10\\
    \end{align*}
\end{minipage}\columnbreak
    \refstepcounter{minipagecount} % increments the counter minipagecount by one.
\noindent{(\theminipagecount)}\hspace{0.1mm} % By default, LaTeX indents the first line of a new paragraph, but \noindent overrides this
% and inserts the current value of the minipagecount counter, enclosed in parentheses
\begin{minipage}[t]{0.45\textwidth} % The [t] option aligns the top of the minipage with the baseline of the surrounding text.
    \vspace{-26pt}  % moves the content of the minipage up, reducing the space between the minipage content and the preceding text.
    \raggedright %  the text lines up on the left side, but the right side will be ragged.
    \begin{align*} % The * align environment does not number the equations- Each line is aligned at the & symbol
        \frac{x + 1}{7} &= 5\\
        \frac{x + 1}{7} \times7 &= 5 \times7\\
        x + 1 &= 35\\
        x + 1 - 1 &= 35 - 1\\
        x &= 34\\
    \end{align*}
\end{minipage}\refstepcounter{minipagecount} % increments the counter minipagecount by one.
\noindent{(\theminipagecount)}\hspace{0.1mm} % By default, LaTeX indents the first line of a new paragraph, but \noindent overrides this
% and inserts the current value of the minipagecount counter, enclosed in parentheses
\begin{minipage}[t]{0.45\textwidth} % The [t] option aligns the top of the minipage with the baseline of the surrounding text.
    \vspace{-26pt}  % moves the content of the minipage up, reducing the space between the minipage content and the preceding text.
    \raggedright %  the text lines up on the left side, but the right side will be ragged.
    \begin{align*} % The * align environment does not number the equations- Each line is aligned at the & symbol
        8x - 6 &= 10\\
        8x - 6 + 6 &= 10 + 6\\
        8x &= 16\\
        \frac{8x}{8} &= \frac{16}{8}\\
        x &= 2\\
    \end{align*}
\end{minipage}\refstepcounter{minipagecount} % increments the counter minipagecount by one.
\noindent{(\theminipagecount)}\hspace{0.1mm} % By default, LaTeX indents the first line of a new paragraph, but \noindent overrides this
% and inserts the current value of the minipagecount counter, enclosed in parentheses
\begin{minipage}[t]{0.45\textwidth} % The [t] option aligns the top of the minipage with the baseline of the surrounding text.
    \vspace{-26pt}  % moves the content of the minipage up, reducing the space between the minipage content and the preceding text.
    \raggedright %  the text lines up on the left side, but the right side will be ragged.
    \begin{align*} % The * align environment does not number the equations- Each line is aligned at the & symbol
        \frac{x}{5} - 2 &= 7\\
        \frac{x}{5} - 2 + 2 &= 7 + 2\\
        \frac{x}{5} &= 9\\
        \frac{x}{5} \times5 &= 9 \times5\\
        x &= 45\\
    \end{align*}
\end{minipage}\refstepcounter{minipagecount} % increments the counter minipagecount by one.
\noindent{(\theminipagecount)}\hspace{0.1mm} % By default, LaTeX indents the first line of a new paragraph, but \noindent overrides this
% and inserts the current value of the minipagecount counter, enclosed in parentheses
\begin{minipage}[t]{0.45\textwidth} % The [t] option aligns the top of the minipage with the baseline of the surrounding text.
    \vspace{-26pt}  % moves the content of the minipage up, reducing the space between the minipage content and the preceding text.
    \raggedright %  the text lines up on the left side, but the right side will be ragged.
    \begin{align*} % The * align environment does not number the equations- Each line is aligned at the & symbol
        \frac{x + 4}{10} &= 10\\
        \frac{x + 4}{10} \times10 &= 10 \times10\\
        x + 4 &= 100\\
        x + 4 - 4 &= 100 - 4\\
        x &= 96\\
    \end{align*}
\end{minipage}\refstepcounter{minipagecount} % increments the counter minipagecount by one.
\noindent{(\theminipagecount)}\hspace{0.1mm} % By default, LaTeX indents the first line of a new paragraph, but \noindent overrides this
% and inserts the current value of the minipagecount counter, enclosed in parentheses
\begin{minipage}[t]{0.45\textwidth} % The [t] option aligns the top of the minipage with the baseline of the surrounding text.
    \vspace{-26pt}  % moves the content of the minipage up, reducing the space between the minipage content and the preceding text.
    \raggedright %  the text lines up on the left side, but the right side will be ragged.
    \begin{align*} % The * align environment does not number the equations- Each line is aligned at the & symbol
        \frac{x - 2}{8} &= 8\\
        \frac{x - 2}{8} \times8 &= 8 \times8\\
        x - 2 &= 64\\
        x - 2 + 2 &= 64 + 2\\
        x &= 18\\
    \end{align*}
\end{minipage}\newpage
    \refstepcounter{minipagecount} % increments the counter minipagecount by one.
\noindent{(\theminipagecount)}\hspace{0.1mm} % By default, LaTeX indents the first line of a new paragraph, but \noindent overrides this
% and inserts the current value of the minipagecount counter, enclosed in parentheses
\begin{minipage}[t]{0.45\textwidth} % The [t] option aligns the top of the minipage with the baseline of the surrounding text.
    \vspace{-26pt}  % moves the content of the minipage up, reducing the space between the minipage content and the preceding text.
    \raggedright %  the text lines up on the left side, but the right side will be ragged.
    \begin{align*} % The * align environment does not number the equations- Each line is aligned at the & symbol
        \frac{x}{7} + 5 &= 11\\
        \frac{x}{7} + 5 - 5 &= 11 - 5\\
        \frac{x}{7} &= 6\\
        \frac{x}{7} \times7 &= 6 \times7\\
        x &= 42\\
    \end{align*}
\end{minipage}\refstepcounter{minipagecount} % increments the counter minipagecount by one.
\noindent{(\theminipagecount)}\hspace{0.1mm} % By default, LaTeX indents the first line of a new paragraph, but \noindent overrides this
% and inserts the current value of the minipagecount counter, enclosed in parentheses
\begin{minipage}[t]{0.45\textwidth} % The [t] option aligns the top of the minipage with the baseline of the surrounding text.
    \vspace{-26pt}  % moves the content of the minipage up, reducing the space between the minipage content and the preceding text.
    \raggedright %  the text lines up on the left side, but the right side will be ragged.
    \begin{align*} % The * align environment does not number the equations- Each line is aligned at the & symbol
        10(x - 6) &= 30\\
        \frac{10(x-6)}{10} &= \frac{30}{10}\\
        x - 6 &= 3\\
        x - 6 + 6 &= 3 + 6\\
        x &= 9\\
    \end{align*}
\end{minipage}\refstepcounter{minipagecount} % increments the counter minipagecount by one.
\noindent{(\theminipagecount)}\hspace{0.1mm} % By default, LaTeX indents the first line of a new paragraph, but \noindent overrides this
% and inserts the current value of the minipagecount counter, enclosed in parentheses
\begin{minipage}[t]{0.45\textwidth} % The [t] option aligns the top of the minipage with the baseline of the surrounding text.
    \vspace{-26pt}  % moves the content of the minipage up, reducing the space between the minipage content and the preceding text.
    \raggedright %  the text lines up on the left side, but the right side will be ragged.
    \begin{align*} % The * align environment does not number the equations- Each line is aligned at the & symbol
        5x + 6 &= 26\\
        5x + 6 - 6 &= 26 - 6\\
        5x &= 20\\
        \frac{5x}{5} &= \frac{20}{5}\\
        x &= 4\\
    \end{align*}
\end{minipage}\refstepcounter{minipagecount} % increments the counter minipagecount by one.
\noindent{(\theminipagecount)}\hspace{0.1mm} % By default, LaTeX indents the first line of a new paragraph, but \noindent overrides this
% and inserts the current value of the minipagecount counter, enclosed in parentheses
\begin{minipage}[t]{0.45\textwidth} % The [t] option aligns the top of the minipage with the baseline of the surrounding text.
    \vspace{-26pt}  % moves the content of the minipage up, reducing the space between the minipage content and the preceding text.
    \raggedright %  the text lines up on the left side, but the right side will be ragged.
    \begin{align*} % The * align environment does not number the equations- Each line is aligned at the & symbol
        3(x - 2) &= 3\\
        \frac{3(x-2)}{3} &= \frac{3}{3}\\
        x - 2 &= 1\\
        x - 2 + 2 &= 1 + 2\\
        x &= 3\\
    \end{align*}
\end{minipage}\refstepcounter{minipagecount} % increments the counter minipagecount by one.
\noindent{(\theminipagecount)}\hspace{0.1mm} % By default, LaTeX indents the first line of a new paragraph, but \noindent overrides this
% and inserts the current value of the minipagecount counter, enclosed in parentheses
\begin{minipage}[t]{0.45\textwidth} % The [t] option aligns the top of the minipage with the baseline of the surrounding text.
    \vspace{-26pt}  % moves the content of the minipage up, reducing the space between the minipage content and the preceding text.
    \raggedright %  the text lines up on the left side, but the right side will be ragged.
    \begin{align*} % The * align environment does not number the equations- Each line is aligned at the & symbol
        \frac{x}{8} + 4 &= 6\\
        \frac{x}{8} + 4 - 4 &= 6 - 4\\
        \frac{x}{8} &= 2\\
        \frac{x}{8} \times8 &= 2 \times8\\
        x &= 16\\
    \end{align*}
\end{minipage}\columnbreak
    \refstepcounter{minipagecount} % increments the counter minipagecount by one.
\noindent{(\theminipagecount)}\hspace{0.1mm} % By default, LaTeX indents the first line of a new paragraph, but \noindent overrides this
% and inserts the current value of the minipagecount counter, enclosed in parentheses
\begin{minipage}[t]{0.45\textwidth} % The [t] option aligns the top of the minipage with the baseline of the surrounding text.
    \vspace{-26pt}  % moves the content of the minipage up, reducing the space between the minipage content and the preceding text.
    \raggedright %  the text lines up on the left side, but the right side will be ragged.
    \begin{align*} % The * align environment does not number the equations- Each line is aligned at the & symbol
        \frac{x - 1}{8} &= 2\\
        \frac{x - 1}{8} \times8 &= 2 \times8\\
        x - 1 &= 16\\
        x - 1 + 1 &= 16 + 1\\
        x &= 9\\
    \end{align*}
\end{minipage}\refstepcounter{minipagecount} % increments the counter minipagecount by one.
\noindent{(\theminipagecount)}\hspace{0.1mm} % By default, LaTeX indents the first line of a new paragraph, but \noindent overrides this
% and inserts the current value of the minipagecount counter, enclosed in parentheses
\begin{minipage}[t]{0.45\textwidth} % The [t] option aligns the top of the minipage with the baseline of the surrounding text.
    \vspace{-26pt}  % moves the content of the minipage up, reducing the space between the minipage content and the preceding text.
    \raggedright %  the text lines up on the left side, but the right side will be ragged.
    \begin{align*} % The * align environment does not number the equations- Each line is aligned at the & symbol
        7x - 3 &= 25\\
        7x - 3 + 3 &= 25 + 3\\
        7x &= 28\\
        \frac{7x}{7} &= \frac{28}{7}\\
        x &= 4\\
    \end{align*}
\end{minipage}\refstepcounter{minipagecount} % increments the counter minipagecount by one.
\noindent{(\theminipagecount)}\hspace{0.1mm} % By default, LaTeX indents the first line of a new paragraph, but \noindent overrides this
% and inserts the current value of the minipagecount counter, enclosed in parentheses
\begin{minipage}[t]{0.45\textwidth} % The [t] option aligns the top of the minipage with the baseline of the surrounding text.
    \vspace{-26pt}  % moves the content of the minipage up, reducing the space between the minipage content and the preceding text.
    \raggedright %  the text lines up on the left side, but the right side will be ragged.
    \begin{align*} % The * align environment does not number the equations- Each line is aligned at the & symbol
        \frac{x + 1}{6} &= 10\\
        \frac{x + 1}{6} \times6 &= 10 \times6\\
        x + 1 &= 60\\
        x + 1 - 1 &= 60 - 1\\
        x &= 59\\
    \end{align*}
\end{minipage}\refstepcounter{minipagecount} % increments the counter minipagecount by one.
\noindent{(\theminipagecount)}\hspace{0.1mm} % By default, LaTeX indents the first line of a new paragraph, but \noindent overrides this
% and inserts the current value of the minipagecount counter, enclosed in parentheses
\begin{minipage}[t]{0.45\textwidth} % The [t] option aligns the top of the minipage with the baseline of the surrounding text.
    \vspace{-26pt}  % moves the content of the minipage up, reducing the space between the minipage content and the preceding text.
    \raggedright %  the text lines up on the left side, but the right side will be ragged.
    \begin{align*} % The * align environment does not number the equations- Each line is aligned at the & symbol
        \frac{x}{9} - 3 &= -1\\
        \frac{x}{9} - 3 + 3 &= -1 + 3\\
        \frac{x}{9} &= 2\\
        \frac{x}{9} \times9 &= 2 \times9\\
        x &= 18\\
    \end{align*}
\end{minipage}\refstepcounter{minipagecount} % increments the counter minipagecount by one.
\noindent{(\theminipagecount)}\hspace{0.1mm} % By default, LaTeX indents the first line of a new paragraph, but \noindent overrides this
% and inserts the current value of the minipagecount counter, enclosed in parentheses
\begin{minipage}[t]{0.45\textwidth} % The [t] option aligns the top of the minipage with the baseline of the surrounding text.
    \vspace{-26pt}  % moves the content of the minipage up, reducing the space between the minipage content and the preceding text.
    \raggedright %  the text lines up on the left side, but the right side will be ragged.
    \begin{align*} % The * align environment does not number the equations- Each line is aligned at the & symbol
        7(x - 5) &= 14\\
        \frac{7(x-5)}{7} &= \frac{14}{7}\\
        x - 5 &= 2\\
        x - 5 + 5 &= 2 + 5\\
        x &= 7\\
    \end{align*}
\end{minipage}\newpage
    \refstepcounter{minipagecount} % increments the counter minipagecount by one.
\noindent{(\theminipagecount)}\hspace{0.1mm} % By default, LaTeX indents the first line of a new paragraph, but \noindent overrides this
% and inserts the current value of the minipagecount counter, enclosed in parentheses
\begin{minipage}[t]{0.45\textwidth} % The [t] option aligns the top of the minipage with the baseline of the surrounding text.
    \vspace{-26pt}  % moves the content of the minipage up, reducing the space between the minipage content and the preceding text.
    \raggedright %  the text lines up on the left side, but the right side will be ragged.
    \begin{align*} % The * align environment does not number the equations- Each line is aligned at the & symbol
        \frac{x - 5}{5} &= 2\\
        \frac{x - 5}{5} \times5 &= 2 \times5\\
        x - 5 &= 10\\
        x - 5 + 5 &= 10 + 5\\
        x &= 30\\
    \end{align*}
\end{minipage}\refstepcounter{minipagecount} % increments the counter minipagecount by one.
\noindent{(\theminipagecount)}\hspace{0.1mm} % By default, LaTeX indents the first line of a new paragraph, but \noindent overrides this
% and inserts the current value of the minipagecount counter, enclosed in parentheses
\begin{minipage}[t]{0.45\textwidth} % The [t] option aligns the top of the minipage with the baseline of the surrounding text.
    \vspace{-26pt}  % moves the content of the minipage up, reducing the space between the minipage content and the preceding text.
    \raggedright %  the text lines up on the left side, but the right side will be ragged.
    \begin{align*} % The * align environment does not number the equations- Each line is aligned at the & symbol
        8(x - 2) &= 8\\
        \frac{8(x-2)}{8} &= \frac{8}{8}\\
        x - 2 &= 1\\
        x - 2 + 2 &= 1 + 2\\
        x &= 3\\
    \end{align*}
\end{minipage}\refstepcounter{minipagecount} % increments the counter minipagecount by one.
\noindent{(\theminipagecount)}\hspace{0.1mm} % By default, LaTeX indents the first line of a new paragraph, but \noindent overrides this
% and inserts the current value of the minipagecount counter, enclosed in parentheses
\begin{minipage}[t]{0.45\textwidth} % The [t] option aligns the top of the minipage with the baseline of the surrounding text.
    \vspace{-26pt}  % moves the content of the minipage up, reducing the space between the minipage content and the preceding text.
    \raggedright %  the text lines up on the left side, but the right side will be ragged.
    \begin{align*} % The * align environment does not number the equations- Each line is aligned at the & symbol
        \frac{x + 3}{2} &= 9\\
        \frac{x + 3}{2} \times2 &= 9 \times2\\
        x + 3 &= 18\\
        x + 3 - 3 &= 18 - 3\\
        x &= 15\\
    \end{align*}
\end{minipage}\refstepcounter{minipagecount} % increments the counter minipagecount by one.
\noindent{(\theminipagecount)}\hspace{0.1mm} % By default, LaTeX indents the first line of a new paragraph, but \noindent overrides this
% and inserts the current value of the minipagecount counter, enclosed in parentheses
\begin{minipage}[t]{0.45\textwidth} % The [t] option aligns the top of the minipage with the baseline of the surrounding text.
    \vspace{-26pt}  % moves the content of the minipage up, reducing the space between the minipage content and the preceding text.
    \raggedright %  the text lines up on the left side, but the right side will be ragged.
    \begin{align*} % The * align environment does not number the equations- Each line is aligned at the & symbol
        10x + 6 &= 16\\
        10x + 6 - 6 &= 16 - 6\\
        10x &= 10\\
        \frac{10x}{10} &= \frac{10}{10}\\
        x &= 1\\
    \end{align*}
\end{minipage}\refstepcounter{minipagecount} % increments the counter minipagecount by one.
\noindent{(\theminipagecount)}\hspace{0.1mm} % By default, LaTeX indents the first line of a new paragraph, but \noindent overrides this
% and inserts the current value of the minipagecount counter, enclosed in parentheses
\begin{minipage}[t]{0.45\textwidth} % The [t] option aligns the top of the minipage with the baseline of the surrounding text.
    \vspace{-26pt}  % moves the content of the minipage up, reducing the space between the minipage content and the preceding text.
    \raggedright %  the text lines up on the left side, but the right side will be ragged.
    \begin{align*} % The * align environment does not number the equations- Each line is aligned at the & symbol
        \frac{x}{10} + 10 &= 15\\
        \frac{x}{10} + 10 - 10 &= 15 - 10\\
        \frac{x}{10} &= 5\\
        \frac{x}{10} \times10 &= 5 \times10\\
        x &= 50\\
    \end{align*}
\end{minipage}\columnbreak
    \refstepcounter{minipagecount} % increments the counter minipagecount by one.
\noindent{(\theminipagecount)}\hspace{0.1mm} % By default, LaTeX indents the first line of a new paragraph, but \noindent overrides this
% and inserts the current value of the minipagecount counter, enclosed in parentheses
\begin{minipage}[t]{0.45\textwidth} % The [t] option aligns the top of the minipage with the baseline of the surrounding text.
    \vspace{-26pt}  % moves the content of the minipage up, reducing the space between the minipage content and the preceding text.
    \raggedright %  the text lines up on the left side, but the right side will be ragged.
    \begin{align*} % The * align environment does not number the equations- Each line is aligned at the & symbol
        \frac{x}{9} + 3 &= 7\\
        \frac{x}{9} + 3 - 3 &= 7 - 3\\
        \frac{x}{9} &= 4\\
        \frac{x}{9} \times9 &= 4 \times9\\
        x &= 36\\
    \end{align*}
\end{minipage}\refstepcounter{minipagecount} % increments the counter minipagecount by one.
\noindent{(\theminipagecount)}\hspace{0.1mm} % By default, LaTeX indents the first line of a new paragraph, but \noindent overrides this
% and inserts the current value of the minipagecount counter, enclosed in parentheses
\begin{minipage}[t]{0.45\textwidth} % The [t] option aligns the top of the minipage with the baseline of the surrounding text.
    \vspace{-26pt}  % moves the content of the minipage up, reducing the space between the minipage content and the preceding text.
    \raggedright %  the text lines up on the left side, but the right side will be ragged.
    \begin{align*} % The * align environment does not number the equations- Each line is aligned at the & symbol
        2(x + 7) &= 30\\
        \frac{2(x+7)}{2} &= \frac{30}{2}\\
        x + 7 &= 15\\
        x + 7 - 7 &= 15 - 7\\
        x &= 8\\
    \end{align*}
\end{minipage}\refstepcounter{minipagecount} % increments the counter minipagecount by one.
\noindent{(\theminipagecount)}\hspace{0.1mm} % By default, LaTeX indents the first line of a new paragraph, but \noindent overrides this
% and inserts the current value of the minipagecount counter, enclosed in parentheses
\begin{minipage}[t]{0.45\textwidth} % The [t] option aligns the top of the minipage with the baseline of the surrounding text.
    \vspace{-26pt}  % moves the content of the minipage up, reducing the space between the minipage content and the preceding text.
    \raggedright %  the text lines up on the left side, but the right side will be ragged.
    \begin{align*} % The * align environment does not number the equations- Each line is aligned at the & symbol
        2x - 6 &= 12\\
        2x - 6 + 6 &= 12 + 6\\
        2x &= 18\\
        \frac{2x}{2} &= \frac{18}{2}\\
        x &= 9\\
    \end{align*}
\end{minipage}\refstepcounter{minipagecount} % increments the counter minipagecount by one.
\noindent{(\theminipagecount)}\hspace{0.1mm} % By default, LaTeX indents the first line of a new paragraph, but \noindent overrides this
% and inserts the current value of the minipagecount counter, enclosed in parentheses
\begin{minipage}[t]{0.45\textwidth} % The [t] option aligns the top of the minipage with the baseline of the surrounding text.
    \vspace{-26pt}  % moves the content of the minipage up, reducing the space between the minipage content and the preceding text.
    \raggedright %  the text lines up on the left side, but the right side will be ragged.
    \begin{align*} % The * align environment does not number the equations- Each line is aligned at the & symbol
        \frac{x - 2}{5} &= 4\\
        \frac{x - 2}{5} \times5 &= 4 \times5\\
        x - 2 &= 20\\
        x - 2 + 2 &= 20 + 2\\
        x &= 12\\
    \end{align*}
\end{minipage}\refstepcounter{minipagecount} % increments the counter minipagecount by one.
\noindent{(\theminipagecount)}\hspace{0.1mm} % By default, LaTeX indents the first line of a new paragraph, but \noindent overrides this
% and inserts the current value of the minipagecount counter, enclosed in parentheses
\begin{minipage}[t]{0.45\textwidth} % The [t] option aligns the top of the minipage with the baseline of the surrounding text.
    \vspace{-26pt}  % moves the content of the minipage up, reducing the space between the minipage content and the preceding text.
    \raggedright %  the text lines up on the left side, but the right side will be ragged.
    \begin{align*} % The * align environment does not number the equations- Each line is aligned at the & symbol
        \frac{x - 8}{2} &= 7\\
        \frac{x - 8}{2} \times2 &= 7 \times2\\
        x - 8 &= 14\\
        x - 8 + 8 &= 14 + 8\\
        x &= 24\\
    \end{align*}
\end{minipage}\newpage
    \refstepcounter{minipagecount} % increments the counter minipagecount by one.
\noindent{(\theminipagecount)}\hspace{0.1mm} % By default, LaTeX indents the first line of a new paragraph, but \noindent overrides this
% and inserts the current value of the minipagecount counter, enclosed in parentheses
\begin{minipage}[t]{0.45\textwidth} % The [t] option aligns the top of the minipage with the baseline of the surrounding text.
    \vspace{-26pt}  % moves the content of the minipage up, reducing the space between the minipage content and the preceding text.
    \raggedright %  the text lines up on the left side, but the right side will be ragged.
    \begin{align*} % The * align environment does not number the equations- Each line is aligned at the & symbol
        5(x - 5) &= -10\\
        \frac{5(x-5)}{5} &= \frac{-10}{5}\\
        x - 5 &= -2\\
        x - 5 + 5 &= -2 + 5\\
        x &= 3\\
    \end{align*}
\end{minipage}\refstepcounter{minipagecount} % increments the counter minipagecount by one.
\noindent{(\theminipagecount)}\hspace{0.1mm} % By default, LaTeX indents the first line of a new paragraph, but \noindent overrides this
% and inserts the current value of the minipagecount counter, enclosed in parentheses
\begin{minipage}[t]{0.45\textwidth} % The [t] option aligns the top of the minipage with the baseline of the surrounding text.
    \vspace{-26pt}  % moves the content of the minipage up, reducing the space between the minipage content and the preceding text.
    \raggedright %  the text lines up on the left side, but the right side will be ragged.
    \begin{align*} % The * align environment does not number the equations- Each line is aligned at the & symbol
        \frac{x - 4}{5} &= 4\\
        \frac{x - 4}{5} \times5 &= 4 \times5\\
        x - 4 &= 20\\
        x - 4 + 4 &= 20 + 4\\
        x &= 24\\
    \end{align*}
\end{minipage}\refstepcounter{minipagecount} % increments the counter minipagecount by one.
\noindent{(\theminipagecount)}\hspace{0.1mm} % By default, LaTeX indents the first line of a new paragraph, but \noindent overrides this
% and inserts the current value of the minipagecount counter, enclosed in parentheses
\begin{minipage}[t]{0.45\textwidth} % The [t] option aligns the top of the minipage with the baseline of the surrounding text.
    \vspace{-26pt}  % moves the content of the minipage up, reducing the space between the minipage content and the preceding text.
    \raggedright %  the text lines up on the left side, but the right side will be ragged.
    \begin{align*} % The * align environment does not number the equations- Each line is aligned at the & symbol
        \frac{x}{6} + 10 &= 15\\
        \frac{x}{6} + 10 - 10 &= 15 - 10\\
        \frac{x}{6} &= 5\\
        \frac{x}{6} \times6 &= 5 \times6\\
        x &= 30\\
    \end{align*}
\end{minipage}\refstepcounter{minipagecount} % increments the counter minipagecount by one.
\noindent{(\theminipagecount)}\hspace{0.1mm} % By default, LaTeX indents the first line of a new paragraph, but \noindent overrides this
% and inserts the current value of the minipagecount counter, enclosed in parentheses
\begin{minipage}[t]{0.45\textwidth} % The [t] option aligns the top of the minipage with the baseline of the surrounding text.
    \vspace{-26pt}  % moves the content of the minipage up, reducing the space between the minipage content and the preceding text.
    \raggedright %  the text lines up on the left side, but the right side will be ragged.
    \begin{align*} % The * align environment does not number the equations- Each line is aligned at the & symbol
        \frac{x}{2} - 6 &= -2\\
        \frac{x}{2} - 6 + 6 &= -2 + 6\\
        \frac{x}{2} &= 4\\
        \frac{x}{2} \times2 &= 4 \times2\\
        x &= 8\\
    \end{align*}
\end{minipage}\refstepcounter{minipagecount} % increments the counter minipagecount by one.
\noindent{(\theminipagecount)}\hspace{0.1mm} % By default, LaTeX indents the first line of a new paragraph, but \noindent overrides this
% and inserts the current value of the minipagecount counter, enclosed in parentheses
\begin{minipage}[t]{0.45\textwidth} % The [t] option aligns the top of the minipage with the baseline of the surrounding text.
    \vspace{-26pt}  % moves the content of the minipage up, reducing the space between the minipage content and the preceding text.
    \raggedright %  the text lines up on the left side, but the right side will be ragged.
    \begin{align*} % The * align environment does not number the equations- Each line is aligned at the & symbol
        \frac{x}{5} - 4 &= -1\\
        \frac{x}{5} - 4 + 4 &= -1 + 4\\
        \frac{x}{5} &= 3\\
        \frac{x}{5} \times5 &= 3 \times5\\
        x &= 15\\
    \end{align*}
\end{minipage}\columnbreak
    \refstepcounter{minipagecount} % increments the counter minipagecount by one.
\noindent{(\theminipagecount)}\hspace{0.1mm} % By default, LaTeX indents the first line of a new paragraph, but \noindent overrides this
% and inserts the current value of the minipagecount counter, enclosed in parentheses
\begin{minipage}[t]{0.45\textwidth} % The [t] option aligns the top of the minipage with the baseline of the surrounding text.
    \vspace{-26pt}  % moves the content of the minipage up, reducing the space between the minipage content and the preceding text.
    \raggedright %  the text lines up on the left side, but the right side will be ragged.
    \begin{align*} % The * align environment does not number the equations- Each line is aligned at the & symbol
        \frac{x + 2}{10} &= 10\\
        \frac{x + 2}{10} \times10 &= 10 \times10\\
        x + 2 &= 100\\
        x + 2 - 2 &= 100 - 2\\
        x &= 98\\
    \end{align*}
\end{minipage}\refstepcounter{minipagecount} % increments the counter minipagecount by one.
\noindent{(\theminipagecount)}\hspace{0.1mm} % By default, LaTeX indents the first line of a new paragraph, but \noindent overrides this
% and inserts the current value of the minipagecount counter, enclosed in parentheses
\begin{minipage}[t]{0.45\textwidth} % The [t] option aligns the top of the minipage with the baseline of the surrounding text.
    \vspace{-26pt}  % moves the content of the minipage up, reducing the space between the minipage content and the preceding text.
    \raggedright %  the text lines up on the left side, but the right side will be ragged.
    \begin{align*} % The * align environment does not number the equations- Each line is aligned at the & symbol
        \frac{x + 5}{7} &= 10\\
        \frac{x + 5}{7} \times7 &= 10 \times7\\
        x + 5 &= 70\\
        x + 5 - 5 &= 70 - 5\\
        x &= 65\\
    \end{align*}
\end{minipage}\refstepcounter{minipagecount} % increments the counter minipagecount by one.
\noindent{(\theminipagecount)}\hspace{0.1mm} % By default, LaTeX indents the first line of a new paragraph, but \noindent overrides this
% and inserts the current value of the minipagecount counter, enclosed in parentheses
\begin{minipage}[t]{0.45\textwidth} % The [t] option aligns the top of the minipage with the baseline of the surrounding text.
    \vspace{-26pt}  % moves the content of the minipage up, reducing the space between the minipage content and the preceding text.
    \raggedright %  the text lines up on the left side, but the right side will be ragged.
    \begin{align*} % The * align environment does not number the equations- Each line is aligned at the & symbol
        \frac{x}{7} + 1 &= 3\\
        \frac{x}{7} + 1 - 1 &= 3 - 1\\
        \frac{x}{7} &= 2\\
        \frac{x}{7} \times7 &= 2 \times7\\
        x &= 14\\
    \end{align*}
\end{minipage}\refstepcounter{minipagecount} % increments the counter minipagecount by one.
\noindent{(\theminipagecount)}\hspace{0.1mm} % By default, LaTeX indents the first line of a new paragraph, but \noindent overrides this
% and inserts the current value of the minipagecount counter, enclosed in parentheses
\begin{minipage}[t]{0.45\textwidth} % The [t] option aligns the top of the minipage with the baseline of the surrounding text.
    \vspace{-26pt}  % moves the content of the minipage up, reducing the space between the minipage content and the preceding text.
    \raggedright %  the text lines up on the left side, but the right side will be ragged.
    \begin{align*} % The * align environment does not number the equations- Each line is aligned at the & symbol
        7(x + 9) &= 119\\
        \frac{7(x+9)}{7} &= \frac{119}{7}\\
        x + 9 &= 17\\
        x + 9 - 9 &= 17 - 9\\
        x &= 8\\
    \end{align*}
\end{minipage}\refstepcounter{minipagecount} % increments the counter minipagecount by one.
\noindent{(\theminipagecount)}\hspace{0.1mm} % By default, LaTeX indents the first line of a new paragraph, but \noindent overrides this
% and inserts the current value of the minipagecount counter, enclosed in parentheses
\begin{minipage}[t]{0.45\textwidth} % The [t] option aligns the top of the minipage with the baseline of the surrounding text.
    \vspace{-26pt}  % moves the content of the minipage up, reducing the space between the minipage content and the preceding text.
    \raggedright %  the text lines up on the left side, but the right side will be ragged.
    \begin{align*} % The * align environment does not number the equations- Each line is aligned at the & symbol
        \frac{x + 6}{3} &= 10\\
        \frac{x + 6}{3} \times3 &= 10 \times3\\
        x + 6 &= 30\\
        x + 6 - 6 &= 30 - 6\\
        x &= 24\\
    \end{align*}
\end{minipage}\newpage
    \refstepcounter{minipagecount} % increments the counter minipagecount by one.
\noindent{(\theminipagecount)}\hspace{0.1mm} % By default, LaTeX indents the first line of a new paragraph, but \noindent overrides this
% and inserts the current value of the minipagecount counter, enclosed in parentheses
\begin{minipage}[t]{0.45\textwidth} % The [t] option aligns the top of the minipage with the baseline of the surrounding text.
    \vspace{-26pt}  % moves the content of the minipage up, reducing the space between the minipage content and the preceding text.
    \raggedright %  the text lines up on the left side, but the right side will be ragged.
    \begin{align*} % The * align environment does not number the equations- Each line is aligned at the & symbol
        9x - 4 &= 77\\
        9x - 4 + 4 &= 77 + 4\\
        9x &= 81\\
        \frac{9x}{9} &= \frac{81}{9}\\
        x &= 9\\
    \end{align*}
\end{minipage}\refstepcounter{minipagecount} % increments the counter minipagecount by one.
\noindent{(\theminipagecount)}\hspace{0.1mm} % By default, LaTeX indents the first line of a new paragraph, but \noindent overrides this
% and inserts the current value of the minipagecount counter, enclosed in parentheses
\begin{minipage}[t]{0.45\textwidth} % The [t] option aligns the top of the minipage with the baseline of the surrounding text.
    \vspace{-26pt}  % moves the content of the minipage up, reducing the space between the minipage content and the preceding text.
    \raggedright %  the text lines up on the left side, but the right side will be ragged.
    \begin{align*} % The * align environment does not number the equations- Each line is aligned at the & symbol
        4(x + 5) &= 24\\
        \frac{4(x+5)}{4} &= \frac{24}{4}\\
        x + 5 &= 6\\
        x + 5 - 5 &= 6 - 5\\
        x &= 1\\
    \end{align*}
\end{minipage}\refstepcounter{minipagecount} % increments the counter minipagecount by one.
\noindent{(\theminipagecount)}\hspace{0.1mm} % By default, LaTeX indents the first line of a new paragraph, but \noindent overrides this
% and inserts the current value of the minipagecount counter, enclosed in parentheses
\begin{minipage}[t]{0.45\textwidth} % The [t] option aligns the top of the minipage with the baseline of the surrounding text.
    \vspace{-26pt}  % moves the content of the minipage up, reducing the space between the minipage content and the preceding text.
    \raggedright %  the text lines up on the left side, but the right side will be ragged.
    \begin{align*} % The * align environment does not number the equations- Each line is aligned at the & symbol
        \frac{x}{9} - 3 &= 6\\
        \frac{x}{9} - 3 + 3 &= 6 + 3\\
        \frac{x}{9} &= 9\\
        \frac{x}{9} \times9 &= 9 \times9\\
        x &= 81\\
    \end{align*}
\end{minipage}\refstepcounter{minipagecount} % increments the counter minipagecount by one.
\noindent{(\theminipagecount)}\hspace{0.1mm} % By default, LaTeX indents the first line of a new paragraph, but \noindent overrides this
% and inserts the current value of the minipagecount counter, enclosed in parentheses
\begin{minipage}[t]{0.45\textwidth} % The [t] option aligns the top of the minipage with the baseline of the surrounding text.
    \vspace{-26pt}  % moves the content of the minipage up, reducing the space between the minipage content and the preceding text.
    \raggedright %  the text lines up on the left side, but the right side will be ragged.
    \begin{align*} % The * align environment does not number the equations- Each line is aligned at the & symbol
        \frac{x}{9} - 4 &= 6\\
        \frac{x}{9} - 4 + 4 &= 6 + 4\\
        \frac{x}{9} &= 10\\
        \frac{x}{9} \times9 &= 10 \times9\\
        x &= 90\\
    \end{align*}
\end{minipage}\refstepcounter{minipagecount} % increments the counter minipagecount by one.
\noindent{(\theminipagecount)}\hspace{0.1mm} % By default, LaTeX indents the first line of a new paragraph, but \noindent overrides this
% and inserts the current value of the minipagecount counter, enclosed in parentheses
\begin{minipage}[t]{0.45\textwidth} % The [t] option aligns the top of the minipage with the baseline of the surrounding text.
    \vspace{-26pt}  % moves the content of the minipage up, reducing the space between the minipage content and the preceding text.
    \raggedright %  the text lines up on the left side, but the right side will be ragged.
    \begin{align*} % The * align environment does not number the equations- Each line is aligned at the & symbol
        \frac{x + 3}{2} &= 3\\
        \frac{x + 3}{2} \times2 &= 3 \times2\\
        x + 3 &= 6\\
        x + 3 - 3 &= 6 - 3\\
        x &= 3\\
    \end{align*}
\end{minipage}\columnbreak
    \refstepcounter{minipagecount} % increments the counter minipagecount by one.
\noindent{(\theminipagecount)}\hspace{0.1mm} % By default, LaTeX indents the first line of a new paragraph, but \noindent overrides this
% and inserts the current value of the minipagecount counter, enclosed in parentheses
\begin{minipage}[t]{0.45\textwidth} % The [t] option aligns the top of the minipage with the baseline of the surrounding text.
    \vspace{-26pt}  % moves the content of the minipage up, reducing the space between the minipage content and the preceding text.
    \raggedright %  the text lines up on the left side, but the right side will be ragged.
    \begin{align*} % The * align environment does not number the equations- Each line is aligned at the & symbol
        7x - 8 &= -1\\
        7x - 8 + 8 &= -1 + 8\\
        7x &= 7\\
        \frac{7x}{7} &= \frac{7}{7}\\
        x &= 1\\
    \end{align*}
\end{minipage}\refstepcounter{minipagecount} % increments the counter minipagecount by one.
\noindent{(\theminipagecount)}\hspace{0.1mm} % By default, LaTeX indents the first line of a new paragraph, but \noindent overrides this
% and inserts the current value of the minipagecount counter, enclosed in parentheses
\begin{minipage}[t]{0.45\textwidth} % The [t] option aligns the top of the minipage with the baseline of the surrounding text.
    \vspace{-26pt}  % moves the content of the minipage up, reducing the space between the minipage content and the preceding text.
    \raggedright %  the text lines up on the left side, but the right side will be ragged.
    \begin{align*} % The * align environment does not number the equations- Each line is aligned at the & symbol
        2x - 2 &= 4\\
        2x - 2 + 2 &= 4 + 2\\
        2x &= 6\\
        \frac{2x}{2} &= \frac{6}{2}\\
        x &= 3\\
    \end{align*}
\end{minipage}\refstepcounter{minipagecount} % increments the counter minipagecount by one.
\noindent{(\theminipagecount)}\hspace{0.1mm} % By default, LaTeX indents the first line of a new paragraph, but \noindent overrides this
% and inserts the current value of the minipagecount counter, enclosed in parentheses
\begin{minipage}[t]{0.45\textwidth} % The [t] option aligns the top of the minipage with the baseline of the surrounding text.
    \vspace{-26pt}  % moves the content of the minipage up, reducing the space between the minipage content and the preceding text.
    \raggedright %  the text lines up on the left side, but the right side will be ragged.
    \begin{align*} % The * align environment does not number the equations- Each line is aligned at the & symbol
        \frac{x + 1}{10} &= 1\\
        \frac{x + 1}{10} \times10 &= 1 \times10\\
        x + 1 &= 10\\
        x + 1 - 1 &= 10 - 1\\
        x &= 9\\
    \end{align*}
\end{minipage}\refstepcounter{minipagecount} % increments the counter minipagecount by one.
\noindent{(\theminipagecount)}\hspace{0.1mm} % By default, LaTeX indents the first line of a new paragraph, but \noindent overrides this
% and inserts the current value of the minipagecount counter, enclosed in parentheses
\begin{minipage}[t]{0.45\textwidth} % The [t] option aligns the top of the minipage with the baseline of the surrounding text.
    \vspace{-26pt}  % moves the content of the minipage up, reducing the space between the minipage content and the preceding text.
    \raggedright %  the text lines up on the left side, but the right side will be ragged.
    \begin{align*} % The * align environment does not number the equations- Each line is aligned at the & symbol
        2(x + 4) &= 26\\
        \frac{2(x+4)}{2} &= \frac{26}{2}\\
        x + 4 &= 13\\
        x + 4 - 4 &= 13 - 4\\
        x &= 9\\
    \end{align*}
\end{minipage}\refstepcounter{minipagecount} % increments the counter minipagecount by one.
\noindent{(\theminipagecount)}\hspace{0.1mm} % By default, LaTeX indents the first line of a new paragraph, but \noindent overrides this
% and inserts the current value of the minipagecount counter, enclosed in parentheses
\begin{minipage}[t]{0.45\textwidth} % The [t] option aligns the top of the minipage with the baseline of the surrounding text.
    \vspace{-26pt}  % moves the content of the minipage up, reducing the space between the minipage content and the preceding text.
    \raggedright %  the text lines up on the left side, but the right side will be ragged.
    \begin{align*} % The * align environment does not number the equations- Each line is aligned at the & symbol
        \frac{x}{5} + 9 &= 18\\
        \frac{x}{5} + 9 - 9 &= 18 - 9\\
        \frac{x}{5} &= 9\\
        \frac{x}{5} \times5 &= 9 \times5\\
        x &= 45\\
    \end{align*}
\end{minipage}\newpage
    \refstepcounter{minipagecount} % increments the counter minipagecount by one.
\noindent{(\theminipagecount)}\hspace{0.1mm} % By default, LaTeX indents the first line of a new paragraph, but \noindent overrides this
% and inserts the current value of the minipagecount counter, enclosed in parentheses
\begin{minipage}[t]{0.45\textwidth} % The [t] option aligns the top of the minipage with the baseline of the surrounding text.
    \vspace{-26pt}  % moves the content of the minipage up, reducing the space between the minipage content and the preceding text.
    \raggedright %  the text lines up on the left side, but the right side will be ragged.
    \begin{align*} % The * align environment does not number the equations- Each line is aligned at the & symbol
        \frac{x + 3}{4} &= 6\\
        \frac{x + 3}{4} \times4 &= 6 \times4\\
        x + 3 &= 24\\
        x + 3 - 3 &= 24 - 3\\
        x &= 21\\
    \end{align*}
\end{minipage}\refstepcounter{minipagecount} % increments the counter minipagecount by one.
\noindent{(\theminipagecount)}\hspace{0.1mm} % By default, LaTeX indents the first line of a new paragraph, but \noindent overrides this
% and inserts the current value of the minipagecount counter, enclosed in parentheses
\begin{minipage}[t]{0.45\textwidth} % The [t] option aligns the top of the minipage with the baseline of the surrounding text.
    \vspace{-26pt}  % moves the content of the minipage up, reducing the space between the minipage content and the preceding text.
    \raggedright %  the text lines up on the left side, but the right side will be ragged.
    \begin{align*} % The * align environment does not number the equations- Each line is aligned at the & symbol
        9x - 4 &= 5\\
        9x - 4 + 4 &= 5 + 4\\
        9x &= 9\\
        \frac{9x}{9} &= \frac{9}{9}\\
        x &= 1\\
    \end{align*}
\end{minipage}\refstepcounter{minipagecount} % increments the counter minipagecount by one.
\noindent{(\theminipagecount)}\hspace{0.1mm} % By default, LaTeX indents the first line of a new paragraph, but \noindent overrides this
% and inserts the current value of the minipagecount counter, enclosed in parentheses
\begin{minipage}[t]{0.45\textwidth} % The [t] option aligns the top of the minipage with the baseline of the surrounding text.
    \vspace{-26pt}  % moves the content of the minipage up, reducing the space between the minipage content and the preceding text.
    \raggedright %  the text lines up on the left side, but the right side will be ragged.
    \begin{align*} % The * align environment does not number the equations- Each line is aligned at the & symbol
        \frac{x + 3}{2} &= 5\\
        \frac{x + 3}{2} \times2 &= 5 \times2\\
        x + 3 &= 10\\
        x + 3 - 3 &= 10 - 3\\
        x &= 7\\
    \end{align*}
\end{minipage}\refstepcounter{minipagecount} % increments the counter minipagecount by one.
\noindent{(\theminipagecount)}\hspace{0.1mm} % By default, LaTeX indents the first line of a new paragraph, but \noindent overrides this
% and inserts the current value of the minipagecount counter, enclosed in parentheses
\begin{minipage}[t]{0.45\textwidth} % The [t] option aligns the top of the minipage with the baseline of the surrounding text.
    \vspace{-26pt}  % moves the content of the minipage up, reducing the space between the minipage content and the preceding text.
    \raggedright %  the text lines up on the left side, but the right side will be ragged.
    \begin{align*} % The * align environment does not number the equations- Each line is aligned at the & symbol
        \frac{x + 2}{9} &= 7\\
        \frac{x + 2}{9} \times9 &= 7 \times9\\
        x + 2 &= 63\\
        x + 2 - 2 &= 63 - 2\\
        x &= 61\\
    \end{align*}
\end{minipage}\refstepcounter{minipagecount} % increments the counter minipagecount by one.
\noindent{(\theminipagecount)}\hspace{0.1mm} % By default, LaTeX indents the first line of a new paragraph, but \noindent overrides this
% and inserts the current value of the minipagecount counter, enclosed in parentheses
\begin{minipage}[t]{0.45\textwidth} % The [t] option aligns the top of the minipage with the baseline of the surrounding text.
    \vspace{-26pt}  % moves the content of the minipage up, reducing the space between the minipage content and the preceding text.
    \raggedright %  the text lines up on the left side, but the right side will be ragged.
    \begin{align*} % The * align environment does not number the equations- Each line is aligned at the & symbol
        \frac{x}{2} + 10 &= 13\\
        \frac{x}{2} + 10 - 10 &= 13 - 10\\
        \frac{x}{2} &= 3\\
        \frac{x}{2} \times2 &= 3 \times2\\
        x &= 6\\
    \end{align*}
\end{minipage}\columnbreak
    \refstepcounter{minipagecount} % increments the counter minipagecount by one.
\noindent{(\theminipagecount)}\hspace{0.1mm} % By default, LaTeX indents the first line of a new paragraph, but \noindent overrides this
% and inserts the current value of the minipagecount counter, enclosed in parentheses
\begin{minipage}[t]{0.45\textwidth} % The [t] option aligns the top of the minipage with the baseline of the surrounding text.
    \vspace{-26pt}  % moves the content of the minipage up, reducing the space between the minipage content and the preceding text.
    \raggedright %  the text lines up on the left side, but the right side will be ragged.
    \begin{align*} % The * align environment does not number the equations- Each line is aligned at the & symbol
        2x + 9 &= 23\\
        2x + 9 - 9 &= 23 - 9\\
        2x &= 14\\
        \frac{2x}{2} &= \frac{14}{2}\\
        x &= 7\\
    \end{align*}
\end{minipage}\refstepcounter{minipagecount} % increments the counter minipagecount by one.
\noindent{(\theminipagecount)}\hspace{0.1mm} % By default, LaTeX indents the first line of a new paragraph, but \noindent overrides this
% and inserts the current value of the minipagecount counter, enclosed in parentheses
\begin{minipage}[t]{0.45\textwidth} % The [t] option aligns the top of the minipage with the baseline of the surrounding text.
    \vspace{-26pt}  % moves the content of the minipage up, reducing the space between the minipage content and the preceding text.
    \raggedright %  the text lines up on the left side, but the right side will be ragged.
    \begin{align*} % The * align environment does not number the equations- Each line is aligned at the & symbol
        \frac{x}{7} + 7 &= 9\\
        \frac{x}{7} + 7 - 7 &= 9 - 7\\
        \frac{x}{7} &= 2\\
        \frac{x}{7} \times7 &= 2 \times7\\
        x &= 14\\
    \end{align*}
\end{minipage}\refstepcounter{minipagecount} % increments the counter minipagecount by one.
\noindent{(\theminipagecount)}\hspace{0.1mm} % By default, LaTeX indents the first line of a new paragraph, but \noindent overrides this
% and inserts the current value of the minipagecount counter, enclosed in parentheses
\begin{minipage}[t]{0.45\textwidth} % The [t] option aligns the top of the minipage with the baseline of the surrounding text.
    \vspace{-26pt}  % moves the content of the minipage up, reducing the space between the minipage content and the preceding text.
    \raggedright %  the text lines up on the left side, but the right side will be ragged.
    \begin{align*} % The * align environment does not number the equations- Each line is aligned at the & symbol
        8x + 6 &= 38\\
        8x + 6 - 6 &= 38 - 6\\
        8x &= 32\\
        \frac{8x}{8} &= \frac{32}{8}\\
        x &= 4\\
    \end{align*}
\end{minipage}\refstepcounter{minipagecount} % increments the counter minipagecount by one.
\noindent{(\theminipagecount)}\hspace{0.1mm} % By default, LaTeX indents the first line of a new paragraph, but \noindent overrides this
% and inserts the current value of the minipagecount counter, enclosed in parentheses
\begin{minipage}[t]{0.45\textwidth} % The [t] option aligns the top of the minipage with the baseline of the surrounding text.
    \vspace{-26pt}  % moves the content of the minipage up, reducing the space between the minipage content and the preceding text.
    \raggedright %  the text lines up on the left side, but the right side will be ragged.
    \begin{align*} % The * align environment does not number the equations- Each line is aligned at the & symbol
        \frac{x}{7} - 10 &= -1\\
        \frac{x}{7} - 10 + 10 &= -1 + 10\\
        \frac{x}{7} &= 9\\
        \frac{x}{7} \times7 &= 9 \times7\\
        x &= 63\\
    \end{align*}
\end{minipage}\refstepcounter{minipagecount} % increments the counter minipagecount by one.
\noindent{(\theminipagecount)}\hspace{0.1mm} % By default, LaTeX indents the first line of a new paragraph, but \noindent overrides this
% and inserts the current value of the minipagecount counter, enclosed in parentheses
\begin{minipage}[t]{0.45\textwidth} % The [t] option aligns the top of the minipage with the baseline of the surrounding text.
    \vspace{-26pt}  % moves the content of the minipage up, reducing the space between the minipage content and the preceding text.
    \raggedright %  the text lines up on the left side, but the right side will be ragged.
    \begin{align*} % The * align environment does not number the equations- Each line is aligned at the & symbol
        \frac{x}{9} - 5 &= 0\\
        \frac{x}{9} - 5 + 5 &= 0 + 5\\
        \frac{x}{9} &= 5\\
        \frac{x}{9} \times9 &= 5 \times9\\
        x &= 45\\
    \end{align*}
\end{minipage}\newpage
    \refstepcounter{minipagecount} % increments the counter minipagecount by one.
\noindent{(\theminipagecount)}\hspace{0.1mm} % By default, LaTeX indents the first line of a new paragraph, but \noindent overrides this
% and inserts the current value of the minipagecount counter, enclosed in parentheses
\begin{minipage}[t]{0.45\textwidth} % The [t] option aligns the top of the minipage with the baseline of the surrounding text.
    \vspace{-26pt}  % moves the content of the minipage up, reducing the space between the minipage content and the preceding text.
    \raggedright %  the text lines up on the left side, but the right side will be ragged.
    \begin{align*} % The * align environment does not number the equations- Each line is aligned at the & symbol
        10(x - 8) &= 10\\
        \frac{10(x-8)}{10} &= \frac{10}{10}\\
        x - 8 &= 1\\
        x - 8 + 8 &= 1 + 8\\
        x &= 9\\
    \end{align*}
\end{minipage}\refstepcounter{minipagecount} % increments the counter minipagecount by one.
\noindent{(\theminipagecount)}\hspace{0.1mm} % By default, LaTeX indents the first line of a new paragraph, but \noindent overrides this
% and inserts the current value of the minipagecount counter, enclosed in parentheses
\begin{minipage}[t]{0.45\textwidth} % The [t] option aligns the top of the minipage with the baseline of the surrounding text.
    \vspace{-26pt}  % moves the content of the minipage up, reducing the space between the minipage content and the preceding text.
    \raggedright %  the text lines up on the left side, but the right side will be ragged.
    \begin{align*} % The * align environment does not number the equations- Each line is aligned at the & symbol
        4(x + 7) &= 48\\
        \frac{4(x+7)}{4} &= \frac{48}{4}\\
        x + 7 &= 12\\
        x + 7 - 7 &= 12 - 7\\
        x &= 5\\
    \end{align*}
\end{minipage}\refstepcounter{minipagecount} % increments the counter minipagecount by one.
\noindent{(\theminipagecount)}\hspace{0.1mm} % By default, LaTeX indents the first line of a new paragraph, but \noindent overrides this
% and inserts the current value of the minipagecount counter, enclosed in parentheses
\begin{minipage}[t]{0.45\textwidth} % The [t] option aligns the top of the minipage with the baseline of the surrounding text.
    \vspace{-26pt}  % moves the content of the minipage up, reducing the space between the minipage content and the preceding text.
    \raggedright %  the text lines up on the left side, but the right side will be ragged.
    \begin{align*} % The * align environment does not number the equations- Each line is aligned at the & symbol
        7(x - 3) &= 21\\
        \frac{7(x-3)}{7} &= \frac{21}{7}\\
        x - 3 &= 3\\
        x - 3 + 3 &= 3 + 3\\
        x &= 6\\
    \end{align*}
\end{minipage}\refstepcounter{minipagecount} % increments the counter minipagecount by one.
\noindent{(\theminipagecount)}\hspace{0.1mm} % By default, LaTeX indents the first line of a new paragraph, but \noindent overrides this
% and inserts the current value of the minipagecount counter, enclosed in parentheses
\begin{minipage}[t]{0.45\textwidth} % The [t] option aligns the top of the minipage with the baseline of the surrounding text.
    \vspace{-26pt}  % moves the content of the minipage up, reducing the space between the minipage content and the preceding text.
    \raggedright %  the text lines up on the left side, but the right side will be ragged.
    \begin{align*} % The * align environment does not number the equations- Each line is aligned at the & symbol
        \frac{x}{3} - 3 &= 7\\
        \frac{x}{3} - 3 + 3 &= 7 + 3\\
        \frac{x}{3} &= 10\\
        \frac{x}{3} \times3 &= 10 \times3\\
        x &= 30\\
    \end{align*}
\end{minipage}\refstepcounter{minipagecount} % increments the counter minipagecount by one.
\noindent{(\theminipagecount)}\hspace{0.1mm} % By default, LaTeX indents the first line of a new paragraph, but \noindent overrides this
% and inserts the current value of the minipagecount counter, enclosed in parentheses
\begin{minipage}[t]{0.45\textwidth} % The [t] option aligns the top of the minipage with the baseline of the surrounding text.
    \vspace{-26pt}  % moves the content of the minipage up, reducing the space between the minipage content and the preceding text.
    \raggedright %  the text lines up on the left side, but the right side will be ragged.
    \begin{align*} % The * align environment does not number the equations- Each line is aligned at the & symbol
        8x + 3 &= 19\\
        8x + 3 - 3 &= 19 - 3\\
        8x &= 16\\
        \frac{8x}{8} &= \frac{16}{8}\\
        x &= 2\\
    \end{align*}
\end{minipage}\columnbreak
    \refstepcounter{minipagecount} % increments the counter minipagecount by one.
\noindent{(\theminipagecount)}\hspace{0.1mm} % By default, LaTeX indents the first line of a new paragraph, but \noindent overrides this
% and inserts the current value of the minipagecount counter, enclosed in parentheses
\begin{minipage}[t]{0.45\textwidth} % The [t] option aligns the top of the minipage with the baseline of the surrounding text.
    \vspace{-26pt}  % moves the content of the minipage up, reducing the space between the minipage content and the preceding text.
    \raggedright %  the text lines up on the left side, but the right side will be ragged.
    \begin{align*} % The * align environment does not number the equations- Each line is aligned at the & symbol
        \frac{x}{5} - 8 &= -4\\
        \frac{x}{5} - 8 + 8 &= -4 + 8\\
        \frac{x}{5} &= 4\\
        \frac{x}{5} \times5 &= 4 \times5\\
        x &= 20\\
    \end{align*}
\end{minipage}\refstepcounter{minipagecount} % increments the counter minipagecount by one.
\noindent{(\theminipagecount)}\hspace{0.1mm} % By default, LaTeX indents the first line of a new paragraph, but \noindent overrides this
% and inserts the current value of the minipagecount counter, enclosed in parentheses
\begin{minipage}[t]{0.45\textwidth} % The [t] option aligns the top of the minipage with the baseline of the surrounding text.
    \vspace{-26pt}  % moves the content of the minipage up, reducing the space between the minipage content and the preceding text.
    \raggedright %  the text lines up on the left side, but the right side will be ragged.
    \begin{align*} % The * align environment does not number the equations- Each line is aligned at the & symbol
        2(x - 10) &= -4\\
        \frac{2(x-10)}{2} &= \frac{-4}{2}\\
        x - 10 &= -2\\
        x - 10 + 10 &= -2 + 10\\
        x &= 8\\
    \end{align*}
\end{minipage}\refstepcounter{minipagecount} % increments the counter minipagecount by one.
\noindent{(\theminipagecount)}\hspace{0.1mm} % By default, LaTeX indents the first line of a new paragraph, but \noindent overrides this
% and inserts the current value of the minipagecount counter, enclosed in parentheses
\begin{minipage}[t]{0.45\textwidth} % The [t] option aligns the top of the minipage with the baseline of the surrounding text.
    \vspace{-26pt}  % moves the content of the minipage up, reducing the space between the minipage content and the preceding text.
    \raggedright %  the text lines up on the left side, but the right side will be ragged.
    \begin{align*} % The * align environment does not number the equations- Each line is aligned at the & symbol
        6x - 9 &= 51\\
        6x - 9 + 9 &= 51 + 9\\
        6x &= 60\\
        \frac{6x}{6} &= \frac{60}{6}\\
        x &= 10\\
    \end{align*}
\end{minipage}\refstepcounter{minipagecount} % increments the counter minipagecount by one.
\noindent{(\theminipagecount)}\hspace{0.1mm} % By default, LaTeX indents the first line of a new paragraph, but \noindent overrides this
% and inserts the current value of the minipagecount counter, enclosed in parentheses
\begin{minipage}[t]{0.45\textwidth} % The [t] option aligns the top of the minipage with the baseline of the surrounding text.
    \vspace{-26pt}  % moves the content of the minipage up, reducing the space between the minipage content and the preceding text.
    \raggedright %  the text lines up on the left side, but the right side will be ragged.
    \begin{align*} % The * align environment does not number the equations- Each line is aligned at the & symbol
        9x - 6 &= 75\\
        9x - 6 + 6 &= 75 + 6\\
        9x &= 81\\
        \frac{9x}{9} &= \frac{81}{9}\\
        x &= 9\\
    \end{align*}
\end{minipage}\refstepcounter{minipagecount} % increments the counter minipagecount by one.
\noindent{(\theminipagecount)}\hspace{0.1mm} % By default, LaTeX indents the first line of a new paragraph, but \noindent overrides this
% and inserts the current value of the minipagecount counter, enclosed in parentheses
\begin{minipage}[t]{0.45\textwidth} % The [t] option aligns the top of the minipage with the baseline of the surrounding text.
    \vspace{-26pt}  % moves the content of the minipage up, reducing the space between the minipage content and the preceding text.
    \raggedright %  the text lines up on the left side, but the right side will be ragged.
    \begin{align*} % The * align environment does not number the equations- Each line is aligned at the & symbol
        \frac{x}{9} - 3 &= 7\\
        \frac{x}{9} - 3 + 3 &= 7 + 3\\
        \frac{x}{9} &= 10\\
        \frac{x}{9} \times9 &= 10 \times9\\
        x &= 90\\
    \end{align*}
\end{minipage}\newpage
    \refstepcounter{minipagecount} % increments the counter minipagecount by one.
\noindent{(\theminipagecount)}\hspace{0.1mm} % By default, LaTeX indents the first line of a new paragraph, but \noindent overrides this
% and inserts the current value of the minipagecount counter, enclosed in parentheses
\begin{minipage}[t]{0.45\textwidth} % The [t] option aligns the top of the minipage with the baseline of the surrounding text.
    \vspace{-26pt}  % moves the content of the minipage up, reducing the space between the minipage content and the preceding text.
    \raggedright %  the text lines up on the left side, but the right side will be ragged.
    \begin{align*} % The * align environment does not number the equations- Each line is aligned at the & symbol
        \frac{x + 3}{8} &= 9\\
        \frac{x + 3}{8} \times8 &= 9 \times8\\
        x + 3 &= 72\\
        x + 3 - 3 &= 72 - 3\\
        x &= 69\\
    \end{align*}
\end{minipage}\refstepcounter{minipagecount} % increments the counter minipagecount by one.
\noindent{(\theminipagecount)}\hspace{0.1mm} % By default, LaTeX indents the first line of a new paragraph, but \noindent overrides this
% and inserts the current value of the minipagecount counter, enclosed in parentheses
\begin{minipage}[t]{0.45\textwidth} % The [t] option aligns the top of the minipage with the baseline of the surrounding text.
    \vspace{-26pt}  % moves the content of the minipage up, reducing the space between the minipage content and the preceding text.
    \raggedright %  the text lines up on the left side, but the right side will be ragged.
    \begin{align*} % The * align environment does not number the equations- Each line is aligned at the & symbol
        \frac{x}{7} + 3 &= 13\\
        \frac{x}{7} + 3 - 3 &= 13 - 3\\
        \frac{x}{7} &= 10\\
        \frac{x}{7} \times7 &= 10 \times7\\
        x &= 70\\
    \end{align*}
\end{minipage}\refstepcounter{minipagecount} % increments the counter minipagecount by one.
\noindent{(\theminipagecount)}\hspace{0.1mm} % By default, LaTeX indents the first line of a new paragraph, but \noindent overrides this
% and inserts the current value of the minipagecount counter, enclosed in parentheses
\begin{minipage}[t]{0.45\textwidth} % The [t] option aligns the top of the minipage with the baseline of the surrounding text.
    \vspace{-26pt}  % moves the content of the minipage up, reducing the space between the minipage content and the preceding text.
    \raggedright %  the text lines up on the left side, but the right side will be ragged.
    \begin{align*} % The * align environment does not number the equations- Each line is aligned at the & symbol
        \frac{x}{5} - 2 &= 3\\
        \frac{x}{5} - 2 + 2 &= 3 + 2\\
        \frac{x}{5} &= 5\\
        \frac{x}{5} \times5 &= 5 \times5\\
        x &= 25\\
    \end{align*}
\end{minipage}\refstepcounter{minipagecount} % increments the counter minipagecount by one.
\noindent{(\theminipagecount)}\hspace{0.1mm} % By default, LaTeX indents the first line of a new paragraph, but \noindent overrides this
% and inserts the current value of the minipagecount counter, enclosed in parentheses
\begin{minipage}[t]{0.45\textwidth} % The [t] option aligns the top of the minipage with the baseline of the surrounding text.
    \vspace{-26pt}  % moves the content of the minipage up, reducing the space between the minipage content and the preceding text.
    \raggedright %  the text lines up on the left side, but the right side will be ragged.
    \begin{align*} % The * align environment does not number the equations- Each line is aligned at the & symbol
        7x + 7 &= 42\\
        7x + 7 - 7 &= 42 - 7\\
        7x &= 35\\
        \frac{7x}{7} &= \frac{35}{7}\\
        x &= 5\\
    \end{align*}
\end{minipage}\refstepcounter{minipagecount} % increments the counter minipagecount by one.
\noindent{(\theminipagecount)}\hspace{0.1mm} % By default, LaTeX indents the first line of a new paragraph, but \noindent overrides this
% and inserts the current value of the minipagecount counter, enclosed in parentheses
\begin{minipage}[t]{0.45\textwidth} % The [t] option aligns the top of the minipage with the baseline of the surrounding text.
    \vspace{-26pt}  % moves the content of the minipage up, reducing the space between the minipage content and the preceding text.
    \raggedright %  the text lines up on the left side, but the right side will be ragged.
    \begin{align*} % The * align environment does not number the equations- Each line is aligned at the & symbol
        \frac{x + 10}{6} &= 4\\
        \frac{x + 10}{6} \times6 &= 4 \times6\\
        x + 10 &= 24\\
        x + 10 - 10 &= 24 - 10\\
        x &= 14\\
    \end{align*}
\end{minipage}\columnbreak
    \refstepcounter{minipagecount} % increments the counter minipagecount by one.
\noindent{(\theminipagecount)}\hspace{0.1mm} % By default, LaTeX indents the first line of a new paragraph, but \noindent overrides this
% and inserts the current value of the minipagecount counter, enclosed in parentheses
\begin{minipage}[t]{0.45\textwidth} % The [t] option aligns the top of the minipage with the baseline of the surrounding text.
    \vspace{-26pt}  % moves the content of the minipage up, reducing the space between the minipage content and the preceding text.
    \raggedright %  the text lines up on the left side, but the right side will be ragged.
    \begin{align*} % The * align environment does not number the equations- Each line is aligned at the & symbol
        \frac{x}{5} - 3 &= 2\\
        \frac{x}{5} - 3 + 3 &= 2 + 3\\
        \frac{x}{5} &= 5\\
        \frac{x}{5} \times5 &= 5 \times5\\
        x &= 25\\
    \end{align*}
\end{minipage}\refstepcounter{minipagecount} % increments the counter minipagecount by one.
\noindent{(\theminipagecount)}\hspace{0.1mm} % By default, LaTeX indents the first line of a new paragraph, but \noindent overrides this
% and inserts the current value of the minipagecount counter, enclosed in parentheses
\begin{minipage}[t]{0.45\textwidth} % The [t] option aligns the top of the minipage with the baseline of the surrounding text.
    \vspace{-26pt}  % moves the content of the minipage up, reducing the space between the minipage content and the preceding text.
    \raggedright %  the text lines up on the left side, but the right side will be ragged.
    \begin{align*} % The * align environment does not number the equations- Each line is aligned at the & symbol
        \frac{x + 5}{8} &= 4\\
        \frac{x + 5}{8} \times8 &= 4 \times8\\
        x + 5 &= 32\\
        x + 5 - 5 &= 32 - 5\\
        x &= 27\\
    \end{align*}
\end{minipage}\refstepcounter{minipagecount} % increments the counter minipagecount by one.
\noindent{(\theminipagecount)}\hspace{0.1mm} % By default, LaTeX indents the first line of a new paragraph, but \noindent overrides this
% and inserts the current value of the minipagecount counter, enclosed in parentheses
\begin{minipage}[t]{0.45\textwidth} % The [t] option aligns the top of the minipage with the baseline of the surrounding text.
    \vspace{-26pt}  % moves the content of the minipage up, reducing the space between the minipage content and the preceding text.
    \raggedright %  the text lines up on the left side, but the right side will be ragged.
    \begin{align*} % The * align environment does not number the equations- Each line is aligned at the & symbol
        6x - 8 &= 34\\
        6x - 8 + 8 &= 34 + 8\\
        6x &= 42\\
        \frac{6x}{6} &= \frac{42}{6}\\
        x &= 7\\
    \end{align*}
\end{minipage}\refstepcounter{minipagecount} % increments the counter minipagecount by one.
\noindent{(\theminipagecount)}\hspace{0.1mm} % By default, LaTeX indents the first line of a new paragraph, but \noindent overrides this
% and inserts the current value of the minipagecount counter, enclosed in parentheses
\begin{minipage}[t]{0.45\textwidth} % The [t] option aligns the top of the minipage with the baseline of the surrounding text.
    \vspace{-26pt}  % moves the content of the minipage up, reducing the space between the minipage content and the preceding text.
    \raggedright %  the text lines up on the left side, but the right side will be ragged.
    \begin{align*} % The * align environment does not number the equations- Each line is aligned at the & symbol
        \frac{x - 9}{9} &= 5\\
        \frac{x - 9}{9} \times9 &= 5 \times9\\
        x - 9 &= 45\\
        x - 9 + 9 &= 45 + 9\\
        x &= 90\\
    \end{align*}
\end{minipage}\refstepcounter{minipagecount} % increments the counter minipagecount by one.
\noindent{(\theminipagecount)}\hspace{0.1mm} % By default, LaTeX indents the first line of a new paragraph, but \noindent overrides this
% and inserts the current value of the minipagecount counter, enclosed in parentheses
\begin{minipage}[t]{0.45\textwidth} % The [t] option aligns the top of the minipage with the baseline of the surrounding text.
    \vspace{-26pt}  % moves the content of the minipage up, reducing the space between the minipage content and the preceding text.
    \raggedright %  the text lines up on the left side, but the right side will be ragged.
    \begin{align*} % The * align environment does not number the equations- Each line is aligned at the & symbol
        \frac{x + 10}{8} &= 4\\
        \frac{x + 10}{8} \times8 &= 4 \times8\\
        x + 10 &= 32\\
        x + 10 - 10 &= 32 - 10\\
        x &= 22\\
    \end{align*}
\end{minipage}\newpage
    
\end{multicols}
\end{document}
