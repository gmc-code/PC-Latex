\documentclass[12pt]{article}
\usepackage{tikz}
\usepackage{amsmath}
% Underlining package
\usepackage[normalem]{ulem} % [normalem] prevents the package from changing the default behavior of \emph to underline.
\usepackage[a4paper, portrait, margin=1cm]{geometry}
\usepackage{multicol}
\usepackage{fancyhdr}

\def \HeadingQuestions {\section*{\Large Name: \underline{\hspace{8cm}} \hfill Date: \underline{\hspace{3cm}}} \vspace{-3mm}
{Inverse operations: Questions} \vspace{1pt}\hrule}

% only in Q due to increased line height with dotted underline
\linespread{0.9} % Adjust line spacing factor to 0.9 if needed to fit 5 in column for 2 div steps

% raise footer with page number; no header
\fancypagestyle{myfancypagestyle}{
  \fancyhf{}% clear all header and footer fields
  \renewcommand{\headrulewidth}{0pt} % no rule under header
  \fancyfoot[C] {\thepage} \setlength{\footskip}{6pt} % raise page number 6pt
}
\pagestyle{myfancypagestyle}  % apply myfancypagestyle
\newcounter{minipagecount}
\begin{document}
\HeadingQuestions
\vspace{1mm}
\begin{multicols}{2}
\refstepcounter{minipagecount} % increments the counter minipagecount by one.
\noindent{(\theminipagecount)}\hspace{0.1mm} % By default, LaTeX indents the first line of a new paragraph, but \noindent overrides this
% and inserts the current value of the minipagecount counter, enclosed in parentheses
\begin{minipage}[t]{0.45\textwidth} % The [t] option aligns the top of the minipage with the baseline of the surrounding text.
    \vspace{-26pt}  % moves the content of the minipage up, reducing the space between the minipage content and the preceding text.
    \raggedright %  the text lines up on the left side, but the right side will be ragged.
    \begin{align*} % The * align environment does not number the equations- Each line is aligned at the & symbol
        \frac{x}{4} + 2 &= 9\\
        \frac{x}{4} + 2 - \dotuline{\hspace{5mm}} &= 9 - \dotuline{\hspace{5mm}}\\
        \frac{x}{4} &= \dotuline{\hspace{5mm}}\\
        \frac{x}{4} \times\dotuline{\hspace{5mm}} &= \dotuline{\hspace{5mm}} \times\dotuline{\hspace{5mm}}\\
        x &= \dotuline{\hspace{5mm}}\\
    \end{align*}
\end{minipage}\refstepcounter{minipagecount} % increments the counter minipagecount by one.
\noindent{(\theminipagecount)}\hspace{0.1mm} % By default, LaTeX indents the first line of a new paragraph, but \noindent overrides this
% and inserts the current value of the minipagecount counter, enclosed in parentheses
\begin{minipage}[t]{0.45\textwidth} % The [t] option aligns the top of the minipage with the baseline of the surrounding text.
    \vspace{-26pt}  % moves the content of the minipage up, reducing the space between the minipage content and the preceding text.
    \raggedright %  the text lines up on the left side, but the right side will be ragged.
    \begin{align*} % The * align environment does not number the equations- Each line is aligned at the & symbol
        \frac{x}{10} + 8 &= 15\\
        \frac{x}{10} + 8 - \dotuline{\hspace{5mm}} &= 15 - \dotuline{\hspace{5mm}}\\
        \frac{x}{10} &= \dotuline{\hspace{5mm}}\\
        \frac{x}{10} \times\dotuline{\hspace{5mm}} &= \dotuline{\hspace{5mm}} \times\dotuline{\hspace{5mm}}\\
        x &= \dotuline{\hspace{5mm}}\\
    \end{align*}
\end{minipage}\refstepcounter{minipagecount} % increments the counter minipagecount by one.
\noindent{(\theminipagecount)}\hspace{0.1mm} % By default, LaTeX indents the first line of a new paragraph, but \noindent overrides this
% and inserts the current value of the minipagecount counter, enclosed in parentheses
\begin{minipage}[t]{0.45\textwidth} % The [t] option aligns the top of the minipage with the baseline of the surrounding text.
    \vspace{-26pt}  % moves the content of the minipage up, reducing the space between the minipage content and the preceding text.
    \raggedright %  the text lines up on the left side, but the right side will be ragged.
    \begin{align*} % The * align environment does not number the equations- Each line is aligned at the & symbol
        \frac{x}{7} + 2 &= 7\\
        \frac{x}{7} + 2 - \dotuline{\hspace{5mm}} &= 7 - \dotuline{\hspace{5mm}}\\
        \frac{x}{7} &= \dotuline{\hspace{5mm}}\\
        \frac{x}{7} \times\dotuline{\hspace{5mm}} &= \dotuline{\hspace{5mm}} \times\dotuline{\hspace{5mm}}\\
        x &= \dotuline{\hspace{5mm}}\\
    \end{align*}
\end{minipage}\refstepcounter{minipagecount} % increments the counter minipagecount by one.
\noindent{(\theminipagecount)}\hspace{0.1mm} % By default, LaTeX indents the first line of a new paragraph, but \noindent overrides this
% and inserts the current value of the minipagecount counter, enclosed in parentheses
\begin{minipage}[t]{0.45\textwidth} % The [t] option aligns the top of the minipage with the baseline of the surrounding text.
    \vspace{-26pt}  % moves the content of the minipage up, reducing the space between the minipage content and the preceding text.
    \raggedright %  the text lines up on the left side, but the right side will be ragged.
    \begin{align*} % The * align environment does not number the equations- Each line is aligned at the & symbol
        \frac{x}{4} + 4 &= 8\\
        \frac{x}{4} + 4 - \dotuline{\hspace{5mm}} &= 8 - \dotuline{\hspace{5mm}}\\
        \frac{x}{4} &= \dotuline{\hspace{5mm}}\\
        \frac{x}{4} \times\dotuline{\hspace{5mm}} &= \dotuline{\hspace{5mm}} \times\dotuline{\hspace{5mm}}\\
        x &= \dotuline{\hspace{5mm}}\\
    \end{align*}
\end{minipage}\refstepcounter{minipagecount} % increments the counter minipagecount by one.
\noindent{(\theminipagecount)}\hspace{0.1mm} % By default, LaTeX indents the first line of a new paragraph, but \noindent overrides this
% and inserts the current value of the minipagecount counter, enclosed in parentheses
\begin{minipage}[t]{0.45\textwidth} % The [t] option aligns the top of the minipage with the baseline of the surrounding text.
    \vspace{-26pt}  % moves the content of the minipage up, reducing the space between the minipage content and the preceding text.
    \raggedright %  the text lines up on the left side, but the right side will be ragged.
    \begin{align*} % The * align environment does not number the equations- Each line is aligned at the & symbol
        \frac{x}{2} + 3 &= 7\\
        \frac{x}{2} + 3 - \dotuline{\hspace{5mm}} &= 7 - \dotuline{\hspace{5mm}}\\
        \frac{x}{2} &= \dotuline{\hspace{5mm}}\\
        \frac{x}{2} \times\dotuline{\hspace{5mm}} &= \dotuline{\hspace{5mm}} \times\dotuline{\hspace{5mm}}\\
        x &= \dotuline{\hspace{5mm}}\\
    \end{align*}
\end{minipage}\columnbreak
    \refstepcounter{minipagecount} % increments the counter minipagecount by one.
\noindent{(\theminipagecount)}\hspace{0.1mm} % By default, LaTeX indents the first line of a new paragraph, but \noindent overrides this
% and inserts the current value of the minipagecount counter, enclosed in parentheses
\begin{minipage}[t]{0.45\textwidth} % The [t] option aligns the top of the minipage with the baseline of the surrounding text.
    \vspace{-26pt}  % moves the content of the minipage up, reducing the space between the minipage content and the preceding text.
    \raggedright %  the text lines up on the left side, but the right side will be ragged.
    \begin{align*} % The * align environment does not number the equations- Each line is aligned at the & symbol
        \frac{x}{6} + 5 &= 9\\
        \frac{x}{6} + 5 - \dotuline{\hspace{5mm}} &= 9 - \dotuline{\hspace{5mm}}\\
        \frac{x}{6} &= \dotuline{\hspace{5mm}}\\
        \frac{x}{6} \times\dotuline{\hspace{5mm}} &= \dotuline{\hspace{5mm}} \times\dotuline{\hspace{5mm}}\\
        x &= \dotuline{\hspace{5mm}}\\
    \end{align*}
\end{minipage}\refstepcounter{minipagecount} % increments the counter minipagecount by one.
\noindent{(\theminipagecount)}\hspace{0.1mm} % By default, LaTeX indents the first line of a new paragraph, but \noindent overrides this
% and inserts the current value of the minipagecount counter, enclosed in parentheses
\begin{minipage}[t]{0.45\textwidth} % The [t] option aligns the top of the minipage with the baseline of the surrounding text.
    \vspace{-26pt}  % moves the content of the minipage up, reducing the space between the minipage content and the preceding text.
    \raggedright %  the text lines up on the left side, but the right side will be ragged.
    \begin{align*} % The * align environment does not number the equations- Each line is aligned at the & symbol
        \frac{x}{10} + 3 &= 10\\
        \frac{x}{10} + 3 - \dotuline{\hspace{5mm}} &= 10 - \dotuline{\hspace{5mm}}\\
        \frac{x}{10} &= \dotuline{\hspace{5mm}}\\
        \frac{x}{10} \times\dotuline{\hspace{5mm}} &= \dotuline{\hspace{5mm}} \times\dotuline{\hspace{5mm}}\\
        x &= \dotuline{\hspace{5mm}}\\
    \end{align*}
\end{minipage}\refstepcounter{minipagecount} % increments the counter minipagecount by one.
\noindent{(\theminipagecount)}\hspace{0.1mm} % By default, LaTeX indents the first line of a new paragraph, but \noindent overrides this
% and inserts the current value of the minipagecount counter, enclosed in parentheses
\begin{minipage}[t]{0.45\textwidth} % The [t] option aligns the top of the minipage with the baseline of the surrounding text.
    \vspace{-26pt}  % moves the content of the minipage up, reducing the space between the minipage content and the preceding text.
    \raggedright %  the text lines up on the left side, but the right side will be ragged.
    \begin{align*} % The * align environment does not number the equations- Each line is aligned at the & symbol
        \frac{x}{3} + 3 &= 13\\
        \frac{x}{3} + 3 - \dotuline{\hspace{5mm}} &= 13 - \dotuline{\hspace{5mm}}\\
        \frac{x}{3} &= \dotuline{\hspace{5mm}}\\
        \frac{x}{3} \times\dotuline{\hspace{5mm}} &= \dotuline{\hspace{5mm}} \times\dotuline{\hspace{5mm}}\\
        x &= \dotuline{\hspace{5mm}}\\
    \end{align*}
\end{minipage}\refstepcounter{minipagecount} % increments the counter minipagecount by one.
\noindent{(\theminipagecount)}\hspace{0.1mm} % By default, LaTeX indents the first line of a new paragraph, but \noindent overrides this
% and inserts the current value of the minipagecount counter, enclosed in parentheses
\begin{minipage}[t]{0.45\textwidth} % The [t] option aligns the top of the minipage with the baseline of the surrounding text.
    \vspace{-26pt}  % moves the content of the minipage up, reducing the space between the minipage content and the preceding text.
    \raggedright %  the text lines up on the left side, but the right side will be ragged.
    \begin{align*} % The * align environment does not number the equations- Each line is aligned at the & symbol
        \frac{x}{10} + 5 &= 13\\
        \frac{x}{10} + 5 - \dotuline{\hspace{5mm}} &= 13 - \dotuline{\hspace{5mm}}\\
        \frac{x}{10} &= \dotuline{\hspace{5mm}}\\
        \frac{x}{10} \times\dotuline{\hspace{5mm}} &= \dotuline{\hspace{5mm}} \times\dotuline{\hspace{5mm}}\\
        x &= \dotuline{\hspace{5mm}}\\
    \end{align*}
\end{minipage}\refstepcounter{minipagecount} % increments the counter minipagecount by one.
\noindent{(\theminipagecount)}\hspace{0.1mm} % By default, LaTeX indents the first line of a new paragraph, but \noindent overrides this
% and inserts the current value of the minipagecount counter, enclosed in parentheses
\begin{minipage}[t]{0.45\textwidth} % The [t] option aligns the top of the minipage with the baseline of the surrounding text.
    \vspace{-26pt}  % moves the content of the minipage up, reducing the space between the minipage content and the preceding text.
    \raggedright %  the text lines up on the left side, but the right side will be ragged.
    \begin{align*} % The * align environment does not number the equations- Each line is aligned at the & symbol
        \frac{x}{7} + 4 &= 12\\
        \frac{x}{7} + 4 - \dotuline{\hspace{5mm}} &= 12 - \dotuline{\hspace{5mm}}\\
        \frac{x}{7} &= \dotuline{\hspace{5mm}}\\
        \frac{x}{7} \times\dotuline{\hspace{5mm}} &= \dotuline{\hspace{5mm}} \times\dotuline{\hspace{5mm}}\\
        x &= \dotuline{\hspace{5mm}}\\
    \end{align*}
\end{minipage}\newpage
    \refstepcounter{minipagecount} % increments the counter minipagecount by one.
\noindent{(\theminipagecount)}\hspace{0.1mm} % By default, LaTeX indents the first line of a new paragraph, but \noindent overrides this
% and inserts the current value of the minipagecount counter, enclosed in parentheses
\begin{minipage}[t]{0.45\textwidth} % The [t] option aligns the top of the minipage with the baseline of the surrounding text.
    \vspace{-26pt}  % moves the content of the minipage up, reducing the space between the minipage content and the preceding text.
    \raggedright %  the text lines up on the left side, but the right side will be ragged.
    \begin{align*} % The * align environment does not number the equations- Each line is aligned at the & symbol
        \frac{x}{3} + 1 &= 9\\
        \frac{x}{3} + 1 - \dotuline{\hspace{5mm}} &= 9 - \dotuline{\hspace{5mm}}\\
        \frac{x}{3} &= \dotuline{\hspace{5mm}}\\
        \frac{x}{3} \times\dotuline{\hspace{5mm}} &= \dotuline{\hspace{5mm}} \times\dotuline{\hspace{5mm}}\\
        x &= \dotuline{\hspace{5mm}}\\
    \end{align*}
\end{minipage}\refstepcounter{minipagecount} % increments the counter minipagecount by one.
\noindent{(\theminipagecount)}\hspace{0.1mm} % By default, LaTeX indents the first line of a new paragraph, but \noindent overrides this
% and inserts the current value of the minipagecount counter, enclosed in parentheses
\begin{minipage}[t]{0.45\textwidth} % The [t] option aligns the top of the minipage with the baseline of the surrounding text.
    \vspace{-26pt}  % moves the content of the minipage up, reducing the space between the minipage content and the preceding text.
    \raggedright %  the text lines up on the left side, but the right side will be ragged.
    \begin{align*} % The * align environment does not number the equations- Each line is aligned at the & symbol
        \frac{x}{9} + 8 &= 16\\
        \frac{x}{9} + 8 - \dotuline{\hspace{5mm}} &= 16 - \dotuline{\hspace{5mm}}\\
        \frac{x}{9} &= \dotuline{\hspace{5mm}}\\
        \frac{x}{9} \times\dotuline{\hspace{5mm}} &= \dotuline{\hspace{5mm}} \times\dotuline{\hspace{5mm}}\\
        x &= \dotuline{\hspace{5mm}}\\
    \end{align*}
\end{minipage}\refstepcounter{minipagecount} % increments the counter minipagecount by one.
\noindent{(\theminipagecount)}\hspace{0.1mm} % By default, LaTeX indents the first line of a new paragraph, but \noindent overrides this
% and inserts the current value of the minipagecount counter, enclosed in parentheses
\begin{minipage}[t]{0.45\textwidth} % The [t] option aligns the top of the minipage with the baseline of the surrounding text.
    \vspace{-26pt}  % moves the content of the minipage up, reducing the space between the minipage content and the preceding text.
    \raggedright %  the text lines up on the left side, but the right side will be ragged.
    \begin{align*} % The * align environment does not number the equations- Each line is aligned at the & symbol
        \frac{x}{3} + 7 &= 9\\
        \frac{x}{3} + 7 - \dotuline{\hspace{5mm}} &= 9 - \dotuline{\hspace{5mm}}\\
        \frac{x}{3} &= \dotuline{\hspace{5mm}}\\
        \frac{x}{3} \times\dotuline{\hspace{5mm}} &= \dotuline{\hspace{5mm}} \times\dotuline{\hspace{5mm}}\\
        x &= \dotuline{\hspace{5mm}}\\
    \end{align*}
\end{minipage}\refstepcounter{minipagecount} % increments the counter minipagecount by one.
\noindent{(\theminipagecount)}\hspace{0.1mm} % By default, LaTeX indents the first line of a new paragraph, but \noindent overrides this
% and inserts the current value of the minipagecount counter, enclosed in parentheses
\begin{minipage}[t]{0.45\textwidth} % The [t] option aligns the top of the minipage with the baseline of the surrounding text.
    \vspace{-26pt}  % moves the content of the minipage up, reducing the space between the minipage content and the preceding text.
    \raggedright %  the text lines up on the left side, but the right side will be ragged.
    \begin{align*} % The * align environment does not number the equations- Each line is aligned at the & symbol
        \frac{x}{10} + 4 &= 8\\
        \frac{x}{10} + 4 - \dotuline{\hspace{5mm}} &= 8 - \dotuline{\hspace{5mm}}\\
        \frac{x}{10} &= \dotuline{\hspace{5mm}}\\
        \frac{x}{10} \times\dotuline{\hspace{5mm}} &= \dotuline{\hspace{5mm}} \times\dotuline{\hspace{5mm}}\\
        x &= \dotuline{\hspace{5mm}}\\
    \end{align*}
\end{minipage}\refstepcounter{minipagecount} % increments the counter minipagecount by one.
\noindent{(\theminipagecount)}\hspace{0.1mm} % By default, LaTeX indents the first line of a new paragraph, but \noindent overrides this
% and inserts the current value of the minipagecount counter, enclosed in parentheses
\begin{minipage}[t]{0.45\textwidth} % The [t] option aligns the top of the minipage with the baseline of the surrounding text.
    \vspace{-26pt}  % moves the content of the minipage up, reducing the space between the minipage content and the preceding text.
    \raggedright %  the text lines up on the left side, but the right side will be ragged.
    \begin{align*} % The * align environment does not number the equations- Each line is aligned at the & symbol
        \frac{x}{2} + 5 &= 10\\
        \frac{x}{2} + 5 - \dotuline{\hspace{5mm}} &= 10 - \dotuline{\hspace{5mm}}\\
        \frac{x}{2} &= \dotuline{\hspace{5mm}}\\
        \frac{x}{2} \times\dotuline{\hspace{5mm}} &= \dotuline{\hspace{5mm}} \times\dotuline{\hspace{5mm}}\\
        x &= \dotuline{\hspace{5mm}}\\
    \end{align*}
\end{minipage}\columnbreak
    \refstepcounter{minipagecount} % increments the counter minipagecount by one.
\noindent{(\theminipagecount)}\hspace{0.1mm} % By default, LaTeX indents the first line of a new paragraph, but \noindent overrides this
% and inserts the current value of the minipagecount counter, enclosed in parentheses
\begin{minipage}[t]{0.45\textwidth} % The [t] option aligns the top of the minipage with the baseline of the surrounding text.
    \vspace{-26pt}  % moves the content of the minipage up, reducing the space between the minipage content and the preceding text.
    \raggedright %  the text lines up on the left side, but the right side will be ragged.
    \begin{align*} % The * align environment does not number the equations- Each line is aligned at the & symbol
        \frac{x}{10} + 8 &= 16\\
        \frac{x}{10} + 8 - \dotuline{\hspace{5mm}} &= 16 - \dotuline{\hspace{5mm}}\\
        \frac{x}{10} &= \dotuline{\hspace{5mm}}\\
        \frac{x}{10} \times\dotuline{\hspace{5mm}} &= \dotuline{\hspace{5mm}} \times\dotuline{\hspace{5mm}}\\
        x &= \dotuline{\hspace{5mm}}\\
    \end{align*}
\end{minipage}\refstepcounter{minipagecount} % increments the counter minipagecount by one.
\noindent{(\theminipagecount)}\hspace{0.1mm} % By default, LaTeX indents the first line of a new paragraph, but \noindent overrides this
% and inserts the current value of the minipagecount counter, enclosed in parentheses
\begin{minipage}[t]{0.45\textwidth} % The [t] option aligns the top of the minipage with the baseline of the surrounding text.
    \vspace{-26pt}  % moves the content of the minipage up, reducing the space between the minipage content and the preceding text.
    \raggedright %  the text lines up on the left side, but the right side will be ragged.
    \begin{align*} % The * align environment does not number the equations- Each line is aligned at the & symbol
        \frac{x}{6} + 5 &= 8\\
        \frac{x}{6} + 5 - \dotuline{\hspace{5mm}} &= 8 - \dotuline{\hspace{5mm}}\\
        \frac{x}{6} &= \dotuline{\hspace{5mm}}\\
        \frac{x}{6} \times\dotuline{\hspace{5mm}} &= \dotuline{\hspace{5mm}} \times\dotuline{\hspace{5mm}}\\
        x &= \dotuline{\hspace{5mm}}\\
    \end{align*}
\end{minipage}\refstepcounter{minipagecount} % increments the counter minipagecount by one.
\noindent{(\theminipagecount)}\hspace{0.1mm} % By default, LaTeX indents the first line of a new paragraph, but \noindent overrides this
% and inserts the current value of the minipagecount counter, enclosed in parentheses
\begin{minipage}[t]{0.45\textwidth} % The [t] option aligns the top of the minipage with the baseline of the surrounding text.
    \vspace{-26pt}  % moves the content of the minipage up, reducing the space between the minipage content and the preceding text.
    \raggedright %  the text lines up on the left side, but the right side will be ragged.
    \begin{align*} % The * align environment does not number the equations- Each line is aligned at the & symbol
        \frac{x}{2} + 10 &= 20\\
        \frac{x}{2} + 10 - \dotuline{\hspace{5mm}} &= 20 - \dotuline{\hspace{5mm}}\\
        \frac{x}{2} &= \dotuline{\hspace{5mm}}\\
        \frac{x}{2} \times\dotuline{\hspace{5mm}} &= \dotuline{\hspace{5mm}} \times\dotuline{\hspace{5mm}}\\
        x &= \dotuline{\hspace{5mm}}\\
    \end{align*}
\end{minipage}\refstepcounter{minipagecount} % increments the counter minipagecount by one.
\noindent{(\theminipagecount)}\hspace{0.1mm} % By default, LaTeX indents the first line of a new paragraph, but \noindent overrides this
% and inserts the current value of the minipagecount counter, enclosed in parentheses
\begin{minipage}[t]{0.45\textwidth} % The [t] option aligns the top of the minipage with the baseline of the surrounding text.
    \vspace{-26pt}  % moves the content of the minipage up, reducing the space between the minipage content and the preceding text.
    \raggedright %  the text lines up on the left side, but the right side will be ragged.
    \begin{align*} % The * align environment does not number the equations- Each line is aligned at the & symbol
        \frac{x}{9} + 4 &= 8\\
        \frac{x}{9} + 4 - \dotuline{\hspace{5mm}} &= 8 - \dotuline{\hspace{5mm}}\\
        \frac{x}{9} &= \dotuline{\hspace{5mm}}\\
        \frac{x}{9} \times\dotuline{\hspace{5mm}} &= \dotuline{\hspace{5mm}} \times\dotuline{\hspace{5mm}}\\
        x &= \dotuline{\hspace{5mm}}\\
    \end{align*}
\end{minipage}\refstepcounter{minipagecount} % increments the counter minipagecount by one.
\noindent{(\theminipagecount)}\hspace{0.1mm} % By default, LaTeX indents the first line of a new paragraph, but \noindent overrides this
% and inserts the current value of the minipagecount counter, enclosed in parentheses
\begin{minipage}[t]{0.45\textwidth} % The [t] option aligns the top of the minipage with the baseline of the surrounding text.
    \vspace{-26pt}  % moves the content of the minipage up, reducing the space between the minipage content and the preceding text.
    \raggedright %  the text lines up on the left side, but the right side will be ragged.
    \begin{align*} % The * align environment does not number the equations- Each line is aligned at the & symbol
        \frac{x}{7} + 4 &= 8\\
        \frac{x}{7} + 4 - \dotuline{\hspace{5mm}} &= 8 - \dotuline{\hspace{5mm}}\\
        \frac{x}{7} &= \dotuline{\hspace{5mm}}\\
        \frac{x}{7} \times\dotuline{\hspace{5mm}} &= \dotuline{\hspace{5mm}} \times\dotuline{\hspace{5mm}}\\
        x &= \dotuline{\hspace{5mm}}\\
    \end{align*}
\end{minipage}\newpage
    
\end{multicols}
\end{document}
