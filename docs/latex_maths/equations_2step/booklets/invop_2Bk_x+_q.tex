\documentclass[12pt]{article}
\usepackage{tikz}
\usepackage{amsmath}
% Underlining package
\usepackage[normalem]{ulem} % [normalem] prevents the package from changing the default behavior of \emph to underline.
\usepackage[a4paper, portrait, margin=1cm]{geometry}
\usepackage{multicol}
\usepackage{fancyhdr}

\def \HeadingQuestions {\section*{\Large Name: \underline{\hspace{8cm}} \hfill Date: \underline{\hspace{3cm}}} \vspace{-3mm}
{Inverse operations: Questions} \vspace{1pt}\hrule}

% only in Q due to increased line height with dotted underline
\linespread{1} % Adjust line spacing factor to 0.9 if needed to fit 5 in column

% raise footer with page number; no header
\fancypagestyle{myfancypagestyle}{
  \fancyhf{}% clear all header and footer fields
  \renewcommand{\headrulewidth}{0pt} % no rule under header
  \fancyfoot[C] {\thepage} \setlength{\footskip}{6pt} % raise page number 6pt
}
\pagestyle{myfancypagestyle}  % apply myfancypagestyle
\newcounter{minipagecount}
\begin{document}
\HeadingQuestions
\vspace{1mm}
\begin{multicols}{2}
\refstepcounter{minipagecount} % increments the counter minipagecount by one.
\noindent{(\theminipagecount)}\hspace{0.1mm} % By default, LaTeX indents the first line of a new paragraph, but \noindent overrides this
% and inserts the current value of the minipagecount counter, enclosed in parentheses
\begin{minipage}[t]{0.45\textwidth} % The [t] option aligns the top of the minipage with the baseline of the surrounding text.
    \vspace{-26pt}  % moves the content of the minipage up, reducing the space between the minipage content and the preceding text.
    \raggedright %  the text lines up on the left side, but the right side will be ragged.
    \begin{align*} % The * align environment does not number the equations- Each line is aligned at the & symbol
        2x + 5 &= 7\\
        2x + 5 - \dotuline{\hspace{5mm}} &= 7 - \dotuline{\hspace{5mm}}\\
        2x &= \dotuline{\hspace{5mm}}\\
        \frac{2x}{\dotuline{\hspace{5mm}}} &= \frac{\dotuline{\hspace{5mm}}}{\dotuline{\hspace{5mm}}}\\
        x &= \dotuline{\hspace{5mm}}\\
    \end{align*}
\end{minipage}\refstepcounter{minipagecount} % increments the counter minipagecount by one.
\noindent{(\theminipagecount)}\hspace{0.1mm} % By default, LaTeX indents the first line of a new paragraph, but \noindent overrides this
% and inserts the current value of the minipagecount counter, enclosed in parentheses
\begin{minipage}[t]{0.45\textwidth} % The [t] option aligns the top of the minipage with the baseline of the surrounding text.
    \vspace{-26pt}  % moves the content of the minipage up, reducing the space between the minipage content and the preceding text.
    \raggedright %  the text lines up on the left side, but the right side will be ragged.
    \begin{align*} % The * align environment does not number the equations- Each line is aligned at the & symbol
        9x + 2 &= 83\\
        9x + 2 - \dotuline{\hspace{5mm}} &= 83 - \dotuline{\hspace{5mm}}\\
        9x &= \dotuline{\hspace{5mm}}\\
        \frac{9x}{\dotuline{\hspace{5mm}}} &= \frac{\dotuline{\hspace{5mm}}}{\dotuline{\hspace{5mm}}}\\
        x &= \dotuline{\hspace{5mm}}\\
    \end{align*}
\end{minipage}\refstepcounter{minipagecount} % increments the counter minipagecount by one.
\noindent{(\theminipagecount)}\hspace{0.1mm} % By default, LaTeX indents the first line of a new paragraph, but \noindent overrides this
% and inserts the current value of the minipagecount counter, enclosed in parentheses
\begin{minipage}[t]{0.45\textwidth} % The [t] option aligns the top of the minipage with the baseline of the surrounding text.
    \vspace{-26pt}  % moves the content of the minipage up, reducing the space between the minipage content and the preceding text.
    \raggedright %  the text lines up on the left side, but the right side will be ragged.
    \begin{align*} % The * align environment does not number the equations- Each line is aligned at the & symbol
        10x + 1 &= 71\\
        10x + 1 - \dotuline{\hspace{5mm}} &= 71 - \dotuline{\hspace{5mm}}\\
        10x &= \dotuline{\hspace{5mm}}\\
        \frac{10x}{\dotuline{\hspace{5mm}}} &= \frac{\dotuline{\hspace{5mm}}}{\dotuline{\hspace{5mm}}}\\
        x &= \dotuline{\hspace{5mm}}\\
    \end{align*}
\end{minipage}\refstepcounter{minipagecount} % increments the counter minipagecount by one.
\noindent{(\theminipagecount)}\hspace{0.1mm} % By default, LaTeX indents the first line of a new paragraph, but \noindent overrides this
% and inserts the current value of the minipagecount counter, enclosed in parentheses
\begin{minipage}[t]{0.45\textwidth} % The [t] option aligns the top of the minipage with the baseline of the surrounding text.
    \vspace{-26pt}  % moves the content of the minipage up, reducing the space between the minipage content and the preceding text.
    \raggedright %  the text lines up on the left side, but the right side will be ragged.
    \begin{align*} % The * align environment does not number the equations- Each line is aligned at the & symbol
        5x + 10 &= 45\\
        5x + 10 - \dotuline{\hspace{5mm}} &= 45 - \dotuline{\hspace{5mm}}\\
        5x &= \dotuline{\hspace{5mm}}\\
        \frac{5x}{\dotuline{\hspace{5mm}}} &= \frac{\dotuline{\hspace{5mm}}}{\dotuline{\hspace{5mm}}}\\
        x &= \dotuline{\hspace{5mm}}\\
    \end{align*}
\end{minipage}\refstepcounter{minipagecount} % increments the counter minipagecount by one.
\noindent{(\theminipagecount)}\hspace{0.1mm} % By default, LaTeX indents the first line of a new paragraph, but \noindent overrides this
% and inserts the current value of the minipagecount counter, enclosed in parentheses
\begin{minipage}[t]{0.45\textwidth} % The [t] option aligns the top of the minipage with the baseline of the surrounding text.
    \vspace{-26pt}  % moves the content of the minipage up, reducing the space between the minipage content and the preceding text.
    \raggedright %  the text lines up on the left side, but the right side will be ragged.
    \begin{align*} % The * align environment does not number the equations- Each line is aligned at the & symbol
        8x + 3 &= 51\\
        8x + 3 - \dotuline{\hspace{5mm}} &= 51 - \dotuline{\hspace{5mm}}\\
        8x &= \dotuline{\hspace{5mm}}\\
        \frac{8x}{\dotuline{\hspace{5mm}}} &= \frac{\dotuline{\hspace{5mm}}}{\dotuline{\hspace{5mm}}}\\
        x &= \dotuline{\hspace{5mm}}\\
    \end{align*}
\end{minipage}\columnbreak
    \refstepcounter{minipagecount} % increments the counter minipagecount by one.
\noindent{(\theminipagecount)}\hspace{0.1mm} % By default, LaTeX indents the first line of a new paragraph, but \noindent overrides this
% and inserts the current value of the minipagecount counter, enclosed in parentheses
\begin{minipage}[t]{0.45\textwidth} % The [t] option aligns the top of the minipage with the baseline of the surrounding text.
    \vspace{-26pt}  % moves the content of the minipage up, reducing the space between the minipage content and the preceding text.
    \raggedright %  the text lines up on the left side, but the right side will be ragged.
    \begin{align*} % The * align environment does not number the equations- Each line is aligned at the & symbol
        7x + 6 &= 13\\
        7x + 6 - \dotuline{\hspace{5mm}} &= 13 - \dotuline{\hspace{5mm}}\\
        7x &= \dotuline{\hspace{5mm}}\\
        \frac{7x}{\dotuline{\hspace{5mm}}} &= \frac{\dotuline{\hspace{5mm}}}{\dotuline{\hspace{5mm}}}\\
        x &= \dotuline{\hspace{5mm}}\\
    \end{align*}
\end{minipage}\refstepcounter{minipagecount} % increments the counter minipagecount by one.
\noindent{(\theminipagecount)}\hspace{0.1mm} % By default, LaTeX indents the first line of a new paragraph, but \noindent overrides this
% and inserts the current value of the minipagecount counter, enclosed in parentheses
\begin{minipage}[t]{0.45\textwidth} % The [t] option aligns the top of the minipage with the baseline of the surrounding text.
    \vspace{-26pt}  % moves the content of the minipage up, reducing the space between the minipage content and the preceding text.
    \raggedright %  the text lines up on the left side, but the right side will be ragged.
    \begin{align*} % The * align environment does not number the equations- Each line is aligned at the & symbol
        3x + 9 &= 39\\
        3x + 9 - \dotuline{\hspace{5mm}} &= 39 - \dotuline{\hspace{5mm}}\\
        3x &= \dotuline{\hspace{5mm}}\\
        \frac{3x}{\dotuline{\hspace{5mm}}} &= \frac{\dotuline{\hspace{5mm}}}{\dotuline{\hspace{5mm}}}\\
        x &= \dotuline{\hspace{5mm}}\\
    \end{align*}
\end{minipage}\refstepcounter{minipagecount} % increments the counter minipagecount by one.
\noindent{(\theminipagecount)}\hspace{0.1mm} % By default, LaTeX indents the first line of a new paragraph, but \noindent overrides this
% and inserts the current value of the minipagecount counter, enclosed in parentheses
\begin{minipage}[t]{0.45\textwidth} % The [t] option aligns the top of the minipage with the baseline of the surrounding text.
    \vspace{-26pt}  % moves the content of the minipage up, reducing the space between the minipage content and the preceding text.
    \raggedright %  the text lines up on the left side, but the right side will be ragged.
    \begin{align*} % The * align environment does not number the equations- Each line is aligned at the & symbol
        9x + 1 &= 19\\
        9x + 1 - \dotuline{\hspace{5mm}} &= 19 - \dotuline{\hspace{5mm}}\\
        9x &= \dotuline{\hspace{5mm}}\\
        \frac{9x}{\dotuline{\hspace{5mm}}} &= \frac{\dotuline{\hspace{5mm}}}{\dotuline{\hspace{5mm}}}\\
        x &= \dotuline{\hspace{5mm}}\\
    \end{align*}
\end{minipage}\refstepcounter{minipagecount} % increments the counter minipagecount by one.
\noindent{(\theminipagecount)}\hspace{0.1mm} % By default, LaTeX indents the first line of a new paragraph, but \noindent overrides this
% and inserts the current value of the minipagecount counter, enclosed in parentheses
\begin{minipage}[t]{0.45\textwidth} % The [t] option aligns the top of the minipage with the baseline of the surrounding text.
    \vspace{-26pt}  % moves the content of the minipage up, reducing the space between the minipage content and the preceding text.
    \raggedright %  the text lines up on the left side, but the right side will be ragged.
    \begin{align*} % The * align environment does not number the equations- Each line is aligned at the & symbol
        2x + 10 &= 20\\
        2x + 10 - \dotuline{\hspace{5mm}} &= 20 - \dotuline{\hspace{5mm}}\\
        2x &= \dotuline{\hspace{5mm}}\\
        \frac{2x}{\dotuline{\hspace{5mm}}} &= \frac{\dotuline{\hspace{5mm}}}{\dotuline{\hspace{5mm}}}\\
        x &= \dotuline{\hspace{5mm}}\\
    \end{align*}
\end{minipage}\refstepcounter{minipagecount} % increments the counter minipagecount by one.
\noindent{(\theminipagecount)}\hspace{0.1mm} % By default, LaTeX indents the first line of a new paragraph, but \noindent overrides this
% and inserts the current value of the minipagecount counter, enclosed in parentheses
\begin{minipage}[t]{0.45\textwidth} % The [t] option aligns the top of the minipage with the baseline of the surrounding text.
    \vspace{-26pt}  % moves the content of the minipage up, reducing the space between the minipage content and the preceding text.
    \raggedright %  the text lines up on the left side, but the right side will be ragged.
    \begin{align*} % The * align environment does not number the equations- Each line is aligned at the & symbol
        3x + 6 &= 21\\
        3x + 6 - \dotuline{\hspace{5mm}} &= 21 - \dotuline{\hspace{5mm}}\\
        3x &= \dotuline{\hspace{5mm}}\\
        \frac{3x}{\dotuline{\hspace{5mm}}} &= \frac{\dotuline{\hspace{5mm}}}{\dotuline{\hspace{5mm}}}\\
        x &= \dotuline{\hspace{5mm}}\\
    \end{align*}
\end{minipage}\newpage
    \refstepcounter{minipagecount} % increments the counter minipagecount by one.
\noindent{(\theminipagecount)}\hspace{0.1mm} % By default, LaTeX indents the first line of a new paragraph, but \noindent overrides this
% and inserts the current value of the minipagecount counter, enclosed in parentheses
\begin{minipage}[t]{0.45\textwidth} % The [t] option aligns the top of the minipage with the baseline of the surrounding text.
    \vspace{-26pt}  % moves the content of the minipage up, reducing the space between the minipage content and the preceding text.
    \raggedright %  the text lines up on the left side, but the right side will be ragged.
    \begin{align*} % The * align environment does not number the equations- Each line is aligned at the & symbol
        6x + 3 &= 39\\
        6x + 3 - \dotuline{\hspace{5mm}} &= 39 - \dotuline{\hspace{5mm}}\\
        6x &= \dotuline{\hspace{5mm}}\\
        \frac{6x}{\dotuline{\hspace{5mm}}} &= \frac{\dotuline{\hspace{5mm}}}{\dotuline{\hspace{5mm}}}\\
        x &= \dotuline{\hspace{5mm}}\\
    \end{align*}
\end{minipage}\refstepcounter{minipagecount} % increments the counter minipagecount by one.
\noindent{(\theminipagecount)}\hspace{0.1mm} % By default, LaTeX indents the first line of a new paragraph, but \noindent overrides this
% and inserts the current value of the minipagecount counter, enclosed in parentheses
\begin{minipage}[t]{0.45\textwidth} % The [t] option aligns the top of the minipage with the baseline of the surrounding text.
    \vspace{-26pt}  % moves the content of the minipage up, reducing the space between the minipage content and the preceding text.
    \raggedright %  the text lines up on the left side, but the right side will be ragged.
    \begin{align*} % The * align environment does not number the equations- Each line is aligned at the & symbol
        2x + 8 &= 10\\
        2x + 8 - \dotuline{\hspace{5mm}} &= 10 - \dotuline{\hspace{5mm}}\\
        2x &= \dotuline{\hspace{5mm}}\\
        \frac{2x}{\dotuline{\hspace{5mm}}} &= \frac{\dotuline{\hspace{5mm}}}{\dotuline{\hspace{5mm}}}\\
        x &= \dotuline{\hspace{5mm}}\\
    \end{align*}
\end{minipage}\refstepcounter{minipagecount} % increments the counter minipagecount by one.
\noindent{(\theminipagecount)}\hspace{0.1mm} % By default, LaTeX indents the first line of a new paragraph, but \noindent overrides this
% and inserts the current value of the minipagecount counter, enclosed in parentheses
\begin{minipage}[t]{0.45\textwidth} % The [t] option aligns the top of the minipage with the baseline of the surrounding text.
    \vspace{-26pt}  % moves the content of the minipage up, reducing the space between the minipage content and the preceding text.
    \raggedright %  the text lines up on the left side, but the right side will be ragged.
    \begin{align*} % The * align environment does not number the equations- Each line is aligned at the & symbol
        7x + 4 &= 46\\
        7x + 4 - \dotuline{\hspace{5mm}} &= 46 - \dotuline{\hspace{5mm}}\\
        7x &= \dotuline{\hspace{5mm}}\\
        \frac{7x}{\dotuline{\hspace{5mm}}} &= \frac{\dotuline{\hspace{5mm}}}{\dotuline{\hspace{5mm}}}\\
        x &= \dotuline{\hspace{5mm}}\\
    \end{align*}
\end{minipage}\refstepcounter{minipagecount} % increments the counter minipagecount by one.
\noindent{(\theminipagecount)}\hspace{0.1mm} % By default, LaTeX indents the first line of a new paragraph, but \noindent overrides this
% and inserts the current value of the minipagecount counter, enclosed in parentheses
\begin{minipage}[t]{0.45\textwidth} % The [t] option aligns the top of the minipage with the baseline of the surrounding text.
    \vspace{-26pt}  % moves the content of the minipage up, reducing the space between the minipage content and the preceding text.
    \raggedright %  the text lines up on the left side, but the right side will be ragged.
    \begin{align*} % The * align environment does not number the equations- Each line is aligned at the & symbol
        9x + 4 &= 40\\
        9x + 4 - \dotuline{\hspace{5mm}} &= 40 - \dotuline{\hspace{5mm}}\\
        9x &= \dotuline{\hspace{5mm}}\\
        \frac{9x}{\dotuline{\hspace{5mm}}} &= \frac{\dotuline{\hspace{5mm}}}{\dotuline{\hspace{5mm}}}\\
        x &= \dotuline{\hspace{5mm}}\\
    \end{align*}
\end{minipage}\refstepcounter{minipagecount} % increments the counter minipagecount by one.
\noindent{(\theminipagecount)}\hspace{0.1mm} % By default, LaTeX indents the first line of a new paragraph, but \noindent overrides this
% and inserts the current value of the minipagecount counter, enclosed in parentheses
\begin{minipage}[t]{0.45\textwidth} % The [t] option aligns the top of the minipage with the baseline of the surrounding text.
    \vspace{-26pt}  % moves the content of the minipage up, reducing the space between the minipage content and the preceding text.
    \raggedright %  the text lines up on the left side, but the right side will be ragged.
    \begin{align*} % The * align environment does not number the equations- Each line is aligned at the & symbol
        4x + 10 &= 34\\
        4x + 10 - \dotuline{\hspace{5mm}} &= 34 - \dotuline{\hspace{5mm}}\\
        4x &= \dotuline{\hspace{5mm}}\\
        \frac{4x}{\dotuline{\hspace{5mm}}} &= \frac{\dotuline{\hspace{5mm}}}{\dotuline{\hspace{5mm}}}\\
        x &= \dotuline{\hspace{5mm}}\\
    \end{align*}
\end{minipage}\columnbreak
    \refstepcounter{minipagecount} % increments the counter minipagecount by one.
\noindent{(\theminipagecount)}\hspace{0.1mm} % By default, LaTeX indents the first line of a new paragraph, but \noindent overrides this
% and inserts the current value of the minipagecount counter, enclosed in parentheses
\begin{minipage}[t]{0.45\textwidth} % The [t] option aligns the top of the minipage with the baseline of the surrounding text.
    \vspace{-26pt}  % moves the content of the minipage up, reducing the space between the minipage content and the preceding text.
    \raggedright %  the text lines up on the left side, but the right side will be ragged.
    \begin{align*} % The * align environment does not number the equations- Each line is aligned at the & symbol
        4x + 8 &= 44\\
        4x + 8 - \dotuline{\hspace{5mm}} &= 44 - \dotuline{\hspace{5mm}}\\
        4x &= \dotuline{\hspace{5mm}}\\
        \frac{4x}{\dotuline{\hspace{5mm}}} &= \frac{\dotuline{\hspace{5mm}}}{\dotuline{\hspace{5mm}}}\\
        x &= \dotuline{\hspace{5mm}}\\
    \end{align*}
\end{minipage}\refstepcounter{minipagecount} % increments the counter minipagecount by one.
\noindent{(\theminipagecount)}\hspace{0.1mm} % By default, LaTeX indents the first line of a new paragraph, but \noindent overrides this
% and inserts the current value of the minipagecount counter, enclosed in parentheses
\begin{minipage}[t]{0.45\textwidth} % The [t] option aligns the top of the minipage with the baseline of the surrounding text.
    \vspace{-26pt}  % moves the content of the minipage up, reducing the space between the minipage content and the preceding text.
    \raggedright %  the text lines up on the left side, but the right side will be ragged.
    \begin{align*} % The * align environment does not number the equations- Each line is aligned at the & symbol
        5x + 5 &= 20\\
        5x + 5 - \dotuline{\hspace{5mm}} &= 20 - \dotuline{\hspace{5mm}}\\
        5x &= \dotuline{\hspace{5mm}}\\
        \frac{5x}{\dotuline{\hspace{5mm}}} &= \frac{\dotuline{\hspace{5mm}}}{\dotuline{\hspace{5mm}}}\\
        x &= \dotuline{\hspace{5mm}}\\
    \end{align*}
\end{minipage}\refstepcounter{minipagecount} % increments the counter minipagecount by one.
\noindent{(\theminipagecount)}\hspace{0.1mm} % By default, LaTeX indents the first line of a new paragraph, but \noindent overrides this
% and inserts the current value of the minipagecount counter, enclosed in parentheses
\begin{minipage}[t]{0.45\textwidth} % The [t] option aligns the top of the minipage with the baseline of the surrounding text.
    \vspace{-26pt}  % moves the content of the minipage up, reducing the space between the minipage content and the preceding text.
    \raggedright %  the text lines up on the left side, but the right side will be ragged.
    \begin{align*} % The * align environment does not number the equations- Each line is aligned at the & symbol
        9x + 6 &= 33\\
        9x + 6 - \dotuline{\hspace{5mm}} &= 33 - \dotuline{\hspace{5mm}}\\
        9x &= \dotuline{\hspace{5mm}}\\
        \frac{9x}{\dotuline{\hspace{5mm}}} &= \frac{\dotuline{\hspace{5mm}}}{\dotuline{\hspace{5mm}}}\\
        x &= \dotuline{\hspace{5mm}}\\
    \end{align*}
\end{minipage}\refstepcounter{minipagecount} % increments the counter minipagecount by one.
\noindent{(\theminipagecount)}\hspace{0.1mm} % By default, LaTeX indents the first line of a new paragraph, but \noindent overrides this
% and inserts the current value of the minipagecount counter, enclosed in parentheses
\begin{minipage}[t]{0.45\textwidth} % The [t] option aligns the top of the minipage with the baseline of the surrounding text.
    \vspace{-26pt}  % moves the content of the minipage up, reducing the space between the minipage content and the preceding text.
    \raggedright %  the text lines up on the left side, but the right side will be ragged.
    \begin{align*} % The * align environment does not number the equations- Each line is aligned at the & symbol
        6x + 6 &= 48\\
        6x + 6 - \dotuline{\hspace{5mm}} &= 48 - \dotuline{\hspace{5mm}}\\
        6x &= \dotuline{\hspace{5mm}}\\
        \frac{6x}{\dotuline{\hspace{5mm}}} &= \frac{\dotuline{\hspace{5mm}}}{\dotuline{\hspace{5mm}}}\\
        x &= \dotuline{\hspace{5mm}}\\
    \end{align*}
\end{minipage}\refstepcounter{minipagecount} % increments the counter minipagecount by one.
\noindent{(\theminipagecount)}\hspace{0.1mm} % By default, LaTeX indents the first line of a new paragraph, but \noindent overrides this
% and inserts the current value of the minipagecount counter, enclosed in parentheses
\begin{minipage}[t]{0.45\textwidth} % The [t] option aligns the top of the minipage with the baseline of the surrounding text.
    \vspace{-26pt}  % moves the content of the minipage up, reducing the space between the minipage content and the preceding text.
    \raggedright %  the text lines up on the left side, but the right side will be ragged.
    \begin{align*} % The * align environment does not number the equations- Each line is aligned at the & symbol
        9x + 8 &= 53\\
        9x + 8 - \dotuline{\hspace{5mm}} &= 53 - \dotuline{\hspace{5mm}}\\
        9x &= \dotuline{\hspace{5mm}}\\
        \frac{9x}{\dotuline{\hspace{5mm}}} &= \frac{\dotuline{\hspace{5mm}}}{\dotuline{\hspace{5mm}}}\\
        x &= \dotuline{\hspace{5mm}}\\
    \end{align*}
\end{minipage}\newpage
    
\end{multicols}
\end{document}
