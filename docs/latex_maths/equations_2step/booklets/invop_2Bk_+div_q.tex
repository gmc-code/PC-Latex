\documentclass[12pt]{article}
\usepackage{tikz}
\usepackage{amsmath}
% Underlining package
\usepackage[normalem]{ulem} % [normalem] prevents the package from changing the default behavior of \emph to underline.
\usepackage[a4paper, portrait, margin=1cm]{geometry}
\usepackage{multicol}
\usepackage{fancyhdr}

\def \HeadingQuestions {\section*{\Large Name: \underline{\hspace{8cm}} \hfill Date: \underline{\hspace{3cm}}} \vspace{-3mm}
{Inverse operations: Questions} \vspace{1pt}\hrule}

% only in Q due to increased line height with dotted underline
\linespread{0.9} % Adjust line spacing factor to 0.9 if needed to fit 5 in column for 2 div steps

% raise footer with page number; no header
\fancypagestyle{myfancypagestyle}{
  \fancyhf{}% clear all header and footer fields
  \renewcommand{\headrulewidth}{0pt} % no rule under header
  \fancyfoot[C] {\thepage} \setlength{\footskip}{14.5pt} % raise page number allowed min 14.5pt
}
\pagestyle{myfancypagestyle}  % apply myfancypagestyle
\newcounter{minipagecount}
\begin{document}
\HeadingQuestions
\vspace{1mm}
\begin{multicols}{2}
\refstepcounter{minipagecount} % increments the counter minipagecount by one.
\noindent{(\theminipagecount)}\hspace{0.1mm} % By default, LaTeX indents the first line of a new paragraph, but \noindent overrides this
% and inserts the current value of the minipagecount counter, enclosed in parentheses
\begin{minipage}[t]{0.45\textwidth} % The [t] option aligns the top of the minipage with the baseline of the surrounding text.
    \vspace{-26pt}  % moves the content of the minipage up, reducing the space between the minipage content and the preceding text.
    \raggedright %  the text lines up on the left side, but the right side will be ragged.
    \begin{align*} % The * align environment does not number the equations- Each line is aligned at the & symbol
        \frac{x + 6}{4} &= 3\\
        \frac{x + 6}{4} \times\dotuline{\hspace{5mm}} &= 3 \times\dotuline{\hspace{5mm}}\\
        x + 6 &= \dotuline{\hspace{5mm}}\\
        x + 6 - \dotuline{\hspace{5mm}} &= \dotuline{\hspace{5mm}} - \dotuline{\hspace{5mm}}\\
        x &= \dotuline{\hspace{5mm}}\\
    \end{align*}
\end{minipage}\refstepcounter{minipagecount} % increments the counter minipagecount by one.
\noindent{(\theminipagecount)}\hspace{0.1mm} % By default, LaTeX indents the first line of a new paragraph, but \noindent overrides this
% and inserts the current value of the minipagecount counter, enclosed in parentheses
\begin{minipage}[t]{0.45\textwidth} % The [t] option aligns the top of the minipage with the baseline of the surrounding text.
    \vspace{-26pt}  % moves the content of the minipage up, reducing the space between the minipage content and the preceding text.
    \raggedright %  the text lines up on the left side, but the right side will be ragged.
    \begin{align*} % The * align environment does not number the equations- Each line is aligned at the & symbol
        \frac{x + 6}{6} &= 3\\
        \frac{x + 6}{6} \times\dotuline{\hspace{5mm}} &= 3 \times\dotuline{\hspace{5mm}}\\
        x + 6 &= \dotuline{\hspace{5mm}}\\
        x + 6 - \dotuline{\hspace{5mm}} &= \dotuline{\hspace{5mm}} - \dotuline{\hspace{5mm}}\\
        x &= \dotuline{\hspace{5mm}}\\
    \end{align*}
\end{minipage}\refstepcounter{minipagecount} % increments the counter minipagecount by one.
\noindent{(\theminipagecount)}\hspace{0.1mm} % By default, LaTeX indents the first line of a new paragraph, but \noindent overrides this
% and inserts the current value of the minipagecount counter, enclosed in parentheses
\begin{minipage}[t]{0.45\textwidth} % The [t] option aligns the top of the minipage with the baseline of the surrounding text.
    \vspace{-26pt}  % moves the content of the minipage up, reducing the space between the minipage content and the preceding text.
    \raggedright %  the text lines up on the left side, but the right side will be ragged.
    \begin{align*} % The * align environment does not number the equations- Each line is aligned at the & symbol
        \frac{x + 2}{10} &= 5\\
        \frac{x + 2}{10} \times\dotuline{\hspace{5mm}} &= 5 \times\dotuline{\hspace{5mm}}\\
        x + 2 &= \dotuline{\hspace{5mm}}\\
        x + 2 - \dotuline{\hspace{5mm}} &= \dotuline{\hspace{5mm}} - \dotuline{\hspace{5mm}}\\
        x &= \dotuline{\hspace{5mm}}\\
    \end{align*}
\end{minipage}\refstepcounter{minipagecount} % increments the counter minipagecount by one.
\noindent{(\theminipagecount)}\hspace{0.1mm} % By default, LaTeX indents the first line of a new paragraph, but \noindent overrides this
% and inserts the current value of the minipagecount counter, enclosed in parentheses
\begin{minipage}[t]{0.45\textwidth} % The [t] option aligns the top of the minipage with the baseline of the surrounding text.
    \vspace{-26pt}  % moves the content of the minipage up, reducing the space between the minipage content and the preceding text.
    \raggedright %  the text lines up on the left side, but the right side will be ragged.
    \begin{align*} % The * align environment does not number the equations- Each line is aligned at the & symbol
        \frac{x + 7}{4} &= 3\\
        \frac{x + 7}{4} \times\dotuline{\hspace{5mm}} &= 3 \times\dotuline{\hspace{5mm}}\\
        x + 7 &= \dotuline{\hspace{5mm}}\\
        x + 7 - \dotuline{\hspace{5mm}} &= \dotuline{\hspace{5mm}} - \dotuline{\hspace{5mm}}\\
        x &= \dotuline{\hspace{5mm}}\\
    \end{align*}
\end{minipage}\refstepcounter{minipagecount} % increments the counter minipagecount by one.
\noindent{(\theminipagecount)}\hspace{0.1mm} % By default, LaTeX indents the first line of a new paragraph, but \noindent overrides this
% and inserts the current value of the minipagecount counter, enclosed in parentheses
\begin{minipage}[t]{0.45\textwidth} % The [t] option aligns the top of the minipage with the baseline of the surrounding text.
    \vspace{-26pt}  % moves the content of the minipage up, reducing the space between the minipage content and the preceding text.
    \raggedright %  the text lines up on the left side, but the right side will be ragged.
    \begin{align*} % The * align environment does not number the equations- Each line is aligned at the & symbol
        \frac{x + 6}{7} &= 2\\
        \frac{x + 6}{7} \times\dotuline{\hspace{5mm}} &= 2 \times\dotuline{\hspace{5mm}}\\
        x + 6 &= \dotuline{\hspace{5mm}}\\
        x + 6 - \dotuline{\hspace{5mm}} &= \dotuline{\hspace{5mm}} - \dotuline{\hspace{5mm}}\\
        x &= \dotuline{\hspace{5mm}}\\
    \end{align*}
\end{minipage}\columnbreak
    \refstepcounter{minipagecount} % increments the counter minipagecount by one.
\noindent{(\theminipagecount)}\hspace{0.1mm} % By default, LaTeX indents the first line of a new paragraph, but \noindent overrides this
% and inserts the current value of the minipagecount counter, enclosed in parentheses
\begin{minipage}[t]{0.45\textwidth} % The [t] option aligns the top of the minipage with the baseline of the surrounding text.
    \vspace{-26pt}  % moves the content of the minipage up, reducing the space between the minipage content and the preceding text.
    \raggedright %  the text lines up on the left side, but the right side will be ragged.
    \begin{align*} % The * align environment does not number the equations- Each line is aligned at the & symbol
        \frac{x + 1}{2} &= 7\\
        \frac{x + 1}{2} \times\dotuline{\hspace{5mm}} &= 7 \times\dotuline{\hspace{5mm}}\\
        x + 1 &= \dotuline{\hspace{5mm}}\\
        x + 1 - \dotuline{\hspace{5mm}} &= \dotuline{\hspace{5mm}} - \dotuline{\hspace{5mm}}\\
        x &= \dotuline{\hspace{5mm}}\\
    \end{align*}
\end{minipage}\refstepcounter{minipagecount} % increments the counter minipagecount by one.
\noindent{(\theminipagecount)}\hspace{0.1mm} % By default, LaTeX indents the first line of a new paragraph, but \noindent overrides this
% and inserts the current value of the minipagecount counter, enclosed in parentheses
\begin{minipage}[t]{0.45\textwidth} % The [t] option aligns the top of the minipage with the baseline of the surrounding text.
    \vspace{-26pt}  % moves the content of the minipage up, reducing the space between the minipage content and the preceding text.
    \raggedright %  the text lines up on the left side, but the right side will be ragged.
    \begin{align*} % The * align environment does not number the equations- Each line is aligned at the & symbol
        \frac{x + 3}{7} &= 8\\
        \frac{x + 3}{7} \times\dotuline{\hspace{5mm}} &= 8 \times\dotuline{\hspace{5mm}}\\
        x + 3 &= \dotuline{\hspace{5mm}}\\
        x + 3 - \dotuline{\hspace{5mm}} &= \dotuline{\hspace{5mm}} - \dotuline{\hspace{5mm}}\\
        x &= \dotuline{\hspace{5mm}}\\
    \end{align*}
\end{minipage}\refstepcounter{minipagecount} % increments the counter minipagecount by one.
\noindent{(\theminipagecount)}\hspace{0.1mm} % By default, LaTeX indents the first line of a new paragraph, but \noindent overrides this
% and inserts the current value of the minipagecount counter, enclosed in parentheses
\begin{minipage}[t]{0.45\textwidth} % The [t] option aligns the top of the minipage with the baseline of the surrounding text.
    \vspace{-26pt}  % moves the content of the minipage up, reducing the space between the minipage content and the preceding text.
    \raggedright %  the text lines up on the left side, but the right side will be ragged.
    \begin{align*} % The * align environment does not number the equations- Each line is aligned at the & symbol
        \frac{x + 1}{10} &= 2\\
        \frac{x + 1}{10} \times\dotuline{\hspace{5mm}} &= 2 \times\dotuline{\hspace{5mm}}\\
        x + 1 &= \dotuline{\hspace{5mm}}\\
        x + 1 - \dotuline{\hspace{5mm}} &= \dotuline{\hspace{5mm}} - \dotuline{\hspace{5mm}}\\
        x &= \dotuline{\hspace{5mm}}\\
    \end{align*}
\end{minipage}\refstepcounter{minipagecount} % increments the counter minipagecount by one.
\noindent{(\theminipagecount)}\hspace{0.1mm} % By default, LaTeX indents the first line of a new paragraph, but \noindent overrides this
% and inserts the current value of the minipagecount counter, enclosed in parentheses
\begin{minipage}[t]{0.45\textwidth} % The [t] option aligns the top of the minipage with the baseline of the surrounding text.
    \vspace{-26pt}  % moves the content of the minipage up, reducing the space between the minipage content and the preceding text.
    \raggedright %  the text lines up on the left side, but the right side will be ragged.
    \begin{align*} % The * align environment does not number the equations- Each line is aligned at the & symbol
        \frac{x + 5}{3} &= 7\\
        \frac{x + 5}{3} \times\dotuline{\hspace{5mm}} &= 7 \times\dotuline{\hspace{5mm}}\\
        x + 5 &= \dotuline{\hspace{5mm}}\\
        x + 5 - \dotuline{\hspace{5mm}} &= \dotuline{\hspace{5mm}} - \dotuline{\hspace{5mm}}\\
        x &= \dotuline{\hspace{5mm}}\\
    \end{align*}
\end{minipage}\refstepcounter{minipagecount} % increments the counter minipagecount by one.
\noindent{(\theminipagecount)}\hspace{0.1mm} % By default, LaTeX indents the first line of a new paragraph, but \noindent overrides this
% and inserts the current value of the minipagecount counter, enclosed in parentheses
\begin{minipage}[t]{0.45\textwidth} % The [t] option aligns the top of the minipage with the baseline of the surrounding text.
    \vspace{-26pt}  % moves the content of the minipage up, reducing the space between the minipage content and the preceding text.
    \raggedright %  the text lines up on the left side, but the right side will be ragged.
    \begin{align*} % The * align environment does not number the equations- Each line is aligned at the & symbol
        \frac{x + 5}{2} &= 10\\
        \frac{x + 5}{2} \times\dotuline{\hspace{5mm}} &= 10 \times\dotuline{\hspace{5mm}}\\
        x + 5 &= \dotuline{\hspace{5mm}}\\
        x + 5 - \dotuline{\hspace{5mm}} &= \dotuline{\hspace{5mm}} - \dotuline{\hspace{5mm}}\\
        x &= \dotuline{\hspace{5mm}}\\
    \end{align*}
\end{minipage}\newpage
    \refstepcounter{minipagecount} % increments the counter minipagecount by one.
\noindent{(\theminipagecount)}\hspace{0.1mm} % By default, LaTeX indents the first line of a new paragraph, but \noindent overrides this
% and inserts the current value of the minipagecount counter, enclosed in parentheses
\begin{minipage}[t]{0.45\textwidth} % The [t] option aligns the top of the minipage with the baseline of the surrounding text.
    \vspace{-26pt}  % moves the content of the minipage up, reducing the space between the minipage content and the preceding text.
    \raggedright %  the text lines up on the left side, but the right side will be ragged.
    \begin{align*} % The * align environment does not number the equations- Each line is aligned at the & symbol
        \frac{x + 8}{5} &= 1\\
        \frac{x + 8}{5} \times\dotuline{\hspace{5mm}} &= 1 \times\dotuline{\hspace{5mm}}\\
        x + 8 &= \dotuline{\hspace{5mm}}\\
        x + 8 - \dotuline{\hspace{5mm}} &= \dotuline{\hspace{5mm}} - \dotuline{\hspace{5mm}}\\
        x &= \dotuline{\hspace{5mm}}\\
    \end{align*}
\end{minipage}\refstepcounter{minipagecount} % increments the counter minipagecount by one.
\noindent{(\theminipagecount)}\hspace{0.1mm} % By default, LaTeX indents the first line of a new paragraph, but \noindent overrides this
% and inserts the current value of the minipagecount counter, enclosed in parentheses
\begin{minipage}[t]{0.45\textwidth} % The [t] option aligns the top of the minipage with the baseline of the surrounding text.
    \vspace{-26pt}  % moves the content of the minipage up, reducing the space between the minipage content and the preceding text.
    \raggedright %  the text lines up on the left side, but the right side will be ragged.
    \begin{align*} % The * align environment does not number the equations- Each line is aligned at the & symbol
        \frac{x + 7}{9} &= 6\\
        \frac{x + 7}{9} \times\dotuline{\hspace{5mm}} &= 6 \times\dotuline{\hspace{5mm}}\\
        x + 7 &= \dotuline{\hspace{5mm}}\\
        x + 7 - \dotuline{\hspace{5mm}} &= \dotuline{\hspace{5mm}} - \dotuline{\hspace{5mm}}\\
        x &= \dotuline{\hspace{5mm}}\\
    \end{align*}
\end{minipage}\refstepcounter{minipagecount} % increments the counter minipagecount by one.
\noindent{(\theminipagecount)}\hspace{0.1mm} % By default, LaTeX indents the first line of a new paragraph, but \noindent overrides this
% and inserts the current value of the minipagecount counter, enclosed in parentheses
\begin{minipage}[t]{0.45\textwidth} % The [t] option aligns the top of the minipage with the baseline of the surrounding text.
    \vspace{-26pt}  % moves the content of the minipage up, reducing the space between the minipage content and the preceding text.
    \raggedright %  the text lines up on the left side, but the right side will be ragged.
    \begin{align*} % The * align environment does not number the equations- Each line is aligned at the & symbol
        \frac{x + 10}{10} &= 2\\
        \frac{x + 10}{10} \times\dotuline{\hspace{5mm}} &= 2 \times\dotuline{\hspace{5mm}}\\
        x + 10 &= \dotuline{\hspace{5mm}}\\
        x + 10 - \dotuline{\hspace{5mm}} &= \dotuline{\hspace{5mm}} - \dotuline{\hspace{5mm}}\\
        x &= \dotuline{\hspace{5mm}}\\
    \end{align*}
\end{minipage}\refstepcounter{minipagecount} % increments the counter minipagecount by one.
\noindent{(\theminipagecount)}\hspace{0.1mm} % By default, LaTeX indents the first line of a new paragraph, but \noindent overrides this
% and inserts the current value of the minipagecount counter, enclosed in parentheses
\begin{minipage}[t]{0.45\textwidth} % The [t] option aligns the top of the minipage with the baseline of the surrounding text.
    \vspace{-26pt}  % moves the content of the minipage up, reducing the space between the minipage content and the preceding text.
    \raggedright %  the text lines up on the left side, but the right side will be ragged.
    \begin{align*} % The * align environment does not number the equations- Each line is aligned at the & symbol
        \frac{x + 10}{8} &= 1\\
        \frac{x + 10}{8} \times\dotuline{\hspace{5mm}} &= 1 \times\dotuline{\hspace{5mm}}\\
        x + 10 &= \dotuline{\hspace{5mm}}\\
        x + 10 - \dotuline{\hspace{5mm}} &= \dotuline{\hspace{5mm}} - \dotuline{\hspace{5mm}}\\
        x &= \dotuline{\hspace{5mm}}\\
    \end{align*}
\end{minipage}\refstepcounter{minipagecount} % increments the counter minipagecount by one.
\noindent{(\theminipagecount)}\hspace{0.1mm} % By default, LaTeX indents the first line of a new paragraph, but \noindent overrides this
% and inserts the current value of the minipagecount counter, enclosed in parentheses
\begin{minipage}[t]{0.45\textwidth} % The [t] option aligns the top of the minipage with the baseline of the surrounding text.
    \vspace{-26pt}  % moves the content of the minipage up, reducing the space between the minipage content and the preceding text.
    \raggedright %  the text lines up on the left side, but the right side will be ragged.
    \begin{align*} % The * align environment does not number the equations- Each line is aligned at the & symbol
        \frac{x + 4}{5} &= 5\\
        \frac{x + 4}{5} \times\dotuline{\hspace{5mm}} &= 5 \times\dotuline{\hspace{5mm}}\\
        x + 4 &= \dotuline{\hspace{5mm}}\\
        x + 4 - \dotuline{\hspace{5mm}} &= \dotuline{\hspace{5mm}} - \dotuline{\hspace{5mm}}\\
        x &= \dotuline{\hspace{5mm}}\\
    \end{align*}
\end{minipage}\columnbreak
    \refstepcounter{minipagecount} % increments the counter minipagecount by one.
\noindent{(\theminipagecount)}\hspace{0.1mm} % By default, LaTeX indents the first line of a new paragraph, but \noindent overrides this
% and inserts the current value of the minipagecount counter, enclosed in parentheses
\begin{minipage}[t]{0.45\textwidth} % The [t] option aligns the top of the minipage with the baseline of the surrounding text.
    \vspace{-26pt}  % moves the content of the minipage up, reducing the space between the minipage content and the preceding text.
    \raggedright %  the text lines up on the left side, but the right side will be ragged.
    \begin{align*} % The * align environment does not number the equations- Each line is aligned at the & symbol
        \frac{x + 5}{8} &= 9\\
        \frac{x + 5}{8} \times\dotuline{\hspace{5mm}} &= 9 \times\dotuline{\hspace{5mm}}\\
        x + 5 &= \dotuline{\hspace{5mm}}\\
        x + 5 - \dotuline{\hspace{5mm}} &= \dotuline{\hspace{5mm}} - \dotuline{\hspace{5mm}}\\
        x &= \dotuline{\hspace{5mm}}\\
    \end{align*}
\end{minipage}\refstepcounter{minipagecount} % increments the counter minipagecount by one.
\noindent{(\theminipagecount)}\hspace{0.1mm} % By default, LaTeX indents the first line of a new paragraph, but \noindent overrides this
% and inserts the current value of the minipagecount counter, enclosed in parentheses
\begin{minipage}[t]{0.45\textwidth} % The [t] option aligns the top of the minipage with the baseline of the surrounding text.
    \vspace{-26pt}  % moves the content of the minipage up, reducing the space between the minipage content and the preceding text.
    \raggedright %  the text lines up on the left side, but the right side will be ragged.
    \begin{align*} % The * align environment does not number the equations- Each line is aligned at the & symbol
        \frac{x + 8}{2} &= 4\\
        \frac{x + 8}{2} \times\dotuline{\hspace{5mm}} &= 4 \times\dotuline{\hspace{5mm}}\\
        x + 8 &= \dotuline{\hspace{5mm}}\\
        x + 8 - \dotuline{\hspace{5mm}} &= \dotuline{\hspace{5mm}} - \dotuline{\hspace{5mm}}\\
        x &= \dotuline{\hspace{5mm}}\\
    \end{align*}
\end{minipage}\refstepcounter{minipagecount} % increments the counter minipagecount by one.
\noindent{(\theminipagecount)}\hspace{0.1mm} % By default, LaTeX indents the first line of a new paragraph, but \noindent overrides this
% and inserts the current value of the minipagecount counter, enclosed in parentheses
\begin{minipage}[t]{0.45\textwidth} % The [t] option aligns the top of the minipage with the baseline of the surrounding text.
    \vspace{-26pt}  % moves the content of the minipage up, reducing the space between the minipage content and the preceding text.
    \raggedright %  the text lines up on the left side, but the right side will be ragged.
    \begin{align*} % The * align environment does not number the equations- Each line is aligned at the & symbol
        \frac{x + 10}{8} &= 3\\
        \frac{x + 10}{8} \times\dotuline{\hspace{5mm}} &= 3 \times\dotuline{\hspace{5mm}}\\
        x + 10 &= \dotuline{\hspace{5mm}}\\
        x + 10 - \dotuline{\hspace{5mm}} &= \dotuline{\hspace{5mm}} - \dotuline{\hspace{5mm}}\\
        x &= \dotuline{\hspace{5mm}}\\
    \end{align*}
\end{minipage}\refstepcounter{minipagecount} % increments the counter minipagecount by one.
\noindent{(\theminipagecount)}\hspace{0.1mm} % By default, LaTeX indents the first line of a new paragraph, but \noindent overrides this
% and inserts the current value of the minipagecount counter, enclosed in parentheses
\begin{minipage}[t]{0.45\textwidth} % The [t] option aligns the top of the minipage with the baseline of the surrounding text.
    \vspace{-26pt}  % moves the content of the minipage up, reducing the space between the minipage content and the preceding text.
    \raggedright %  the text lines up on the left side, but the right side will be ragged.
    \begin{align*} % The * align environment does not number the equations- Each line is aligned at the & symbol
        \frac{x + 4}{9} &= 8\\
        \frac{x + 4}{9} \times\dotuline{\hspace{5mm}} &= 8 \times\dotuline{\hspace{5mm}}\\
        x + 4 &= \dotuline{\hspace{5mm}}\\
        x + 4 - \dotuline{\hspace{5mm}} &= \dotuline{\hspace{5mm}} - \dotuline{\hspace{5mm}}\\
        x &= \dotuline{\hspace{5mm}}\\
    \end{align*}
\end{minipage}\refstepcounter{minipagecount} % increments the counter minipagecount by one.
\noindent{(\theminipagecount)}\hspace{0.1mm} % By default, LaTeX indents the first line of a new paragraph, but \noindent overrides this
% and inserts the current value of the minipagecount counter, enclosed in parentheses
\begin{minipage}[t]{0.45\textwidth} % The [t] option aligns the top of the minipage with the baseline of the surrounding text.
    \vspace{-26pt}  % moves the content of the minipage up, reducing the space between the minipage content and the preceding text.
    \raggedright %  the text lines up on the left side, but the right side will be ragged.
    \begin{align*} % The * align environment does not number the equations- Each line is aligned at the & symbol
        \frac{x + 3}{9} &= 5\\
        \frac{x + 3}{9} \times\dotuline{\hspace{5mm}} &= 5 \times\dotuline{\hspace{5mm}}\\
        x + 3 &= \dotuline{\hspace{5mm}}\\
        x + 3 - \dotuline{\hspace{5mm}} &= \dotuline{\hspace{5mm}} - \dotuline{\hspace{5mm}}\\
        x &= \dotuline{\hspace{5mm}}\\
    \end{align*}
\end{minipage}\newpage

\end{multicols}
\end{document}
