\documentclass[12pt]{article}
\usepackage{tikz}
\usepackage{amsmath}
% Underlining package
\usepackage[normalem]{ulem} % [normalem] prevents the package from changing the default behavior of \emph to underline.
\usepackage[a4paper, portrait, margin=1cm]{geometry}
\usepackage{multicol}
\usepackage{fancyhdr}

\def \HeadingQuestions {\section*{\Large Name: \underline{\hspace{8cm}} \hfill Date: \underline{\hspace{3cm}}} \vspace{-3mm}
{Inverse operations: Questions} \vspace{1pt}\hrule}

% only in Q due to increased line height with dotted underline
\linespread{0.9} % Adjust line spacing factor to 0.9 if needed to fit 5 in column for 2 div steps

% raise footer with page number; no header
\fancypagestyle{myfancypagestyle}{
  \fancyhf{}% clear all header and footer fields
  \renewcommand{\headrulewidth}{0pt} % no rule under header
  \fancyfoot[C] {\thepage} \setlength{\footskip}{14.5pt} % raise page number allowed min 14.5pt
}
\pagestyle{myfancypagestyle}  % apply myfancypagestyle
\newcounter{minipagecount}
\begin{document}
\HeadingQuestions
\vspace{1mm}
\begin{multicols}{2}
\refstepcounter{minipagecount} % increments the counter minipagecount by one.
\noindent{(\theminipagecount)}\hspace{0.1mm} % By default, LaTeX indents the first line of a new paragraph, but \noindent overrides this
% and inserts the current value of the minipagecount counter, enclosed in parentheses
\begin{minipage}[t]{0.45\textwidth} % The [t] option aligns the top of the minipage with the baseline of the surrounding text.
    \vspace{-26pt}  % moves the content of the minipage up, reducing the space between the minipage content and the preceding text.
    \raggedright %  the text lines up on the left side, but the right side will be ragged.
    \begin{align*} % The * align environment does not number the equations- Each line is aligned at the & symbol
        4(x + 9) &= 68\\
        \frac{4(x+9)}{\dotuline{\hspace{5mm}}} &= \frac{68}{\dotuline{\hspace{5mm}}}\\
        x + 9 &= \dotuline{\hspace{5mm}}\\
        x + 9 - \dotuline{\hspace{5mm}} &= \dotuline{\hspace{5mm}} - \dotuline{\hspace{5mm}}\\
        x &= \dotuline{\hspace{5mm}}\\
    \end{align*}
\end{minipage}\refstepcounter{minipagecount} % increments the counter minipagecount by one.
\noindent{(\theminipagecount)}\hspace{0.1mm} % By default, LaTeX indents the first line of a new paragraph, but \noindent overrides this
% and inserts the current value of the minipagecount counter, enclosed in parentheses
\begin{minipage}[t]{0.45\textwidth} % The [t] option aligns the top of the minipage with the baseline of the surrounding text.
    \vspace{-26pt}  % moves the content of the minipage up, reducing the space between the minipage content and the preceding text.
    \raggedright %  the text lines up on the left side, but the right side will be ragged.
    \begin{align*} % The * align environment does not number the equations- Each line is aligned at the & symbol
        \frac{x}{3} + 6 &= 10\\
        \frac{x}{3} + 6 - \dotuline{\hspace{5mm}} &= 10 - \dotuline{\hspace{5mm}}\\
        \frac{x}{3} &= \dotuline{\hspace{5mm}}\\
        \frac{x}{3} \times\dotuline{\hspace{5mm}} &= \dotuline{\hspace{5mm}} \times\dotuline{\hspace{5mm}}\\
        x &= \dotuline{\hspace{5mm}}\\
    \end{align*}
\end{minipage}\refstepcounter{minipagecount} % increments the counter minipagecount by one.
\noindent{(\theminipagecount)}\hspace{0.1mm} % By default, LaTeX indents the first line of a new paragraph, but \noindent overrides this
% and inserts the current value of the minipagecount counter, enclosed in parentheses
\begin{minipage}[t]{0.45\textwidth} % The [t] option aligns the top of the minipage with the baseline of the surrounding text.
    \vspace{-26pt}  % moves the content of the minipage up, reducing the space between the minipage content and the preceding text.
    \raggedright %  the text lines up on the left side, but the right side will be ragged.
    \begin{align*} % The * align environment does not number the equations- Each line is aligned at the & symbol
        \frac{x + 5}{10} &= 8\\
        \frac{x + 5}{10} \times\dotuline{\hspace{5mm}} &= 8 \times\dotuline{\hspace{5mm}}\\
        x + 5 &= \dotuline{\hspace{5mm}}\\
        x + 5 - \dotuline{\hspace{5mm}} &= \dotuline{\hspace{5mm}} - \dotuline{\hspace{5mm}}\\
        x &= \dotuline{\hspace{5mm}}\\
    \end{align*}
\end{minipage}\refstepcounter{minipagecount} % increments the counter minipagecount by one.
\noindent{(\theminipagecount)}\hspace{0.1mm} % By default, LaTeX indents the first line of a new paragraph, but \noindent overrides this
% and inserts the current value of the minipagecount counter, enclosed in parentheses
\begin{minipage}[t]{0.45\textwidth} % The [t] option aligns the top of the minipage with the baseline of the surrounding text.
    \vspace{-26pt}  % moves the content of the minipage up, reducing the space between the minipage content and the preceding text.
    \raggedright %  the text lines up on the left side, but the right side will be ragged.
    \begin{align*} % The * align environment does not number the equations- Each line is aligned at the & symbol
        \frac{x + 7}{6} &= 9\\
        \frac{x + 7}{6} \times\dotuline{\hspace{5mm}} &= 9 \times\dotuline{\hspace{5mm}}\\
        x + 7 &= \dotuline{\hspace{5mm}}\\
        x + 7 - \dotuline{\hspace{5mm}} &= \dotuline{\hspace{5mm}} - \dotuline{\hspace{5mm}}\\
        x &= \dotuline{\hspace{5mm}}\\
    \end{align*}
\end{minipage}\refstepcounter{minipagecount} % increments the counter minipagecount by one.
\noindent{(\theminipagecount)}\hspace{0.1mm} % By default, LaTeX indents the first line of a new paragraph, but \noindent overrides this
% and inserts the current value of the minipagecount counter, enclosed in parentheses
\begin{minipage}[t]{0.45\textwidth} % The [t] option aligns the top of the minipage with the baseline of the surrounding text.
    \vspace{-26pt}  % moves the content of the minipage up, reducing the space between the minipage content and the preceding text.
    \raggedright %  the text lines up on the left side, but the right side will be ragged.
    \begin{align*} % The * align environment does not number the equations- Each line is aligned at the & symbol
        4x - 7 &= 33\\
        4x - 7 + \dotuline{\hspace{5mm}} &= 33 + \dotuline{\hspace{5mm}}\\
        4x &= \dotuline{\hspace{5mm}}\\
        \frac{4x}{\dotuline{\hspace{5mm}}} &= \frac{\dotuline{\hspace{5mm}}}{\dotuline{\hspace{5mm}}}\\
        x &= \dotuline{\hspace{5mm}}\\
    \end{align*}
\end{minipage}\columnbreak
    \refstepcounter{minipagecount} % increments the counter minipagecount by one.
\noindent{(\theminipagecount)}\hspace{0.1mm} % By default, LaTeX indents the first line of a new paragraph, but \noindent overrides this
% and inserts the current value of the minipagecount counter, enclosed in parentheses
\begin{minipage}[t]{0.45\textwidth} % The [t] option aligns the top of the minipage with the baseline of the surrounding text.
    \vspace{-26pt}  % moves the content of the minipage up, reducing the space between the minipage content and the preceding text.
    \raggedright %  the text lines up on the left side, but the right side will be ragged.
    \begin{align*} % The * align environment does not number the equations- Each line is aligned at the & symbol
        \frac{x + 1}{7} &= 5\\
        \frac{x + 1}{7} \times\dotuline{\hspace{5mm}} &= 5 \times\dotuline{\hspace{5mm}}\\
        x + 1 &= \dotuline{\hspace{5mm}}\\
        x + 1 - \dotuline{\hspace{5mm}} &= \dotuline{\hspace{5mm}} - \dotuline{\hspace{5mm}}\\
        x &= \dotuline{\hspace{5mm}}\\
    \end{align*}
\end{minipage}\refstepcounter{minipagecount} % increments the counter minipagecount by one.
\noindent{(\theminipagecount)}\hspace{0.1mm} % By default, LaTeX indents the first line of a new paragraph, but \noindent overrides this
% and inserts the current value of the minipagecount counter, enclosed in parentheses
\begin{minipage}[t]{0.45\textwidth} % The [t] option aligns the top of the minipage with the baseline of the surrounding text.
    \vspace{-26pt}  % moves the content of the minipage up, reducing the space between the minipage content and the preceding text.
    \raggedright %  the text lines up on the left side, but the right side will be ragged.
    \begin{align*} % The * align environment does not number the equations- Each line is aligned at the & symbol
        8x - 6 &= 10\\
        8x - 6 + \dotuline{\hspace{5mm}} &= 10 + \dotuline{\hspace{5mm}}\\
        8x &= \dotuline{\hspace{5mm}}\\
        \frac{8x}{\dotuline{\hspace{5mm}}} &= \frac{\dotuline{\hspace{5mm}}}{\dotuline{\hspace{5mm}}}\\
        x &= \dotuline{\hspace{5mm}}\\
    \end{align*}
\end{minipage}\refstepcounter{minipagecount} % increments the counter minipagecount by one.
\noindent{(\theminipagecount)}\hspace{0.1mm} % By default, LaTeX indents the first line of a new paragraph, but \noindent overrides this
% and inserts the current value of the minipagecount counter, enclosed in parentheses
\begin{minipage}[t]{0.45\textwidth} % The [t] option aligns the top of the minipage with the baseline of the surrounding text.
    \vspace{-26pt}  % moves the content of the minipage up, reducing the space between the minipage content and the preceding text.
    \raggedright %  the text lines up on the left side, but the right side will be ragged.
    \begin{align*} % The * align environment does not number the equations- Each line is aligned at the & symbol
        \frac{x}{5} - 2 &= 7\\
        \frac{x}{5} - 2 + \dotuline{\hspace{5mm}} &= 7 + \dotuline{\hspace{5mm}}\\
        \frac{x}{5} &= \dotuline{\hspace{5mm}}\\
        \frac{x}{5} \times\dotuline{\hspace{5mm}} &= \dotuline{\hspace{5mm}} \times\dotuline{\hspace{5mm}}\\
        x &= \dotuline{\hspace{5mm}}\\
    \end{align*}
\end{minipage}\refstepcounter{minipagecount} % increments the counter minipagecount by one.
\noindent{(\theminipagecount)}\hspace{0.1mm} % By default, LaTeX indents the first line of a new paragraph, but \noindent overrides this
% and inserts the current value of the minipagecount counter, enclosed in parentheses
\begin{minipage}[t]{0.45\textwidth} % The [t] option aligns the top of the minipage with the baseline of the surrounding text.
    \vspace{-26pt}  % moves the content of the minipage up, reducing the space between the minipage content and the preceding text.
    \raggedright %  the text lines up on the left side, but the right side will be ragged.
    \begin{align*} % The * align environment does not number the equations- Each line is aligned at the & symbol
        \frac{x + 4}{10} &= 10\\
        \frac{x + 4}{10} \times\dotuline{\hspace{5mm}} &= 10 \times\dotuline{\hspace{5mm}}\\
        x + 4 &= \dotuline{\hspace{5mm}}\\
        x + 4 - \dotuline{\hspace{5mm}} &= \dotuline{\hspace{5mm}} - \dotuline{\hspace{5mm}}\\
        x &= \dotuline{\hspace{5mm}}\\
    \end{align*}
\end{minipage}\refstepcounter{minipagecount} % increments the counter minipagecount by one.
\noindent{(\theminipagecount)}\hspace{0.1mm} % By default, LaTeX indents the first line of a new paragraph, but \noindent overrides this
% and inserts the current value of the minipagecount counter, enclosed in parentheses
\begin{minipage}[t]{0.45\textwidth} % The [t] option aligns the top of the minipage with the baseline of the surrounding text.
    \vspace{-26pt}  % moves the content of the minipage up, reducing the space between the minipage content and the preceding text.
    \raggedright %  the text lines up on the left side, but the right side will be ragged.
    \begin{align*} % The * align environment does not number the equations- Each line is aligned at the & symbol
        \frac{x - 2}{8} &= 8\\
        \frac{x - 2}{8} \times\dotuline{\hspace{5mm}} &= 8 \times\dotuline{\hspace{5mm}}\\
        x - 2 &= \dotuline{\hspace{5mm}}\\
        x - 2 + \dotuline{\hspace{5mm}} &= \dotuline{\hspace{5mm}} + \dotuline{\hspace{5mm}}\\
        x &= \dotuline{\hspace{5mm}}\\
    \end{align*}
\end{minipage}\newpage
    \refstepcounter{minipagecount} % increments the counter minipagecount by one.
\noindent{(\theminipagecount)}\hspace{0.1mm} % By default, LaTeX indents the first line of a new paragraph, but \noindent overrides this
% and inserts the current value of the minipagecount counter, enclosed in parentheses
\begin{minipage}[t]{0.45\textwidth} % The [t] option aligns the top of the minipage with the baseline of the surrounding text.
    \vspace{-26pt}  % moves the content of the minipage up, reducing the space between the minipage content and the preceding text.
    \raggedright %  the text lines up on the left side, but the right side will be ragged.
    \begin{align*} % The * align environment does not number the equations- Each line is aligned at the & symbol
        \frac{x}{7} + 5 &= 11\\
        \frac{x}{7} + 5 - \dotuline{\hspace{5mm}} &= 11 - \dotuline{\hspace{5mm}}\\
        \frac{x}{7} &= \dotuline{\hspace{5mm}}\\
        \frac{x}{7} \times\dotuline{\hspace{5mm}} &= \dotuline{\hspace{5mm}} \times\dotuline{\hspace{5mm}}\\
        x &= \dotuline{\hspace{5mm}}\\
    \end{align*}
\end{minipage}\refstepcounter{minipagecount} % increments the counter minipagecount by one.
\noindent{(\theminipagecount)}\hspace{0.1mm} % By default, LaTeX indents the first line of a new paragraph, but \noindent overrides this
% and inserts the current value of the minipagecount counter, enclosed in parentheses
\begin{minipage}[t]{0.45\textwidth} % The [t] option aligns the top of the minipage with the baseline of the surrounding text.
    \vspace{-26pt}  % moves the content of the minipage up, reducing the space between the minipage content and the preceding text.
    \raggedright %  the text lines up on the left side, but the right side will be ragged.
    \begin{align*} % The * align environment does not number the equations- Each line is aligned at the & symbol
        10(x - 6) &= 30\\
        \frac{10(x-6)}{\dotuline{\hspace{5mm}}} &= \frac{30}{\dotuline{\hspace{5mm}}}\\
        x - 6 &= \dotuline{\hspace{5mm}}\\
        x - 6 + \dotuline{\hspace{5mm}} &= \dotuline{\hspace{5mm}} + \dotuline{\hspace{5mm}}\\
        x &= \dotuline{\hspace{5mm}}\\
    \end{align*}
\end{minipage}\refstepcounter{minipagecount} % increments the counter minipagecount by one.
\noindent{(\theminipagecount)}\hspace{0.1mm} % By default, LaTeX indents the first line of a new paragraph, but \noindent overrides this
% and inserts the current value of the minipagecount counter, enclosed in parentheses
\begin{minipage}[t]{0.45\textwidth} % The [t] option aligns the top of the minipage with the baseline of the surrounding text.
    \vspace{-26pt}  % moves the content of the minipage up, reducing the space between the minipage content and the preceding text.
    \raggedright %  the text lines up on the left side, but the right side will be ragged.
    \begin{align*} % The * align environment does not number the equations- Each line is aligned at the & symbol
        5x + 6 &= 26\\
        5x + 6 - \dotuline{\hspace{5mm}} &= 26 - \dotuline{\hspace{5mm}}\\
        5x &= \dotuline{\hspace{5mm}}\\
        \frac{5x}{\dotuline{\hspace{5mm}}} &= \frac{\dotuline{\hspace{5mm}}}{\dotuline{\hspace{5mm}}}\\
        x &= \dotuline{\hspace{5mm}}\\
    \end{align*}
\end{minipage}\refstepcounter{minipagecount} % increments the counter minipagecount by one.
\noindent{(\theminipagecount)}\hspace{0.1mm} % By default, LaTeX indents the first line of a new paragraph, but \noindent overrides this
% and inserts the current value of the minipagecount counter, enclosed in parentheses
\begin{minipage}[t]{0.45\textwidth} % The [t] option aligns the top of the minipage with the baseline of the surrounding text.
    \vspace{-26pt}  % moves the content of the minipage up, reducing the space between the minipage content and the preceding text.
    \raggedright %  the text lines up on the left side, but the right side will be ragged.
    \begin{align*} % The * align environment does not number the equations- Each line is aligned at the & symbol
        3(x - 2) &= 3\\
        \frac{3(x-2)}{\dotuline{\hspace{5mm}}} &= \frac{3}{\dotuline{\hspace{5mm}}}\\
        x - 2 &= \dotuline{\hspace{5mm}}\\
        x - 2 + \dotuline{\hspace{5mm}} &= \dotuline{\hspace{5mm}} + \dotuline{\hspace{5mm}}\\
        x &= \dotuline{\hspace{5mm}}\\
    \end{align*}
\end{minipage}\refstepcounter{minipagecount} % increments the counter minipagecount by one.
\noindent{(\theminipagecount)}\hspace{0.1mm} % By default, LaTeX indents the first line of a new paragraph, but \noindent overrides this
% and inserts the current value of the minipagecount counter, enclosed in parentheses
\begin{minipage}[t]{0.45\textwidth} % The [t] option aligns the top of the minipage with the baseline of the surrounding text.
    \vspace{-26pt}  % moves the content of the minipage up, reducing the space between the minipage content and the preceding text.
    \raggedright %  the text lines up on the left side, but the right side will be ragged.
    \begin{align*} % The * align environment does not number the equations- Each line is aligned at the & symbol
        \frac{x}{8} + 4 &= 6\\
        \frac{x}{8} + 4 - \dotuline{\hspace{5mm}} &= 6 - \dotuline{\hspace{5mm}}\\
        \frac{x}{8} &= \dotuline{\hspace{5mm}}\\
        \frac{x}{8} \times\dotuline{\hspace{5mm}} &= \dotuline{\hspace{5mm}} \times\dotuline{\hspace{5mm}}\\
        x &= \dotuline{\hspace{5mm}}\\
    \end{align*}
\end{minipage}\columnbreak
    \refstepcounter{minipagecount} % increments the counter minipagecount by one.
\noindent{(\theminipagecount)}\hspace{0.1mm} % By default, LaTeX indents the first line of a new paragraph, but \noindent overrides this
% and inserts the current value of the minipagecount counter, enclosed in parentheses
\begin{minipage}[t]{0.45\textwidth} % The [t] option aligns the top of the minipage with the baseline of the surrounding text.
    \vspace{-26pt}  % moves the content of the minipage up, reducing the space between the minipage content and the preceding text.
    \raggedright %  the text lines up on the left side, but the right side will be ragged.
    \begin{align*} % The * align environment does not number the equations- Each line is aligned at the & symbol
        \frac{x - 1}{8} &= 2\\
        \frac{x - 1}{8} \times\dotuline{\hspace{5mm}} &= 2 \times\dotuline{\hspace{5mm}}\\
        x - 1 &= \dotuline{\hspace{5mm}}\\
        x - 1 + \dotuline{\hspace{5mm}} &= \dotuline{\hspace{5mm}} + \dotuline{\hspace{5mm}}\\
        x &= \dotuline{\hspace{5mm}}\\
    \end{align*}
\end{minipage}\refstepcounter{minipagecount} % increments the counter minipagecount by one.
\noindent{(\theminipagecount)}\hspace{0.1mm} % By default, LaTeX indents the first line of a new paragraph, but \noindent overrides this
% and inserts the current value of the minipagecount counter, enclosed in parentheses
\begin{minipage}[t]{0.45\textwidth} % The [t] option aligns the top of the minipage with the baseline of the surrounding text.
    \vspace{-26pt}  % moves the content of the minipage up, reducing the space between the minipage content and the preceding text.
    \raggedright %  the text lines up on the left side, but the right side will be ragged.
    \begin{align*} % The * align environment does not number the equations- Each line is aligned at the & symbol
        7x - 3 &= 25\\
        7x - 3 + \dotuline{\hspace{5mm}} &= 25 + \dotuline{\hspace{5mm}}\\
        7x &= \dotuline{\hspace{5mm}}\\
        \frac{7x}{\dotuline{\hspace{5mm}}} &= \frac{\dotuline{\hspace{5mm}}}{\dotuline{\hspace{5mm}}}\\
        x &= \dotuline{\hspace{5mm}}\\
    \end{align*}
\end{minipage}\refstepcounter{minipagecount} % increments the counter minipagecount by one.
\noindent{(\theminipagecount)}\hspace{0.1mm} % By default, LaTeX indents the first line of a new paragraph, but \noindent overrides this
% and inserts the current value of the minipagecount counter, enclosed in parentheses
\begin{minipage}[t]{0.45\textwidth} % The [t] option aligns the top of the minipage with the baseline of the surrounding text.
    \vspace{-26pt}  % moves the content of the minipage up, reducing the space between the minipage content and the preceding text.
    \raggedright %  the text lines up on the left side, but the right side will be ragged.
    \begin{align*} % The * align environment does not number the equations- Each line is aligned at the & symbol
        \frac{x + 1}{6} &= 10\\
        \frac{x + 1}{6} \times\dotuline{\hspace{5mm}} &= 10 \times\dotuline{\hspace{5mm}}\\
        x + 1 &= \dotuline{\hspace{5mm}}\\
        x + 1 - \dotuline{\hspace{5mm}} &= \dotuline{\hspace{5mm}} - \dotuline{\hspace{5mm}}\\
        x &= \dotuline{\hspace{5mm}}\\
    \end{align*}
\end{minipage}\refstepcounter{minipagecount} % increments the counter minipagecount by one.
\noindent{(\theminipagecount)}\hspace{0.1mm} % By default, LaTeX indents the first line of a new paragraph, but \noindent overrides this
% and inserts the current value of the minipagecount counter, enclosed in parentheses
\begin{minipage}[t]{0.45\textwidth} % The [t] option aligns the top of the minipage with the baseline of the surrounding text.
    \vspace{-26pt}  % moves the content of the minipage up, reducing the space between the minipage content and the preceding text.
    \raggedright %  the text lines up on the left side, but the right side will be ragged.
    \begin{align*} % The * align environment does not number the equations- Each line is aligned at the & symbol
        \frac{x}{9} - 3 &= -1\\
        \frac{x}{9} - 3 + \dotuline{\hspace{5mm}} &= -1 + \dotuline{\hspace{5mm}}\\
        \frac{x}{9} &= \dotuline{\hspace{5mm}}\\
        \frac{x}{9} \times\dotuline{\hspace{5mm}} &= \dotuline{\hspace{5mm}} \times\dotuline{\hspace{5mm}}\\
        x &= \dotuline{\hspace{5mm}}\\
    \end{align*}
\end{minipage}\refstepcounter{minipagecount} % increments the counter minipagecount by one.
\noindent{(\theminipagecount)}\hspace{0.1mm} % By default, LaTeX indents the first line of a new paragraph, but \noindent overrides this
% and inserts the current value of the minipagecount counter, enclosed in parentheses
\begin{minipage}[t]{0.45\textwidth} % The [t] option aligns the top of the minipage with the baseline of the surrounding text.
    \vspace{-26pt}  % moves the content of the minipage up, reducing the space between the minipage content and the preceding text.
    \raggedright %  the text lines up on the left side, but the right side will be ragged.
    \begin{align*} % The * align environment does not number the equations- Each line is aligned at the & symbol
        7(x - 5) &= 14\\
        \frac{7(x-5)}{\dotuline{\hspace{5mm}}} &= \frac{14}{\dotuline{\hspace{5mm}}}\\
        x - 5 &= \dotuline{\hspace{5mm}}\\
        x - 5 + \dotuline{\hspace{5mm}} &= \dotuline{\hspace{5mm}} + \dotuline{\hspace{5mm}}\\
        x &= \dotuline{\hspace{5mm}}\\
    \end{align*}
\end{minipage}\newpage
    \refstepcounter{minipagecount} % increments the counter minipagecount by one.
\noindent{(\theminipagecount)}\hspace{0.1mm} % By default, LaTeX indents the first line of a new paragraph, but \noindent overrides this
% and inserts the current value of the minipagecount counter, enclosed in parentheses
\begin{minipage}[t]{0.45\textwidth} % The [t] option aligns the top of the minipage with the baseline of the surrounding text.
    \vspace{-26pt}  % moves the content of the minipage up, reducing the space between the minipage content and the preceding text.
    \raggedright %  the text lines up on the left side, but the right side will be ragged.
    \begin{align*} % The * align environment does not number the equations- Each line is aligned at the & symbol
        \frac{x - 5}{5} &= 2\\
        \frac{x - 5}{5} \times\dotuline{\hspace{5mm}} &= 2 \times\dotuline{\hspace{5mm}}\\
        x - 5 &= \dotuline{\hspace{5mm}}\\
        x - 5 + \dotuline{\hspace{5mm}} &= \dotuline{\hspace{5mm}} + \dotuline{\hspace{5mm}}\\
        x &= \dotuline{\hspace{5mm}}\\
    \end{align*}
\end{minipage}\refstepcounter{minipagecount} % increments the counter minipagecount by one.
\noindent{(\theminipagecount)}\hspace{0.1mm} % By default, LaTeX indents the first line of a new paragraph, but \noindent overrides this
% and inserts the current value of the minipagecount counter, enclosed in parentheses
\begin{minipage}[t]{0.45\textwidth} % The [t] option aligns the top of the minipage with the baseline of the surrounding text.
    \vspace{-26pt}  % moves the content of the minipage up, reducing the space between the minipage content and the preceding text.
    \raggedright %  the text lines up on the left side, but the right side will be ragged.
    \begin{align*} % The * align environment does not number the equations- Each line is aligned at the & symbol
        8(x - 2) &= 8\\
        \frac{8(x-2)}{\dotuline{\hspace{5mm}}} &= \frac{8}{\dotuline{\hspace{5mm}}}\\
        x - 2 &= \dotuline{\hspace{5mm}}\\
        x - 2 + \dotuline{\hspace{5mm}} &= \dotuline{\hspace{5mm}} + \dotuline{\hspace{5mm}}\\
        x &= \dotuline{\hspace{5mm}}\\
    \end{align*}
\end{minipage}\refstepcounter{minipagecount} % increments the counter minipagecount by one.
\noindent{(\theminipagecount)}\hspace{0.1mm} % By default, LaTeX indents the first line of a new paragraph, but \noindent overrides this
% and inserts the current value of the minipagecount counter, enclosed in parentheses
\begin{minipage}[t]{0.45\textwidth} % The [t] option aligns the top of the minipage with the baseline of the surrounding text.
    \vspace{-26pt}  % moves the content of the minipage up, reducing the space between the minipage content and the preceding text.
    \raggedright %  the text lines up on the left side, but the right side will be ragged.
    \begin{align*} % The * align environment does not number the equations- Each line is aligned at the & symbol
        \frac{x + 3}{2} &= 9\\
        \frac{x + 3}{2} \times\dotuline{\hspace{5mm}} &= 9 \times\dotuline{\hspace{5mm}}\\
        x + 3 &= \dotuline{\hspace{5mm}}\\
        x + 3 - \dotuline{\hspace{5mm}} &= \dotuline{\hspace{5mm}} - \dotuline{\hspace{5mm}}\\
        x &= \dotuline{\hspace{5mm}}\\
    \end{align*}
\end{minipage}\refstepcounter{minipagecount} % increments the counter minipagecount by one.
\noindent{(\theminipagecount)}\hspace{0.1mm} % By default, LaTeX indents the first line of a new paragraph, but \noindent overrides this
% and inserts the current value of the minipagecount counter, enclosed in parentheses
\begin{minipage}[t]{0.45\textwidth} % The [t] option aligns the top of the minipage with the baseline of the surrounding text.
    \vspace{-26pt}  % moves the content of the minipage up, reducing the space between the minipage content and the preceding text.
    \raggedright %  the text lines up on the left side, but the right side will be ragged.
    \begin{align*} % The * align environment does not number the equations- Each line is aligned at the & symbol
        10x + 6 &= 16\\
        10x + 6 - \dotuline{\hspace{5mm}} &= 16 - \dotuline{\hspace{5mm}}\\
        10x &= \dotuline{\hspace{5mm}}\\
        \frac{10x}{\dotuline{\hspace{5mm}}} &= \frac{\dotuline{\hspace{5mm}}}{\dotuline{\hspace{5mm}}}\\
        x &= \dotuline{\hspace{5mm}}\\
    \end{align*}
\end{minipage}\refstepcounter{minipagecount} % increments the counter minipagecount by one.
\noindent{(\theminipagecount)}\hspace{0.1mm} % By default, LaTeX indents the first line of a new paragraph, but \noindent overrides this
% and inserts the current value of the minipagecount counter, enclosed in parentheses
\begin{minipage}[t]{0.45\textwidth} % The [t] option aligns the top of the minipage with the baseline of the surrounding text.
    \vspace{-26pt}  % moves the content of the minipage up, reducing the space between the minipage content and the preceding text.
    \raggedright %  the text lines up on the left side, but the right side will be ragged.
    \begin{align*} % The * align environment does not number the equations- Each line is aligned at the & symbol
        \frac{x}{10} + 10 &= 15\\
        \frac{x}{10} + 10 - \dotuline{\hspace{5mm}} &= 15 - \dotuline{\hspace{5mm}}\\
        \frac{x}{10} &= \dotuline{\hspace{5mm}}\\
        \frac{x}{10} \times\dotuline{\hspace{5mm}} &= \dotuline{\hspace{5mm}} \times\dotuline{\hspace{5mm}}\\
        x &= \dotuline{\hspace{5mm}}\\
    \end{align*}
\end{minipage}\columnbreak
    \refstepcounter{minipagecount} % increments the counter minipagecount by one.
\noindent{(\theminipagecount)}\hspace{0.1mm} % By default, LaTeX indents the first line of a new paragraph, but \noindent overrides this
% and inserts the current value of the minipagecount counter, enclosed in parentheses
\begin{minipage}[t]{0.45\textwidth} % The [t] option aligns the top of the minipage with the baseline of the surrounding text.
    \vspace{-26pt}  % moves the content of the minipage up, reducing the space between the minipage content and the preceding text.
    \raggedright %  the text lines up on the left side, but the right side will be ragged.
    \begin{align*} % The * align environment does not number the equations- Each line is aligned at the & symbol
        \frac{x}{9} + 3 &= 7\\
        \frac{x}{9} + 3 - \dotuline{\hspace{5mm}} &= 7 - \dotuline{\hspace{5mm}}\\
        \frac{x}{9} &= \dotuline{\hspace{5mm}}\\
        \frac{x}{9} \times\dotuline{\hspace{5mm}} &= \dotuline{\hspace{5mm}} \times\dotuline{\hspace{5mm}}\\
        x &= \dotuline{\hspace{5mm}}\\
    \end{align*}
\end{minipage}\refstepcounter{minipagecount} % increments the counter minipagecount by one.
\noindent{(\theminipagecount)}\hspace{0.1mm} % By default, LaTeX indents the first line of a new paragraph, but \noindent overrides this
% and inserts the current value of the minipagecount counter, enclosed in parentheses
\begin{minipage}[t]{0.45\textwidth} % The [t] option aligns the top of the minipage with the baseline of the surrounding text.
    \vspace{-26pt}  % moves the content of the minipage up, reducing the space between the minipage content and the preceding text.
    \raggedright %  the text lines up on the left side, but the right side will be ragged.
    \begin{align*} % The * align environment does not number the equations- Each line is aligned at the & symbol
        2(x + 7) &= 30\\
        \frac{2(x+7)}{\dotuline{\hspace{5mm}}} &= \frac{30}{\dotuline{\hspace{5mm}}}\\
        x + 7 &= \dotuline{\hspace{5mm}}\\
        x + 7 - \dotuline{\hspace{5mm}} &= \dotuline{\hspace{5mm}} - \dotuline{\hspace{5mm}}\\
        x &= \dotuline{\hspace{5mm}}\\
    \end{align*}
\end{minipage}\refstepcounter{minipagecount} % increments the counter minipagecount by one.
\noindent{(\theminipagecount)}\hspace{0.1mm} % By default, LaTeX indents the first line of a new paragraph, but \noindent overrides this
% and inserts the current value of the minipagecount counter, enclosed in parentheses
\begin{minipage}[t]{0.45\textwidth} % The [t] option aligns the top of the minipage with the baseline of the surrounding text.
    \vspace{-26pt}  % moves the content of the minipage up, reducing the space between the minipage content and the preceding text.
    \raggedright %  the text lines up on the left side, but the right side will be ragged.
    \begin{align*} % The * align environment does not number the equations- Each line is aligned at the & symbol
        2x - 6 &= 12\\
        2x - 6 + \dotuline{\hspace{5mm}} &= 12 + \dotuline{\hspace{5mm}}\\
        2x &= \dotuline{\hspace{5mm}}\\
        \frac{2x}{\dotuline{\hspace{5mm}}} &= \frac{\dotuline{\hspace{5mm}}}{\dotuline{\hspace{5mm}}}\\
        x &= \dotuline{\hspace{5mm}}\\
    \end{align*}
\end{minipage}\refstepcounter{minipagecount} % increments the counter minipagecount by one.
\noindent{(\theminipagecount)}\hspace{0.1mm} % By default, LaTeX indents the first line of a new paragraph, but \noindent overrides this
% and inserts the current value of the minipagecount counter, enclosed in parentheses
\begin{minipage}[t]{0.45\textwidth} % The [t] option aligns the top of the minipage with the baseline of the surrounding text.
    \vspace{-26pt}  % moves the content of the minipage up, reducing the space between the minipage content and the preceding text.
    \raggedright %  the text lines up on the left side, but the right side will be ragged.
    \begin{align*} % The * align environment does not number the equations- Each line is aligned at the & symbol
        \frac{x - 2}{5} &= 4\\
        \frac{x - 2}{5} \times\dotuline{\hspace{5mm}} &= 4 \times\dotuline{\hspace{5mm}}\\
        x - 2 &= \dotuline{\hspace{5mm}}\\
        x - 2 + \dotuline{\hspace{5mm}} &= \dotuline{\hspace{5mm}} + \dotuline{\hspace{5mm}}\\
        x &= \dotuline{\hspace{5mm}}\\
    \end{align*}
\end{minipage}\refstepcounter{minipagecount} % increments the counter minipagecount by one.
\noindent{(\theminipagecount)}\hspace{0.1mm} % By default, LaTeX indents the first line of a new paragraph, but \noindent overrides this
% and inserts the current value of the minipagecount counter, enclosed in parentheses
\begin{minipage}[t]{0.45\textwidth} % The [t] option aligns the top of the minipage with the baseline of the surrounding text.
    \vspace{-26pt}  % moves the content of the minipage up, reducing the space between the minipage content and the preceding text.
    \raggedright %  the text lines up on the left side, but the right side will be ragged.
    \begin{align*} % The * align environment does not number the equations- Each line is aligned at the & symbol
        \frac{x - 8}{2} &= 7\\
        \frac{x - 8}{2} \times\dotuline{\hspace{5mm}} &= 7 \times\dotuline{\hspace{5mm}}\\
        x - 8 &= \dotuline{\hspace{5mm}}\\
        x - 8 + \dotuline{\hspace{5mm}} &= \dotuline{\hspace{5mm}} + \dotuline{\hspace{5mm}}\\
        x &= \dotuline{\hspace{5mm}}\\
    \end{align*}
\end{minipage}\newpage
    \refstepcounter{minipagecount} % increments the counter minipagecount by one.
\noindent{(\theminipagecount)}\hspace{0.1mm} % By default, LaTeX indents the first line of a new paragraph, but \noindent overrides this
% and inserts the current value of the minipagecount counter, enclosed in parentheses
\begin{minipage}[t]{0.45\textwidth} % The [t] option aligns the top of the minipage with the baseline of the surrounding text.
    \vspace{-26pt}  % moves the content of the minipage up, reducing the space between the minipage content and the preceding text.
    \raggedright %  the text lines up on the left side, but the right side will be ragged.
    \begin{align*} % The * align environment does not number the equations- Each line is aligned at the & symbol
        5(x - 5) &= -10\\
        \frac{5(x-5)}{\dotuline{\hspace{5mm}}} &= \frac{-10}{\dotuline{\hspace{5mm}}}\\
        x - 5 &= \dotuline{\hspace{5mm}}\\
        x - 5 + \dotuline{\hspace{5mm}} &= \dotuline{\hspace{5mm}} + \dotuline{\hspace{5mm}}\\
        x &= \dotuline{\hspace{5mm}}\\
    \end{align*}
\end{minipage}\refstepcounter{minipagecount} % increments the counter minipagecount by one.
\noindent{(\theminipagecount)}\hspace{0.1mm} % By default, LaTeX indents the first line of a new paragraph, but \noindent overrides this
% and inserts the current value of the minipagecount counter, enclosed in parentheses
\begin{minipage}[t]{0.45\textwidth} % The [t] option aligns the top of the minipage with the baseline of the surrounding text.
    \vspace{-26pt}  % moves the content of the minipage up, reducing the space between the minipage content and the preceding text.
    \raggedright %  the text lines up on the left side, but the right side will be ragged.
    \begin{align*} % The * align environment does not number the equations- Each line is aligned at the & symbol
        \frac{x - 4}{5} &= 4\\
        \frac{x - 4}{5} \times\dotuline{\hspace{5mm}} &= 4 \times\dotuline{\hspace{5mm}}\\
        x - 4 &= \dotuline{\hspace{5mm}}\\
        x - 4 + \dotuline{\hspace{5mm}} &= \dotuline{\hspace{5mm}} + \dotuline{\hspace{5mm}}\\
        x &= \dotuline{\hspace{5mm}}\\
    \end{align*}
\end{minipage}\refstepcounter{minipagecount} % increments the counter minipagecount by one.
\noindent{(\theminipagecount)}\hspace{0.1mm} % By default, LaTeX indents the first line of a new paragraph, but \noindent overrides this
% and inserts the current value of the minipagecount counter, enclosed in parentheses
\begin{minipage}[t]{0.45\textwidth} % The [t] option aligns the top of the minipage with the baseline of the surrounding text.
    \vspace{-26pt}  % moves the content of the minipage up, reducing the space between the minipage content and the preceding text.
    \raggedright %  the text lines up on the left side, but the right side will be ragged.
    \begin{align*} % The * align environment does not number the equations- Each line is aligned at the & symbol
        \frac{x}{6} + 10 &= 15\\
        \frac{x}{6} + 10 - \dotuline{\hspace{5mm}} &= 15 - \dotuline{\hspace{5mm}}\\
        \frac{x}{6} &= \dotuline{\hspace{5mm}}\\
        \frac{x}{6} \times\dotuline{\hspace{5mm}} &= \dotuline{\hspace{5mm}} \times\dotuline{\hspace{5mm}}\\
        x &= \dotuline{\hspace{5mm}}\\
    \end{align*}
\end{minipage}\refstepcounter{minipagecount} % increments the counter minipagecount by one.
\noindent{(\theminipagecount)}\hspace{0.1mm} % By default, LaTeX indents the first line of a new paragraph, but \noindent overrides this
% and inserts the current value of the minipagecount counter, enclosed in parentheses
\begin{minipage}[t]{0.45\textwidth} % The [t] option aligns the top of the minipage with the baseline of the surrounding text.
    \vspace{-26pt}  % moves the content of the minipage up, reducing the space between the minipage content and the preceding text.
    \raggedright %  the text lines up on the left side, but the right side will be ragged.
    \begin{align*} % The * align environment does not number the equations- Each line is aligned at the & symbol
        \frac{x}{2} - 6 &= -2\\
        \frac{x}{2} - 6 + \dotuline{\hspace{5mm}} &= -2 + \dotuline{\hspace{5mm}}\\
        \frac{x}{2} &= \dotuline{\hspace{5mm}}\\
        \frac{x}{2} \times\dotuline{\hspace{5mm}} &= \dotuline{\hspace{5mm}} \times\dotuline{\hspace{5mm}}\\
        x &= \dotuline{\hspace{5mm}}\\
    \end{align*}
\end{minipage}\refstepcounter{minipagecount} % increments the counter minipagecount by one.
\noindent{(\theminipagecount)}\hspace{0.1mm} % By default, LaTeX indents the first line of a new paragraph, but \noindent overrides this
% and inserts the current value of the minipagecount counter, enclosed in parentheses
\begin{minipage}[t]{0.45\textwidth} % The [t] option aligns the top of the minipage with the baseline of the surrounding text.
    \vspace{-26pt}  % moves the content of the minipage up, reducing the space between the minipage content and the preceding text.
    \raggedright %  the text lines up on the left side, but the right side will be ragged.
    \begin{align*} % The * align environment does not number the equations- Each line is aligned at the & symbol
        \frac{x}{5} - 4 &= -1\\
        \frac{x}{5} - 4 + \dotuline{\hspace{5mm}} &= -1 + \dotuline{\hspace{5mm}}\\
        \frac{x}{5} &= \dotuline{\hspace{5mm}}\\
        \frac{x}{5} \times\dotuline{\hspace{5mm}} &= \dotuline{\hspace{5mm}} \times\dotuline{\hspace{5mm}}\\
        x &= \dotuline{\hspace{5mm}}\\
    \end{align*}
\end{minipage}\columnbreak
    \refstepcounter{minipagecount} % increments the counter minipagecount by one.
\noindent{(\theminipagecount)}\hspace{0.1mm} % By default, LaTeX indents the first line of a new paragraph, but \noindent overrides this
% and inserts the current value of the minipagecount counter, enclosed in parentheses
\begin{minipage}[t]{0.45\textwidth} % The [t] option aligns the top of the minipage with the baseline of the surrounding text.
    \vspace{-26pt}  % moves the content of the minipage up, reducing the space between the minipage content and the preceding text.
    \raggedright %  the text lines up on the left side, but the right side will be ragged.
    \begin{align*} % The * align environment does not number the equations- Each line is aligned at the & symbol
        \frac{x + 2}{10} &= 10\\
        \frac{x + 2}{10} \times\dotuline{\hspace{5mm}} &= 10 \times\dotuline{\hspace{5mm}}\\
        x + 2 &= \dotuline{\hspace{5mm}}\\
        x + 2 - \dotuline{\hspace{5mm}} &= \dotuline{\hspace{5mm}} - \dotuline{\hspace{5mm}}\\
        x &= \dotuline{\hspace{5mm}}\\
    \end{align*}
\end{minipage}\refstepcounter{minipagecount} % increments the counter minipagecount by one.
\noindent{(\theminipagecount)}\hspace{0.1mm} % By default, LaTeX indents the first line of a new paragraph, but \noindent overrides this
% and inserts the current value of the minipagecount counter, enclosed in parentheses
\begin{minipage}[t]{0.45\textwidth} % The [t] option aligns the top of the minipage with the baseline of the surrounding text.
    \vspace{-26pt}  % moves the content of the minipage up, reducing the space between the minipage content and the preceding text.
    \raggedright %  the text lines up on the left side, but the right side will be ragged.
    \begin{align*} % The * align environment does not number the equations- Each line is aligned at the & symbol
        \frac{x + 5}{7} &= 10\\
        \frac{x + 5}{7} \times\dotuline{\hspace{5mm}} &= 10 \times\dotuline{\hspace{5mm}}\\
        x + 5 &= \dotuline{\hspace{5mm}}\\
        x + 5 - \dotuline{\hspace{5mm}} &= \dotuline{\hspace{5mm}} - \dotuline{\hspace{5mm}}\\
        x &= \dotuline{\hspace{5mm}}\\
    \end{align*}
\end{minipage}\refstepcounter{minipagecount} % increments the counter minipagecount by one.
\noindent{(\theminipagecount)}\hspace{0.1mm} % By default, LaTeX indents the first line of a new paragraph, but \noindent overrides this
% and inserts the current value of the minipagecount counter, enclosed in parentheses
\begin{minipage}[t]{0.45\textwidth} % The [t] option aligns the top of the minipage with the baseline of the surrounding text.
    \vspace{-26pt}  % moves the content of the minipage up, reducing the space between the minipage content and the preceding text.
    \raggedright %  the text lines up on the left side, but the right side will be ragged.
    \begin{align*} % The * align environment does not number the equations- Each line is aligned at the & symbol
        \frac{x}{7} + 1 &= 3\\
        \frac{x}{7} + 1 - \dotuline{\hspace{5mm}} &= 3 - \dotuline{\hspace{5mm}}\\
        \frac{x}{7} &= \dotuline{\hspace{5mm}}\\
        \frac{x}{7} \times\dotuline{\hspace{5mm}} &= \dotuline{\hspace{5mm}} \times\dotuline{\hspace{5mm}}\\
        x &= \dotuline{\hspace{5mm}}\\
    \end{align*}
\end{minipage}\refstepcounter{minipagecount} % increments the counter minipagecount by one.
\noindent{(\theminipagecount)}\hspace{0.1mm} % By default, LaTeX indents the first line of a new paragraph, but \noindent overrides this
% and inserts the current value of the minipagecount counter, enclosed in parentheses
\begin{minipage}[t]{0.45\textwidth} % The [t] option aligns the top of the minipage with the baseline of the surrounding text.
    \vspace{-26pt}  % moves the content of the minipage up, reducing the space between the minipage content and the preceding text.
    \raggedright %  the text lines up on the left side, but the right side will be ragged.
    \begin{align*} % The * align environment does not number the equations- Each line is aligned at the & symbol
        7(x + 9) &= 119\\
        \frac{7(x+9)}{\dotuline{\hspace{5mm}}} &= \frac{119}{\dotuline{\hspace{5mm}}}\\
        x + 9 &= \dotuline{\hspace{5mm}}\\
        x + 9 - \dotuline{\hspace{5mm}} &= \dotuline{\hspace{5mm}} - \dotuline{\hspace{5mm}}\\
        x &= \dotuline{\hspace{5mm}}\\
    \end{align*}
\end{minipage}\refstepcounter{minipagecount} % increments the counter minipagecount by one.
\noindent{(\theminipagecount)}\hspace{0.1mm} % By default, LaTeX indents the first line of a new paragraph, but \noindent overrides this
% and inserts the current value of the minipagecount counter, enclosed in parentheses
\begin{minipage}[t]{0.45\textwidth} % The [t] option aligns the top of the minipage with the baseline of the surrounding text.
    \vspace{-26pt}  % moves the content of the minipage up, reducing the space between the minipage content and the preceding text.
    \raggedright %  the text lines up on the left side, but the right side will be ragged.
    \begin{align*} % The * align environment does not number the equations- Each line is aligned at the & symbol
        \frac{x + 6}{3} &= 10\\
        \frac{x + 6}{3} \times\dotuline{\hspace{5mm}} &= 10 \times\dotuline{\hspace{5mm}}\\
        x + 6 &= \dotuline{\hspace{5mm}}\\
        x + 6 - \dotuline{\hspace{5mm}} &= \dotuline{\hspace{5mm}} - \dotuline{\hspace{5mm}}\\
        x &= \dotuline{\hspace{5mm}}\\
    \end{align*}
\end{minipage}\newpage
    \refstepcounter{minipagecount} % increments the counter minipagecount by one.
\noindent{(\theminipagecount)}\hspace{0.1mm} % By default, LaTeX indents the first line of a new paragraph, but \noindent overrides this
% and inserts the current value of the minipagecount counter, enclosed in parentheses
\begin{minipage}[t]{0.45\textwidth} % The [t] option aligns the top of the minipage with the baseline of the surrounding text.
    \vspace{-26pt}  % moves the content of the minipage up, reducing the space between the minipage content and the preceding text.
    \raggedright %  the text lines up on the left side, but the right side will be ragged.
    \begin{align*} % The * align environment does not number the equations- Each line is aligned at the & symbol
        9x - 4 &= 77\\
        9x - 4 + \dotuline{\hspace{5mm}} &= 77 + \dotuline{\hspace{5mm}}\\
        9x &= \dotuline{\hspace{5mm}}\\
        \frac{9x}{\dotuline{\hspace{5mm}}} &= \frac{\dotuline{\hspace{5mm}}}{\dotuline{\hspace{5mm}}}\\
        x &= \dotuline{\hspace{5mm}}\\
    \end{align*}
\end{minipage}\refstepcounter{minipagecount} % increments the counter minipagecount by one.
\noindent{(\theminipagecount)}\hspace{0.1mm} % By default, LaTeX indents the first line of a new paragraph, but \noindent overrides this
% and inserts the current value of the minipagecount counter, enclosed in parentheses
\begin{minipage}[t]{0.45\textwidth} % The [t] option aligns the top of the minipage with the baseline of the surrounding text.
    \vspace{-26pt}  % moves the content of the minipage up, reducing the space between the minipage content and the preceding text.
    \raggedright %  the text lines up on the left side, but the right side will be ragged.
    \begin{align*} % The * align environment does not number the equations- Each line is aligned at the & symbol
        4(x + 5) &= 24\\
        \frac{4(x+5)}{\dotuline{\hspace{5mm}}} &= \frac{24}{\dotuline{\hspace{5mm}}}\\
        x + 5 &= \dotuline{\hspace{5mm}}\\
        x + 5 - \dotuline{\hspace{5mm}} &= \dotuline{\hspace{5mm}} - \dotuline{\hspace{5mm}}\\
        x &= \dotuline{\hspace{5mm}}\\
    \end{align*}
\end{minipage}\refstepcounter{minipagecount} % increments the counter minipagecount by one.
\noindent{(\theminipagecount)}\hspace{0.1mm} % By default, LaTeX indents the first line of a new paragraph, but \noindent overrides this
% and inserts the current value of the minipagecount counter, enclosed in parentheses
\begin{minipage}[t]{0.45\textwidth} % The [t] option aligns the top of the minipage with the baseline of the surrounding text.
    \vspace{-26pt}  % moves the content of the minipage up, reducing the space between the minipage content and the preceding text.
    \raggedright %  the text lines up on the left side, but the right side will be ragged.
    \begin{align*} % The * align environment does not number the equations- Each line is aligned at the & symbol
        \frac{x}{9} - 3 &= 6\\
        \frac{x}{9} - 3 + \dotuline{\hspace{5mm}} &= 6 + \dotuline{\hspace{5mm}}\\
        \frac{x}{9} &= \dotuline{\hspace{5mm}}\\
        \frac{x}{9} \times\dotuline{\hspace{5mm}} &= \dotuline{\hspace{5mm}} \times\dotuline{\hspace{5mm}}\\
        x &= \dotuline{\hspace{5mm}}\\
    \end{align*}
\end{minipage}\refstepcounter{minipagecount} % increments the counter minipagecount by one.
\noindent{(\theminipagecount)}\hspace{0.1mm} % By default, LaTeX indents the first line of a new paragraph, but \noindent overrides this
% and inserts the current value of the minipagecount counter, enclosed in parentheses
\begin{minipage}[t]{0.45\textwidth} % The [t] option aligns the top of the minipage with the baseline of the surrounding text.
    \vspace{-26pt}  % moves the content of the minipage up, reducing the space between the minipage content and the preceding text.
    \raggedright %  the text lines up on the left side, but the right side will be ragged.
    \begin{align*} % The * align environment does not number the equations- Each line is aligned at the & symbol
        \frac{x}{9} - 4 &= 6\\
        \frac{x}{9} - 4 + \dotuline{\hspace{5mm}} &= 6 + \dotuline{\hspace{5mm}}\\
        \frac{x}{9} &= \dotuline{\hspace{5mm}}\\
        \frac{x}{9} \times\dotuline{\hspace{5mm}} &= \dotuline{\hspace{5mm}} \times\dotuline{\hspace{5mm}}\\
        x &= \dotuline{\hspace{5mm}}\\
    \end{align*}
\end{minipage}\refstepcounter{minipagecount} % increments the counter minipagecount by one.
\noindent{(\theminipagecount)}\hspace{0.1mm} % By default, LaTeX indents the first line of a new paragraph, but \noindent overrides this
% and inserts the current value of the minipagecount counter, enclosed in parentheses
\begin{minipage}[t]{0.45\textwidth} % The [t] option aligns the top of the minipage with the baseline of the surrounding text.
    \vspace{-26pt}  % moves the content of the minipage up, reducing the space between the minipage content and the preceding text.
    \raggedright %  the text lines up on the left side, but the right side will be ragged.
    \begin{align*} % The * align environment does not number the equations- Each line is aligned at the & symbol
        \frac{x + 3}{2} &= 3\\
        \frac{x + 3}{2} \times\dotuline{\hspace{5mm}} &= 3 \times\dotuline{\hspace{5mm}}\\
        x + 3 &= \dotuline{\hspace{5mm}}\\
        x + 3 - \dotuline{\hspace{5mm}} &= \dotuline{\hspace{5mm}} - \dotuline{\hspace{5mm}}\\
        x &= \dotuline{\hspace{5mm}}\\
    \end{align*}
\end{minipage}\columnbreak
    \refstepcounter{minipagecount} % increments the counter minipagecount by one.
\noindent{(\theminipagecount)}\hspace{0.1mm} % By default, LaTeX indents the first line of a new paragraph, but \noindent overrides this
% and inserts the current value of the minipagecount counter, enclosed in parentheses
\begin{minipage}[t]{0.45\textwidth} % The [t] option aligns the top of the minipage with the baseline of the surrounding text.
    \vspace{-26pt}  % moves the content of the minipage up, reducing the space between the minipage content and the preceding text.
    \raggedright %  the text lines up on the left side, but the right side will be ragged.
    \begin{align*} % The * align environment does not number the equations- Each line is aligned at the & symbol
        7x - 8 &= -1\\
        7x - 8 + \dotuline{\hspace{5mm}} &= -1 + \dotuline{\hspace{5mm}}\\
        7x &= \dotuline{\hspace{5mm}}\\
        \frac{7x}{\dotuline{\hspace{5mm}}} &= \frac{\dotuline{\hspace{5mm}}}{\dotuline{\hspace{5mm}}}\\
        x &= \dotuline{\hspace{5mm}}\\
    \end{align*}
\end{minipage}\refstepcounter{minipagecount} % increments the counter minipagecount by one.
\noindent{(\theminipagecount)}\hspace{0.1mm} % By default, LaTeX indents the first line of a new paragraph, but \noindent overrides this
% and inserts the current value of the minipagecount counter, enclosed in parentheses
\begin{minipage}[t]{0.45\textwidth} % The [t] option aligns the top of the minipage with the baseline of the surrounding text.
    \vspace{-26pt}  % moves the content of the minipage up, reducing the space between the minipage content and the preceding text.
    \raggedright %  the text lines up on the left side, but the right side will be ragged.
    \begin{align*} % The * align environment does not number the equations- Each line is aligned at the & symbol
        2x - 2 &= 4\\
        2x - 2 + \dotuline{\hspace{5mm}} &= 4 + \dotuline{\hspace{5mm}}\\
        2x &= \dotuline{\hspace{5mm}}\\
        \frac{2x}{\dotuline{\hspace{5mm}}} &= \frac{\dotuline{\hspace{5mm}}}{\dotuline{\hspace{5mm}}}\\
        x &= \dotuline{\hspace{5mm}}\\
    \end{align*}
\end{minipage}\refstepcounter{minipagecount} % increments the counter minipagecount by one.
\noindent{(\theminipagecount)}\hspace{0.1mm} % By default, LaTeX indents the first line of a new paragraph, but \noindent overrides this
% and inserts the current value of the minipagecount counter, enclosed in parentheses
\begin{minipage}[t]{0.45\textwidth} % The [t] option aligns the top of the minipage with the baseline of the surrounding text.
    \vspace{-26pt}  % moves the content of the minipage up, reducing the space between the minipage content and the preceding text.
    \raggedright %  the text lines up on the left side, but the right side will be ragged.
    \begin{align*} % The * align environment does not number the equations- Each line is aligned at the & symbol
        \frac{x + 1}{10} &= 1\\
        \frac{x + 1}{10} \times\dotuline{\hspace{5mm}} &= 1 \times\dotuline{\hspace{5mm}}\\
        x + 1 &= \dotuline{\hspace{5mm}}\\
        x + 1 - \dotuline{\hspace{5mm}} &= \dotuline{\hspace{5mm}} - \dotuline{\hspace{5mm}}\\
        x &= \dotuline{\hspace{5mm}}\\
    \end{align*}
\end{minipage}\refstepcounter{minipagecount} % increments the counter minipagecount by one.
\noindent{(\theminipagecount)}\hspace{0.1mm} % By default, LaTeX indents the first line of a new paragraph, but \noindent overrides this
% and inserts the current value of the minipagecount counter, enclosed in parentheses
\begin{minipage}[t]{0.45\textwidth} % The [t] option aligns the top of the minipage with the baseline of the surrounding text.
    \vspace{-26pt}  % moves the content of the minipage up, reducing the space between the minipage content and the preceding text.
    \raggedright %  the text lines up on the left side, but the right side will be ragged.
    \begin{align*} % The * align environment does not number the equations- Each line is aligned at the & symbol
        2(x + 4) &= 26\\
        \frac{2(x+4)}{\dotuline{\hspace{5mm}}} &= \frac{26}{\dotuline{\hspace{5mm}}}\\
        x + 4 &= \dotuline{\hspace{5mm}}\\
        x + 4 - \dotuline{\hspace{5mm}} &= \dotuline{\hspace{5mm}} - \dotuline{\hspace{5mm}}\\
        x &= \dotuline{\hspace{5mm}}\\
    \end{align*}
\end{minipage}\refstepcounter{minipagecount} % increments the counter minipagecount by one.
\noindent{(\theminipagecount)}\hspace{0.1mm} % By default, LaTeX indents the first line of a new paragraph, but \noindent overrides this
% and inserts the current value of the minipagecount counter, enclosed in parentheses
\begin{minipage}[t]{0.45\textwidth} % The [t] option aligns the top of the minipage with the baseline of the surrounding text.
    \vspace{-26pt}  % moves the content of the minipage up, reducing the space between the minipage content and the preceding text.
    \raggedright %  the text lines up on the left side, but the right side will be ragged.
    \begin{align*} % The * align environment does not number the equations- Each line is aligned at the & symbol
        \frac{x}{5} + 9 &= 18\\
        \frac{x}{5} + 9 - \dotuline{\hspace{5mm}} &= 18 - \dotuline{\hspace{5mm}}\\
        \frac{x}{5} &= \dotuline{\hspace{5mm}}\\
        \frac{x}{5} \times\dotuline{\hspace{5mm}} &= \dotuline{\hspace{5mm}} \times\dotuline{\hspace{5mm}}\\
        x &= \dotuline{\hspace{5mm}}\\
    \end{align*}
\end{minipage}\newpage
    \refstepcounter{minipagecount} % increments the counter minipagecount by one.
\noindent{(\theminipagecount)}\hspace{0.1mm} % By default, LaTeX indents the first line of a new paragraph, but \noindent overrides this
% and inserts the current value of the minipagecount counter, enclosed in parentheses
\begin{minipage}[t]{0.45\textwidth} % The [t] option aligns the top of the minipage with the baseline of the surrounding text.
    \vspace{-26pt}  % moves the content of the minipage up, reducing the space between the minipage content and the preceding text.
    \raggedright %  the text lines up on the left side, but the right side will be ragged.
    \begin{align*} % The * align environment does not number the equations- Each line is aligned at the & symbol
        \frac{x + 3}{4} &= 6\\
        \frac{x + 3}{4} \times\dotuline{\hspace{5mm}} &= 6 \times\dotuline{\hspace{5mm}}\\
        x + 3 &= \dotuline{\hspace{5mm}}\\
        x + 3 - \dotuline{\hspace{5mm}} &= \dotuline{\hspace{5mm}} - \dotuline{\hspace{5mm}}\\
        x &= \dotuline{\hspace{5mm}}\\
    \end{align*}
\end{minipage}\refstepcounter{minipagecount} % increments the counter minipagecount by one.
\noindent{(\theminipagecount)}\hspace{0.1mm} % By default, LaTeX indents the first line of a new paragraph, but \noindent overrides this
% and inserts the current value of the minipagecount counter, enclosed in parentheses
\begin{minipage}[t]{0.45\textwidth} % The [t] option aligns the top of the minipage with the baseline of the surrounding text.
    \vspace{-26pt}  % moves the content of the minipage up, reducing the space between the minipage content and the preceding text.
    \raggedright %  the text lines up on the left side, but the right side will be ragged.
    \begin{align*} % The * align environment does not number the equations- Each line is aligned at the & symbol
        9x - 4 &= 5\\
        9x - 4 + \dotuline{\hspace{5mm}} &= 5 + \dotuline{\hspace{5mm}}\\
        9x &= \dotuline{\hspace{5mm}}\\
        \frac{9x}{\dotuline{\hspace{5mm}}} &= \frac{\dotuline{\hspace{5mm}}}{\dotuline{\hspace{5mm}}}\\
        x &= \dotuline{\hspace{5mm}}\\
    \end{align*}
\end{minipage}\refstepcounter{minipagecount} % increments the counter minipagecount by one.
\noindent{(\theminipagecount)}\hspace{0.1mm} % By default, LaTeX indents the first line of a new paragraph, but \noindent overrides this
% and inserts the current value of the minipagecount counter, enclosed in parentheses
\begin{minipage}[t]{0.45\textwidth} % The [t] option aligns the top of the minipage with the baseline of the surrounding text.
    \vspace{-26pt}  % moves the content of the minipage up, reducing the space between the minipage content and the preceding text.
    \raggedright %  the text lines up on the left side, but the right side will be ragged.
    \begin{align*} % The * align environment does not number the equations- Each line is aligned at the & symbol
        \frac{x + 3}{2} &= 5\\
        \frac{x + 3}{2} \times\dotuline{\hspace{5mm}} &= 5 \times\dotuline{\hspace{5mm}}\\
        x + 3 &= \dotuline{\hspace{5mm}}\\
        x + 3 - \dotuline{\hspace{5mm}} &= \dotuline{\hspace{5mm}} - \dotuline{\hspace{5mm}}\\
        x &= \dotuline{\hspace{5mm}}\\
    \end{align*}
\end{minipage}\refstepcounter{minipagecount} % increments the counter minipagecount by one.
\noindent{(\theminipagecount)}\hspace{0.1mm} % By default, LaTeX indents the first line of a new paragraph, but \noindent overrides this
% and inserts the current value of the minipagecount counter, enclosed in parentheses
\begin{minipage}[t]{0.45\textwidth} % The [t] option aligns the top of the minipage with the baseline of the surrounding text.
    \vspace{-26pt}  % moves the content of the minipage up, reducing the space between the minipage content and the preceding text.
    \raggedright %  the text lines up on the left side, but the right side will be ragged.
    \begin{align*} % The * align environment does not number the equations- Each line is aligned at the & symbol
        \frac{x + 2}{9} &= 7\\
        \frac{x + 2}{9} \times\dotuline{\hspace{5mm}} &= 7 \times\dotuline{\hspace{5mm}}\\
        x + 2 &= \dotuline{\hspace{5mm}}\\
        x + 2 - \dotuline{\hspace{5mm}} &= \dotuline{\hspace{5mm}} - \dotuline{\hspace{5mm}}\\
        x &= \dotuline{\hspace{5mm}}\\
    \end{align*}
\end{minipage}\refstepcounter{minipagecount} % increments the counter minipagecount by one.
\noindent{(\theminipagecount)}\hspace{0.1mm} % By default, LaTeX indents the first line of a new paragraph, but \noindent overrides this
% and inserts the current value of the minipagecount counter, enclosed in parentheses
\begin{minipage}[t]{0.45\textwidth} % The [t] option aligns the top of the minipage with the baseline of the surrounding text.
    \vspace{-26pt}  % moves the content of the minipage up, reducing the space between the minipage content and the preceding text.
    \raggedright %  the text lines up on the left side, but the right side will be ragged.
    \begin{align*} % The * align environment does not number the equations- Each line is aligned at the & symbol
        \frac{x}{2} + 10 &= 13\\
        \frac{x}{2} + 10 - \dotuline{\hspace{5mm}} &= 13 - \dotuline{\hspace{5mm}}\\
        \frac{x}{2} &= \dotuline{\hspace{5mm}}\\
        \frac{x}{2} \times\dotuline{\hspace{5mm}} &= \dotuline{\hspace{5mm}} \times\dotuline{\hspace{5mm}}\\
        x &= \dotuline{\hspace{5mm}}\\
    \end{align*}
\end{minipage}\columnbreak
    \refstepcounter{minipagecount} % increments the counter minipagecount by one.
\noindent{(\theminipagecount)}\hspace{0.1mm} % By default, LaTeX indents the first line of a new paragraph, but \noindent overrides this
% and inserts the current value of the minipagecount counter, enclosed in parentheses
\begin{minipage}[t]{0.45\textwidth} % The [t] option aligns the top of the minipage with the baseline of the surrounding text.
    \vspace{-26pt}  % moves the content of the minipage up, reducing the space between the minipage content and the preceding text.
    \raggedright %  the text lines up on the left side, but the right side will be ragged.
    \begin{align*} % The * align environment does not number the equations- Each line is aligned at the & symbol
        2x + 9 &= 23\\
        2x + 9 - \dotuline{\hspace{5mm}} &= 23 - \dotuline{\hspace{5mm}}\\
        2x &= \dotuline{\hspace{5mm}}\\
        \frac{2x}{\dotuline{\hspace{5mm}}} &= \frac{\dotuline{\hspace{5mm}}}{\dotuline{\hspace{5mm}}}\\
        x &= \dotuline{\hspace{5mm}}\\
    \end{align*}
\end{minipage}\refstepcounter{minipagecount} % increments the counter minipagecount by one.
\noindent{(\theminipagecount)}\hspace{0.1mm} % By default, LaTeX indents the first line of a new paragraph, but \noindent overrides this
% and inserts the current value of the minipagecount counter, enclosed in parentheses
\begin{minipage}[t]{0.45\textwidth} % The [t] option aligns the top of the minipage with the baseline of the surrounding text.
    \vspace{-26pt}  % moves the content of the minipage up, reducing the space between the minipage content and the preceding text.
    \raggedright %  the text lines up on the left side, but the right side will be ragged.
    \begin{align*} % The * align environment does not number the equations- Each line is aligned at the & symbol
        \frac{x}{7} + 7 &= 9\\
        \frac{x}{7} + 7 - \dotuline{\hspace{5mm}} &= 9 - \dotuline{\hspace{5mm}}\\
        \frac{x}{7} &= \dotuline{\hspace{5mm}}\\
        \frac{x}{7} \times\dotuline{\hspace{5mm}} &= \dotuline{\hspace{5mm}} \times\dotuline{\hspace{5mm}}\\
        x &= \dotuline{\hspace{5mm}}\\
    \end{align*}
\end{minipage}\refstepcounter{minipagecount} % increments the counter minipagecount by one.
\noindent{(\theminipagecount)}\hspace{0.1mm} % By default, LaTeX indents the first line of a new paragraph, but \noindent overrides this
% and inserts the current value of the minipagecount counter, enclosed in parentheses
\begin{minipage}[t]{0.45\textwidth} % The [t] option aligns the top of the minipage with the baseline of the surrounding text.
    \vspace{-26pt}  % moves the content of the minipage up, reducing the space between the minipage content and the preceding text.
    \raggedright %  the text lines up on the left side, but the right side will be ragged.
    \begin{align*} % The * align environment does not number the equations- Each line is aligned at the & symbol
        8x + 6 &= 38\\
        8x + 6 - \dotuline{\hspace{5mm}} &= 38 - \dotuline{\hspace{5mm}}\\
        8x &= \dotuline{\hspace{5mm}}\\
        \frac{8x}{\dotuline{\hspace{5mm}}} &= \frac{\dotuline{\hspace{5mm}}}{\dotuline{\hspace{5mm}}}\\
        x &= \dotuline{\hspace{5mm}}\\
    \end{align*}
\end{minipage}\refstepcounter{minipagecount} % increments the counter minipagecount by one.
\noindent{(\theminipagecount)}\hspace{0.1mm} % By default, LaTeX indents the first line of a new paragraph, but \noindent overrides this
% and inserts the current value of the minipagecount counter, enclosed in parentheses
\begin{minipage}[t]{0.45\textwidth} % The [t] option aligns the top of the minipage with the baseline of the surrounding text.
    \vspace{-26pt}  % moves the content of the minipage up, reducing the space between the minipage content and the preceding text.
    \raggedright %  the text lines up on the left side, but the right side will be ragged.
    \begin{align*} % The * align environment does not number the equations- Each line is aligned at the & symbol
        \frac{x}{7} - 10 &= -1\\
        \frac{x}{7} - 10 + \dotuline{\hspace{5mm}} &= -1 + \dotuline{\hspace{5mm}}\\
        \frac{x}{7} &= \dotuline{\hspace{5mm}}\\
        \frac{x}{7} \times\dotuline{\hspace{5mm}} &= \dotuline{\hspace{5mm}} \times\dotuline{\hspace{5mm}}\\
        x &= \dotuline{\hspace{5mm}}\\
    \end{align*}
\end{minipage}\refstepcounter{minipagecount} % increments the counter minipagecount by one.
\noindent{(\theminipagecount)}\hspace{0.1mm} % By default, LaTeX indents the first line of a new paragraph, but \noindent overrides this
% and inserts the current value of the minipagecount counter, enclosed in parentheses
\begin{minipage}[t]{0.45\textwidth} % The [t] option aligns the top of the minipage with the baseline of the surrounding text.
    \vspace{-26pt}  % moves the content of the minipage up, reducing the space between the minipage content and the preceding text.
    \raggedright %  the text lines up on the left side, but the right side will be ragged.
    \begin{align*} % The * align environment does not number the equations- Each line is aligned at the & symbol
        \frac{x}{9} - 5 &= 0\\
        \frac{x}{9} - 5 + \dotuline{\hspace{5mm}} &= 0 + \dotuline{\hspace{5mm}}\\
        \frac{x}{9} &= \dotuline{\hspace{5mm}}\\
        \frac{x}{9} \times\dotuline{\hspace{5mm}} &= \dotuline{\hspace{5mm}} \times\dotuline{\hspace{5mm}}\\
        x &= \dotuline{\hspace{5mm}}\\
    \end{align*}
\end{minipage}\newpage
    \refstepcounter{minipagecount} % increments the counter minipagecount by one.
\noindent{(\theminipagecount)}\hspace{0.1mm} % By default, LaTeX indents the first line of a new paragraph, but \noindent overrides this
% and inserts the current value of the minipagecount counter, enclosed in parentheses
\begin{minipage}[t]{0.45\textwidth} % The [t] option aligns the top of the minipage with the baseline of the surrounding text.
    \vspace{-26pt}  % moves the content of the minipage up, reducing the space between the minipage content and the preceding text.
    \raggedright %  the text lines up on the left side, but the right side will be ragged.
    \begin{align*} % The * align environment does not number the equations- Each line is aligned at the & symbol
        10(x - 8) &= 10\\
        \frac{10(x-8)}{\dotuline{\hspace{5mm}}} &= \frac{10}{\dotuline{\hspace{5mm}}}\\
        x - 8 &= \dotuline{\hspace{5mm}}\\
        x - 8 + \dotuline{\hspace{5mm}} &= \dotuline{\hspace{5mm}} + \dotuline{\hspace{5mm}}\\
        x &= \dotuline{\hspace{5mm}}\\
    \end{align*}
\end{minipage}\refstepcounter{minipagecount} % increments the counter minipagecount by one.
\noindent{(\theminipagecount)}\hspace{0.1mm} % By default, LaTeX indents the first line of a new paragraph, but \noindent overrides this
% and inserts the current value of the minipagecount counter, enclosed in parentheses
\begin{minipage}[t]{0.45\textwidth} % The [t] option aligns the top of the minipage with the baseline of the surrounding text.
    \vspace{-26pt}  % moves the content of the minipage up, reducing the space between the minipage content and the preceding text.
    \raggedright %  the text lines up on the left side, but the right side will be ragged.
    \begin{align*} % The * align environment does not number the equations- Each line is aligned at the & symbol
        4(x + 7) &= 48\\
        \frac{4(x+7)}{\dotuline{\hspace{5mm}}} &= \frac{48}{\dotuline{\hspace{5mm}}}\\
        x + 7 &= \dotuline{\hspace{5mm}}\\
        x + 7 - \dotuline{\hspace{5mm}} &= \dotuline{\hspace{5mm}} - \dotuline{\hspace{5mm}}\\
        x &= \dotuline{\hspace{5mm}}\\
    \end{align*}
\end{minipage}\refstepcounter{minipagecount} % increments the counter minipagecount by one.
\noindent{(\theminipagecount)}\hspace{0.1mm} % By default, LaTeX indents the first line of a new paragraph, but \noindent overrides this
% and inserts the current value of the minipagecount counter, enclosed in parentheses
\begin{minipage}[t]{0.45\textwidth} % The [t] option aligns the top of the minipage with the baseline of the surrounding text.
    \vspace{-26pt}  % moves the content of the minipage up, reducing the space between the minipage content and the preceding text.
    \raggedright %  the text lines up on the left side, but the right side will be ragged.
    \begin{align*} % The * align environment does not number the equations- Each line is aligned at the & symbol
        7(x - 3) &= 21\\
        \frac{7(x-3)}{\dotuline{\hspace{5mm}}} &= \frac{21}{\dotuline{\hspace{5mm}}}\\
        x - 3 &= \dotuline{\hspace{5mm}}\\
        x - 3 + \dotuline{\hspace{5mm}} &= \dotuline{\hspace{5mm}} + \dotuline{\hspace{5mm}}\\
        x &= \dotuline{\hspace{5mm}}\\
    \end{align*}
\end{minipage}\refstepcounter{minipagecount} % increments the counter minipagecount by one.
\noindent{(\theminipagecount)}\hspace{0.1mm} % By default, LaTeX indents the first line of a new paragraph, but \noindent overrides this
% and inserts the current value of the minipagecount counter, enclosed in parentheses
\begin{minipage}[t]{0.45\textwidth} % The [t] option aligns the top of the minipage with the baseline of the surrounding text.
    \vspace{-26pt}  % moves the content of the minipage up, reducing the space between the minipage content and the preceding text.
    \raggedright %  the text lines up on the left side, but the right side will be ragged.
    \begin{align*} % The * align environment does not number the equations- Each line is aligned at the & symbol
        \frac{x}{3} - 3 &= 7\\
        \frac{x}{3} - 3 + \dotuline{\hspace{5mm}} &= 7 + \dotuline{\hspace{5mm}}\\
        \frac{x}{3} &= \dotuline{\hspace{5mm}}\\
        \frac{x}{3} \times\dotuline{\hspace{5mm}} &= \dotuline{\hspace{5mm}} \times\dotuline{\hspace{5mm}}\\
        x &= \dotuline{\hspace{5mm}}\\
    \end{align*}
\end{minipage}\refstepcounter{minipagecount} % increments the counter minipagecount by one.
\noindent{(\theminipagecount)}\hspace{0.1mm} % By default, LaTeX indents the first line of a new paragraph, but \noindent overrides this
% and inserts the current value of the minipagecount counter, enclosed in parentheses
\begin{minipage}[t]{0.45\textwidth} % The [t] option aligns the top of the minipage with the baseline of the surrounding text.
    \vspace{-26pt}  % moves the content of the minipage up, reducing the space between the minipage content and the preceding text.
    \raggedright %  the text lines up on the left side, but the right side will be ragged.
    \begin{align*} % The * align environment does not number the equations- Each line is aligned at the & symbol
        8x + 3 &= 19\\
        8x + 3 - \dotuline{\hspace{5mm}} &= 19 - \dotuline{\hspace{5mm}}\\
        8x &= \dotuline{\hspace{5mm}}\\
        \frac{8x}{\dotuline{\hspace{5mm}}} &= \frac{\dotuline{\hspace{5mm}}}{\dotuline{\hspace{5mm}}}\\
        x &= \dotuline{\hspace{5mm}}\\
    \end{align*}
\end{minipage}\columnbreak
    \refstepcounter{minipagecount} % increments the counter minipagecount by one.
\noindent{(\theminipagecount)}\hspace{0.1mm} % By default, LaTeX indents the first line of a new paragraph, but \noindent overrides this
% and inserts the current value of the minipagecount counter, enclosed in parentheses
\begin{minipage}[t]{0.45\textwidth} % The [t] option aligns the top of the minipage with the baseline of the surrounding text.
    \vspace{-26pt}  % moves the content of the minipage up, reducing the space between the minipage content and the preceding text.
    \raggedright %  the text lines up on the left side, but the right side will be ragged.
    \begin{align*} % The * align environment does not number the equations- Each line is aligned at the & symbol
        \frac{x}{5} - 8 &= -4\\
        \frac{x}{5} - 8 + \dotuline{\hspace{5mm}} &= -4 + \dotuline{\hspace{5mm}}\\
        \frac{x}{5} &= \dotuline{\hspace{5mm}}\\
        \frac{x}{5} \times\dotuline{\hspace{5mm}} &= \dotuline{\hspace{5mm}} \times\dotuline{\hspace{5mm}}\\
        x &= \dotuline{\hspace{5mm}}\\
    \end{align*}
\end{minipage}\refstepcounter{minipagecount} % increments the counter minipagecount by one.
\noindent{(\theminipagecount)}\hspace{0.1mm} % By default, LaTeX indents the first line of a new paragraph, but \noindent overrides this
% and inserts the current value of the minipagecount counter, enclosed in parentheses
\begin{minipage}[t]{0.45\textwidth} % The [t] option aligns the top of the minipage with the baseline of the surrounding text.
    \vspace{-26pt}  % moves the content of the minipage up, reducing the space between the minipage content and the preceding text.
    \raggedright %  the text lines up on the left side, but the right side will be ragged.
    \begin{align*} % The * align environment does not number the equations- Each line is aligned at the & symbol
        2(x - 10) &= -4\\
        \frac{2(x-10)}{\dotuline{\hspace{5mm}}} &= \frac{-4}{\dotuline{\hspace{5mm}}}\\
        x - 10 &= \dotuline{\hspace{5mm}}\\
        x - 10 + \dotuline{\hspace{5mm}} &= \dotuline{\hspace{5mm}} + \dotuline{\hspace{5mm}}\\
        x &= \dotuline{\hspace{5mm}}\\
    \end{align*}
\end{minipage}\refstepcounter{minipagecount} % increments the counter minipagecount by one.
\noindent{(\theminipagecount)}\hspace{0.1mm} % By default, LaTeX indents the first line of a new paragraph, but \noindent overrides this
% and inserts the current value of the minipagecount counter, enclosed in parentheses
\begin{minipage}[t]{0.45\textwidth} % The [t] option aligns the top of the minipage with the baseline of the surrounding text.
    \vspace{-26pt}  % moves the content of the minipage up, reducing the space between the minipage content and the preceding text.
    \raggedright %  the text lines up on the left side, but the right side will be ragged.
    \begin{align*} % The * align environment does not number the equations- Each line is aligned at the & symbol
        6x - 9 &= 51\\
        6x - 9 + \dotuline{\hspace{5mm}} &= 51 + \dotuline{\hspace{5mm}}\\
        6x &= \dotuline{\hspace{5mm}}\\
        \frac{6x}{\dotuline{\hspace{5mm}}} &= \frac{\dotuline{\hspace{5mm}}}{\dotuline{\hspace{5mm}}}\\
        x &= \dotuline{\hspace{5mm}}\\
    \end{align*}
\end{minipage}\refstepcounter{minipagecount} % increments the counter minipagecount by one.
\noindent{(\theminipagecount)}\hspace{0.1mm} % By default, LaTeX indents the first line of a new paragraph, but \noindent overrides this
% and inserts the current value of the minipagecount counter, enclosed in parentheses
\begin{minipage}[t]{0.45\textwidth} % The [t] option aligns the top of the minipage with the baseline of the surrounding text.
    \vspace{-26pt}  % moves the content of the minipage up, reducing the space between the minipage content and the preceding text.
    \raggedright %  the text lines up on the left side, but the right side will be ragged.
    \begin{align*} % The * align environment does not number the equations- Each line is aligned at the & symbol
        9x - 6 &= 75\\
        9x - 6 + \dotuline{\hspace{5mm}} &= 75 + \dotuline{\hspace{5mm}}\\
        9x &= \dotuline{\hspace{5mm}}\\
        \frac{9x}{\dotuline{\hspace{5mm}}} &= \frac{\dotuline{\hspace{5mm}}}{\dotuline{\hspace{5mm}}}\\
        x &= \dotuline{\hspace{5mm}}\\
    \end{align*}
\end{minipage}\refstepcounter{minipagecount} % increments the counter minipagecount by one.
\noindent{(\theminipagecount)}\hspace{0.1mm} % By default, LaTeX indents the first line of a new paragraph, but \noindent overrides this
% and inserts the current value of the minipagecount counter, enclosed in parentheses
\begin{minipage}[t]{0.45\textwidth} % The [t] option aligns the top of the minipage with the baseline of the surrounding text.
    \vspace{-26pt}  % moves the content of the minipage up, reducing the space between the minipage content and the preceding text.
    \raggedright %  the text lines up on the left side, but the right side will be ragged.
    \begin{align*} % The * align environment does not number the equations- Each line is aligned at the & symbol
        \frac{x}{9} - 3 &= 7\\
        \frac{x}{9} - 3 + \dotuline{\hspace{5mm}} &= 7 + \dotuline{\hspace{5mm}}\\
        \frac{x}{9} &= \dotuline{\hspace{5mm}}\\
        \frac{x}{9} \times\dotuline{\hspace{5mm}} &= \dotuline{\hspace{5mm}} \times\dotuline{\hspace{5mm}}\\
        x &= \dotuline{\hspace{5mm}}\\
    \end{align*}
\end{minipage}\newpage
    \refstepcounter{minipagecount} % increments the counter minipagecount by one.
\noindent{(\theminipagecount)}\hspace{0.1mm} % By default, LaTeX indents the first line of a new paragraph, but \noindent overrides this
% and inserts the current value of the minipagecount counter, enclosed in parentheses
\begin{minipage}[t]{0.45\textwidth} % The [t] option aligns the top of the minipage with the baseline of the surrounding text.
    \vspace{-26pt}  % moves the content of the minipage up, reducing the space between the minipage content and the preceding text.
    \raggedright %  the text lines up on the left side, but the right side will be ragged.
    \begin{align*} % The * align environment does not number the equations- Each line is aligned at the & symbol
        \frac{x + 3}{8} &= 9\\
        \frac{x + 3}{8} \times\dotuline{\hspace{5mm}} &= 9 \times\dotuline{\hspace{5mm}}\\
        x + 3 &= \dotuline{\hspace{5mm}}\\
        x + 3 - \dotuline{\hspace{5mm}} &= \dotuline{\hspace{5mm}} - \dotuline{\hspace{5mm}}\\
        x &= \dotuline{\hspace{5mm}}\\
    \end{align*}
\end{minipage}\refstepcounter{minipagecount} % increments the counter minipagecount by one.
\noindent{(\theminipagecount)}\hspace{0.1mm} % By default, LaTeX indents the first line of a new paragraph, but \noindent overrides this
% and inserts the current value of the minipagecount counter, enclosed in parentheses
\begin{minipage}[t]{0.45\textwidth} % The [t] option aligns the top of the minipage with the baseline of the surrounding text.
    \vspace{-26pt}  % moves the content of the minipage up, reducing the space between the minipage content and the preceding text.
    \raggedright %  the text lines up on the left side, but the right side will be ragged.
    \begin{align*} % The * align environment does not number the equations- Each line is aligned at the & symbol
        \frac{x}{7} + 3 &= 13\\
        \frac{x}{7} + 3 - \dotuline{\hspace{5mm}} &= 13 - \dotuline{\hspace{5mm}}\\
        \frac{x}{7} &= \dotuline{\hspace{5mm}}\\
        \frac{x}{7} \times\dotuline{\hspace{5mm}} &= \dotuline{\hspace{5mm}} \times\dotuline{\hspace{5mm}}\\
        x &= \dotuline{\hspace{5mm}}\\
    \end{align*}
\end{minipage}\refstepcounter{minipagecount} % increments the counter minipagecount by one.
\noindent{(\theminipagecount)}\hspace{0.1mm} % By default, LaTeX indents the first line of a new paragraph, but \noindent overrides this
% and inserts the current value of the minipagecount counter, enclosed in parentheses
\begin{minipage}[t]{0.45\textwidth} % The [t] option aligns the top of the minipage with the baseline of the surrounding text.
    \vspace{-26pt}  % moves the content of the minipage up, reducing the space between the minipage content and the preceding text.
    \raggedright %  the text lines up on the left side, but the right side will be ragged.
    \begin{align*} % The * align environment does not number the equations- Each line is aligned at the & symbol
        \frac{x}{5} - 2 &= 3\\
        \frac{x}{5} - 2 + \dotuline{\hspace{5mm}} &= 3 + \dotuline{\hspace{5mm}}\\
        \frac{x}{5} &= \dotuline{\hspace{5mm}}\\
        \frac{x}{5} \times\dotuline{\hspace{5mm}} &= \dotuline{\hspace{5mm}} \times\dotuline{\hspace{5mm}}\\
        x &= \dotuline{\hspace{5mm}}\\
    \end{align*}
\end{minipage}\refstepcounter{minipagecount} % increments the counter minipagecount by one.
\noindent{(\theminipagecount)}\hspace{0.1mm} % By default, LaTeX indents the first line of a new paragraph, but \noindent overrides this
% and inserts the current value of the minipagecount counter, enclosed in parentheses
\begin{minipage}[t]{0.45\textwidth} % The [t] option aligns the top of the minipage with the baseline of the surrounding text.
    \vspace{-26pt}  % moves the content of the minipage up, reducing the space between the minipage content and the preceding text.
    \raggedright %  the text lines up on the left side, but the right side will be ragged.
    \begin{align*} % The * align environment does not number the equations- Each line is aligned at the & symbol
        7x + 7 &= 42\\
        7x + 7 - \dotuline{\hspace{5mm}} &= 42 - \dotuline{\hspace{5mm}}\\
        7x &= \dotuline{\hspace{5mm}}\\
        \frac{7x}{\dotuline{\hspace{5mm}}} &= \frac{\dotuline{\hspace{5mm}}}{\dotuline{\hspace{5mm}}}\\
        x &= \dotuline{\hspace{5mm}}\\
    \end{align*}
\end{minipage}\refstepcounter{minipagecount} % increments the counter minipagecount by one.
\noindent{(\theminipagecount)}\hspace{0.1mm} % By default, LaTeX indents the first line of a new paragraph, but \noindent overrides this
% and inserts the current value of the minipagecount counter, enclosed in parentheses
\begin{minipage}[t]{0.45\textwidth} % The [t] option aligns the top of the minipage with the baseline of the surrounding text.
    \vspace{-26pt}  % moves the content of the minipage up, reducing the space between the minipage content and the preceding text.
    \raggedright %  the text lines up on the left side, but the right side will be ragged.
    \begin{align*} % The * align environment does not number the equations- Each line is aligned at the & symbol
        \frac{x + 10}{6} &= 4\\
        \frac{x + 10}{6} \times\dotuline{\hspace{5mm}} &= 4 \times\dotuline{\hspace{5mm}}\\
        x + 10 &= \dotuline{\hspace{5mm}}\\
        x + 10 - \dotuline{\hspace{5mm}} &= \dotuline{\hspace{5mm}} - \dotuline{\hspace{5mm}}\\
        x &= \dotuline{\hspace{5mm}}\\
    \end{align*}
\end{minipage}\columnbreak
    \refstepcounter{minipagecount} % increments the counter minipagecount by one.
\noindent{(\theminipagecount)}\hspace{0.1mm} % By default, LaTeX indents the first line of a new paragraph, but \noindent overrides this
% and inserts the current value of the minipagecount counter, enclosed in parentheses
\begin{minipage}[t]{0.45\textwidth} % The [t] option aligns the top of the minipage with the baseline of the surrounding text.
    \vspace{-26pt}  % moves the content of the minipage up, reducing the space between the minipage content and the preceding text.
    \raggedright %  the text lines up on the left side, but the right side will be ragged.
    \begin{align*} % The * align environment does not number the equations- Each line is aligned at the & symbol
        \frac{x}{5} - 3 &= 2\\
        \frac{x}{5} - 3 + \dotuline{\hspace{5mm}} &= 2 + \dotuline{\hspace{5mm}}\\
        \frac{x}{5} &= \dotuline{\hspace{5mm}}\\
        \frac{x}{5} \times\dotuline{\hspace{5mm}} &= \dotuline{\hspace{5mm}} \times\dotuline{\hspace{5mm}}\\
        x &= \dotuline{\hspace{5mm}}\\
    \end{align*}
\end{minipage}\refstepcounter{minipagecount} % increments the counter minipagecount by one.
\noindent{(\theminipagecount)}\hspace{0.1mm} % By default, LaTeX indents the first line of a new paragraph, but \noindent overrides this
% and inserts the current value of the minipagecount counter, enclosed in parentheses
\begin{minipage}[t]{0.45\textwidth} % The [t] option aligns the top of the minipage with the baseline of the surrounding text.
    \vspace{-26pt}  % moves the content of the minipage up, reducing the space between the minipage content and the preceding text.
    \raggedright %  the text lines up on the left side, but the right side will be ragged.
    \begin{align*} % The * align environment does not number the equations- Each line is aligned at the & symbol
        \frac{x + 5}{8} &= 4\\
        \frac{x + 5}{8} \times\dotuline{\hspace{5mm}} &= 4 \times\dotuline{\hspace{5mm}}\\
        x + 5 &= \dotuline{\hspace{5mm}}\\
        x + 5 - \dotuline{\hspace{5mm}} &= \dotuline{\hspace{5mm}} - \dotuline{\hspace{5mm}}\\
        x &= \dotuline{\hspace{5mm}}\\
    \end{align*}
\end{minipage}\refstepcounter{minipagecount} % increments the counter minipagecount by one.
\noindent{(\theminipagecount)}\hspace{0.1mm} % By default, LaTeX indents the first line of a new paragraph, but \noindent overrides this
% and inserts the current value of the minipagecount counter, enclosed in parentheses
\begin{minipage}[t]{0.45\textwidth} % The [t] option aligns the top of the minipage with the baseline of the surrounding text.
    \vspace{-26pt}  % moves the content of the minipage up, reducing the space between the minipage content and the preceding text.
    \raggedright %  the text lines up on the left side, but the right side will be ragged.
    \begin{align*} % The * align environment does not number the equations- Each line is aligned at the & symbol
        6x - 8 &= 34\\
        6x - 8 + \dotuline{\hspace{5mm}} &= 34 + \dotuline{\hspace{5mm}}\\
        6x &= \dotuline{\hspace{5mm}}\\
        \frac{6x}{\dotuline{\hspace{5mm}}} &= \frac{\dotuline{\hspace{5mm}}}{\dotuline{\hspace{5mm}}}\\
        x &= \dotuline{\hspace{5mm}}\\
    \end{align*}
\end{minipage}\refstepcounter{minipagecount} % increments the counter minipagecount by one.
\noindent{(\theminipagecount)}\hspace{0.1mm} % By default, LaTeX indents the first line of a new paragraph, but \noindent overrides this
% and inserts the current value of the minipagecount counter, enclosed in parentheses
\begin{minipage}[t]{0.45\textwidth} % The [t] option aligns the top of the minipage with the baseline of the surrounding text.
    \vspace{-26pt}  % moves the content of the minipage up, reducing the space between the minipage content and the preceding text.
    \raggedright %  the text lines up on the left side, but the right side will be ragged.
    \begin{align*} % The * align environment does not number the equations- Each line is aligned at the & symbol
        \frac{x - 9}{9} &= 5\\
        \frac{x - 9}{9} \times\dotuline{\hspace{5mm}} &= 5 \times\dotuline{\hspace{5mm}}\\
        x - 9 &= \dotuline{\hspace{5mm}}\\
        x - 9 + \dotuline{\hspace{5mm}} &= \dotuline{\hspace{5mm}} + \dotuline{\hspace{5mm}}\\
        x &= \dotuline{\hspace{5mm}}\\
    \end{align*}
\end{minipage}\refstepcounter{minipagecount} % increments the counter minipagecount by one.
\noindent{(\theminipagecount)}\hspace{0.1mm} % By default, LaTeX indents the first line of a new paragraph, but \noindent overrides this
% and inserts the current value of the minipagecount counter, enclosed in parentheses
\begin{minipage}[t]{0.45\textwidth} % The [t] option aligns the top of the minipage with the baseline of the surrounding text.
    \vspace{-26pt}  % moves the content of the minipage up, reducing the space between the minipage content and the preceding text.
    \raggedright %  the text lines up on the left side, but the right side will be ragged.
    \begin{align*} % The * align environment does not number the equations- Each line is aligned at the & symbol
        \frac{x + 10}{8} &= 4\\
        \frac{x + 10}{8} \times\dotuline{\hspace{5mm}} &= 4 \times\dotuline{\hspace{5mm}}\\
        x + 10 &= \dotuline{\hspace{5mm}}\\
        x + 10 - \dotuline{\hspace{5mm}} &= \dotuline{\hspace{5mm}} - \dotuline{\hspace{5mm}}\\
        x &= \dotuline{\hspace{5mm}}\\
    \end{align*}
\end{minipage}\newpage

\end{multicols}
\end{document}
