\documentclass[12pt]{article}
\usepackage{tikz}
\usepackage{amsmath}
% Underlining package
\usepackage[normalem]{ulem} % [normalem] prevents the package from changing the default behavior of \emph to underline.
\usepackage[a4paper, portrait, margin=1cm]{geometry}
\usepackage{multicol}
\usepackage{fancyhdr}

\def \HeadingQuestions {\section*{\Large Name: \underline{\hspace{8cm}} \hfill Date: \underline{\hspace{3cm}}} \vspace{-3mm}
{Inverse operations: Questions} \vspace{1pt}\hrule}

% only in Q due to increased line height with dotted underline
\linespread{0.9} % Adjust line spacing factor to 0.9 if needed to fit 5 in column for 2 div steps

% raise footer with page number; no header
\fancypagestyle{myfancypagestyle}{
  \fancyhf{}% clear all header and footer fields
  \renewcommand{\headrulewidth}{0pt} % no rule under header
  \fancyfoot[C] {\thepage} \setlength{\footskip}{6pt} % raise page number 6pt
}
\pagestyle{myfancypagestyle}  % apply myfancypagestyle
\newcounter{minipagecount}
\begin{document}
\HeadingQuestions
\vspace{1mm}
\begin{multicols}{2}
\refstepcounter{minipagecount} % increments the counter minipagecount by one.
\noindent{(\theminipagecount)}\hspace{0.1mm} % By default, LaTeX indents the first line of a new paragraph, but \noindent overrides this
% and inserts the current value of the minipagecount counter, enclosed in parentheses
\begin{minipage}[t]{0.45\textwidth} % The [t] option aligns the top of the minipage with the baseline of the surrounding text.
    \vspace{-26pt}  % moves the content of the minipage up, reducing the space between the minipage content and the preceding text.
    \raggedright %  the text lines up on the left side, but the right side will be ragged.
    \begin{align*} % The * align environment does not number the equations- Each line is aligned at the & symbol
        \frac{x}{10} - 10 &= -2\\
        \frac{x}{10} - 10 + \dotuline{\hspace{5mm}} &= -2 + \dotuline{\hspace{5mm}}\\
        \frac{x}{10} &= \dotuline{\hspace{5mm}}\\
        \frac{x}{10} \times\dotuline{\hspace{5mm}} &= \dotuline{\hspace{5mm}} \times\dotuline{\hspace{5mm}}\\
        x &= \dotuline{\hspace{5mm}}\\
    \end{align*}
\end{minipage}\refstepcounter{minipagecount} % increments the counter minipagecount by one.
\noindent{(\theminipagecount)}\hspace{0.1mm} % By default, LaTeX indents the first line of a new paragraph, but \noindent overrides this
% and inserts the current value of the minipagecount counter, enclosed in parentheses
\begin{minipage}[t]{0.45\textwidth} % The [t] option aligns the top of the minipage with the baseline of the surrounding text.
    \vspace{-26pt}  % moves the content of the minipage up, reducing the space between the minipage content and the preceding text.
    \raggedright %  the text lines up on the left side, but the right side will be ragged.
    \begin{align*} % The * align environment does not number the equations- Each line is aligned at the & symbol
        \frac{x}{8} - 5 &= -2\\
        \frac{x}{8} - 5 + \dotuline{\hspace{5mm}} &= -2 + \dotuline{\hspace{5mm}}\\
        \frac{x}{8} &= \dotuline{\hspace{5mm}}\\
        \frac{x}{8} \times\dotuline{\hspace{5mm}} &= \dotuline{\hspace{5mm}} \times\dotuline{\hspace{5mm}}\\
        x &= \dotuline{\hspace{5mm}}\\
    \end{align*}
\end{minipage}\refstepcounter{minipagecount} % increments the counter minipagecount by one.
\noindent{(\theminipagecount)}\hspace{0.1mm} % By default, LaTeX indents the first line of a new paragraph, but \noindent overrides this
% and inserts the current value of the minipagecount counter, enclosed in parentheses
\begin{minipage}[t]{0.45\textwidth} % The [t] option aligns the top of the minipage with the baseline of the surrounding text.
    \vspace{-26pt}  % moves the content of the minipage up, reducing the space between the minipage content and the preceding text.
    \raggedright %  the text lines up on the left side, but the right side will be ragged.
    \begin{align*} % The * align environment does not number the equations- Each line is aligned at the & symbol
        \frac{x}{7} - 2 &= 2\\
        \frac{x}{7} - 2 + \dotuline{\hspace{5mm}} &= 2 + \dotuline{\hspace{5mm}}\\
        \frac{x}{7} &= \dotuline{\hspace{5mm}}\\
        \frac{x}{7} \times\dotuline{\hspace{5mm}} &= \dotuline{\hspace{5mm}} \times\dotuline{\hspace{5mm}}\\
        x &= \dotuline{\hspace{5mm}}\\
    \end{align*}
\end{minipage}\refstepcounter{minipagecount} % increments the counter minipagecount by one.
\noindent{(\theminipagecount)}\hspace{0.1mm} % By default, LaTeX indents the first line of a new paragraph, but \noindent overrides this
% and inserts the current value of the minipagecount counter, enclosed in parentheses
\begin{minipage}[t]{0.45\textwidth} % The [t] option aligns the top of the minipage with the baseline of the surrounding text.
    \vspace{-26pt}  % moves the content of the minipage up, reducing the space between the minipage content and the preceding text.
    \raggedright %  the text lines up on the left side, but the right side will be ragged.
    \begin{align*} % The * align environment does not number the equations- Each line is aligned at the & symbol
        \frac{x}{8} - 4 &= 0\\
        \frac{x}{8} - 4 + \dotuline{\hspace{5mm}} &= 0 + \dotuline{\hspace{5mm}}\\
        \frac{x}{8} &= \dotuline{\hspace{5mm}}\\
        \frac{x}{8} \times\dotuline{\hspace{5mm}} &= \dotuline{\hspace{5mm}} \times\dotuline{\hspace{5mm}}\\
        x &= \dotuline{\hspace{5mm}}\\
    \end{align*}
\end{minipage}\refstepcounter{minipagecount} % increments the counter minipagecount by one.
\noindent{(\theminipagecount)}\hspace{0.1mm} % By default, LaTeX indents the first line of a new paragraph, but \noindent overrides this
% and inserts the current value of the minipagecount counter, enclosed in parentheses
\begin{minipage}[t]{0.45\textwidth} % The [t] option aligns the top of the minipage with the baseline of the surrounding text.
    \vspace{-26pt}  % moves the content of the minipage up, reducing the space between the minipage content and the preceding text.
    \raggedright %  the text lines up on the left side, but the right side will be ragged.
    \begin{align*} % The * align environment does not number the equations- Each line is aligned at the & symbol
        \frac{x}{3} - 10 &= -4\\
        \frac{x}{3} - 10 + \dotuline{\hspace{5mm}} &= -4 + \dotuline{\hspace{5mm}}\\
        \frac{x}{3} &= \dotuline{\hspace{5mm}}\\
        \frac{x}{3} \times\dotuline{\hspace{5mm}} &= \dotuline{\hspace{5mm}} \times\dotuline{\hspace{5mm}}\\
        x &= \dotuline{\hspace{5mm}}\\
    \end{align*}
\end{minipage}\columnbreak
    \refstepcounter{minipagecount} % increments the counter minipagecount by one.
\noindent{(\theminipagecount)}\hspace{0.1mm} % By default, LaTeX indents the first line of a new paragraph, but \noindent overrides this
% and inserts the current value of the minipagecount counter, enclosed in parentheses
\begin{minipage}[t]{0.45\textwidth} % The [t] option aligns the top of the minipage with the baseline of the surrounding text.
    \vspace{-26pt}  % moves the content of the minipage up, reducing the space between the minipage content and the preceding text.
    \raggedright %  the text lines up on the left side, but the right side will be ragged.
    \begin{align*} % The * align environment does not number the equations- Each line is aligned at the & symbol
        \frac{x}{5} - 3 &= 2\\
        \frac{x}{5} - 3 + \dotuline{\hspace{5mm}} &= 2 + \dotuline{\hspace{5mm}}\\
        \frac{x}{5} &= \dotuline{\hspace{5mm}}\\
        \frac{x}{5} \times\dotuline{\hspace{5mm}} &= \dotuline{\hspace{5mm}} \times\dotuline{\hspace{5mm}}\\
        x &= \dotuline{\hspace{5mm}}\\
    \end{align*}
\end{minipage}\refstepcounter{minipagecount} % increments the counter minipagecount by one.
\noindent{(\theminipagecount)}\hspace{0.1mm} % By default, LaTeX indents the first line of a new paragraph, but \noindent overrides this
% and inserts the current value of the minipagecount counter, enclosed in parentheses
\begin{minipage}[t]{0.45\textwidth} % The [t] option aligns the top of the minipage with the baseline of the surrounding text.
    \vspace{-26pt}  % moves the content of the minipage up, reducing the space between the minipage content and the preceding text.
    \raggedright %  the text lines up on the left side, but the right side will be ragged.
    \begin{align*} % The * align environment does not number the equations- Each line is aligned at the & symbol
        \frac{x}{7} - 4 &= 2\\
        \frac{x}{7} - 4 + \dotuline{\hspace{5mm}} &= 2 + \dotuline{\hspace{5mm}}\\
        \frac{x}{7} &= \dotuline{\hspace{5mm}}\\
        \frac{x}{7} \times\dotuline{\hspace{5mm}} &= \dotuline{\hspace{5mm}} \times\dotuline{\hspace{5mm}}\\
        x &= \dotuline{\hspace{5mm}}\\
    \end{align*}
\end{minipage}\refstepcounter{minipagecount} % increments the counter minipagecount by one.
\noindent{(\theminipagecount)}\hspace{0.1mm} % By default, LaTeX indents the first line of a new paragraph, but \noindent overrides this
% and inserts the current value of the minipagecount counter, enclosed in parentheses
\begin{minipage}[t]{0.45\textwidth} % The [t] option aligns the top of the minipage with the baseline of the surrounding text.
    \vspace{-26pt}  % moves the content of the minipage up, reducing the space between the minipage content and the preceding text.
    \raggedright %  the text lines up on the left side, but the right side will be ragged.
    \begin{align*} % The * align environment does not number the equations- Each line is aligned at the & symbol
        \frac{x}{8} - 7 &= 3\\
        \frac{x}{8} - 7 + \dotuline{\hspace{5mm}} &= 3 + \dotuline{\hspace{5mm}}\\
        \frac{x}{8} &= \dotuline{\hspace{5mm}}\\
        \frac{x}{8} \times\dotuline{\hspace{5mm}} &= \dotuline{\hspace{5mm}} \times\dotuline{\hspace{5mm}}\\
        x &= \dotuline{\hspace{5mm}}\\
    \end{align*}
\end{minipage}\refstepcounter{minipagecount} % increments the counter minipagecount by one.
\noindent{(\theminipagecount)}\hspace{0.1mm} % By default, LaTeX indents the first line of a new paragraph, but \noindent overrides this
% and inserts the current value of the minipagecount counter, enclosed in parentheses
\begin{minipage}[t]{0.45\textwidth} % The [t] option aligns the top of the minipage with the baseline of the surrounding text.
    \vspace{-26pt}  % moves the content of the minipage up, reducing the space between the minipage content and the preceding text.
    \raggedright %  the text lines up on the left side, but the right side will be ragged.
    \begin{align*} % The * align environment does not number the equations- Each line is aligned at the & symbol
        \frac{x}{4} - 1 &= 8\\
        \frac{x}{4} - 1 + \dotuline{\hspace{5mm}} &= 8 + \dotuline{\hspace{5mm}}\\
        \frac{x}{4} &= \dotuline{\hspace{5mm}}\\
        \frac{x}{4} \times\dotuline{\hspace{5mm}} &= \dotuline{\hspace{5mm}} \times\dotuline{\hspace{5mm}}\\
        x &= \dotuline{\hspace{5mm}}\\
    \end{align*}
\end{minipage}\refstepcounter{minipagecount} % increments the counter minipagecount by one.
\noindent{(\theminipagecount)}\hspace{0.1mm} % By default, LaTeX indents the first line of a new paragraph, but \noindent overrides this
% and inserts the current value of the minipagecount counter, enclosed in parentheses
\begin{minipage}[t]{0.45\textwidth} % The [t] option aligns the top of the minipage with the baseline of the surrounding text.
    \vspace{-26pt}  % moves the content of the minipage up, reducing the space between the minipage content and the preceding text.
    \raggedright %  the text lines up on the left side, but the right side will be ragged.
    \begin{align*} % The * align environment does not number the equations- Each line is aligned at the & symbol
        \frac{x}{5} - 5 &= -2\\
        \frac{x}{5} - 5 + \dotuline{\hspace{5mm}} &= -2 + \dotuline{\hspace{5mm}}\\
        \frac{x}{5} &= \dotuline{\hspace{5mm}}\\
        \frac{x}{5} \times\dotuline{\hspace{5mm}} &= \dotuline{\hspace{5mm}} \times\dotuline{\hspace{5mm}}\\
        x &= \dotuline{\hspace{5mm}}\\
    \end{align*}
\end{minipage}\newpage
    \refstepcounter{minipagecount} % increments the counter minipagecount by one.
\noindent{(\theminipagecount)}\hspace{0.1mm} % By default, LaTeX indents the first line of a new paragraph, but \noindent overrides this
% and inserts the current value of the minipagecount counter, enclosed in parentheses
\begin{minipage}[t]{0.45\textwidth} % The [t] option aligns the top of the minipage with the baseline of the surrounding text.
    \vspace{-26pt}  % moves the content of the minipage up, reducing the space between the minipage content and the preceding text.
    \raggedright %  the text lines up on the left side, but the right side will be ragged.
    \begin{align*} % The * align environment does not number the equations- Each line is aligned at the & symbol
        \frac{x}{6} - 10 &= -3\\
        \frac{x}{6} - 10 + \dotuline{\hspace{5mm}} &= -3 + \dotuline{\hspace{5mm}}\\
        \frac{x}{6} &= \dotuline{\hspace{5mm}}\\
        \frac{x}{6} \times\dotuline{\hspace{5mm}} &= \dotuline{\hspace{5mm}} \times\dotuline{\hspace{5mm}}\\
        x &= \dotuline{\hspace{5mm}}\\
    \end{align*}
\end{minipage}\refstepcounter{minipagecount} % increments the counter minipagecount by one.
\noindent{(\theminipagecount)}\hspace{0.1mm} % By default, LaTeX indents the first line of a new paragraph, but \noindent overrides this
% and inserts the current value of the minipagecount counter, enclosed in parentheses
\begin{minipage}[t]{0.45\textwidth} % The [t] option aligns the top of the minipage with the baseline of the surrounding text.
    \vspace{-26pt}  % moves the content of the minipage up, reducing the space between the minipage content and the preceding text.
    \raggedright %  the text lines up on the left side, but the right side will be ragged.
    \begin{align*} % The * align environment does not number the equations- Each line is aligned at the & symbol
        \frac{x}{2} - 10 &= -4\\
        \frac{x}{2} - 10 + \dotuline{\hspace{5mm}} &= -4 + \dotuline{\hspace{5mm}}\\
        \frac{x}{2} &= \dotuline{\hspace{5mm}}\\
        \frac{x}{2} \times\dotuline{\hspace{5mm}} &= \dotuline{\hspace{5mm}} \times\dotuline{\hspace{5mm}}\\
        x &= \dotuline{\hspace{5mm}}\\
    \end{align*}
\end{minipage}\refstepcounter{minipagecount} % increments the counter minipagecount by one.
\noindent{(\theminipagecount)}\hspace{0.1mm} % By default, LaTeX indents the first line of a new paragraph, but \noindent overrides this
% and inserts the current value of the minipagecount counter, enclosed in parentheses
\begin{minipage}[t]{0.45\textwidth} % The [t] option aligns the top of the minipage with the baseline of the surrounding text.
    \vspace{-26pt}  % moves the content of the minipage up, reducing the space between the minipage content and the preceding text.
    \raggedright %  the text lines up on the left side, but the right side will be ragged.
    \begin{align*} % The * align environment does not number the equations- Each line is aligned at the & symbol
        \frac{x}{2} - 9 &= 1\\
        \frac{x}{2} - 9 + \dotuline{\hspace{5mm}} &= 1 + \dotuline{\hspace{5mm}}\\
        \frac{x}{2} &= \dotuline{\hspace{5mm}}\\
        \frac{x}{2} \times\dotuline{\hspace{5mm}} &= \dotuline{\hspace{5mm}} \times\dotuline{\hspace{5mm}}\\
        x &= \dotuline{\hspace{5mm}}\\
    \end{align*}
\end{minipage}\refstepcounter{minipagecount} % increments the counter minipagecount by one.
\noindent{(\theminipagecount)}\hspace{0.1mm} % By default, LaTeX indents the first line of a new paragraph, but \noindent overrides this
% and inserts the current value of the minipagecount counter, enclosed in parentheses
\begin{minipage}[t]{0.45\textwidth} % The [t] option aligns the top of the minipage with the baseline of the surrounding text.
    \vspace{-26pt}  % moves the content of the minipage up, reducing the space between the minipage content and the preceding text.
    \raggedright %  the text lines up on the left side, but the right side will be ragged.
    \begin{align*} % The * align environment does not number the equations- Each line is aligned at the & symbol
        \frac{x}{10} - 9 &= -4\\
        \frac{x}{10} - 9 + \dotuline{\hspace{5mm}} &= -4 + \dotuline{\hspace{5mm}}\\
        \frac{x}{10} &= \dotuline{\hspace{5mm}}\\
        \frac{x}{10} \times\dotuline{\hspace{5mm}} &= \dotuline{\hspace{5mm}} \times\dotuline{\hspace{5mm}}\\
        x &= \dotuline{\hspace{5mm}}\\
    \end{align*}
\end{minipage}\refstepcounter{minipagecount} % increments the counter minipagecount by one.
\noindent{(\theminipagecount)}\hspace{0.1mm} % By default, LaTeX indents the first line of a new paragraph, but \noindent overrides this
% and inserts the current value of the minipagecount counter, enclosed in parentheses
\begin{minipage}[t]{0.45\textwidth} % The [t] option aligns the top of the minipage with the baseline of the surrounding text.
    \vspace{-26pt}  % moves the content of the minipage up, reducing the space between the minipage content and the preceding text.
    \raggedright %  the text lines up on the left side, but the right side will be ragged.
    \begin{align*} % The * align environment does not number the equations- Each line is aligned at the & symbol
        \frac{x}{7} - 6 &= -3\\
        \frac{x}{7} - 6 + \dotuline{\hspace{5mm}} &= -3 + \dotuline{\hspace{5mm}}\\
        \frac{x}{7} &= \dotuline{\hspace{5mm}}\\
        \frac{x}{7} \times\dotuline{\hspace{5mm}} &= \dotuline{\hspace{5mm}} \times\dotuline{\hspace{5mm}}\\
        x &= \dotuline{\hspace{5mm}}\\
    \end{align*}
\end{minipage}\columnbreak
    \refstepcounter{minipagecount} % increments the counter minipagecount by one.
\noindent{(\theminipagecount)}\hspace{0.1mm} % By default, LaTeX indents the first line of a new paragraph, but \noindent overrides this
% and inserts the current value of the minipagecount counter, enclosed in parentheses
\begin{minipage}[t]{0.45\textwidth} % The [t] option aligns the top of the minipage with the baseline of the surrounding text.
    \vspace{-26pt}  % moves the content of the minipage up, reducing the space between the minipage content and the preceding text.
    \raggedright %  the text lines up on the left side, but the right side will be ragged.
    \begin{align*} % The * align environment does not number the equations- Each line is aligned at the & symbol
        \frac{x}{8} - 4 &= 5\\
        \frac{x}{8} - 4 + \dotuline{\hspace{5mm}} &= 5 + \dotuline{\hspace{5mm}}\\
        \frac{x}{8} &= \dotuline{\hspace{5mm}}\\
        \frac{x}{8} \times\dotuline{\hspace{5mm}} &= \dotuline{\hspace{5mm}} \times\dotuline{\hspace{5mm}}\\
        x &= \dotuline{\hspace{5mm}}\\
    \end{align*}
\end{minipage}\refstepcounter{minipagecount} % increments the counter minipagecount by one.
\noindent{(\theminipagecount)}\hspace{0.1mm} % By default, LaTeX indents the first line of a new paragraph, but \noindent overrides this
% and inserts the current value of the minipagecount counter, enclosed in parentheses
\begin{minipage}[t]{0.45\textwidth} % The [t] option aligns the top of the minipage with the baseline of the surrounding text.
    \vspace{-26pt}  % moves the content of the minipage up, reducing the space between the minipage content and the preceding text.
    \raggedright %  the text lines up on the left side, but the right side will be ragged.
    \begin{align*} % The * align environment does not number the equations- Each line is aligned at the & symbol
        \frac{x}{2} - 9 &= -6\\
        \frac{x}{2} - 9 + \dotuline{\hspace{5mm}} &= -6 + \dotuline{\hspace{5mm}}\\
        \frac{x}{2} &= \dotuline{\hspace{5mm}}\\
        \frac{x}{2} \times\dotuline{\hspace{5mm}} &= \dotuline{\hspace{5mm}} \times\dotuline{\hspace{5mm}}\\
        x &= \dotuline{\hspace{5mm}}\\
    \end{align*}
\end{minipage}\refstepcounter{minipagecount} % increments the counter minipagecount by one.
\noindent{(\theminipagecount)}\hspace{0.1mm} % By default, LaTeX indents the first line of a new paragraph, but \noindent overrides this
% and inserts the current value of the minipagecount counter, enclosed in parentheses
\begin{minipage}[t]{0.45\textwidth} % The [t] option aligns the top of the minipage with the baseline of the surrounding text.
    \vspace{-26pt}  % moves the content of the minipage up, reducing the space between the minipage content and the preceding text.
    \raggedright %  the text lines up on the left side, but the right side will be ragged.
    \begin{align*} % The * align environment does not number the equations- Each line is aligned at the & symbol
        \frac{x}{7} - 3 &= 2\\
        \frac{x}{7} - 3 + \dotuline{\hspace{5mm}} &= 2 + \dotuline{\hspace{5mm}}\\
        \frac{x}{7} &= \dotuline{\hspace{5mm}}\\
        \frac{x}{7} \times\dotuline{\hspace{5mm}} &= \dotuline{\hspace{5mm}} \times\dotuline{\hspace{5mm}}\\
        x &= \dotuline{\hspace{5mm}}\\
    \end{align*}
\end{minipage}\refstepcounter{minipagecount} % increments the counter minipagecount by one.
\noindent{(\theminipagecount)}\hspace{0.1mm} % By default, LaTeX indents the first line of a new paragraph, but \noindent overrides this
% and inserts the current value of the minipagecount counter, enclosed in parentheses
\begin{minipage}[t]{0.45\textwidth} % The [t] option aligns the top of the minipage with the baseline of the surrounding text.
    \vspace{-26pt}  % moves the content of the minipage up, reducing the space between the minipage content and the preceding text.
    \raggedright %  the text lines up on the left side, but the right side will be ragged.
    \begin{align*} % The * align environment does not number the equations- Each line is aligned at the & symbol
        \frac{x}{9} - 8 &= -3\\
        \frac{x}{9} - 8 + \dotuline{\hspace{5mm}} &= -3 + \dotuline{\hspace{5mm}}\\
        \frac{x}{9} &= \dotuline{\hspace{5mm}}\\
        \frac{x}{9} \times\dotuline{\hspace{5mm}} &= \dotuline{\hspace{5mm}} \times\dotuline{\hspace{5mm}}\\
        x &= \dotuline{\hspace{5mm}}\\
    \end{align*}
\end{minipage}\refstepcounter{minipagecount} % increments the counter minipagecount by one.
\noindent{(\theminipagecount)}\hspace{0.1mm} % By default, LaTeX indents the first line of a new paragraph, but \noindent overrides this
% and inserts the current value of the minipagecount counter, enclosed in parentheses
\begin{minipage}[t]{0.45\textwidth} % The [t] option aligns the top of the minipage with the baseline of the surrounding text.
    \vspace{-26pt}  % moves the content of the minipage up, reducing the space between the minipage content and the preceding text.
    \raggedright %  the text lines up on the left side, but the right side will be ragged.
    \begin{align*} % The * align environment does not number the equations- Each line is aligned at the & symbol
        \frac{x}{5} - 1 &= 2\\
        \frac{x}{5} - 1 + \dotuline{\hspace{5mm}} &= 2 + \dotuline{\hspace{5mm}}\\
        \frac{x}{5} &= \dotuline{\hspace{5mm}}\\
        \frac{x}{5} \times\dotuline{\hspace{5mm}} &= \dotuline{\hspace{5mm}} \times\dotuline{\hspace{5mm}}\\
        x &= \dotuline{\hspace{5mm}}\\
    \end{align*}
\end{minipage}\newpage
    
\end{multicols}
\end{document}
