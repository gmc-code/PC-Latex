\documentclass[12pt]{article}
\usepackage{tikz}
\usepackage{amsmath}
% Underlining package
\usepackage[normalem]{ulem} % [normalem] prevents the package from changing the default behavior of \emph to underline.
\usepackage[a4paper, portrait, margin=1cm]{geometry}
\usepackage{multicol}
\usepackage{fancyhdr}

\def \HeadingQuestions {\section*{\Large Name: \underline{\hspace{8cm}} \hfill Date: \underline{\hspace{3cm}}} \vspace{-3mm}
{Inverse operations: Questions} \vspace{1pt}\hrule}

% only in Q due to increased line height with dotted underline
\linespread{1} % Adjust line spacing factor to 0.9 if needed to fit 5 in column

% raise footer with page number; no header
\fancypagestyle{myfancypagestyle}{
  \fancyhf{}% clear all header and footer fields
  \renewcommand{\headrulewidth}{0pt} % no rule under header
  \fancyfoot[C] {\thepage} \setlength{\footskip}{14.5pt} % raise page number allowed min 14.5pt
}
\pagestyle{myfancypagestyle}  % apply myfancypagestyle
\newcounter{minipagecount}
\begin{document}
\HeadingQuestions
\vspace{1mm}
\begin{multicols}{2}
\refstepcounter{minipagecount} % increments the counter minipagecount by one.
\noindent{(\theminipagecount)}\hspace{0.1mm} % By default, LaTeX indents the first line of a new paragraph, but \noindent overrides this
% and inserts the current value of the minipagecount counter, enclosed in parentheses
\begin{minipage}[t]{0.45\textwidth} % The [t] option aligns the top of the minipage with the baseline of the surrounding text.
    \vspace{-26pt}  % moves the content of the minipage up, reducing the space between the minipage content and the preceding text.
    \raggedright %  the text lines up on the left side, but the right side will be ragged.
    \begin{align*} % The * align environment does not number the equations- Each line is aligned at the & symbol
        7x - 10 &= 18\\
        7x - 10 + \dotuline{\hspace{5mm}} &= 18 + \dotuline{\hspace{5mm}}\\
        7x &= \dotuline{\hspace{5mm}}\\
        \frac{7x}{\dotuline{\hspace{5mm}}} &= \frac{\dotuline{\hspace{5mm}}}{\dotuline{\hspace{5mm}}}\\
        x &= \dotuline{\hspace{5mm}}\\
    \end{align*}
\end{minipage}\refstepcounter{minipagecount} % increments the counter minipagecount by one.
\noindent{(\theminipagecount)}\hspace{0.1mm} % By default, LaTeX indents the first line of a new paragraph, but \noindent overrides this
% and inserts the current value of the minipagecount counter, enclosed in parentheses
\begin{minipage}[t]{0.45\textwidth} % The [t] option aligns the top of the minipage with the baseline of the surrounding text.
    \vspace{-26pt}  % moves the content of the minipage up, reducing the space between the minipage content and the preceding text.
    \raggedright %  the text lines up on the left side, but the right side will be ragged.
    \begin{align*} % The * align environment does not number the equations- Each line is aligned at the & symbol
        6x - 7 &= 29\\
        6x - 7 + \dotuline{\hspace{5mm}} &= 29 + \dotuline{\hspace{5mm}}\\
        6x &= \dotuline{\hspace{5mm}}\\
        \frac{6x}{\dotuline{\hspace{5mm}}} &= \frac{\dotuline{\hspace{5mm}}}{\dotuline{\hspace{5mm}}}\\
        x &= \dotuline{\hspace{5mm}}\\
    \end{align*}
\end{minipage}\refstepcounter{minipagecount} % increments the counter minipagecount by one.
\noindent{(\theminipagecount)}\hspace{0.1mm} % By default, LaTeX indents the first line of a new paragraph, but \noindent overrides this
% and inserts the current value of the minipagecount counter, enclosed in parentheses
\begin{minipage}[t]{0.45\textwidth} % The [t] option aligns the top of the minipage with the baseline of the surrounding text.
    \vspace{-26pt}  % moves the content of the minipage up, reducing the space between the minipage content and the preceding text.
    \raggedright %  the text lines up on the left side, but the right side will be ragged.
    \begin{align*} % The * align environment does not number the equations- Each line is aligned at the & symbol
        10x - 4 &= 16\\
        10x - 4 + \dotuline{\hspace{5mm}} &= 16 + \dotuline{\hspace{5mm}}\\
        10x &= \dotuline{\hspace{5mm}}\\
        \frac{10x}{\dotuline{\hspace{5mm}}} &= \frac{\dotuline{\hspace{5mm}}}{\dotuline{\hspace{5mm}}}\\
        x &= \dotuline{\hspace{5mm}}\\
    \end{align*}
\end{minipage}\refstepcounter{minipagecount} % increments the counter minipagecount by one.
\noindent{(\theminipagecount)}\hspace{0.1mm} % By default, LaTeX indents the first line of a new paragraph, but \noindent overrides this
% and inserts the current value of the minipagecount counter, enclosed in parentheses
\begin{minipage}[t]{0.45\textwidth} % The [t] option aligns the top of the minipage with the baseline of the surrounding text.
    \vspace{-26pt}  % moves the content of the minipage up, reducing the space between the minipage content and the preceding text.
    \raggedright %  the text lines up on the left side, but the right side will be ragged.
    \begin{align*} % The * align environment does not number the equations- Each line is aligned at the & symbol
        3x - 9 &= 9\\
        3x - 9 + \dotuline{\hspace{5mm}} &= 9 + \dotuline{\hspace{5mm}}\\
        3x &= \dotuline{\hspace{5mm}}\\
        \frac{3x}{\dotuline{\hspace{5mm}}} &= \frac{\dotuline{\hspace{5mm}}}{\dotuline{\hspace{5mm}}}\\
        x &= \dotuline{\hspace{5mm}}\\
    \end{align*}
\end{minipage}\refstepcounter{minipagecount} % increments the counter minipagecount by one.
\noindent{(\theminipagecount)}\hspace{0.1mm} % By default, LaTeX indents the first line of a new paragraph, but \noindent overrides this
% and inserts the current value of the minipagecount counter, enclosed in parentheses
\begin{minipage}[t]{0.45\textwidth} % The [t] option aligns the top of the minipage with the baseline of the surrounding text.
    \vspace{-26pt}  % moves the content of the minipage up, reducing the space between the minipage content and the preceding text.
    \raggedright %  the text lines up on the left side, but the right side will be ragged.
    \begin{align*} % The * align environment does not number the equations- Each line is aligned at the & symbol
        5x - 3 &= 7\\
        5x - 3 + \dotuline{\hspace{5mm}} &= 7 + \dotuline{\hspace{5mm}}\\
        5x &= \dotuline{\hspace{5mm}}\\
        \frac{5x}{\dotuline{\hspace{5mm}}} &= \frac{\dotuline{\hspace{5mm}}}{\dotuline{\hspace{5mm}}}\\
        x &= \dotuline{\hspace{5mm}}\\
    \end{align*}
\end{minipage}\columnbreak
    \refstepcounter{minipagecount} % increments the counter minipagecount by one.
\noindent{(\theminipagecount)}\hspace{0.1mm} % By default, LaTeX indents the first line of a new paragraph, but \noindent overrides this
% and inserts the current value of the minipagecount counter, enclosed in parentheses
\begin{minipage}[t]{0.45\textwidth} % The [t] option aligns the top of the minipage with the baseline of the surrounding text.
    \vspace{-26pt}  % moves the content of the minipage up, reducing the space between the minipage content and the preceding text.
    \raggedright %  the text lines up on the left side, but the right side will be ragged.
    \begin{align*} % The * align environment does not number the equations- Each line is aligned at the & symbol
        7x - 5 &= 58\\
        7x - 5 + \dotuline{\hspace{5mm}} &= 58 + \dotuline{\hspace{5mm}}\\
        7x &= \dotuline{\hspace{5mm}}\\
        \frac{7x}{\dotuline{\hspace{5mm}}} &= \frac{\dotuline{\hspace{5mm}}}{\dotuline{\hspace{5mm}}}\\
        x &= \dotuline{\hspace{5mm}}\\
    \end{align*}
\end{minipage}\refstepcounter{minipagecount} % increments the counter minipagecount by one.
\noindent{(\theminipagecount)}\hspace{0.1mm} % By default, LaTeX indents the first line of a new paragraph, but \noindent overrides this
% and inserts the current value of the minipagecount counter, enclosed in parentheses
\begin{minipage}[t]{0.45\textwidth} % The [t] option aligns the top of the minipage with the baseline of the surrounding text.
    \vspace{-26pt}  % moves the content of the minipage up, reducing the space between the minipage content and the preceding text.
    \raggedright %  the text lines up on the left side, but the right side will be ragged.
    \begin{align*} % The * align environment does not number the equations- Each line is aligned at the & symbol
        2x - 5 &= 1\\
        2x - 5 + \dotuline{\hspace{5mm}} &= 1 + \dotuline{\hspace{5mm}}\\
        2x &= \dotuline{\hspace{5mm}}\\
        \frac{2x}{\dotuline{\hspace{5mm}}} &= \frac{\dotuline{\hspace{5mm}}}{\dotuline{\hspace{5mm}}}\\
        x &= \dotuline{\hspace{5mm}}\\
    \end{align*}
\end{minipage}\refstepcounter{minipagecount} % increments the counter minipagecount by one.
\noindent{(\theminipagecount)}\hspace{0.1mm} % By default, LaTeX indents the first line of a new paragraph, but \noindent overrides this
% and inserts the current value of the minipagecount counter, enclosed in parentheses
\begin{minipage}[t]{0.45\textwidth} % The [t] option aligns the top of the minipage with the baseline of the surrounding text.
    \vspace{-26pt}  % moves the content of the minipage up, reducing the space between the minipage content and the preceding text.
    \raggedright %  the text lines up on the left side, but the right side will be ragged.
    \begin{align*} % The * align environment does not number the equations- Each line is aligned at the & symbol
        3x - 8 &= 10\\
        3x - 8 + \dotuline{\hspace{5mm}} &= 10 + \dotuline{\hspace{5mm}}\\
        3x &= \dotuline{\hspace{5mm}}\\
        \frac{3x}{\dotuline{\hspace{5mm}}} &= \frac{\dotuline{\hspace{5mm}}}{\dotuline{\hspace{5mm}}}\\
        x &= \dotuline{\hspace{5mm}}\\
    \end{align*}
\end{minipage}\refstepcounter{minipagecount} % increments the counter minipagecount by one.
\noindent{(\theminipagecount)}\hspace{0.1mm} % By default, LaTeX indents the first line of a new paragraph, but \noindent overrides this
% and inserts the current value of the minipagecount counter, enclosed in parentheses
\begin{minipage}[t]{0.45\textwidth} % The [t] option aligns the top of the minipage with the baseline of the surrounding text.
    \vspace{-26pt}  % moves the content of the minipage up, reducing the space between the minipage content and the preceding text.
    \raggedright %  the text lines up on the left side, but the right side will be ragged.
    \begin{align*} % The * align environment does not number the equations- Each line is aligned at the & symbol
        6x - 1 &= 23\\
        6x - 1 + \dotuline{\hspace{5mm}} &= 23 + \dotuline{\hspace{5mm}}\\
        6x &= \dotuline{\hspace{5mm}}\\
        \frac{6x}{\dotuline{\hspace{5mm}}} &= \frac{\dotuline{\hspace{5mm}}}{\dotuline{\hspace{5mm}}}\\
        x &= \dotuline{\hspace{5mm}}\\
    \end{align*}
\end{minipage}\refstepcounter{minipagecount} % increments the counter minipagecount by one.
\noindent{(\theminipagecount)}\hspace{0.1mm} % By default, LaTeX indents the first line of a new paragraph, but \noindent overrides this
% and inserts the current value of the minipagecount counter, enclosed in parentheses
\begin{minipage}[t]{0.45\textwidth} % The [t] option aligns the top of the minipage with the baseline of the surrounding text.
    \vspace{-26pt}  % moves the content of the minipage up, reducing the space between the minipage content and the preceding text.
    \raggedright %  the text lines up on the left side, but the right side will be ragged.
    \begin{align*} % The * align environment does not number the equations- Each line is aligned at the & symbol
        5x - 4 &= 11\\
        5x - 4 + \dotuline{\hspace{5mm}} &= 11 + \dotuline{\hspace{5mm}}\\
        5x &= \dotuline{\hspace{5mm}}\\
        \frac{5x}{\dotuline{\hspace{5mm}}} &= \frac{\dotuline{\hspace{5mm}}}{\dotuline{\hspace{5mm}}}\\
        x &= \dotuline{\hspace{5mm}}\\
    \end{align*}
\end{minipage}\newpage
    \refstepcounter{minipagecount} % increments the counter minipagecount by one.
\noindent{(\theminipagecount)}\hspace{0.1mm} % By default, LaTeX indents the first line of a new paragraph, but \noindent overrides this
% and inserts the current value of the minipagecount counter, enclosed in parentheses
\begin{minipage}[t]{0.45\textwidth} % The [t] option aligns the top of the minipage with the baseline of the surrounding text.
    \vspace{-26pt}  % moves the content of the minipage up, reducing the space between the minipage content and the preceding text.
    \raggedright %  the text lines up on the left side, but the right side will be ragged.
    \begin{align*} % The * align environment does not number the equations- Each line is aligned at the & symbol
        8x - 1 &= 23\\
        8x - 1 + \dotuline{\hspace{5mm}} &= 23 + \dotuline{\hspace{5mm}}\\
        8x &= \dotuline{\hspace{5mm}}\\
        \frac{8x}{\dotuline{\hspace{5mm}}} &= \frac{\dotuline{\hspace{5mm}}}{\dotuline{\hspace{5mm}}}\\
        x &= \dotuline{\hspace{5mm}}\\
    \end{align*}
\end{minipage}\refstepcounter{minipagecount} % increments the counter minipagecount by one.
\noindent{(\theminipagecount)}\hspace{0.1mm} % By default, LaTeX indents the first line of a new paragraph, but \noindent overrides this
% and inserts the current value of the minipagecount counter, enclosed in parentheses
\begin{minipage}[t]{0.45\textwidth} % The [t] option aligns the top of the minipage with the baseline of the surrounding text.
    \vspace{-26pt}  % moves the content of the minipage up, reducing the space between the minipage content and the preceding text.
    \raggedright %  the text lines up on the left side, but the right side will be ragged.
    \begin{align*} % The * align environment does not number the equations- Each line is aligned at the & symbol
        10x - 4 &= 36\\
        10x - 4 + \dotuline{\hspace{5mm}} &= 36 + \dotuline{\hspace{5mm}}\\
        10x &= \dotuline{\hspace{5mm}}\\
        \frac{10x}{\dotuline{\hspace{5mm}}} &= \frac{\dotuline{\hspace{5mm}}}{\dotuline{\hspace{5mm}}}\\
        x &= \dotuline{\hspace{5mm}}\\
    \end{align*}
\end{minipage}\refstepcounter{minipagecount} % increments the counter minipagecount by one.
\noindent{(\theminipagecount)}\hspace{0.1mm} % By default, LaTeX indents the first line of a new paragraph, but \noindent overrides this
% and inserts the current value of the minipagecount counter, enclosed in parentheses
\begin{minipage}[t]{0.45\textwidth} % The [t] option aligns the top of the minipage with the baseline of the surrounding text.
    \vspace{-26pt}  % moves the content of the minipage up, reducing the space between the minipage content and the preceding text.
    \raggedright %  the text lines up on the left side, but the right side will be ragged.
    \begin{align*} % The * align environment does not number the equations- Each line is aligned at the & symbol
        9x - 2 &= 52\\
        9x - 2 + \dotuline{\hspace{5mm}} &= 52 + \dotuline{\hspace{5mm}}\\
        9x &= \dotuline{\hspace{5mm}}\\
        \frac{9x}{\dotuline{\hspace{5mm}}} &= \frac{\dotuline{\hspace{5mm}}}{\dotuline{\hspace{5mm}}}\\
        x &= \dotuline{\hspace{5mm}}\\
    \end{align*}
\end{minipage}\refstepcounter{minipagecount} % increments the counter minipagecount by one.
\noindent{(\theminipagecount)}\hspace{0.1mm} % By default, LaTeX indents the first line of a new paragraph, but \noindent overrides this
% and inserts the current value of the minipagecount counter, enclosed in parentheses
\begin{minipage}[t]{0.45\textwidth} % The [t] option aligns the top of the minipage with the baseline of the surrounding text.
    \vspace{-26pt}  % moves the content of the minipage up, reducing the space between the minipage content and the preceding text.
    \raggedright %  the text lines up on the left side, but the right side will be ragged.
    \begin{align*} % The * align environment does not number the equations- Each line is aligned at the & symbol
        9x - 3 &= 60\\
        9x - 3 + \dotuline{\hspace{5mm}} &= 60 + \dotuline{\hspace{5mm}}\\
        9x &= \dotuline{\hspace{5mm}}\\
        \frac{9x}{\dotuline{\hspace{5mm}}} &= \frac{\dotuline{\hspace{5mm}}}{\dotuline{\hspace{5mm}}}\\
        x &= \dotuline{\hspace{5mm}}\\
    \end{align*}
\end{minipage}\refstepcounter{minipagecount} % increments the counter minipagecount by one.
\noindent{(\theminipagecount)}\hspace{0.1mm} % By default, LaTeX indents the first line of a new paragraph, but \noindent overrides this
% and inserts the current value of the minipagecount counter, enclosed in parentheses
\begin{minipage}[t]{0.45\textwidth} % The [t] option aligns the top of the minipage with the baseline of the surrounding text.
    \vspace{-26pt}  % moves the content of the minipage up, reducing the space between the minipage content and the preceding text.
    \raggedright %  the text lines up on the left side, but the right side will be ragged.
    \begin{align*} % The * align environment does not number the equations- Each line is aligned at the & symbol
        4x - 5 &= 23\\
        4x - 5 + \dotuline{\hspace{5mm}} &= 23 + \dotuline{\hspace{5mm}}\\
        4x &= \dotuline{\hspace{5mm}}\\
        \frac{4x}{\dotuline{\hspace{5mm}}} &= \frac{\dotuline{\hspace{5mm}}}{\dotuline{\hspace{5mm}}}\\
        x &= \dotuline{\hspace{5mm}}\\
    \end{align*}
\end{minipage}\columnbreak
    \refstepcounter{minipagecount} % increments the counter minipagecount by one.
\noindent{(\theminipagecount)}\hspace{0.1mm} % By default, LaTeX indents the first line of a new paragraph, but \noindent overrides this
% and inserts the current value of the minipagecount counter, enclosed in parentheses
\begin{minipage}[t]{0.45\textwidth} % The [t] option aligns the top of the minipage with the baseline of the surrounding text.
    \vspace{-26pt}  % moves the content of the minipage up, reducing the space between the minipage content and the preceding text.
    \raggedright %  the text lines up on the left side, but the right side will be ragged.
    \begin{align*} % The * align environment does not number the equations- Each line is aligned at the & symbol
        7x - 6 &= 57\\
        7x - 6 + \dotuline{\hspace{5mm}} &= 57 + \dotuline{\hspace{5mm}}\\
        7x &= \dotuline{\hspace{5mm}}\\
        \frac{7x}{\dotuline{\hspace{5mm}}} &= \frac{\dotuline{\hspace{5mm}}}{\dotuline{\hspace{5mm}}}\\
        x &= \dotuline{\hspace{5mm}}\\
    \end{align*}
\end{minipage}\refstepcounter{minipagecount} % increments the counter minipagecount by one.
\noindent{(\theminipagecount)}\hspace{0.1mm} % By default, LaTeX indents the first line of a new paragraph, but \noindent overrides this
% and inserts the current value of the minipagecount counter, enclosed in parentheses
\begin{minipage}[t]{0.45\textwidth} % The [t] option aligns the top of the minipage with the baseline of the surrounding text.
    \vspace{-26pt}  % moves the content of the minipage up, reducing the space between the minipage content and the preceding text.
    \raggedright %  the text lines up on the left side, but the right side will be ragged.
    \begin{align*} % The * align environment does not number the equations- Each line is aligned at the & symbol
        6x - 3 &= 33\\
        6x - 3 + \dotuline{\hspace{5mm}} &= 33 + \dotuline{\hspace{5mm}}\\
        6x &= \dotuline{\hspace{5mm}}\\
        \frac{6x}{\dotuline{\hspace{5mm}}} &= \frac{\dotuline{\hspace{5mm}}}{\dotuline{\hspace{5mm}}}\\
        x &= \dotuline{\hspace{5mm}}\\
    \end{align*}
\end{minipage}\refstepcounter{minipagecount} % increments the counter minipagecount by one.
\noindent{(\theminipagecount)}\hspace{0.1mm} % By default, LaTeX indents the first line of a new paragraph, but \noindent overrides this
% and inserts the current value of the minipagecount counter, enclosed in parentheses
\begin{minipage}[t]{0.45\textwidth} % The [t] option aligns the top of the minipage with the baseline of the surrounding text.
    \vspace{-26pt}  % moves the content of the minipage up, reducing the space between the minipage content and the preceding text.
    \raggedright %  the text lines up on the left side, but the right side will be ragged.
    \begin{align*} % The * align environment does not number the equations- Each line is aligned at the & symbol
        2x - 1 &= 13\\
        2x - 1 + \dotuline{\hspace{5mm}} &= 13 + \dotuline{\hspace{5mm}}\\
        2x &= \dotuline{\hspace{5mm}}\\
        \frac{2x}{\dotuline{\hspace{5mm}}} &= \frac{\dotuline{\hspace{5mm}}}{\dotuline{\hspace{5mm}}}\\
        x &= \dotuline{\hspace{5mm}}\\
    \end{align*}
\end{minipage}\refstepcounter{minipagecount} % increments the counter minipagecount by one.
\noindent{(\theminipagecount)}\hspace{0.1mm} % By default, LaTeX indents the first line of a new paragraph, but \noindent overrides this
% and inserts the current value of the minipagecount counter, enclosed in parentheses
\begin{minipage}[t]{0.45\textwidth} % The [t] option aligns the top of the minipage with the baseline of the surrounding text.
    \vspace{-26pt}  % moves the content of the minipage up, reducing the space between the minipage content and the preceding text.
    \raggedright %  the text lines up on the left side, but the right side will be ragged.
    \begin{align*} % The * align environment does not number the equations- Each line is aligned at the & symbol
        6x - 8 &= 10\\
        6x - 8 + \dotuline{\hspace{5mm}} &= 10 + \dotuline{\hspace{5mm}}\\
        6x &= \dotuline{\hspace{5mm}}\\
        \frac{6x}{\dotuline{\hspace{5mm}}} &= \frac{\dotuline{\hspace{5mm}}}{\dotuline{\hspace{5mm}}}\\
        x &= \dotuline{\hspace{5mm}}\\
    \end{align*}
\end{minipage}\refstepcounter{minipagecount} % increments the counter minipagecount by one.
\noindent{(\theminipagecount)}\hspace{0.1mm} % By default, LaTeX indents the first line of a new paragraph, but \noindent overrides this
% and inserts the current value of the minipagecount counter, enclosed in parentheses
\begin{minipage}[t]{0.45\textwidth} % The [t] option aligns the top of the minipage with the baseline of the surrounding text.
    \vspace{-26pt}  % moves the content of the minipage up, reducing the space between the minipage content and the preceding text.
    \raggedright %  the text lines up on the left side, but the right side will be ragged.
    \begin{align*} % The * align environment does not number the equations- Each line is aligned at the & symbol
        4x - 7 &= -3\\
        4x - 7 + \dotuline{\hspace{5mm}} &= -3 + \dotuline{\hspace{5mm}}\\
        4x &= \dotuline{\hspace{5mm}}\\
        \frac{4x}{\dotuline{\hspace{5mm}}} &= \frac{\dotuline{\hspace{5mm}}}{\dotuline{\hspace{5mm}}}\\
        x &= \dotuline{\hspace{5mm}}\\
    \end{align*}
\end{minipage}\newpage

\end{multicols}
\end{document}
