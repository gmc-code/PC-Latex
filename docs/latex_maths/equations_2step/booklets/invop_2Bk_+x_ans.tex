\documentclass[12pt]{article}
\usepackage{tikz}
\usepackage{amsmath}
% Underlining package
\usepackage[normalem]{ulem} % [normalem] prevents the package from changing the default behavior of `\\emph` to underline.
\usepackage[a4paper, portrait, margin=1cm]{geometry}
\usepackage{multicol}
\usepackage{fancyhdr}

\def \HeadingAnswers {\section*{\Large Name: \underline{\hspace{8cm}} \hfill Date: \underline{\hspace{3cm}}} \vspace{-3mm}
{Inverse operations: Answers} \vspace{1pt}\hrule}

% raise footer with page number; no header
\fancypagestyle{myfancypagestyle}{
  \fancyhf{}% clear all header and footer fields
  \renewcommand{\headrulewidth}{0pt} % no rule under header
  \fancyfoot[C] {\thepage} \setlength{\footskip}{14.5pt} % raise page number 6pt
}
\pagestyle{myfancypagestyle}  % apply myfancypagestyle
\newcounter{minipagecount}
\begin{document}
\HeadingAnswers
\vspace{1mm}
\begin{multicols}{2}
  \refstepcounter{minipagecount} % increments the counter minipagecount by one.
\noindent{(\theminipagecount)}\hspace{0.1mm} % By default, LaTeX indents the first line of a new paragraph, but \noindent overrides this
% and inserts the current value of the minipagecount counter, enclosed in parentheses
\begin{minipage}[t]{0.45\textwidth} % The [t] option aligns the top of the minipage with the baseline of the surrounding text.
    \vspace{-26pt}  % moves the content of the minipage up, reducing the space between the minipage content and the preceding text.
    \raggedright %  the text lines up on the left side, but the right side will be ragged.
    \begin{align*} % The * align environment does not number the equations- Each line is aligned at the & symbol
        5(x + 5) &= 60\\
        \frac{5(x+5)}{5} &= \frac{60}{5}\\
        x + 5 &= 12\\
        x + 5 - 5 &= 12 - 5\\
        x &= 7\\
    \end{align*}
\end{minipage}\refstepcounter{minipagecount} % increments the counter minipagecount by one.
\noindent{(\theminipagecount)}\hspace{0.1mm} % By default, LaTeX indents the first line of a new paragraph, but \noindent overrides this
% and inserts the current value of the minipagecount counter, enclosed in parentheses
\begin{minipage}[t]{0.45\textwidth} % The [t] option aligns the top of the minipage with the baseline of the surrounding text.
    \vspace{-26pt}  % moves the content of the minipage up, reducing the space between the minipage content and the preceding text.
    \raggedright %  the text lines up on the left side, but the right side will be ragged.
    \begin{align*} % The * align environment does not number the equations- Each line is aligned at the & symbol
        8(x + 2) &= 72\\
        \frac{8(x+2)}{8} &= \frac{72}{8}\\
        x + 2 &= 9\\
        x + 2 - 2 &= 9 - 2\\
        x &= 7\\
    \end{align*}
\end{minipage}\refstepcounter{minipagecount} % increments the counter minipagecount by one.
\noindent{(\theminipagecount)}\hspace{0.1mm} % By default, LaTeX indents the first line of a new paragraph, but \noindent overrides this
% and inserts the current value of the minipagecount counter, enclosed in parentheses
\begin{minipage}[t]{0.45\textwidth} % The [t] option aligns the top of the minipage with the baseline of the surrounding text.
    \vspace{-26pt}  % moves the content of the minipage up, reducing the space between the minipage content and the preceding text.
    \raggedright %  the text lines up on the left side, but the right side will be ragged.
    \begin{align*} % The * align environment does not number the equations- Each line is aligned at the & symbol
        6(x + 5) &= 48\\
        \frac{6(x+5)}{6} &= \frac{48}{6}\\
        x + 5 &= 8\\
        x + 5 - 5 &= 8 - 5\\
        x &= 3\\
    \end{align*}
\end{minipage}\refstepcounter{minipagecount} % increments the counter minipagecount by one.
\noindent{(\theminipagecount)}\hspace{0.1mm} % By default, LaTeX indents the first line of a new paragraph, but \noindent overrides this
% and inserts the current value of the minipagecount counter, enclosed in parentheses
\begin{minipage}[t]{0.45\textwidth} % The [t] option aligns the top of the minipage with the baseline of the surrounding text.
    \vspace{-26pt}  % moves the content of the minipage up, reducing the space between the minipage content and the preceding text.
    \raggedright %  the text lines up on the left side, but the right side will be ragged.
    \begin{align*} % The * align environment does not number the equations- Each line is aligned at the & symbol
        3(x + 9) &= 33\\
        \frac{3(x+9)}{3} &= \frac{33}{3}\\
        x + 9 &= 11\\
        x + 9 - 9 &= 11 - 9\\
        x &= 2\\
    \end{align*}
\end{minipage}\refstepcounter{minipagecount} % increments the counter minipagecount by one.
\noindent{(\theminipagecount)}\hspace{0.1mm} % By default, LaTeX indents the first line of a new paragraph, but \noindent overrides this
% and inserts the current value of the minipagecount counter, enclosed in parentheses
\begin{minipage}[t]{0.45\textwidth} % The [t] option aligns the top of the minipage with the baseline of the surrounding text.
    \vspace{-26pt}  % moves the content of the minipage up, reducing the space between the minipage content and the preceding text.
    \raggedright %  the text lines up on the left side, but the right side will be ragged.
    \begin{align*} % The * align environment does not number the equations- Each line is aligned at the & symbol
        8(x + 7) &= 72\\
        \frac{8(x+7)}{8} &= \frac{72}{8}\\
        x + 7 &= 9\\
        x + 7 - 7 &= 9 - 7\\
        x &= 2\\
    \end{align*}
\end{minipage}\columnbreak
    \refstepcounter{minipagecount} % increments the counter minipagecount by one.
\noindent{(\theminipagecount)}\hspace{0.1mm} % By default, LaTeX indents the first line of a new paragraph, but \noindent overrides this
% and inserts the current value of the minipagecount counter, enclosed in parentheses
\begin{minipage}[t]{0.45\textwidth} % The [t] option aligns the top of the minipage with the baseline of the surrounding text.
    \vspace{-26pt}  % moves the content of the minipage up, reducing the space between the minipage content and the preceding text.
    \raggedright %  the text lines up on the left side, but the right side will be ragged.
    \begin{align*} % The * align environment does not number the equations- Each line is aligned at the & symbol
        4(x + 1) &= 8\\
        \frac{4(x+1)}{4} &= \frac{8}{4}\\
        x + 1 &= 2\\
        x + 1 - 1 &= 2 - 1\\
        x &= 1\\
    \end{align*}
\end{minipage}\refstepcounter{minipagecount} % increments the counter minipagecount by one.
\noindent{(\theminipagecount)}\hspace{0.1mm} % By default, LaTeX indents the first line of a new paragraph, but \noindent overrides this
% and inserts the current value of the minipagecount counter, enclosed in parentheses
\begin{minipage}[t]{0.45\textwidth} % The [t] option aligns the top of the minipage with the baseline of the surrounding text.
    \vspace{-26pt}  % moves the content of the minipage up, reducing the space between the minipage content and the preceding text.
    \raggedright %  the text lines up on the left side, but the right side will be ragged.
    \begin{align*} % The * align environment does not number the equations- Each line is aligned at the & symbol
        9(x + 8) &= 153\\
        \frac{9(x+8)}{9} &= \frac{153}{9}\\
        x + 8 &= 17\\
        x + 8 - 8 &= 17 - 8\\
        x &= 9\\
    \end{align*}
\end{minipage}\refstepcounter{minipagecount} % increments the counter minipagecount by one.
\noindent{(\theminipagecount)}\hspace{0.1mm} % By default, LaTeX indents the first line of a new paragraph, but \noindent overrides this
% and inserts the current value of the minipagecount counter, enclosed in parentheses
\begin{minipage}[t]{0.45\textwidth} % The [t] option aligns the top of the minipage with the baseline of the surrounding text.
    \vspace{-26pt}  % moves the content of the minipage up, reducing the space between the minipage content and the preceding text.
    \raggedright %  the text lines up on the left side, but the right side will be ragged.
    \begin{align*} % The * align environment does not number the equations- Each line is aligned at the & symbol
        2(x + 6) &= 26\\
        \frac{2(x+6)}{2} &= \frac{26}{2}\\
        x + 6 &= 13\\
        x + 6 - 6 &= 13 - 6\\
        x &= 7\\
    \end{align*}
\end{minipage}\refstepcounter{minipagecount} % increments the counter minipagecount by one.
\noindent{(\theminipagecount)}\hspace{0.1mm} % By default, LaTeX indents the first line of a new paragraph, but \noindent overrides this
% and inserts the current value of the minipagecount counter, enclosed in parentheses
\begin{minipage}[t]{0.45\textwidth} % The [t] option aligns the top of the minipage with the baseline of the surrounding text.
    \vspace{-26pt}  % moves the content of the minipage up, reducing the space between the minipage content and the preceding text.
    \raggedright %  the text lines up on the left side, but the right side will be ragged.
    \begin{align*} % The * align environment does not number the equations- Each line is aligned at the & symbol
        2(x + 5) &= 20\\
        \frac{2(x+5)}{2} &= \frac{20}{2}\\
        x + 5 &= 10\\
        x + 5 - 5 &= 10 - 5\\
        x &= 5\\
    \end{align*}
\end{minipage}\refstepcounter{minipagecount} % increments the counter minipagecount by one.
\noindent{(\theminipagecount)}\hspace{0.1mm} % By default, LaTeX indents the first line of a new paragraph, but \noindent overrides this
% and inserts the current value of the minipagecount counter, enclosed in parentheses
\begin{minipage}[t]{0.45\textwidth} % The [t] option aligns the top of the minipage with the baseline of the surrounding text.
    \vspace{-26pt}  % moves the content of the minipage up, reducing the space between the minipage content and the preceding text.
    \raggedright %  the text lines up on the left side, but the right side will be ragged.
    \begin{align*} % The * align environment does not number the equations- Each line is aligned at the & symbol
        4(x + 6) &= 64\\
        \frac{4(x+6)}{4} &= \frac{64}{4}\\
        x + 6 &= 16\\
        x + 6 - 6 &= 16 - 6\\
        x &= 10\\
    \end{align*}
\end{minipage}\newpage
    \refstepcounter{minipagecount} % increments the counter minipagecount by one.
\noindent{(\theminipagecount)}\hspace{0.1mm} % By default, LaTeX indents the first line of a new paragraph, but \noindent overrides this
% and inserts the current value of the minipagecount counter, enclosed in parentheses
\begin{minipage}[t]{0.45\textwidth} % The [t] option aligns the top of the minipage with the baseline of the surrounding text.
    \vspace{-26pt}  % moves the content of the minipage up, reducing the space between the minipage content and the preceding text.
    \raggedright %  the text lines up on the left side, but the right side will be ragged.
    \begin{align*} % The * align environment does not number the equations- Each line is aligned at the & symbol
        10(x + 9) &= 190\\
        \frac{10(x+9)}{10} &= \frac{190}{10}\\
        x + 9 &= 19\\
        x + 9 - 9 &= 19 - 9\\
        x &= 10\\
    \end{align*}
\end{minipage}\refstepcounter{minipagecount} % increments the counter minipagecount by one.
\noindent{(\theminipagecount)}\hspace{0.1mm} % By default, LaTeX indents the first line of a new paragraph, but \noindent overrides this
% and inserts the current value of the minipagecount counter, enclosed in parentheses
\begin{minipage}[t]{0.45\textwidth} % The [t] option aligns the top of the minipage with the baseline of the surrounding text.
    \vspace{-26pt}  % moves the content of the minipage up, reducing the space between the minipage content and the preceding text.
    \raggedright %  the text lines up on the left side, but the right side will be ragged.
    \begin{align*} % The * align environment does not number the equations- Each line is aligned at the & symbol
        10(x + 4) &= 130\\
        \frac{10(x+4)}{10} &= \frac{130}{10}\\
        x + 4 &= 13\\
        x + 4 - 4 &= 13 - 4\\
        x &= 9\\
    \end{align*}
\end{minipage}\refstepcounter{minipagecount} % increments the counter minipagecount by one.
\noindent{(\theminipagecount)}\hspace{0.1mm} % By default, LaTeX indents the first line of a new paragraph, but \noindent overrides this
% and inserts the current value of the minipagecount counter, enclosed in parentheses
\begin{minipage}[t]{0.45\textwidth} % The [t] option aligns the top of the minipage with the baseline of the surrounding text.
    \vspace{-26pt}  % moves the content of the minipage up, reducing the space between the minipage content and the preceding text.
    \raggedright %  the text lines up on the left side, but the right side will be ragged.
    \begin{align*} % The * align environment does not number the equations- Each line is aligned at the & symbol
        5(x + 4) &= 60\\
        \frac{5(x+4)}{5} &= \frac{60}{5}\\
        x + 4 &= 12\\
        x + 4 - 4 &= 12 - 4\\
        x &= 8\\
    \end{align*}
\end{minipage}\refstepcounter{minipagecount} % increments the counter minipagecount by one.
\noindent{(\theminipagecount)}\hspace{0.1mm} % By default, LaTeX indents the first line of a new paragraph, but \noindent overrides this
% and inserts the current value of the minipagecount counter, enclosed in parentheses
\begin{minipage}[t]{0.45\textwidth} % The [t] option aligns the top of the minipage with the baseline of the surrounding text.
    \vspace{-26pt}  % moves the content of the minipage up, reducing the space between the minipage content and the preceding text.
    \raggedright %  the text lines up on the left side, but the right side will be ragged.
    \begin{align*} % The * align environment does not number the equations- Each line is aligned at the & symbol
        3(x + 10) &= 57\\
        \frac{3(x+10)}{3} &= \frac{57}{3}\\
        x + 10 &= 19\\
        x + 10 - 10 &= 19 - 10\\
        x &= 9\\
    \end{align*}
\end{minipage}\refstepcounter{minipagecount} % increments the counter minipagecount by one.
\noindent{(\theminipagecount)}\hspace{0.1mm} % By default, LaTeX indents the first line of a new paragraph, but \noindent overrides this
% and inserts the current value of the minipagecount counter, enclosed in parentheses
\begin{minipage}[t]{0.45\textwidth} % The [t] option aligns the top of the minipage with the baseline of the surrounding text.
    \vspace{-26pt}  % moves the content of the minipage up, reducing the space between the minipage content and the preceding text.
    \raggedright %  the text lines up on the left side, but the right side will be ragged.
    \begin{align*} % The * align environment does not number the equations- Each line is aligned at the & symbol
        5(x + 3) &= 30\\
        \frac{5(x+3)}{5} &= \frac{30}{5}\\
        x + 3 &= 6\\
        x + 3 - 3 &= 6 - 3\\
        x &= 3\\
    \end{align*}
\end{minipage}\columnbreak
    \refstepcounter{minipagecount} % increments the counter minipagecount by one.
\noindent{(\theminipagecount)}\hspace{0.1mm} % By default, LaTeX indents the first line of a new paragraph, but \noindent overrides this
% and inserts the current value of the minipagecount counter, enclosed in parentheses
\begin{minipage}[t]{0.45\textwidth} % The [t] option aligns the top of the minipage with the baseline of the surrounding text.
    \vspace{-26pt}  % moves the content of the minipage up, reducing the space between the minipage content and the preceding text.
    \raggedright %  the text lines up on the left side, but the right side will be ragged.
    \begin{align*} % The * align environment does not number the equations- Each line is aligned at the & symbol
        3(x + 3) &= 21\\
        \frac{3(x+3)}{3} &= \frac{21}{3}\\
        x + 3 &= 7\\
        x + 3 - 3 &= 7 - 3\\
        x &= 4\\
    \end{align*}
\end{minipage}\refstepcounter{minipagecount} % increments the counter minipagecount by one.
\noindent{(\theminipagecount)}\hspace{0.1mm} % By default, LaTeX indents the first line of a new paragraph, but \noindent overrides this
% and inserts the current value of the minipagecount counter, enclosed in parentheses
\begin{minipage}[t]{0.45\textwidth} % The [t] option aligns the top of the minipage with the baseline of the surrounding text.
    \vspace{-26pt}  % moves the content of the minipage up, reducing the space between the minipage content and the preceding text.
    \raggedright %  the text lines up on the left side, but the right side will be ragged.
    \begin{align*} % The * align environment does not number the equations- Each line is aligned at the & symbol
        10(x + 2) &= 80\\
        \frac{10(x+2)}{10} &= \frac{80}{10}\\
        x + 2 &= 8\\
        x + 2 - 2 &= 8 - 2\\
        x &= 6\\
    \end{align*}
\end{minipage}\refstepcounter{minipagecount} % increments the counter minipagecount by one.
\noindent{(\theminipagecount)}\hspace{0.1mm} % By default, LaTeX indents the first line of a new paragraph, but \noindent overrides this
% and inserts the current value of the minipagecount counter, enclosed in parentheses
\begin{minipage}[t]{0.45\textwidth} % The [t] option aligns the top of the minipage with the baseline of the surrounding text.
    \vspace{-26pt}  % moves the content of the minipage up, reducing the space between the minipage content and the preceding text.
    \raggedright %  the text lines up on the left side, but the right side will be ragged.
    \begin{align*} % The * align environment does not number the equations- Each line is aligned at the & symbol
        9(x + 7) &= 99\\
        \frac{9(x+7)}{9} &= \frac{99}{9}\\
        x + 7 &= 11\\
        x + 7 - 7 &= 11 - 7\\
        x &= 4\\
    \end{align*}
\end{minipage}\refstepcounter{minipagecount} % increments the counter minipagecount by one.
\noindent{(\theminipagecount)}\hspace{0.1mm} % By default, LaTeX indents the first line of a new paragraph, but \noindent overrides this
% and inserts the current value of the minipagecount counter, enclosed in parentheses
\begin{minipage}[t]{0.45\textwidth} % The [t] option aligns the top of the minipage with the baseline of the surrounding text.
    \vspace{-26pt}  % moves the content of the minipage up, reducing the space between the minipage content and the preceding text.
    \raggedright %  the text lines up on the left side, but the right side will be ragged.
    \begin{align*} % The * align environment does not number the equations- Each line is aligned at the & symbol
        2(x + 7) &= 30\\
        \frac{2(x+7)}{2} &= \frac{30}{2}\\
        x + 7 &= 15\\
        x + 7 - 7 &= 15 - 7\\
        x &= 8\\
    \end{align*}
\end{minipage}\refstepcounter{minipagecount} % increments the counter minipagecount by one.
\noindent{(\theminipagecount)}\hspace{0.1mm} % By default, LaTeX indents the first line of a new paragraph, but \noindent overrides this
% and inserts the current value of the minipagecount counter, enclosed in parentheses
\begin{minipage}[t]{0.45\textwidth} % The [t] option aligns the top of the minipage with the baseline of the surrounding text.
    \vspace{-26pt}  % moves the content of the minipage up, reducing the space between the minipage content and the preceding text.
    \raggedright %  the text lines up on the left side, but the right side will be ragged.
    \begin{align*} % The * align environment does not number the equations- Each line is aligned at the & symbol
        8(x + 3) &= 88\\
        \frac{8(x+3)}{8} &= \frac{88}{8}\\
        x + 3 &= 11\\
        x + 3 - 3 &= 11 - 3\\
        x &= 8\\
    \end{align*}
\end{minipage}\newpage

\end{multicols}
\end{document}
