\documentclass{article}
\usepackage{tikz}
\usepackage{geometry}
\pagestyle{empty}

% === Paper and Margin Variables ===
\newcommand{\paperwidthcm}{21.0}    % A4 width in cm
\newcommand{\paperheightcm}{29.7}   % A4 height in cm
\newcommand{\hmargin}{1.5}          % horizontal margin in cm
\newcommand{\vmargin}{1.5}          % vertical margin in cm

\geometry{
  paperwidth=\paperwidthcm cm,
  paperheight=\paperheightcm cm,
  hmargin=\hmargin cm,
  vmargin=\vmargin cm
}

% === Dot Style Variables ===
\newcommand{\dotspacing}{0.8}       % spacing between nearest neighbours (cm)
\newcommand{\dotsize}{0.8 pt}        % radius of each dot
\newcommand{\dotstyle}{fill}        % options: fill, draw, filldraw
\newcommand{\dotlinewidth}{0.4pt}   % outline thickness

% === Dot color variable ===
\newcommand{\dotblackpercent}{50}    % 0-100 for black!<percent>
\newcommand{\dotcolor}{black!\dotblackpercent}

% Helper to draw one dot
\newcommand{\drawdot}[2]{%
  \ifx\dotstyle\fill
    \fill[\dotcolor] (#1,#2) circle (\dotsize);
  \else\ifx\dotstyle\draw
    \draw[\dotcolor, line width=\dotlinewidth] (#1,#2) circle (\dotsize);
  \else
    \filldraw[fill=\dotcolor, draw=\dotcolor, line width=\dotlinewidth] (#1,#2) circle (\dotsize);
  \fi\fi
}

\begin{document}
\begin{tikzpicture}[remember picture, overlay]
  % Anchor to bottom-left of the physical page
  \begin{scope}[shift={(current page.south west)}]
    % Work in centimeters and shift to content area
    \begin{scope}[x=1cm, y=1cm, shift={(\hmargin,\vmargin)}]

      % Usable area
      \pgfmathsetmacro{\usablewidth}{\paperwidthcm - 2*\hmargin}
      \pgfmathsetmacro{\usableheight}{\paperheightcm - 2*\vmargin}

      % Convert pt to cm for padding
      \pgfmathsetmacro{\pad}{(\dotsize + 0.5*\dotlinewidth)/28.45274}

      % Clip to content area (+ tiny pad)
      \clip (-\pad,-\pad) rectangle (\usablewidth+\pad, \usableheight+\pad);

      % Loop counts
      \pgfmathtruncatemacro{\xcount}{ceil((\usablewidth+\pad)/\dotspacing)}
      \pgfmathtruncatemacro{\ycount}{ceil((\usableheight+\pad)/\dotspacing)}

      % Draw square grid of dots
      \foreach \i in {0,...,\xcount} {
        \foreach \j in {0,...,\ycount} {
          \pgfmathsetmacro{\x}{\i * \dotspacing}
          \pgfmathsetmacro{\y}{\j * \dotspacing}
          \drawdot{\x}{\y}
        }
      }

    \end{scope}
  \end{scope}
\end{tikzpicture}
\end{document}
