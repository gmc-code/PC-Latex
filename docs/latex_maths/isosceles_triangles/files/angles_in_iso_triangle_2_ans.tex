\documentclass[12pt, varwidth, border=5mm]{standalone}
\usepackage{tikz}
\usepackage{amsmath}
% Underlining package
\usepackage{ulem}
\usetikzlibrary{calc}
% \usepackage[a4paper, portrait, margin=1cm]{geometry}

\begin{document}
\section*{ }
    \begin{minipage}{0.55\textwidth}
  \begin{tikzpicture}[scale=1.5, baseline=(current bounding box.north)]

    \begin{scope}[rotate=290]
      \coordinate (A) at (0,0);
      \coordinate (B) at (3.48374949739156,0);
      \coordinate (C) at (intersection cs: first line={(A)--($(A)+(57:4cm)$)}, second line={(B)--($(B)+(180-57:4cm)$)});
      \draw (A) -- (B) -- (C) -- cycle;

      % Mark angles with arcs; double arc for equal angles
      \draw ($(A)!0.23cm!(B)$) arc [start angle=0, end angle=57, radius=0.23cm];
      \draw ($(A)!0.3cm!(B)$) arc [start angle=0, end angle=57, radius=0.3cm];
      \draw ($(B)!0.23cm!(C)$) arc [start angle=180-57, end angle=180, radius=0.23cm];
      \draw ($(B)!0.3cm!(C)$) arc [start angle=180-57, end angle=180, radius=0.3cm];
      \draw ($(C)!0.3cm!(A)$) arc [start angle=180+57, end angle=360-57, radius=0.3cm];

      % Label angles
      \node at ($(A)!-0.15cm!(B)$) {Y};
      \node at ($(B)!-0.15cm!(C)$) {X};
      \node at ($(C)!-0.15cm!(A)$) {Z};

      % Mark angles in degrees
      \coordinate (midBC) at ($(B)!0.5!(C)$);
      \node at ($(A)!0.55cm!(midBC)$) {};

      \coordinate (midAC) at ($(A)!0.5!(C)$);
      \node at ($(B)!0.55cm!(midAC)$) {$\theta^\circ$};

      \coordinate (midAB) at ($(A)!0.5!(B)$);
      \node at ($(C)!0.55cm!(midAB)$) {$66^\circ$};

      % Draw hash mark perpendicular to line AB at its midpoint
      \draw[black] ($(midAC)!1.5mm!90:(A)$)--($(midAC)!1.5mm!-90:(A)$);

      % Add hash marks on side AC
      \draw[black] ($(midBC)!1.5mm!90:(B)$)--($(midBC)!1.5mm!-90:(B)$);

    \end{scope}
  \end{tikzpicture}
\end{minipage}%
\hfill
\begin{minipage}{0.4\textwidth}
  \begin{align*}
    \text{$\theta^\circ$} &= \frac{(180^\circ - \angle \text{Z})}{2} \\
    &= \frac{(180^\circ - 66^\circ)}{2} \\
    &= \frac{114^\circ}{2} \\
    &= 57^\circ
  \end{align*}
\end{minipage}

\end{document}
