% inv op template
\documentclass[12pt, varwidth, border=5mm]{standalone}
\usepackage{amsmath}
\usepackage{amssymb}  % For \therefore symbol
\usepackage[normalem]{ulem}  % Prevents \emph from underlining text

\begin{document}
    \noindent \textbf{Determine whether \(x = 63\) is a solution to the equation \(\frac{x}{7} = 9\):}
\vspace{2pt}  % Ensure spacing between problem statement and solution

\noindent
\renewcommand{\arraystretch}{1.3} % Adjust line spacing in the aligned environment
\begin{tabular}{@{}p{0.45\linewidth}@{}p{0.45\linewidth}@{}}
    \(\begin{aligned}
        \text{LHS} &= \frac{x}{7} \\
                   &= \frac{63}{7} \\
                   &= 9
    \end{aligned}\) &
    \(\begin{aligned}
        \text{RHS} &= 9\\
                   & \\
                   &
    \end{aligned}\)
\end{tabular}
\renewcommand{\arraystretch}{1.0} % Adjust line spacing in the aligned environment
\vspace{2pt}  % Optional spacing

\noindent \(\therefore\) Since \(\text{LHS} = \text{RHS}\), \(x = 63\) is  a solution to the equation.
\end{document}
