\documentclass[12pt]{article}
\usepackage{tikz}
\usepackage{amssymb}  % For \therefore symbol
\usepackage{amsmath}
% Underlining package
\usepackage[normalem]{ulem} % [normalem] prevents the package from changing the default behavior of `\\emph` to underline.
\usepackage[a4paper, portrait, margin=1cm]{geometry}
\usepackage{multicol}
\usepackage{fancyhdr}
\usepackage[none]{hyphenat}

\def \HeadingAnswers {\section*{\Large Name: \underline{\hspace{8cm}} \hfill Date: \underline{\hspace{3cm}}} \vspace{-3mm}
{r5 Check Solution: Answers} \vspace{1pt}\hrule}

% \linespread{1.5} % Adjust line spacing factor
\raggedbottom

% raise footer with page number; no header
\fancypagestyle{myfancypagestyle}{
  \fancyhf{}% clear all header and footer fields
  \renewcommand{\headrulewidth}{0pt} % no rule under header
  \fancyfoot[C] {\thepage} \setlength{\footskip}{14.5pt} % raise page number 6pt
}
\pagestyle{myfancypagestyle}  % apply myfancypagestyle
\newcounter{minipagecount}
\begin{document}
\HeadingAnswers
\vspace{1pt}
\begin{multicols}{2}
  %  \begin{enumerate}
\refstepcounter{minipagecount} % increments the counter minipagecount by one.
\noindent{(\theminipagecount)}\hspace{0.1mm} % By default, LaTeX indents the first line of a new paragraph, but \noindent overrides this
% and inserts the current value of the minipagecount counter, enclosed in parentheses
\begin{minipage}[t]{0.40\textwidth} % The [t] option aligns the top of the minipage with the baseline of the surrounding text.

    \noindent Determine whether \(x = 16\) is a solution to the equation \(x - 6 = 10\):
    \vspace{4pt}  % Ensure spacing between problem statement and solution

    \noindent
    \renewcommand{\arraystretch}{1.3} % Adjust line spacing in the aligned environment
    \begin{tabular}{@{}p{0.60\linewidth}@{}p{0.40\linewidth}@{}}
        \(\begin{aligned}
            \text{LHS} &= x - 6 \\
                    &= 16 - 6 \\
                    &= 10 
        \end{aligned}\) &
        \(\begin{aligned}
            \text{RHS} &= 10\\
                    & \\
                    &
        \end{aligned}\)
    \end{tabular}
    \renewcommand{\arraystretch}{1.0} % Adjust line spacing in the aligned environment
    \vspace{2pt}  % Optional spacing

    \noindent \(\therefore\) Since \(\text{LHS} = \text{RHS}\), \(x = 16\) is  a solution to the equation.

\end{minipage}

 \vspace*{16pt}
%  \begin{enumerate}
\refstepcounter{minipagecount} % increments the counter minipagecount by one.
\noindent{(\theminipagecount)}\hspace{0.1mm} % By default, LaTeX indents the first line of a new paragraph, but \noindent overrides this
% and inserts the current value of the minipagecount counter, enclosed in parentheses
\begin{minipage}[t]{0.40\textwidth} % The [t] option aligns the top of the minipage with the baseline of the surrounding text.

    \noindent Determine whether \(x = 7\) is a solution to the equation \(5x = 45\):
    \vspace{4pt}  % Ensure spacing between problem statement and solution

    \noindent
    \renewcommand{\arraystretch}{1.3} % Adjust line spacing in the aligned environment
    \begin{tabular}{@{}p{0.60\linewidth}@{}p{0.40\linewidth}@{}}
        \(\begin{aligned}
            \text{LHS} &= 5x \\
                    &= 5 \times7 \\
                    &= 35 
        \end{aligned}\) &
        \(\begin{aligned}
            \text{RHS} &= 45\\
                    & \\
                    &
        \end{aligned}\)
    \end{tabular}
    \renewcommand{\arraystretch}{1.0} % Adjust line spacing in the aligned environment
    \vspace{2pt}  % Optional spacing

    \noindent \(\therefore\) Since \(\text{LHS} \neq \text{RHS}\), \(x = 7\) is not  a solution to the equation.

\end{minipage}

 \vspace*{16pt}
%  \begin{enumerate}
\refstepcounter{minipagecount} % increments the counter minipagecount by one.
\noindent{(\theminipagecount)}\hspace{0.1mm} % By default, LaTeX indents the first line of a new paragraph, but \noindent overrides this
% and inserts the current value of the minipagecount counter, enclosed in parentheses
\begin{minipage}[t]{0.40\textwidth} % The [t] option aligns the top of the minipage with the baseline of the surrounding text.

    \noindent Determine whether \(x = 16\) is a solution to the equation \(\frac{x}{4} = 2\):
    \vspace{4pt}  % Ensure spacing between problem statement and solution

    \noindent
    \renewcommand{\arraystretch}{1.3} % Adjust line spacing in the aligned environment
    \begin{tabular}{@{}p{0.60\linewidth}@{}p{0.40\linewidth}@{}}
        \(\begin{aligned}
            \text{LHS} &= \frac{x}{4} \\
                    &= \frac{16}{4} \\
                    &= 4
        \end{aligned}\) &
        \(\begin{aligned}
            \text{RHS} &= 2\\
                    & \\
                    &
        \end{aligned}\)
    \end{tabular}
    \renewcommand{\arraystretch}{1.0} % Adjust line spacing in the aligned environment
    \vspace{2pt}  % Optional spacing

    \noindent \(\therefore\) Since \(\text{LHS} \neq \text{RHS}\), \(x = 16\) is not  a solution to the equation.

\end{minipage}

 \vspace*{16pt}
%  \begin{enumerate}
\refstepcounter{minipagecount} % increments the counter minipagecount by one.
\noindent{(\theminipagecount)}\hspace{0.1mm} % By default, LaTeX indents the first line of a new paragraph, but \noindent overrides this
% and inserts the current value of the minipagecount counter, enclosed in parentheses
\begin{minipage}[t]{0.40\textwidth} % The [t] option aligns the top of the minipage with the baseline of the surrounding text.

    \noindent Determine whether \(x = 5\) is a solution to the equation \(4x = 20\):
    \vspace{4pt}  % Ensure spacing between problem statement and solution

    \noindent
    \renewcommand{\arraystretch}{1.3} % Adjust line spacing in the aligned environment
    \begin{tabular}{@{}p{0.60\linewidth}@{}p{0.40\linewidth}@{}}
        \(\begin{aligned}
            \text{LHS} &= 4x \\
                    &= 4 \times5 \\
                    &= 20 
        \end{aligned}\) &
        \(\begin{aligned}
            \text{RHS} &= 20\\
                    & \\
                    &
        \end{aligned}\)
    \end{tabular}
    \renewcommand{\arraystretch}{1.0} % Adjust line spacing in the aligned environment
    \vspace{2pt}  % Optional spacing

    \noindent \(\therefore\) Since \(\text{LHS} = \text{RHS}\), \(x = 5\) is  a solution to the equation.

\end{minipage}

 \vspace*{16pt}
%  \begin{enumerate}
\refstepcounter{minipagecount} % increments the counter minipagecount by one.
\noindent{(\theminipagecount)}\hspace{0.1mm} % By default, LaTeX indents the first line of a new paragraph, but \noindent overrides this
% and inserts the current value of the minipagecount counter, enclosed in parentheses
\begin{minipage}[t]{0.40\textwidth} % The [t] option aligns the top of the minipage with the baseline of the surrounding text.

    \noindent Determine whether \(x = 7\) is a solution to the equation \(x + 2 = 9\):
    \vspace{4pt}  % Ensure spacing between problem statement and solution

    \noindent
    \renewcommand{\arraystretch}{1.3} % Adjust line spacing in the aligned environment
    \begin{tabular}{@{}p{0.60\linewidth}@{}p{0.40\linewidth}@{}}
        \(\begin{aligned}
            \text{LHS} &= x + 2 \\
                    &= 7 + 2 \\
                    &= 9 
        \end{aligned}\) &
        \(\begin{aligned}
            \text{RHS} &= 9\\
                    & \\
                    &
        \end{aligned}\)
    \end{tabular}
    \renewcommand{\arraystretch}{1.0} % Adjust line spacing in the aligned environment
    \vspace{2pt}  % Optional spacing

    \noindent \(\therefore\) Since \(\text{LHS} = \text{RHS}\), \(x = 7\) is  a solution to the equation.

\end{minipage}

 \vspace*{16pt}
\columnbreak
    %  \begin{enumerate}
\refstepcounter{minipagecount} % increments the counter minipagecount by one.
\noindent{(\theminipagecount)}\hspace{0.1mm} % By default, LaTeX indents the first line of a new paragraph, but \noindent overrides this
% and inserts the current value of the minipagecount counter, enclosed in parentheses
\begin{minipage}[t]{0.40\textwidth} % The [t] option aligns the top of the minipage with the baseline of the surrounding text.

    \noindent Determine whether \(x = 5\) is a solution to the equation \(x + 3 = 9\):
    \vspace{4pt}  % Ensure spacing between problem statement and solution

    \noindent
    \renewcommand{\arraystretch}{1.3} % Adjust line spacing in the aligned environment
    \begin{tabular}{@{}p{0.60\linewidth}@{}p{0.40\linewidth}@{}}
        \(\begin{aligned}
            \text{LHS} &= x + 3 \\
                    &= 5 + 3 \\
                    &= 8 
        \end{aligned}\) &
        \(\begin{aligned}
            \text{RHS} &= 9\\
                    & \\
                    &
        \end{aligned}\)
    \end{tabular}
    \renewcommand{\arraystretch}{1.0} % Adjust line spacing in the aligned environment
    \vspace{2pt}  % Optional spacing

    \noindent \(\therefore\) Since \(\text{LHS} \neq \text{RHS}\), \(x = 5\) is not  a solution to the equation.

\end{minipage}

 \vspace*{16pt}
%  \begin{enumerate}
\refstepcounter{minipagecount} % increments the counter minipagecount by one.
\noindent{(\theminipagecount)}\hspace{0.1mm} % By default, LaTeX indents the first line of a new paragraph, but \noindent overrides this
% and inserts the current value of the minipagecount counter, enclosed in parentheses
\begin{minipage}[t]{0.40\textwidth} % The [t] option aligns the top of the minipage with the baseline of the surrounding text.

    \noindent Determine whether \(x = 117\) is a solution to the equation \(\frac{x}{9} = 10\):
    \vspace{4pt}  % Ensure spacing between problem statement and solution

    \noindent
    \renewcommand{\arraystretch}{1.3} % Adjust line spacing in the aligned environment
    \begin{tabular}{@{}p{0.60\linewidth}@{}p{0.40\linewidth}@{}}
        \(\begin{aligned}
            \text{LHS} &= \frac{x}{9} \\
                    &= \frac{117}{9} \\
                    &= 13
        \end{aligned}\) &
        \(\begin{aligned}
            \text{RHS} &= 10\\
                    & \\
                    &
        \end{aligned}\)
    \end{tabular}
    \renewcommand{\arraystretch}{1.0} % Adjust line spacing in the aligned environment
    \vspace{2pt}  % Optional spacing

    \noindent \(\therefore\) Since \(\text{LHS} \neq \text{RHS}\), \(x = 117\) is not  a solution to the equation.

\end{minipage}

 \vspace*{16pt}
%  \begin{enumerate}
\refstepcounter{minipagecount} % increments the counter minipagecount by one.
\noindent{(\theminipagecount)}\hspace{0.1mm} % By default, LaTeX indents the first line of a new paragraph, but \noindent overrides this
% and inserts the current value of the minipagecount counter, enclosed in parentheses
\begin{minipage}[t]{0.40\textwidth} % The [t] option aligns the top of the minipage with the baseline of the surrounding text.

    \noindent Determine whether \(x = -3\) is a solution to the equation \(x + 7 = 4\):
    \vspace{4pt}  % Ensure spacing between problem statement and solution

    \noindent
    \renewcommand{\arraystretch}{1.3} % Adjust line spacing in the aligned environment
    \begin{tabular}{@{}p{0.60\linewidth}@{}p{0.40\linewidth}@{}}
        \(\begin{aligned}
            \text{LHS} &= x + 7 \\
                    &= -3 + 7 \\
                    &= 4 
        \end{aligned}\) &
        \(\begin{aligned}
            \text{RHS} &= 4\\
                    & \\
                    &
        \end{aligned}\)
    \end{tabular}
    \renewcommand{\arraystretch}{1.0} % Adjust line spacing in the aligned environment
    \vspace{2pt}  % Optional spacing

    \noindent \(\therefore\) Since \(\text{LHS} = \text{RHS}\), \(x = -3\) is  a solution to the equation.

\end{minipage}

 \vspace*{16pt}
%  \begin{enumerate}
\refstepcounter{minipagecount} % increments the counter minipagecount by one.
\noindent{(\theminipagecount)}\hspace{0.1mm} % By default, LaTeX indents the first line of a new paragraph, but \noindent overrides this
% and inserts the current value of the minipagecount counter, enclosed in parentheses
\begin{minipage}[t]{0.40\textwidth} % The [t] option aligns the top of the minipage with the baseline of the surrounding text.

    \noindent Determine whether \(x = 10\) is a solution to the equation \(10x = 100\):
    \vspace{4pt}  % Ensure spacing between problem statement and solution

    \noindent
    \renewcommand{\arraystretch}{1.3} % Adjust line spacing in the aligned environment
    \begin{tabular}{@{}p{0.60\linewidth}@{}p{0.40\linewidth}@{}}
        \(\begin{aligned}
            \text{LHS} &= 10x \\
                    &= 10 \times10 \\
                    &= 100 
        \end{aligned}\) &
        \(\begin{aligned}
            \text{RHS} &= 100\\
                    & \\
                    &
        \end{aligned}\)
    \end{tabular}
    \renewcommand{\arraystretch}{1.0} % Adjust line spacing in the aligned environment
    \vspace{2pt}  % Optional spacing

    \noindent \(\therefore\) Since \(\text{LHS} = \text{RHS}\), \(x = 10\) is  a solution to the equation.

\end{minipage}

 \vspace*{16pt}
%  \begin{enumerate}
\refstepcounter{minipagecount} % increments the counter minipagecount by one.
\noindent{(\theminipagecount)}\hspace{0.1mm} % By default, LaTeX indents the first line of a new paragraph, but \noindent overrides this
% and inserts the current value of the minipagecount counter, enclosed in parentheses
\begin{minipage}[t]{0.40\textwidth} % The [t] option aligns the top of the minipage with the baseline of the surrounding text.

    \noindent Determine whether \(x = 28\) is a solution to the equation \(\frac{x}{4} = 7\):
    \vspace{4pt}  % Ensure spacing between problem statement and solution

    \noindent
    \renewcommand{\arraystretch}{1.3} % Adjust line spacing in the aligned environment
    \begin{tabular}{@{}p{0.60\linewidth}@{}p{0.40\linewidth}@{}}
        \(\begin{aligned}
            \text{LHS} &= \frac{x}{4} \\
                    &= \frac{28}{4} \\
                    &= 7
        \end{aligned}\) &
        \(\begin{aligned}
            \text{RHS} &= 7\\
                    & \\
                    &
        \end{aligned}\)
    \end{tabular}
    \renewcommand{\arraystretch}{1.0} % Adjust line spacing in the aligned environment
    \vspace{2pt}  % Optional spacing

    \noindent \(\therefore\) Since \(\text{LHS} = \text{RHS}\), \(x = 28\) is  a solution to the equation.

\end{minipage}

 \vspace*{16pt}
\newpage
    %  \begin{enumerate}
\refstepcounter{minipagecount} % increments the counter minipagecount by one.
\noindent{(\theminipagecount)}\hspace{0.1mm} % By default, LaTeX indents the first line of a new paragraph, but \noindent overrides this
% and inserts the current value of the minipagecount counter, enclosed in parentheses
\begin{minipage}[t]{0.40\textwidth} % The [t] option aligns the top of the minipage with the baseline of the surrounding text.

    \noindent Determine whether \(x = 16\) is a solution to the equation \(x - 10 = 6\):
    \vspace{4pt}  % Ensure spacing between problem statement and solution

    \noindent
    \renewcommand{\arraystretch}{1.3} % Adjust line spacing in the aligned environment
    \begin{tabular}{@{}p{0.60\linewidth}@{}p{0.40\linewidth}@{}}
        \(\begin{aligned}
            \text{LHS} &= x - 10 \\
                    &= 16 - 10 \\
                    &= 6 
        \end{aligned}\) &
        \(\begin{aligned}
            \text{RHS} &= 6\\
                    & \\
                    &
        \end{aligned}\)
    \end{tabular}
    \renewcommand{\arraystretch}{1.0} % Adjust line spacing in the aligned environment
    \vspace{2pt}  % Optional spacing

    \noindent \(\therefore\) Since \(\text{LHS} = \text{RHS}\), \(x = 16\) is  a solution to the equation.

\end{minipage}

 \vspace*{16pt}
%  \begin{enumerate}
\refstepcounter{minipagecount} % increments the counter minipagecount by one.
\noindent{(\theminipagecount)}\hspace{0.1mm} % By default, LaTeX indents the first line of a new paragraph, but \noindent overrides this
% and inserts the current value of the minipagecount counter, enclosed in parentheses
\begin{minipage}[t]{0.40\textwidth} % The [t] option aligns the top of the minipage with the baseline of the surrounding text.

    \noindent Determine whether \(x = 0\) is a solution to the equation \(x + 5 = 5\):
    \vspace{4pt}  % Ensure spacing between problem statement and solution

    \noindent
    \renewcommand{\arraystretch}{1.3} % Adjust line spacing in the aligned environment
    \begin{tabular}{@{}p{0.60\linewidth}@{}p{0.40\linewidth}@{}}
        \(\begin{aligned}
            \text{LHS} &= x + 5 \\
                    &= 0 + 5 \\
                    &= 5 
        \end{aligned}\) &
        \(\begin{aligned}
            \text{RHS} &= 5\\
                    & \\
                    &
        \end{aligned}\)
    \end{tabular}
    \renewcommand{\arraystretch}{1.0} % Adjust line spacing in the aligned environment
    \vspace{2pt}  % Optional spacing

    \noindent \(\therefore\) Since \(\text{LHS} = \text{RHS}\), \(x = 0\) is  a solution to the equation.

\end{minipage}

 \vspace*{16pt}
%  \begin{enumerate}
\refstepcounter{minipagecount} % increments the counter minipagecount by one.
\noindent{(\theminipagecount)}\hspace{0.1mm} % By default, LaTeX indents the first line of a new paragraph, but \noindent overrides this
% and inserts the current value of the minipagecount counter, enclosed in parentheses
\begin{minipage}[t]{0.40\textwidth} % The [t] option aligns the top of the minipage with the baseline of the surrounding text.

    \noindent Determine whether \(x = 5\) is a solution to the equation \(10x = 50\):
    \vspace{4pt}  % Ensure spacing between problem statement and solution

    \noindent
    \renewcommand{\arraystretch}{1.3} % Adjust line spacing in the aligned environment
    \begin{tabular}{@{}p{0.60\linewidth}@{}p{0.40\linewidth}@{}}
        \(\begin{aligned}
            \text{LHS} &= 10x \\
                    &= 10 \times5 \\
                    &= 50 
        \end{aligned}\) &
        \(\begin{aligned}
            \text{RHS} &= 50\\
                    & \\
                    &
        \end{aligned}\)
    \end{tabular}
    \renewcommand{\arraystretch}{1.0} % Adjust line spacing in the aligned environment
    \vspace{2pt}  % Optional spacing

    \noindent \(\therefore\) Since \(\text{LHS} = \text{RHS}\), \(x = 5\) is  a solution to the equation.

\end{minipage}

 \vspace*{16pt}
%  \begin{enumerate}
\refstepcounter{minipagecount} % increments the counter minipagecount by one.
\noindent{(\theminipagecount)}\hspace{0.1mm} % By default, LaTeX indents the first line of a new paragraph, but \noindent overrides this
% and inserts the current value of the minipagecount counter, enclosed in parentheses
\begin{minipage}[t]{0.40\textwidth} % The [t] option aligns the top of the minipage with the baseline of the surrounding text.

    \noindent Determine whether \(x = 45\) is a solution to the equation \(\frac{x}{5} = 9\):
    \vspace{4pt}  % Ensure spacing between problem statement and solution

    \noindent
    \renewcommand{\arraystretch}{1.3} % Adjust line spacing in the aligned environment
    \begin{tabular}{@{}p{0.60\linewidth}@{}p{0.40\linewidth}@{}}
        \(\begin{aligned}
            \text{LHS} &= \frac{x}{5} \\
                    &= \frac{45}{5} \\
                    &= 9
        \end{aligned}\) &
        \(\begin{aligned}
            \text{RHS} &= 9\\
                    & \\
                    &
        \end{aligned}\)
    \end{tabular}
    \renewcommand{\arraystretch}{1.0} % Adjust line spacing in the aligned environment
    \vspace{2pt}  % Optional spacing

    \noindent \(\therefore\) Since \(\text{LHS} = \text{RHS}\), \(x = 45\) is  a solution to the equation.

\end{minipage}

 \vspace*{16pt}
%  \begin{enumerate}
\refstepcounter{minipagecount} % increments the counter minipagecount by one.
\noindent{(\theminipagecount)}\hspace{0.1mm} % By default, LaTeX indents the first line of a new paragraph, but \noindent overrides this
% and inserts the current value of the minipagecount counter, enclosed in parentheses
\begin{minipage}[t]{0.40\textwidth} % The [t] option aligns the top of the minipage with the baseline of the surrounding text.

    \noindent Determine whether \(x = 4\) is a solution to the equation \(10x = 40\):
    \vspace{4pt}  % Ensure spacing between problem statement and solution

    \noindent
    \renewcommand{\arraystretch}{1.3} % Adjust line spacing in the aligned environment
    \begin{tabular}{@{}p{0.60\linewidth}@{}p{0.40\linewidth}@{}}
        \(\begin{aligned}
            \text{LHS} &= 10x \\
                    &= 10 \times4 \\
                    &= 40 
        \end{aligned}\) &
        \(\begin{aligned}
            \text{RHS} &= 40\\
                    & \\
                    &
        \end{aligned}\)
    \end{tabular}
    \renewcommand{\arraystretch}{1.0} % Adjust line spacing in the aligned environment
    \vspace{2pt}  % Optional spacing

    \noindent \(\therefore\) Since \(\text{LHS} = \text{RHS}\), \(x = 4\) is  a solution to the equation.

\end{minipage}

 \vspace*{16pt}
\columnbreak
    %  \begin{enumerate}
\refstepcounter{minipagecount} % increments the counter minipagecount by one.
\noindent{(\theminipagecount)}\hspace{0.1mm} % By default, LaTeX indents the first line of a new paragraph, but \noindent overrides this
% and inserts the current value of the minipagecount counter, enclosed in parentheses
\begin{minipage}[t]{0.40\textwidth} % The [t] option aligns the top of the minipage with the baseline of the surrounding text.

    \noindent Determine whether \(x = 13\) is a solution to the equation \(x - 1 = 9\):
    \vspace{4pt}  % Ensure spacing between problem statement and solution

    \noindent
    \renewcommand{\arraystretch}{1.3} % Adjust line spacing in the aligned environment
    \begin{tabular}{@{}p{0.60\linewidth}@{}p{0.40\linewidth}@{}}
        \(\begin{aligned}
            \text{LHS} &= x - 1 \\
                    &= 13 - 1 \\
                    &= 12 
        \end{aligned}\) &
        \(\begin{aligned}
            \text{RHS} &= 9\\
                    & \\
                    &
        \end{aligned}\)
    \end{tabular}
    \renewcommand{\arraystretch}{1.0} % Adjust line spacing in the aligned environment
    \vspace{2pt}  % Optional spacing

    \noindent \(\therefore\) Since \(\text{LHS} \neq \text{RHS}\), \(x = 13\) is not  a solution to the equation.

\end{minipage}

 \vspace*{16pt}
%  \begin{enumerate}
\refstepcounter{minipagecount} % increments the counter minipagecount by one.
\noindent{(\theminipagecount)}\hspace{0.1mm} % By default, LaTeX indents the first line of a new paragraph, but \noindent overrides this
% and inserts the current value of the minipagecount counter, enclosed in parentheses
\begin{minipage}[t]{0.40\textwidth} % The [t] option aligns the top of the minipage with the baseline of the surrounding text.

    \noindent Determine whether \(x = 6\) is a solution to the equation \(\frac{x}{3} = 2\):
    \vspace{4pt}  % Ensure spacing between problem statement and solution

    \noindent
    \renewcommand{\arraystretch}{1.3} % Adjust line spacing in the aligned environment
    \begin{tabular}{@{}p{0.60\linewidth}@{}p{0.40\linewidth}@{}}
        \(\begin{aligned}
            \text{LHS} &= \frac{x}{3} \\
                    &= \frac{6}{3} \\
                    &= 2
        \end{aligned}\) &
        \(\begin{aligned}
            \text{RHS} &= 2\\
                    & \\
                    &
        \end{aligned}\)
    \end{tabular}
    \renewcommand{\arraystretch}{1.0} % Adjust line spacing in the aligned environment
    \vspace{2pt}  % Optional spacing

    \noindent \(\therefore\) Since \(\text{LHS} = \text{RHS}\), \(x = 6\) is  a solution to the equation.

\end{minipage}

 \vspace*{16pt}
%  \begin{enumerate}
\refstepcounter{minipagecount} % increments the counter minipagecount by one.
\noindent{(\theminipagecount)}\hspace{0.1mm} % By default, LaTeX indents the first line of a new paragraph, but \noindent overrides this
% and inserts the current value of the minipagecount counter, enclosed in parentheses
\begin{minipage}[t]{0.40\textwidth} % The [t] option aligns the top of the minipage with the baseline of the surrounding text.

    \noindent Determine whether \(x = 2\) is a solution to the equation \(x - 1 = 1\):
    \vspace{4pt}  % Ensure spacing between problem statement and solution

    \noindent
    \renewcommand{\arraystretch}{1.3} % Adjust line spacing in the aligned environment
    \begin{tabular}{@{}p{0.60\linewidth}@{}p{0.40\linewidth}@{}}
        \(\begin{aligned}
            \text{LHS} &= x - 1 \\
                    &= 2 - 1 \\
                    &= 1 
        \end{aligned}\) &
        \(\begin{aligned}
            \text{RHS} &= 1\\
                    & \\
                    &
        \end{aligned}\)
    \end{tabular}
    \renewcommand{\arraystretch}{1.0} % Adjust line spacing in the aligned environment
    \vspace{2pt}  % Optional spacing

    \noindent \(\therefore\) Since \(\text{LHS} = \text{RHS}\), \(x = 2\) is  a solution to the equation.

\end{minipage}

 \vspace*{16pt}
%  \begin{enumerate}
\refstepcounter{minipagecount} % increments the counter minipagecount by one.
\noindent{(\theminipagecount)}\hspace{0.1mm} % By default, LaTeX indents the first line of a new paragraph, but \noindent overrides this
% and inserts the current value of the minipagecount counter, enclosed in parentheses
\begin{minipage}[t]{0.40\textwidth} % The [t] option aligns the top of the minipage with the baseline of the surrounding text.

    \noindent Determine whether \(x = 12\) is a solution to the equation \(\frac{x}{6} = 3\):
    \vspace{4pt}  % Ensure spacing between problem statement and solution

    \noindent
    \renewcommand{\arraystretch}{1.3} % Adjust line spacing in the aligned environment
    \begin{tabular}{@{}p{0.60\linewidth}@{}p{0.40\linewidth}@{}}
        \(\begin{aligned}
            \text{LHS} &= \frac{x}{6} \\
                    &= \frac{12}{6} \\
                    &= 2
        \end{aligned}\) &
        \(\begin{aligned}
            \text{RHS} &= 3\\
                    & \\
                    &
        \end{aligned}\)
    \end{tabular}
    \renewcommand{\arraystretch}{1.0} % Adjust line spacing in the aligned environment
    \vspace{2pt}  % Optional spacing

    \noindent \(\therefore\) Since \(\text{LHS} \neq \text{RHS}\), \(x = 12\) is not  a solution to the equation.

\end{minipage}

 \vspace*{16pt}
%  \begin{enumerate}
\refstepcounter{minipagecount} % increments the counter minipagecount by one.
\noindent{(\theminipagecount)}\hspace{0.1mm} % By default, LaTeX indents the first line of a new paragraph, but \noindent overrides this
% and inserts the current value of the minipagecount counter, enclosed in parentheses
\begin{minipage}[t]{0.40\textwidth} % The [t] option aligns the top of the minipage with the baseline of the surrounding text.

    \noindent Determine whether \(x = 8\) is a solution to the equation \(x - 1 = 7\):
    \vspace{4pt}  % Ensure spacing between problem statement and solution

    \noindent
    \renewcommand{\arraystretch}{1.3} % Adjust line spacing in the aligned environment
    \begin{tabular}{@{}p{0.60\linewidth}@{}p{0.40\linewidth}@{}}
        \(\begin{aligned}
            \text{LHS} &= x - 1 \\
                    &= 8 - 1 \\
                    &= 7 
        \end{aligned}\) &
        \(\begin{aligned}
            \text{RHS} &= 7\\
                    & \\
                    &
        \end{aligned}\)
    \end{tabular}
    \renewcommand{\arraystretch}{1.0} % Adjust line spacing in the aligned environment
    \vspace{2pt}  % Optional spacing

    \noindent \(\therefore\) Since \(\text{LHS} = \text{RHS}\), \(x = 8\) is  a solution to the equation.

\end{minipage}

 \vspace*{16pt}
\newpage
    
\end{multicols}
\end{document}
