\documentclass[12pt]{article}
\usepackage{tikz}
\usepackage{amssymb}  % For \therefore symbol
\usepackage{amsmath}
% Underlining package
\usepackage[normalem]{ulem} % [normalem] prevents the package from changing the default behavior of `\\emph` to underline.
\usepackage[a4paper, portrait, margin=1cm]{geometry}
\usepackage{multicol}
\usepackage{fancyhdr}

\def \HeadingAnswers {\section*{\Large Name: \underline{\hspace{8cm}} \hfill Date: \underline{\hspace{3cm}}} \vspace{-3mm}
{Check Solution: Answers} \vspace{1pt}\hrule}

% \linespread{1.5} % Adjust line spacing factor
\raggedbottom

% raise footer with page number; no header
\fancypagestyle{myfancypagestyle}{
  \fancyhf{}% clear all header and footer fields
  \renewcommand{\headrulewidth}{0pt} % no rule under header
  \fancyfoot[C] {\thepage} \setlength{\footskip}{14.5pt} % raise page number 6pt
}
\pagestyle{myfancypagestyle}  % apply myfancypagestyle
\newcounter{minipagecount}
\begin{document}
\HeadingAnswers
\vspace{4mm}
\begin{multicols}{2}
  %  \begin{enumerate}
\refstepcounter{minipagecount} % increments the counter minipagecount by one.
\noindent{(\theminipagecount)}\hspace{0.1mm} % By default, LaTeX indents the first line of a new paragraph, but \noindent overrides this
% and inserts the current value of the minipagecount counter, enclosed in parentheses
\begin{minipage}[t]{0.40\textwidth} % The [t] option aligns the top of the minipage with the baseline of the surrounding text.

    \noindent Determine whether \(x = 32\) is a solution to the equation \(\frac{x}{4} = 8\):
    \vspace{2pt}  % Ensure spacing between problem statement and solution

    \noindent
    \renewcommand{\arraystretch}{1.3} % Adjust line spacing in the aligned environment
    \begin{tabular}{@{}p{0.60\linewidth}@{}p{0.40\linewidth}@{}}
        \(\begin{aligned}
            \text{LHS} &= \frac{x}{4} \\
                    &= \frac{32}{4} \\
                    &= 8
        \end{aligned}\) &
        \(\begin{aligned}
            \text{RHS} &= 8\\
                    & \\
                    &
        \end{aligned}\)
    \end{tabular}
    \renewcommand{\arraystretch}{1.0} % Adjust line spacing in the aligned environment
    \vspace{2pt}  % Optional spacing

    \noindent \(\therefore\) Since \(\text{LHS} = \text{RHS}\), \(x = 32\) is  a solution to the equation.

\end{minipage}

\vspace*{0.5ex}
\vfill{}
%  \begin{enumerate}
\refstepcounter{minipagecount} % increments the counter minipagecount by one.
\noindent{(\theminipagecount)}\hspace{0.1mm} % By default, LaTeX indents the first line of a new paragraph, but \noindent overrides this
% and inserts the current value of the minipagecount counter, enclosed in parentheses
\begin{minipage}[t]{0.40\textwidth} % The [t] option aligns the top of the minipage with the baseline of the surrounding text.

    \noindent Determine whether \(x = 20\) is a solution to the equation \(\frac{x}{10} = 2\):
    \vspace{2pt}  % Ensure spacing between problem statement and solution

    \noindent
    \renewcommand{\arraystretch}{1.3} % Adjust line spacing in the aligned environment
    \begin{tabular}{@{}p{0.60\linewidth}@{}p{0.40\linewidth}@{}}
        \(\begin{aligned}
            \text{LHS} &= \frac{x}{10} \\
                    &= \frac{20}{10} \\
                    &= 2
        \end{aligned}\) &
        \(\begin{aligned}
            \text{RHS} &= 2\\
                    & \\
                    &
        \end{aligned}\)
    \end{tabular}
    \renewcommand{\arraystretch}{1.0} % Adjust line spacing in the aligned environment
    \vspace{2pt}  % Optional spacing

    \noindent \(\therefore\) Since \(\text{LHS} = \text{RHS}\), \(x = 20\) is  a solution to the equation.

\end{minipage}

\vspace*{0.5ex}
\vfill{}
%  \begin{enumerate}
\refstepcounter{minipagecount} % increments the counter minipagecount by one.
\noindent{(\theminipagecount)}\hspace{0.1mm} % By default, LaTeX indents the first line of a new paragraph, but \noindent overrides this
% and inserts the current value of the minipagecount counter, enclosed in parentheses
\begin{minipage}[t]{0.40\textwidth} % The [t] option aligns the top of the minipage with the baseline of the surrounding text.

    \noindent Determine whether \(x = 73\) is a solution to the equation \(\frac{x}{8} = 9\):
    \vspace{2pt}  % Ensure spacing between problem statement and solution

    \noindent
    \renewcommand{\arraystretch}{1.3} % Adjust line spacing in the aligned environment
    \begin{tabular}{@{}p{0.60\linewidth}@{}p{0.40\linewidth}@{}}
        \(\begin{aligned}
            \text{LHS} &= \frac{x}{8} \\
                    &= \frac{73}{8} \\
                    &= 9.125
        \end{aligned}\) &
        \(\begin{aligned}
            \text{RHS} &= 9\\
                    & \\
                    &
        \end{aligned}\)
    \end{tabular}
    \renewcommand{\arraystretch}{1.0} % Adjust line spacing in the aligned environment
    \vspace{2pt}  % Optional spacing

    \noindent \(\therefore\) Since \(\text{LHS} \neq \text{RHS}\), \(x = 73\) is not  a solution to the equation.

\end{minipage}

\vspace*{0.5ex}
\vfill{}
%  \begin{enumerate}
\refstepcounter{minipagecount} % increments the counter minipagecount by one.
\noindent{(\theminipagecount)}\hspace{0.1mm} % By default, LaTeX indents the first line of a new paragraph, but \noindent overrides this
% and inserts the current value of the minipagecount counter, enclosed in parentheses
\begin{minipage}[t]{0.40\textwidth} % The [t] option aligns the top of the minipage with the baseline of the surrounding text.

    \noindent Determine whether \(x = 21\) is a solution to the equation \(\frac{x}{4} = 5\):
    \vspace{2pt}  % Ensure spacing between problem statement and solution

    \noindent
    \renewcommand{\arraystretch}{1.3} % Adjust line spacing in the aligned environment
    \begin{tabular}{@{}p{0.60\linewidth}@{}p{0.40\linewidth}@{}}
        \(\begin{aligned}
            \text{LHS} &= \frac{x}{4} \\
                    &= \frac{21}{4} \\
                    &= 5.25
        \end{aligned}\) &
        \(\begin{aligned}
            \text{RHS} &= 5\\
                    & \\
                    &
        \end{aligned}\)
    \end{tabular}
    \renewcommand{\arraystretch}{1.0} % Adjust line spacing in the aligned environment
    \vspace{2pt}  % Optional spacing

    \noindent \(\therefore\) Since \(\text{LHS} \neq \text{RHS}\), \(x = 21\) is not  a solution to the equation.

\end{minipage}

\vspace*{0.5ex}
\vfill{}
%  \begin{enumerate}
\refstepcounter{minipagecount} % increments the counter minipagecount by one.
\noindent{(\theminipagecount)}\hspace{0.1mm} % By default, LaTeX indents the first line of a new paragraph, but \noindent overrides this
% and inserts the current value of the minipagecount counter, enclosed in parentheses
\begin{minipage}[t]{0.40\textwidth} % The [t] option aligns the top of the minipage with the baseline of the surrounding text.

    \noindent Determine whether \(x = 40\) is a solution to the equation \(\frac{x}{4} = 10\):
    \vspace{2pt}  % Ensure spacing between problem statement and solution

    \noindent
    \renewcommand{\arraystretch}{1.3} % Adjust line spacing in the aligned environment
    \begin{tabular}{@{}p{0.60\linewidth}@{}p{0.40\linewidth}@{}}
        \(\begin{aligned}
            \text{LHS} &= \frac{x}{4} \\
                    &= \frac{40}{4} \\
                    &= 10
        \end{aligned}\) &
        \(\begin{aligned}
            \text{RHS} &= 10\\
                    & \\
                    &
        \end{aligned}\)
    \end{tabular}
    \renewcommand{\arraystretch}{1.0} % Adjust line spacing in the aligned environment
    \vspace{2pt}  % Optional spacing

    \noindent \(\therefore\) Since \(\text{LHS} = \text{RHS}\), \(x = 40\) is  a solution to the equation.

\end{minipage}

\vspace*{0.5ex}
\vfill{}
\columnbreak
    %  \begin{enumerate}
\refstepcounter{minipagecount} % increments the counter minipagecount by one.
\noindent{(\theminipagecount)}\hspace{0.1mm} % By default, LaTeX indents the first line of a new paragraph, but \noindent overrides this
% and inserts the current value of the minipagecount counter, enclosed in parentheses
\begin{minipage}[t]{0.40\textwidth} % The [t] option aligns the top of the minipage with the baseline of the surrounding text.

    \noindent Determine whether \(x = 21\) is a solution to the equation \(\frac{x}{4} = 6\):
    \vspace{2pt}  % Ensure spacing between problem statement and solution

    \noindent
    \renewcommand{\arraystretch}{1.3} % Adjust line spacing in the aligned environment
    \begin{tabular}{@{}p{0.60\linewidth}@{}p{0.40\linewidth}@{}}
        \(\begin{aligned}
            \text{LHS} &= \frac{x}{4} \\
                    &= \frac{21}{4} \\
                    &= 5.25
        \end{aligned}\) &
        \(\begin{aligned}
            \text{RHS} &= 6\\
                    & \\
                    &
        \end{aligned}\)
    \end{tabular}
    \renewcommand{\arraystretch}{1.0} % Adjust line spacing in the aligned environment
    \vspace{2pt}  % Optional spacing

    \noindent \(\therefore\) Since \(\text{LHS} \neq \text{RHS}\), \(x = 21\) is not  a solution to the equation.

\end{minipage}

\vspace*{0.5ex}
\vfill{}
%  \begin{enumerate}
\refstepcounter{minipagecount} % increments the counter minipagecount by one.
\noindent{(\theminipagecount)}\hspace{0.1mm} % By default, LaTeX indents the first line of a new paragraph, but \noindent overrides this
% and inserts the current value of the minipagecount counter, enclosed in parentheses
\begin{minipage}[t]{0.40\textwidth} % The [t] option aligns the top of the minipage with the baseline of the surrounding text.

    \noindent Determine whether \(x = 16\) is a solution to the equation \(\frac{x}{8} = 2\):
    \vspace{2pt}  % Ensure spacing between problem statement and solution

    \noindent
    \renewcommand{\arraystretch}{1.3} % Adjust line spacing in the aligned environment
    \begin{tabular}{@{}p{0.60\linewidth}@{}p{0.40\linewidth}@{}}
        \(\begin{aligned}
            \text{LHS} &= \frac{x}{8} \\
                    &= \frac{16}{8} \\
                    &= 2
        \end{aligned}\) &
        \(\begin{aligned}
            \text{RHS} &= 2\\
                    & \\
                    &
        \end{aligned}\)
    \end{tabular}
    \renewcommand{\arraystretch}{1.0} % Adjust line spacing in the aligned environment
    \vspace{2pt}  % Optional spacing

    \noindent \(\therefore\) Since \(\text{LHS} = \text{RHS}\), \(x = 16\) is  a solution to the equation.

\end{minipage}

\vspace*{0.5ex}
\vfill{}
%  \begin{enumerate}
\refstepcounter{minipagecount} % increments the counter minipagecount by one.
\noindent{(\theminipagecount)}\hspace{0.1mm} % By default, LaTeX indents the first line of a new paragraph, but \noindent overrides this
% and inserts the current value of the minipagecount counter, enclosed in parentheses
\begin{minipage}[t]{0.40\textwidth} % The [t] option aligns the top of the minipage with the baseline of the surrounding text.

    \noindent Determine whether \(x = 9\) is a solution to the equation \(\frac{x}{3} = 3\):
    \vspace{2pt}  % Ensure spacing between problem statement and solution

    \noindent
    \renewcommand{\arraystretch}{1.3} % Adjust line spacing in the aligned environment
    \begin{tabular}{@{}p{0.60\linewidth}@{}p{0.40\linewidth}@{}}
        \(\begin{aligned}
            \text{LHS} &= \frac{x}{3} \\
                    &= \frac{9}{3} \\
                    &= 3
        \end{aligned}\) &
        \(\begin{aligned}
            \text{RHS} &= 3\\
                    & \\
                    &
        \end{aligned}\)
    \end{tabular}
    \renewcommand{\arraystretch}{1.0} % Adjust line spacing in the aligned environment
    \vspace{2pt}  % Optional spacing

    \noindent \(\therefore\) Since \(\text{LHS} = \text{RHS}\), \(x = 9\) is  a solution to the equation.

\end{minipage}

\vspace*{0.5ex}
\vfill{}
%  \begin{enumerate}
\refstepcounter{minipagecount} % increments the counter minipagecount by one.
\noindent{(\theminipagecount)}\hspace{0.1mm} % By default, LaTeX indents the first line of a new paragraph, but \noindent overrides this
% and inserts the current value of the minipagecount counter, enclosed in parentheses
\begin{minipage}[t]{0.40\textwidth} % The [t] option aligns the top of the minipage with the baseline of the surrounding text.

    \noindent Determine whether \(x = 30\) is a solution to the equation \(\frac{x}{10} = 3\):
    \vspace{2pt}  % Ensure spacing between problem statement and solution

    \noindent
    \renewcommand{\arraystretch}{1.3} % Adjust line spacing in the aligned environment
    \begin{tabular}{@{}p{0.60\linewidth}@{}p{0.40\linewidth}@{}}
        \(\begin{aligned}
            \text{LHS} &= \frac{x}{10} \\
                    &= \frac{30}{10} \\
                    &= 3
        \end{aligned}\) &
        \(\begin{aligned}
            \text{RHS} &= 3\\
                    & \\
                    &
        \end{aligned}\)
    \end{tabular}
    \renewcommand{\arraystretch}{1.0} % Adjust line spacing in the aligned environment
    \vspace{2pt}  % Optional spacing

    \noindent \(\therefore\) Since \(\text{LHS} = \text{RHS}\), \(x = 30\) is  a solution to the equation.

\end{minipage}

\vspace*{0.5ex}
\vfill{}
%  \begin{enumerate}
\refstepcounter{minipagecount} % increments the counter minipagecount by one.
\noindent{(\theminipagecount)}\hspace{0.1mm} % By default, LaTeX indents the first line of a new paragraph, but \noindent overrides this
% and inserts the current value of the minipagecount counter, enclosed in parentheses
\begin{minipage}[t]{0.40\textwidth} % The [t] option aligns the top of the minipage with the baseline of the surrounding text.

    \noindent Determine whether \(x = 21\) is a solution to the equation \(\frac{x}{3} = 7\):
    \vspace{2pt}  % Ensure spacing between problem statement and solution

    \noindent
    \renewcommand{\arraystretch}{1.3} % Adjust line spacing in the aligned environment
    \begin{tabular}{@{}p{0.60\linewidth}@{}p{0.40\linewidth}@{}}
        \(\begin{aligned}
            \text{LHS} &= \frac{x}{3} \\
                    &= \frac{21}{3} \\
                    &= 7
        \end{aligned}\) &
        \(\begin{aligned}
            \text{RHS} &= 7\\
                    & \\
                    &
        \end{aligned}\)
    \end{tabular}
    \renewcommand{\arraystretch}{1.0} % Adjust line spacing in the aligned environment
    \vspace{2pt}  % Optional spacing

    \noindent \(\therefore\) Since \(\text{LHS} = \text{RHS}\), \(x = 21\) is  a solution to the equation.

\end{minipage}

\vspace*{0.5ex}
\vfill{}
\newpage
    %  \begin{enumerate}
\refstepcounter{minipagecount} % increments the counter minipagecount by one.
\noindent{(\theminipagecount)}\hspace{0.1mm} % By default, LaTeX indents the first line of a new paragraph, but \noindent overrides this
% and inserts the current value of the minipagecount counter, enclosed in parentheses
\begin{minipage}[t]{0.40\textwidth} % The [t] option aligns the top of the minipage with the baseline of the surrounding text.

    \noindent Determine whether \(x = 25\) is a solution to the equation \(\frac{x}{9} = 3\):
    \vspace{2pt}  % Ensure spacing between problem statement and solution

    \noindent
    \renewcommand{\arraystretch}{1.3} % Adjust line spacing in the aligned environment
    \begin{tabular}{@{}p{0.60\linewidth}@{}p{0.40\linewidth}@{}}
        \(\begin{aligned}
            \text{LHS} &= \frac{x}{9} \\
                    &= \frac{25}{9} \\
                    &= 2.778
        \end{aligned}\) &
        \(\begin{aligned}
            \text{RHS} &= 3\\
                    & \\
                    &
        \end{aligned}\)
    \end{tabular}
    \renewcommand{\arraystretch}{1.0} % Adjust line spacing in the aligned environment
    \vspace{2pt}  % Optional spacing

    \noindent \(\therefore\) Since \(\text{LHS} \neq \text{RHS}\), \(x = 25\) is not  a solution to the equation.

\end{minipage}

\vspace*{0.5ex}
\vfill{}
%  \begin{enumerate}
\refstepcounter{minipagecount} % increments the counter minipagecount by one.
\noindent{(\theminipagecount)}\hspace{0.1mm} % By default, LaTeX indents the first line of a new paragraph, but \noindent overrides this
% and inserts the current value of the minipagecount counter, enclosed in parentheses
\begin{minipage}[t]{0.40\textwidth} % The [t] option aligns the top of the minipage with the baseline of the surrounding text.

    \noindent Determine whether \(x = 30\) is a solution to the equation \(\frac{x}{3} = 10\):
    \vspace{2pt}  % Ensure spacing between problem statement and solution

    \noindent
    \renewcommand{\arraystretch}{1.3} % Adjust line spacing in the aligned environment
    \begin{tabular}{@{}p{0.60\linewidth}@{}p{0.40\linewidth}@{}}
        \(\begin{aligned}
            \text{LHS} &= \frac{x}{3} \\
                    &= \frac{30}{3} \\
                    &= 10
        \end{aligned}\) &
        \(\begin{aligned}
            \text{RHS} &= 10\\
                    & \\
                    &
        \end{aligned}\)
    \end{tabular}
    \renewcommand{\arraystretch}{1.0} % Adjust line spacing in the aligned environment
    \vspace{2pt}  % Optional spacing

    \noindent \(\therefore\) Since \(\text{LHS} = \text{RHS}\), \(x = 30\) is  a solution to the equation.

\end{minipage}

\vspace*{0.5ex}
\vfill{}
%  \begin{enumerate}
\refstepcounter{minipagecount} % increments the counter minipagecount by one.
\noindent{(\theminipagecount)}\hspace{0.1mm} % By default, LaTeX indents the first line of a new paragraph, but \noindent overrides this
% and inserts the current value of the minipagecount counter, enclosed in parentheses
\begin{minipage}[t]{0.40\textwidth} % The [t] option aligns the top of the minipage with the baseline of the surrounding text.

    \noindent Determine whether \(x = 5\) is a solution to the equation \(\frac{x}{4} = 2\):
    \vspace{2pt}  % Ensure spacing between problem statement and solution

    \noindent
    \renewcommand{\arraystretch}{1.3} % Adjust line spacing in the aligned environment
    \begin{tabular}{@{}p{0.60\linewidth}@{}p{0.40\linewidth}@{}}
        \(\begin{aligned}
            \text{LHS} &= \frac{x}{4} \\
                    &= \frac{5}{4} \\
                    &= 1.25
        \end{aligned}\) &
        \(\begin{aligned}
            \text{RHS} &= 2\\
                    & \\
                    &
        \end{aligned}\)
    \end{tabular}
    \renewcommand{\arraystretch}{1.0} % Adjust line spacing in the aligned environment
    \vspace{2pt}  % Optional spacing

    \noindent \(\therefore\) Since \(\text{LHS} \neq \text{RHS}\), \(x = 5\) is not  a solution to the equation.

\end{minipage}

\vspace*{0.5ex}
\vfill{}
%  \begin{enumerate}
\refstepcounter{minipagecount} % increments the counter minipagecount by one.
\noindent{(\theminipagecount)}\hspace{0.1mm} % By default, LaTeX indents the first line of a new paragraph, but \noindent overrides this
% and inserts the current value of the minipagecount counter, enclosed in parentheses
\begin{minipage}[t]{0.40\textwidth} % The [t] option aligns the top of the minipage with the baseline of the surrounding text.

    \noindent Determine whether \(x = 54\) is a solution to the equation \(\frac{x}{9} = 6\):
    \vspace{2pt}  % Ensure spacing between problem statement and solution

    \noindent
    \renewcommand{\arraystretch}{1.3} % Adjust line spacing in the aligned environment
    \begin{tabular}{@{}p{0.60\linewidth}@{}p{0.40\linewidth}@{}}
        \(\begin{aligned}
            \text{LHS} &= \frac{x}{9} \\
                    &= \frac{54}{9} \\
                    &= 6
        \end{aligned}\) &
        \(\begin{aligned}
            \text{RHS} &= 6\\
                    & \\
                    &
        \end{aligned}\)
    \end{tabular}
    \renewcommand{\arraystretch}{1.0} % Adjust line spacing in the aligned environment
    \vspace{2pt}  % Optional spacing

    \noindent \(\therefore\) Since \(\text{LHS} = \text{RHS}\), \(x = 54\) is  a solution to the equation.

\end{minipage}

\vspace*{0.5ex}
\vfill{}
%  \begin{enumerate}
\refstepcounter{minipagecount} % increments the counter minipagecount by one.
\noindent{(\theminipagecount)}\hspace{0.1mm} % By default, LaTeX indents the first line of a new paragraph, but \noindent overrides this
% and inserts the current value of the minipagecount counter, enclosed in parentheses
\begin{minipage}[t]{0.40\textwidth} % The [t] option aligns the top of the minipage with the baseline of the surrounding text.

    \noindent Determine whether \(x = 18\) is a solution to the equation \(\frac{x}{2} = 9\):
    \vspace{2pt}  % Ensure spacing between problem statement and solution

    \noindent
    \renewcommand{\arraystretch}{1.3} % Adjust line spacing in the aligned environment
    \begin{tabular}{@{}p{0.60\linewidth}@{}p{0.40\linewidth}@{}}
        \(\begin{aligned}
            \text{LHS} &= \frac{x}{2} \\
                    &= \frac{18}{2} \\
                    &= 9
        \end{aligned}\) &
        \(\begin{aligned}
            \text{RHS} &= 9\\
                    & \\
                    &
        \end{aligned}\)
    \end{tabular}
    \renewcommand{\arraystretch}{1.0} % Adjust line spacing in the aligned environment
    \vspace{2pt}  % Optional spacing

    \noindent \(\therefore\) Since \(\text{LHS} = \text{RHS}\), \(x = 18\) is  a solution to the equation.

\end{minipage}

\vspace*{0.5ex}
\vfill{}
\columnbreak
    %  \begin{enumerate}
\refstepcounter{minipagecount} % increments the counter minipagecount by one.
\noindent{(\theminipagecount)}\hspace{0.1mm} % By default, LaTeX indents the first line of a new paragraph, but \noindent overrides this
% and inserts the current value of the minipagecount counter, enclosed in parentheses
\begin{minipage}[t]{0.40\textwidth} % The [t] option aligns the top of the minipage with the baseline of the surrounding text.

    \noindent Determine whether \(x = 40\) is a solution to the equation \(\frac{x}{5} = 8\):
    \vspace{2pt}  % Ensure spacing between problem statement and solution

    \noindent
    \renewcommand{\arraystretch}{1.3} % Adjust line spacing in the aligned environment
    \begin{tabular}{@{}p{0.60\linewidth}@{}p{0.40\linewidth}@{}}
        \(\begin{aligned}
            \text{LHS} &= \frac{x}{5} \\
                    &= \frac{40}{5} \\
                    &= 8
        \end{aligned}\) &
        \(\begin{aligned}
            \text{RHS} &= 8\\
                    & \\
                    &
        \end{aligned}\)
    \end{tabular}
    \renewcommand{\arraystretch}{1.0} % Adjust line spacing in the aligned environment
    \vspace{2pt}  % Optional spacing

    \noindent \(\therefore\) Since \(\text{LHS} = \text{RHS}\), \(x = 40\) is  a solution to the equation.

\end{minipage}

\vspace*{0.5ex}
\vfill{}
%  \begin{enumerate}
\refstepcounter{minipagecount} % increments the counter minipagecount by one.
\noindent{(\theminipagecount)}\hspace{0.1mm} % By default, LaTeX indents the first line of a new paragraph, but \noindent overrides this
% and inserts the current value of the minipagecount counter, enclosed in parentheses
\begin{minipage}[t]{0.40\textwidth} % The [t] option aligns the top of the minipage with the baseline of the surrounding text.

    \noindent Determine whether \(x = 41\) is a solution to the equation \(\frac{x}{10} = 4\):
    \vspace{2pt}  % Ensure spacing between problem statement and solution

    \noindent
    \renewcommand{\arraystretch}{1.3} % Adjust line spacing in the aligned environment
    \begin{tabular}{@{}p{0.60\linewidth}@{}p{0.40\linewidth}@{}}
        \(\begin{aligned}
            \text{LHS} &= \frac{x}{10} \\
                    &= \frac{41}{10} \\
                    &= 4.1
        \end{aligned}\) &
        \(\begin{aligned}
            \text{RHS} &= 4\\
                    & \\
                    &
        \end{aligned}\)
    \end{tabular}
    \renewcommand{\arraystretch}{1.0} % Adjust line spacing in the aligned environment
    \vspace{2pt}  % Optional spacing

    \noindent \(\therefore\) Since \(\text{LHS} \neq \text{RHS}\), \(x = 41\) is not  a solution to the equation.

\end{minipage}

\vspace*{0.5ex}
\vfill{}
%  \begin{enumerate}
\refstepcounter{minipagecount} % increments the counter minipagecount by one.
\noindent{(\theminipagecount)}\hspace{0.1mm} % By default, LaTeX indents the first line of a new paragraph, but \noindent overrides this
% and inserts the current value of the minipagecount counter, enclosed in parentheses
\begin{minipage}[t]{0.40\textwidth} % The [t] option aligns the top of the minipage with the baseline of the surrounding text.

    \noindent Determine whether \(x = 36\) is a solution to the equation \(\frac{x}{9} = 4\):
    \vspace{2pt}  % Ensure spacing between problem statement and solution

    \noindent
    \renewcommand{\arraystretch}{1.3} % Adjust line spacing in the aligned environment
    \begin{tabular}{@{}p{0.60\linewidth}@{}p{0.40\linewidth}@{}}
        \(\begin{aligned}
            \text{LHS} &= \frac{x}{9} \\
                    &= \frac{36}{9} \\
                    &= 4
        \end{aligned}\) &
        \(\begin{aligned}
            \text{RHS} &= 4\\
                    & \\
                    &
        \end{aligned}\)
    \end{tabular}
    \renewcommand{\arraystretch}{1.0} % Adjust line spacing in the aligned environment
    \vspace{2pt}  % Optional spacing

    \noindent \(\therefore\) Since \(\text{LHS} = \text{RHS}\), \(x = 36\) is  a solution to the equation.

\end{minipage}

\vspace*{0.5ex}
\vfill{}
%  \begin{enumerate}
\refstepcounter{minipagecount} % increments the counter minipagecount by one.
\noindent{(\theminipagecount)}\hspace{0.1mm} % By default, LaTeX indents the first line of a new paragraph, but \noindent overrides this
% and inserts the current value of the minipagecount counter, enclosed in parentheses
\begin{minipage}[t]{0.40\textwidth} % The [t] option aligns the top of the minipage with the baseline of the surrounding text.

    \noindent Determine whether \(x = 20\) is a solution to the equation \(\frac{x}{5} = 4\):
    \vspace{2pt}  % Ensure spacing between problem statement and solution

    \noindent
    \renewcommand{\arraystretch}{1.3} % Adjust line spacing in the aligned environment
    \begin{tabular}{@{}p{0.60\linewidth}@{}p{0.40\linewidth}@{}}
        \(\begin{aligned}
            \text{LHS} &= \frac{x}{5} \\
                    &= \frac{20}{5} \\
                    &= 4
        \end{aligned}\) &
        \(\begin{aligned}
            \text{RHS} &= 4\\
                    & \\
                    &
        \end{aligned}\)
    \end{tabular}
    \renewcommand{\arraystretch}{1.0} % Adjust line spacing in the aligned environment
    \vspace{2pt}  % Optional spacing

    \noindent \(\therefore\) Since \(\text{LHS} = \text{RHS}\), \(x = 20\) is  a solution to the equation.

\end{minipage}

\vspace*{0.5ex}
\vfill{}
%  \begin{enumerate}
\refstepcounter{minipagecount} % increments the counter minipagecount by one.
\noindent{(\theminipagecount)}\hspace{0.1mm} % By default, LaTeX indents the first line of a new paragraph, but \noindent overrides this
% and inserts the current value of the minipagecount counter, enclosed in parentheses
\begin{minipage}[t]{0.40\textwidth} % The [t] option aligns the top of the minipage with the baseline of the surrounding text.

    \noindent Determine whether \(x = 20\) is a solution to the equation \(\frac{x}{5} = 4\):
    \vspace{2pt}  % Ensure spacing between problem statement and solution

    \noindent
    \renewcommand{\arraystretch}{1.3} % Adjust line spacing in the aligned environment
    \begin{tabular}{@{}p{0.60\linewidth}@{}p{0.40\linewidth}@{}}
        \(\begin{aligned}
            \text{LHS} &= \frac{x}{5} \\
                    &= \frac{20}{5} \\
                    &= 4
        \end{aligned}\) &
        \(\begin{aligned}
            \text{RHS} &= 4\\
                    & \\
                    &
        \end{aligned}\)
    \end{tabular}
    \renewcommand{\arraystretch}{1.0} % Adjust line spacing in the aligned environment
    \vspace{2pt}  % Optional spacing

    \noindent \(\therefore\) Since \(\text{LHS} = \text{RHS}\), \(x = 20\) is  a solution to the equation.

\end{minipage}

\vspace*{0.5ex}
\vfill{}
\newpage
    
\end{multicols}
\end{document}
