\documentclass[12pt]{article}
\usepackage{tikz}
\usepackage{amssymb}  % For \therefore symbol
\usepackage{amsmath}
% Underlining package
\usepackage[normalem]{ulem} % [normalem] prevents the package from changing the default behavior of \emph to underline.
\usepackage[a4paper, portrait, margin=1cm]{geometry}
\usepackage{multicol}
\usepackage{fancyhdr}
\usepackage[none]{hyphenat}

\def \HeadingQuestions {\section*{\Large Name: \underline{\hspace{8cm}} \hfill Date: \underline{\hspace{3cm}}} \vspace{-3mm}
{- Check Solution: Questions} \vspace{1pt}\hrule}

% \linespread{1.5} % Adjust line spacing factor
\raggedbottom

% raise footer with page number; no header
\fancypagestyle{myfancypagestyle}{
  \fancyhf{}% clear all header and footer fields
  \renewcommand{\headrulewidth}{0pt} % no rule under header
  \fancyfoot[C] {\thepage} \setlength{\footskip}{14.5pt} % raise page number allowed min 14.5pt
}
\pagestyle{myfancypagestyle}  % apply myfancypagestyle
\newcounter{minipagecount}
\begin{document}
\HeadingQuestions
\vspace{1pt}
\begin{multicols}{2}
%  \begin{enumerate}
\refstepcounter{minipagecount} % increments the counter minipagecount by one.
\noindent{(\theminipagecount)}\hspace{0.1mm} % By default, LaTeX indents the first line of a new paragraph, but \noindent overrides this
% and inserts the current value of the minipagecount counter, enclosed in parentheses
\begin{minipage}[t]{0.40\textwidth} % The [t] option aligns the top of the minipage with the baseline of the surrounding text.

    \noindent Determine whether \(x = 11\) is a solution to the equation \(x - 1 = 10\):
    \vspace{4pt}  % Ensure spacing between problem statement and solution

    \noindent
    \renewcommand{\arraystretch}{1.3} % Adjust line spacing in the aligned environment
    \begin{tabular}{@{}p{0.60\linewidth}@{}p{0.40\linewidth}@{}}
        \(\begin{aligned}
            \text{LHS} &=  \\
                    &=  \\
                    &=
        \end{aligned}\) &
        \(\begin{aligned}
            \text{RHS} &= \\
                    & \\
                    &
        \end{aligned}\)
    \end{tabular}
    \renewcommand{\arraystretch}{1.0} % Adjust line spacing in the aligned environment
    \vspace{2pt}  % Optional spacing

    \noindent \(\therefore\) Since \(\text{LHS} \dotuline{\hspace{5mm}} \text{RHS}\), \(x = 11\) \dotuline{\hspace{12mm}} a solution to the equation.

\end{minipage}

 \vspace*{16pt}
%  \begin{enumerate}
\refstepcounter{minipagecount} % increments the counter minipagecount by one.
\noindent{(\theminipagecount)}\hspace{0.1mm} % By default, LaTeX indents the first line of a new paragraph, but \noindent overrides this
% and inserts the current value of the minipagecount counter, enclosed in parentheses
\begin{minipage}[t]{0.40\textwidth} % The [t] option aligns the top of the minipage with the baseline of the surrounding text.

    \noindent Determine whether \(x = 9\) is a solution to the equation \(x - 10 = 2\):
    \vspace{4pt}  % Ensure spacing between problem statement and solution

    \noindent
    \renewcommand{\arraystretch}{1.3} % Adjust line spacing in the aligned environment
    \begin{tabular}{@{}p{0.60\linewidth}@{}p{0.40\linewidth}@{}}
        \(\begin{aligned}
            \text{LHS} &=  \\
                    &=  \\
                    &=
        \end{aligned}\) &
        \(\begin{aligned}
            \text{RHS} &= \\
                    & \\
                    &
        \end{aligned}\)
    \end{tabular}
    \renewcommand{\arraystretch}{1.0} % Adjust line spacing in the aligned environment
    \vspace{2pt}  % Optional spacing

    \noindent \(\therefore\) Since \(\text{LHS} \dotuline{\hspace{5mm}} \text{RHS}\), \(x = 9\) \dotuline{\hspace{12mm}} a solution to the equation.

\end{minipage}

 \vspace*{16pt}
%  \begin{enumerate}
\refstepcounter{minipagecount} % increments the counter minipagecount by one.
\noindent{(\theminipagecount)}\hspace{0.1mm} % By default, LaTeX indents the first line of a new paragraph, but \noindent overrides this
% and inserts the current value of the minipagecount counter, enclosed in parentheses
\begin{minipage}[t]{0.40\textwidth} % The [t] option aligns the top of the minipage with the baseline of the surrounding text.

    \noindent Determine whether \(x = 11\) is a solution to the equation \(x - 4 = 5\):
    \vspace{4pt}  % Ensure spacing between problem statement and solution

    \noindent
    \renewcommand{\arraystretch}{1.3} % Adjust line spacing in the aligned environment
    \begin{tabular}{@{}p{0.60\linewidth}@{}p{0.40\linewidth}@{}}
        \(\begin{aligned}
            \text{LHS} &=  \\
                    &=  \\
                    &=
        \end{aligned}\) &
        \(\begin{aligned}
            \text{RHS} &= \\
                    & \\
                    &
        \end{aligned}\)
    \end{tabular}
    \renewcommand{\arraystretch}{1.0} % Adjust line spacing in the aligned environment
    \vspace{2pt}  % Optional spacing

    \noindent \(\therefore\) Since \(\text{LHS} \dotuline{\hspace{5mm}} \text{RHS}\), \(x = 11\) \dotuline{\hspace{12mm}} a solution to the equation.

\end{minipage}

 \vspace*{16pt}
%  \begin{enumerate}
\refstepcounter{minipagecount} % increments the counter minipagecount by one.
\noindent{(\theminipagecount)}\hspace{0.1mm} % By default, LaTeX indents the first line of a new paragraph, but \noindent overrides this
% and inserts the current value of the minipagecount counter, enclosed in parentheses
\begin{minipage}[t]{0.40\textwidth} % The [t] option aligns the top of the minipage with the baseline of the surrounding text.

    \noindent Determine whether \(x = 17\) is a solution to the equation \(x - 10 = 7\):
    \vspace{4pt}  % Ensure spacing between problem statement and solution

    \noindent
    \renewcommand{\arraystretch}{1.3} % Adjust line spacing in the aligned environment
    \begin{tabular}{@{}p{0.60\linewidth}@{}p{0.40\linewidth}@{}}
        \(\begin{aligned}
            \text{LHS} &=  \\
                    &=  \\
                    &=
        \end{aligned}\) &
        \(\begin{aligned}
            \text{RHS} &= \\
                    & \\
                    &
        \end{aligned}\)
    \end{tabular}
    \renewcommand{\arraystretch}{1.0} % Adjust line spacing in the aligned environment
    \vspace{2pt}  % Optional spacing

    \noindent \(\therefore\) Since \(\text{LHS} \dotuline{\hspace{5mm}} \text{RHS}\), \(x = 17\) \dotuline{\hspace{12mm}} a solution to the equation.

\end{minipage}

 \vspace*{16pt}
%  \begin{enumerate}
\refstepcounter{minipagecount} % increments the counter minipagecount by one.
\noindent{(\theminipagecount)}\hspace{0.1mm} % By default, LaTeX indents the first line of a new paragraph, but \noindent overrides this
% and inserts the current value of the minipagecount counter, enclosed in parentheses
\begin{minipage}[t]{0.40\textwidth} % The [t] option aligns the top of the minipage with the baseline of the surrounding text.

    \noindent Determine whether \(x = 5\) is a solution to the equation \(x - 4 = 1\):
    \vspace{4pt}  % Ensure spacing between problem statement and solution

    \noindent
    \renewcommand{\arraystretch}{1.3} % Adjust line spacing in the aligned environment
    \begin{tabular}{@{}p{0.60\linewidth}@{}p{0.40\linewidth}@{}}
        \(\begin{aligned}
            \text{LHS} &=  \\
                    &=  \\
                    &=
        \end{aligned}\) &
        \(\begin{aligned}
            \text{RHS} &= \\
                    & \\
                    &
        \end{aligned}\)
    \end{tabular}
    \renewcommand{\arraystretch}{1.0} % Adjust line spacing in the aligned environment
    \vspace{2pt}  % Optional spacing

    \noindent \(\therefore\) Since \(\text{LHS} \dotuline{\hspace{5mm}} \text{RHS}\), \(x = 5\) \dotuline{\hspace{12mm}} a solution to the equation.

\end{minipage}

 \vspace*{16pt}
\columnbreak
    %  \begin{enumerate}
\refstepcounter{minipagecount} % increments the counter minipagecount by one.
\noindent{(\theminipagecount)}\hspace{0.1mm} % By default, LaTeX indents the first line of a new paragraph, but \noindent overrides this
% and inserts the current value of the minipagecount counter, enclosed in parentheses
\begin{minipage}[t]{0.40\textwidth} % The [t] option aligns the top of the minipage with the baseline of the surrounding text.

    \noindent Determine whether \(x = 2\) is a solution to the equation \(x - 1 = 1\):
    \vspace{4pt}  % Ensure spacing between problem statement and solution

    \noindent
    \renewcommand{\arraystretch}{1.3} % Adjust line spacing in the aligned environment
    \begin{tabular}{@{}p{0.60\linewidth}@{}p{0.40\linewidth}@{}}
        \(\begin{aligned}
            \text{LHS} &=  \\
                    &=  \\
                    &=
        \end{aligned}\) &
        \(\begin{aligned}
            \text{RHS} &= \\
                    & \\
                    &
        \end{aligned}\)
    \end{tabular}
    \renewcommand{\arraystretch}{1.0} % Adjust line spacing in the aligned environment
    \vspace{2pt}  % Optional spacing

    \noindent \(\therefore\) Since \(\text{LHS} \dotuline{\hspace{5mm}} \text{RHS}\), \(x = 2\) \dotuline{\hspace{12mm}} a solution to the equation.

\end{minipage}

 \vspace*{16pt}
%  \begin{enumerate}
\refstepcounter{minipagecount} % increments the counter minipagecount by one.
\noindent{(\theminipagecount)}\hspace{0.1mm} % By default, LaTeX indents the first line of a new paragraph, but \noindent overrides this
% and inserts the current value of the minipagecount counter, enclosed in parentheses
\begin{minipage}[t]{0.40\textwidth} % The [t] option aligns the top of the minipage with the baseline of the surrounding text.

    \noindent Determine whether \(x = 13\) is a solution to the equation \(x - 6 = 8\):
    \vspace{4pt}  % Ensure spacing between problem statement and solution

    \noindent
    \renewcommand{\arraystretch}{1.3} % Adjust line spacing in the aligned environment
    \begin{tabular}{@{}p{0.60\linewidth}@{}p{0.40\linewidth}@{}}
        \(\begin{aligned}
            \text{LHS} &=  \\
                    &=  \\
                    &=
        \end{aligned}\) &
        \(\begin{aligned}
            \text{RHS} &= \\
                    & \\
                    &
        \end{aligned}\)
    \end{tabular}
    \renewcommand{\arraystretch}{1.0} % Adjust line spacing in the aligned environment
    \vspace{2pt}  % Optional spacing

    \noindent \(\therefore\) Since \(\text{LHS} \dotuline{\hspace{5mm}} \text{RHS}\), \(x = 13\) \dotuline{\hspace{12mm}} a solution to the equation.

\end{minipage}

 \vspace*{16pt}
%  \begin{enumerate}
\refstepcounter{minipagecount} % increments the counter minipagecount by one.
\noindent{(\theminipagecount)}\hspace{0.1mm} % By default, LaTeX indents the first line of a new paragraph, but \noindent overrides this
% and inserts the current value of the minipagecount counter, enclosed in parentheses
\begin{minipage}[t]{0.40\textwidth} % The [t] option aligns the top of the minipage with the baseline of the surrounding text.

    \noindent Determine whether \(x = 14\) is a solution to the equation \(x - 6 = 8\):
    \vspace{4pt}  % Ensure spacing between problem statement and solution

    \noindent
    \renewcommand{\arraystretch}{1.3} % Adjust line spacing in the aligned environment
    \begin{tabular}{@{}p{0.60\linewidth}@{}p{0.40\linewidth}@{}}
        \(\begin{aligned}
            \text{LHS} &=  \\
                    &=  \\
                    &=
        \end{aligned}\) &
        \(\begin{aligned}
            \text{RHS} &= \\
                    & \\
                    &
        \end{aligned}\)
    \end{tabular}
    \renewcommand{\arraystretch}{1.0} % Adjust line spacing in the aligned environment
    \vspace{2pt}  % Optional spacing

    \noindent \(\therefore\) Since \(\text{LHS} \dotuline{\hspace{5mm}} \text{RHS}\), \(x = 14\) \dotuline{\hspace{12mm}} a solution to the equation.

\end{minipage}

 \vspace*{16pt}
%  \begin{enumerate}
\refstepcounter{minipagecount} % increments the counter minipagecount by one.
\noindent{(\theminipagecount)}\hspace{0.1mm} % By default, LaTeX indents the first line of a new paragraph, but \noindent overrides this
% and inserts the current value of the minipagecount counter, enclosed in parentheses
\begin{minipage}[t]{0.40\textwidth} % The [t] option aligns the top of the minipage with the baseline of the surrounding text.

    \noindent Determine whether \(x = 3\) is a solution to the equation \(x - 1 = 2\):
    \vspace{4pt}  % Ensure spacing between problem statement and solution

    \noindent
    \renewcommand{\arraystretch}{1.3} % Adjust line spacing in the aligned environment
    \begin{tabular}{@{}p{0.60\linewidth}@{}p{0.40\linewidth}@{}}
        \(\begin{aligned}
            \text{LHS} &=  \\
                    &=  \\
                    &=
        \end{aligned}\) &
        \(\begin{aligned}
            \text{RHS} &= \\
                    & \\
                    &
        \end{aligned}\)
    \end{tabular}
    \renewcommand{\arraystretch}{1.0} % Adjust line spacing in the aligned environment
    \vspace{2pt}  % Optional spacing

    \noindent \(\therefore\) Since \(\text{LHS} \dotuline{\hspace{5mm}} \text{RHS}\), \(x = 3\) \dotuline{\hspace{12mm}} a solution to the equation.

\end{minipage}

 \vspace*{16pt}
%  \begin{enumerate}
\refstepcounter{minipagecount} % increments the counter minipagecount by one.
\noindent{(\theminipagecount)}\hspace{0.1mm} % By default, LaTeX indents the first line of a new paragraph, but \noindent overrides this
% and inserts the current value of the minipagecount counter, enclosed in parentheses
\begin{minipage}[t]{0.40\textwidth} % The [t] option aligns the top of the minipage with the baseline of the surrounding text.

    \noindent Determine whether \(x = 11\) is a solution to the equation \(x - 7 = 4\):
    \vspace{4pt}  % Ensure spacing between problem statement and solution

    \noindent
    \renewcommand{\arraystretch}{1.3} % Adjust line spacing in the aligned environment
    \begin{tabular}{@{}p{0.60\linewidth}@{}p{0.40\linewidth}@{}}
        \(\begin{aligned}
            \text{LHS} &=  \\
                    &=  \\
                    &=
        \end{aligned}\) &
        \(\begin{aligned}
            \text{RHS} &= \\
                    & \\
                    &
        \end{aligned}\)
    \end{tabular}
    \renewcommand{\arraystretch}{1.0} % Adjust line spacing in the aligned environment
    \vspace{2pt}  % Optional spacing

    \noindent \(\therefore\) Since \(\text{LHS} \dotuline{\hspace{5mm}} \text{RHS}\), \(x = 11\) \dotuline{\hspace{12mm}} a solution to the equation.

\end{minipage}

 \vspace*{16pt}
\newpage
    %  \begin{enumerate}
\refstepcounter{minipagecount} % increments the counter minipagecount by one.
\noindent{(\theminipagecount)}\hspace{0.1mm} % By default, LaTeX indents the first line of a new paragraph, but \noindent overrides this
% and inserts the current value of the minipagecount counter, enclosed in parentheses
\begin{minipage}[t]{0.40\textwidth} % The [t] option aligns the top of the minipage with the baseline of the surrounding text.

    \noindent Determine whether \(x = 8\) is a solution to the equation \(x - 1 = 5\):
    \vspace{4pt}  % Ensure spacing between problem statement and solution

    \noindent
    \renewcommand{\arraystretch}{1.3} % Adjust line spacing in the aligned environment
    \begin{tabular}{@{}p{0.60\linewidth}@{}p{0.40\linewidth}@{}}
        \(\begin{aligned}
            \text{LHS} &=  \\
                    &=  \\
                    &=
        \end{aligned}\) &
        \(\begin{aligned}
            \text{RHS} &= \\
                    & \\
                    &
        \end{aligned}\)
    \end{tabular}
    \renewcommand{\arraystretch}{1.0} % Adjust line spacing in the aligned environment
    \vspace{2pt}  % Optional spacing

    \noindent \(\therefore\) Since \(\text{LHS} \dotuline{\hspace{5mm}} \text{RHS}\), \(x = 8\) \dotuline{\hspace{12mm}} a solution to the equation.

\end{minipage}

 \vspace*{16pt}
%  \begin{enumerate}
\refstepcounter{minipagecount} % increments the counter minipagecount by one.
\noindent{(\theminipagecount)}\hspace{0.1mm} % By default, LaTeX indents the first line of a new paragraph, but \noindent overrides this
% and inserts the current value of the minipagecount counter, enclosed in parentheses
\begin{minipage}[t]{0.40\textwidth} % The [t] option aligns the top of the minipage with the baseline of the surrounding text.

    \noindent Determine whether \(x = 8\) is a solution to the equation \(x - 4 = 7\):
    \vspace{4pt}  % Ensure spacing between problem statement and solution

    \noindent
    \renewcommand{\arraystretch}{1.3} % Adjust line spacing in the aligned environment
    \begin{tabular}{@{}p{0.60\linewidth}@{}p{0.40\linewidth}@{}}
        \(\begin{aligned}
            \text{LHS} &=  \\
                    &=  \\
                    &=
        \end{aligned}\) &
        \(\begin{aligned}
            \text{RHS} &= \\
                    & \\
                    &
        \end{aligned}\)
    \end{tabular}
    \renewcommand{\arraystretch}{1.0} % Adjust line spacing in the aligned environment
    \vspace{2pt}  % Optional spacing

    \noindent \(\therefore\) Since \(\text{LHS} \dotuline{\hspace{5mm}} \text{RHS}\), \(x = 8\) \dotuline{\hspace{12mm}} a solution to the equation.

\end{minipage}

 \vspace*{16pt}
%  \begin{enumerate}
\refstepcounter{minipagecount} % increments the counter minipagecount by one.
\noindent{(\theminipagecount)}\hspace{0.1mm} % By default, LaTeX indents the first line of a new paragraph, but \noindent overrides this
% and inserts the current value of the minipagecount counter, enclosed in parentheses
\begin{minipage}[t]{0.40\textwidth} % The [t] option aligns the top of the minipage with the baseline of the surrounding text.

    \noindent Determine whether \(x = 6\) is a solution to the equation \(x - 7 = 2\):
    \vspace{4pt}  % Ensure spacing between problem statement and solution

    \noindent
    \renewcommand{\arraystretch}{1.3} % Adjust line spacing in the aligned environment
    \begin{tabular}{@{}p{0.60\linewidth}@{}p{0.40\linewidth}@{}}
        \(\begin{aligned}
            \text{LHS} &=  \\
                    &=  \\
                    &=
        \end{aligned}\) &
        \(\begin{aligned}
            \text{RHS} &= \\
                    & \\
                    &
        \end{aligned}\)
    \end{tabular}
    \renewcommand{\arraystretch}{1.0} % Adjust line spacing in the aligned environment
    \vspace{2pt}  % Optional spacing

    \noindent \(\therefore\) Since \(\text{LHS} \dotuline{\hspace{5mm}} \text{RHS}\), \(x = 6\) \dotuline{\hspace{12mm}} a solution to the equation.

\end{minipage}

 \vspace*{16pt}
%  \begin{enumerate}
\refstepcounter{minipagecount} % increments the counter minipagecount by one.
\noindent{(\theminipagecount)}\hspace{0.1mm} % By default, LaTeX indents the first line of a new paragraph, but \noindent overrides this
% and inserts the current value of the minipagecount counter, enclosed in parentheses
\begin{minipage}[t]{0.40\textwidth} % The [t] option aligns the top of the minipage with the baseline of the surrounding text.

    \noindent Determine whether \(x = 16\) is a solution to the equation \(x - 8 = 9\):
    \vspace{4pt}  % Ensure spacing between problem statement and solution

    \noindent
    \renewcommand{\arraystretch}{1.3} % Adjust line spacing in the aligned environment
    \begin{tabular}{@{}p{0.60\linewidth}@{}p{0.40\linewidth}@{}}
        \(\begin{aligned}
            \text{LHS} &=  \\
                    &=  \\
                    &=
        \end{aligned}\) &
        \(\begin{aligned}
            \text{RHS} &= \\
                    & \\
                    &
        \end{aligned}\)
    \end{tabular}
    \renewcommand{\arraystretch}{1.0} % Adjust line spacing in the aligned environment
    \vspace{2pt}  % Optional spacing

    \noindent \(\therefore\) Since \(\text{LHS} \dotuline{\hspace{5mm}} \text{RHS}\), \(x = 16\) \dotuline{\hspace{12mm}} a solution to the equation.

\end{minipage}

 \vspace*{16pt}
%  \begin{enumerate}
\refstepcounter{minipagecount} % increments the counter minipagecount by one.
\noindent{(\theminipagecount)}\hspace{0.1mm} % By default, LaTeX indents the first line of a new paragraph, but \noindent overrides this
% and inserts the current value of the minipagecount counter, enclosed in parentheses
\begin{minipage}[t]{0.40\textwidth} % The [t] option aligns the top of the minipage with the baseline of the surrounding text.

    \noindent Determine whether \(x = 13\) is a solution to the equation \(x - 6 = 6\):
    \vspace{4pt}  % Ensure spacing between problem statement and solution

    \noindent
    \renewcommand{\arraystretch}{1.3} % Adjust line spacing in the aligned environment
    \begin{tabular}{@{}p{0.60\linewidth}@{}p{0.40\linewidth}@{}}
        \(\begin{aligned}
            \text{LHS} &=  \\
                    &=  \\
                    &=
        \end{aligned}\) &
        \(\begin{aligned}
            \text{RHS} &= \\
                    & \\
                    &
        \end{aligned}\)
    \end{tabular}
    \renewcommand{\arraystretch}{1.0} % Adjust line spacing in the aligned environment
    \vspace{2pt}  % Optional spacing

    \noindent \(\therefore\) Since \(\text{LHS} \dotuline{\hspace{5mm}} \text{RHS}\), \(x = 13\) \dotuline{\hspace{12mm}} a solution to the equation.

\end{minipage}

 \vspace*{16pt}
\columnbreak
    %  \begin{enumerate}
\refstepcounter{minipagecount} % increments the counter minipagecount by one.
\noindent{(\theminipagecount)}\hspace{0.1mm} % By default, LaTeX indents the first line of a new paragraph, but \noindent overrides this
% and inserts the current value of the minipagecount counter, enclosed in parentheses
\begin{minipage}[t]{0.40\textwidth} % The [t] option aligns the top of the minipage with the baseline of the surrounding text.

    \noindent Determine whether \(x = 15\) is a solution to the equation \(x - 6 = 6\):
    \vspace{4pt}  % Ensure spacing between problem statement and solution

    \noindent
    \renewcommand{\arraystretch}{1.3} % Adjust line spacing in the aligned environment
    \begin{tabular}{@{}p{0.60\linewidth}@{}p{0.40\linewidth}@{}}
        \(\begin{aligned}
            \text{LHS} &=  \\
                    &=  \\
                    &=
        \end{aligned}\) &
        \(\begin{aligned}
            \text{RHS} &= \\
                    & \\
                    &
        \end{aligned}\)
    \end{tabular}
    \renewcommand{\arraystretch}{1.0} % Adjust line spacing in the aligned environment
    \vspace{2pt}  % Optional spacing

    \noindent \(\therefore\) Since \(\text{LHS} \dotuline{\hspace{5mm}} \text{RHS}\), \(x = 15\) \dotuline{\hspace{12mm}} a solution to the equation.

\end{minipage}

 \vspace*{16pt}
%  \begin{enumerate}
\refstepcounter{minipagecount} % increments the counter minipagecount by one.
\noindent{(\theminipagecount)}\hspace{0.1mm} % By default, LaTeX indents the first line of a new paragraph, but \noindent overrides this
% and inserts the current value of the minipagecount counter, enclosed in parentheses
\begin{minipage}[t]{0.40\textwidth} % The [t] option aligns the top of the minipage with the baseline of the surrounding text.

    \noindent Determine whether \(x = 6\) is a solution to the equation \(x - 5 = 1\):
    \vspace{4pt}  % Ensure spacing between problem statement and solution

    \noindent
    \renewcommand{\arraystretch}{1.3} % Adjust line spacing in the aligned environment
    \begin{tabular}{@{}p{0.60\linewidth}@{}p{0.40\linewidth}@{}}
        \(\begin{aligned}
            \text{LHS} &=  \\
                    &=  \\
                    &=
        \end{aligned}\) &
        \(\begin{aligned}
            \text{RHS} &= \\
                    & \\
                    &
        \end{aligned}\)
    \end{tabular}
    \renewcommand{\arraystretch}{1.0} % Adjust line spacing in the aligned environment
    \vspace{2pt}  % Optional spacing

    \noindent \(\therefore\) Since \(\text{LHS} \dotuline{\hspace{5mm}} \text{RHS}\), \(x = 6\) \dotuline{\hspace{12mm}} a solution to the equation.

\end{minipage}

 \vspace*{16pt}
%  \begin{enumerate}
\refstepcounter{minipagecount} % increments the counter minipagecount by one.
\noindent{(\theminipagecount)}\hspace{0.1mm} % By default, LaTeX indents the first line of a new paragraph, but \noindent overrides this
% and inserts the current value of the minipagecount counter, enclosed in parentheses
\begin{minipage}[t]{0.40\textwidth} % The [t] option aligns the top of the minipage with the baseline of the surrounding text.

    \noindent Determine whether \(x = 3\) is a solution to the equation \(x - 3 = 2\):
    \vspace{4pt}  % Ensure spacing between problem statement and solution

    \noindent
    \renewcommand{\arraystretch}{1.3} % Adjust line spacing in the aligned environment
    \begin{tabular}{@{}p{0.60\linewidth}@{}p{0.40\linewidth}@{}}
        \(\begin{aligned}
            \text{LHS} &=  \\
                    &=  \\
                    &=
        \end{aligned}\) &
        \(\begin{aligned}
            \text{RHS} &= \\
                    & \\
                    &
        \end{aligned}\)
    \end{tabular}
    \renewcommand{\arraystretch}{1.0} % Adjust line spacing in the aligned environment
    \vspace{2pt}  % Optional spacing

    \noindent \(\therefore\) Since \(\text{LHS} \dotuline{\hspace{5mm}} \text{RHS}\), \(x = 3\) \dotuline{\hspace{12mm}} a solution to the equation.

\end{minipage}

 \vspace*{16pt}
%  \begin{enumerate}
\refstepcounter{minipagecount} % increments the counter minipagecount by one.
\noindent{(\theminipagecount)}\hspace{0.1mm} % By default, LaTeX indents the first line of a new paragraph, but \noindent overrides this
% and inserts the current value of the minipagecount counter, enclosed in parentheses
\begin{minipage}[t]{0.40\textwidth} % The [t] option aligns the top of the minipage with the baseline of the surrounding text.

    \noindent Determine whether \(x = 10\) is a solution to the equation \(x - 1 = 9\):
    \vspace{4pt}  % Ensure spacing between problem statement and solution

    \noindent
    \renewcommand{\arraystretch}{1.3} % Adjust line spacing in the aligned environment
    \begin{tabular}{@{}p{0.60\linewidth}@{}p{0.40\linewidth}@{}}
        \(\begin{aligned}
            \text{LHS} &=  \\
                    &=  \\
                    &=
        \end{aligned}\) &
        \(\begin{aligned}
            \text{RHS} &= \\
                    & \\
                    &
        \end{aligned}\)
    \end{tabular}
    \renewcommand{\arraystretch}{1.0} % Adjust line spacing in the aligned environment
    \vspace{2pt}  % Optional spacing

    \noindent \(\therefore\) Since \(\text{LHS} \dotuline{\hspace{5mm}} \text{RHS}\), \(x = 10\) \dotuline{\hspace{12mm}} a solution to the equation.

\end{minipage}

 \vspace*{16pt}
%  \begin{enumerate}
\refstepcounter{minipagecount} % increments the counter minipagecount by one.
\noindent{(\theminipagecount)}\hspace{0.1mm} % By default, LaTeX indents the first line of a new paragraph, but \noindent overrides this
% and inserts the current value of the minipagecount counter, enclosed in parentheses
\begin{minipage}[t]{0.40\textwidth} % The [t] option aligns the top of the minipage with the baseline of the surrounding text.

    \noindent Determine whether \(x = 6\) is a solution to the equation \(x - 4 = 2\):
    \vspace{4pt}  % Ensure spacing between problem statement and solution

    \noindent
    \renewcommand{\arraystretch}{1.3} % Adjust line spacing in the aligned environment
    \begin{tabular}{@{}p{0.60\linewidth}@{}p{0.40\linewidth}@{}}
        \(\begin{aligned}
            \text{LHS} &=  \\
                    &=  \\
                    &=
        \end{aligned}\) &
        \(\begin{aligned}
            \text{RHS} &= \\
                    & \\
                    &
        \end{aligned}\)
    \end{tabular}
    \renewcommand{\arraystretch}{1.0} % Adjust line spacing in the aligned environment
    \vspace{2pt}  % Optional spacing

    \noindent \(\therefore\) Since \(\text{LHS} \dotuline{\hspace{5mm}} \text{RHS}\), \(x = 6\) \dotuline{\hspace{12mm}} a solution to the equation.

\end{minipage}

 \vspace*{16pt}
\newpage

\end{multicols}
\end{document}
