\documentclass[12pt]{article}
\usepackage{tikz}
\usepackage{amsmath}
\usepackage{graphicx}
\usepackage[normalem]{ulem} % Underlining package
\usepackage[a4paper, portrait, margin=1cm]{geometry}
\usepackage{multicol}
\usepackage{fancyhdr}

\def \Heading {\section*{\Large Name: \underline{\hspace{8cm}} \hfill Date: \underline{\hspace{3cm}}} \vspace{-3mm}
{Times Tables} \vspace{1pt}\hrule}

\fancypagestyle{myfancypagestyle}{
  \fancyhf{} % clear all header and footer fields
  \renewcommand{\headrulewidth}{0pt} % no rule under header
  \fancyfoot[C] {\thepage} \setlength{\footskip}{14.5pt} % raise page number
}
\pagestyle{myfancypagestyle}

% Define five dots symbol for missing numbers
\newcommand{\fivedots}{\textsubscript{. . . .}}

\begin{document}
\Heading

\vspace{5mm}
\Large % Set font size to large

Study the list of multiples and the times table, as quickly as possible, 5 times. \\

Write in the missing multiples.
% Full list of multiples
\[ 3, 6, 9, 12, 15, 18, 21, 24, 27, 30 \]
% Every second number missing
\[ 3, \textsubscript{. . . .}, 9, \textsubscript{. . . .}, 15, \textsubscript{. . . .}, 21, \textsubscript{. . . .}, 27, \textsubscript{. . . .} \]
% Two out of every three missing
\[ 3, \textsubscript{. . . .}, \textsubscript{. . . .}, 12, \textsubscript{. . . .}, \textsubscript{. . . .}, 21, \textsubscript{. . . .}, \textsubscript{. . . .}, 30 \]
% All missing
\[ 3, \textsubscript{. . . .}, \textsubscript{. . . .}, \textsubscript{. . . .}, \textsubscript{. . . .}, \textsubscript{. . . .}, \textsubscript{. . . .}, \textsubscript{. . . .}, \textsubscript{. . . .}, \textsubscript{. . . .} \]

Write in the missing values in each multiplication equation.
% Times Table with variations
\begin{center}
\begin{tabular}{|c|c|c|c|}
\hline
$ 3 \times 1 = 3 $ & $ 3 \times 1 = \textsubscript{. . . .} $ & $ 3 \times \textsubscript{. . . .} = \textsubscript{. . . .} $ & $ \textsubscript{. . . .} \times \textsubscript{. . . .} = \textsubscript{. . . .} $ \\ \hline
$ 3 \times 2 = 6 $ & $ 3 \times 2 = \textsubscript{. . . .} $ & $ 3 \times \textsubscript{. . . .} = \textsubscript{. . . .} $ & $ \textsubscript{. . . .} \times \textsubscript{. . . .} = \textsubscript{. . . .} $ \\ \hline
$ 3 \times 3 = 9 $ & $ 3 \times 3 = \textsubscript{. . . .} $ & $ 3 \times \textsubscript{. . . .} = \textsubscript{. . . .} $ & $ \textsubscript{. . . .} \times \textsubscript{. . . .} = \textsubscript{. . . .} $ \\ \hline
$ 3 \times 4 = 12 $ & $ 3 \times 4 = \textsubscript{. . . .} $ & $ 3 \times \textsubscript{. . . .} = \textsubscript{. . . .} $ & $ \textsubscript{. . . .} \times \textsubscript{. . . .} = \textsubscript{. . . .} $ \\ \hline
$ 3 \times 5 = 15 $ & $ 3 \times 5 = \textsubscript{. . . .} $ & $ 3 \times \textsubscript{. . . .} = \textsubscript{. . . .} $ & $ \textsubscript{. . . .} \times \textsubscript{. . . .} = \textsubscript{. . . .} $ \\ \hline
$ 3 \times 6 = 18 $ & $ 3 \times 6 = \textsubscript{. . . .} $ & $ 3 \times \textsubscript{. . . .} = \textsubscript{. . . .} $ & $ \textsubscript{. . . .} \times \textsubscript{. . . .} = \textsubscript{. . . .} $ \\ \hline
$ 3 \times 7 = 21 $ & $ 3 \times 7 = \textsubscript{. . . .} $ & $ 3 \times \textsubscript{. . . .} = \textsubscript{. . . .} $ & $ \textsubscript{. . . .} \times \textsubscript{. . . .} = \textsubscript{. . . .} $ \\ \hline
$ 3 \times 8 = 24 $ & $ 3 \times 8 = \textsubscript{. . . .} $ & $ 3 \times \textsubscript{. . . .} = \textsubscript{. . . .} $ & $ \textsubscript{. . . .} \times \textsubscript{. . . .} = \textsubscript{. . . .} $ \\ \hline
$ 3 \times 9 = 27 $ & $ 3 \times 9 = \textsubscript{. . . .} $ & $ 3 \times \textsubscript{. . . .} = \textsubscript{. . . .} $ & $ \textsubscript{. . . .} \times \textsubscript{. . . .} = \textsubscript{. . . .} $ \\ \hline
$ 3 \times 10 = 30 $ & $ 3 \times 10 = \textsubscript{. . . .} $ & $ 3 \times \textsubscript{. . . .} = \textsubscript{. . . .} $ & $ \textsubscript{. . . .} \times \textsubscript{. . . .} = \textsubscript{. . . .} $ \\ \hline

\end{tabular}
\end{center}

Write in the missing values in each multiplication equation.
% New table with 4 columns, randomized equations
\begin{center}
\begin{tabular}{|c|c|c|c|}
\hline
$3 \times 6 = \fivedots$ & $3 \times 9 = \fivedots$ & $3 \times 10 = \fivedots$ & $3 \times 3 = \fivedots$ \\ \hline
$3 \times 1 = \fivedots$ & $3 \times 6 = \fivedots$ & $3 \times 1 = \fivedots$ & $3 \times 5 = \fivedots$ \\ \hline
$3 \times 9 = \fivedots$ & $3 \times 8 = \fivedots$ & $3 \times 6 = \fivedots$ & $3 \times 6 = \fivedots$ \\ \hline
$3 \times 10 = \fivedots$ & $3 \times 2 = \fivedots$ & $3 \times 8 = \fivedots$ & $3 \times 7 = \fivedots$ \\ \hline
$3 \times 5 = \fivedots$ & $3 \times 1 = \fivedots$ & $3 \times 7 = \fivedots$ & $3 \times 8 = \fivedots$ \\ \hline
$3 \times 3 = \fivedots$ & $3 \times 7 = \fivedots$ & $3 \times 5 = \fivedots$ & $3 \times 10 = \fivedots$ \\ \hline
$3 \times 2 = \fivedots$ & $3 \times 4 = \fivedots$ & $3 \times 9 = \fivedots$ & $3 \times 9 = \fivedots$ \\ \hline
$3 \times 8 = \fivedots$ & $3 \times 3 = \fivedots$ & $3 \times 2 = \fivedots$ & $3 \times 4 = \fivedots$ \\ \hline
$3 \times 7 = \fivedots$ & $3 \times 5 = \fivedots$ & $3 \times 3 = \fivedots$ & $3 \times 1 = \fivedots$ \\ \hline
$3 \times 4 = \fivedots$ & $3 \times 10 = \fivedots$ & $3 \times 4 = \fivedots$ & $3 \times 2 = \fivedots$ \\ \hline
\end{tabular}
\end{center}
\end{tabular}
\end{center}

\end{document}
