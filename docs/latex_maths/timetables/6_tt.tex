\documentclass[12pt]{article}
\usepackage{tikz}
\usepackage{amsmath}
\usepackage{graphicx}
% Underlining package
\usepackage[normalem]{ulem} % [normalem] prevents the package from changing the default behavior of `\\emph` to underline.
\usepackage[a4paper, portrait, margin=1cm]{geometry}
\usepackage{multicol}
\usepackage{fancyhdr}

\def \Heading {\section*{\Large Name: \underline{\hspace{8cm}} \hfill Date: \underline{\hspace{3cm}}} \vspace{-3mm}
{Times Tables} \vspace{1pt}\hrule}

% raise footer with page number; no header
\fancypagestyle{myfancypagestyle}{
  \fancyhf{}% clear all header and footer fields
  \renewcommand{\headrulewidth}{0pt} % no rule under header
  \fancyfoot[C] {\thepage} \setlength{\footskip}{14.5pt} % raise page number allowed min 14.5pt
}
\pagestyle{myfancypagestyle}  % apply myfancypagestyle




% Define five dots symbol for missing numbers
\newcommand{\fivedots}{\textsubscript{. . . .}}

\begin{document}
\Heading

\vspace{5mm}
\Large % Set font size to large

Study the list of multiples and the times table, as quickly as possible, 5 times. \\

Write in the missing multiples.
% Full list of multiples of 6 from 6 to 60
\[ 6, 12, 18, 24, 30, 36, 42, 48, 54, 60 \]
% Every second number missing
\[ 6, \fivedots, 18, \fivedots, 30, \fivedots, 42, \fivedots, 54, \fivedots \]
% Two out of every three missing
\[ 6, \fivedots, \fivedots, 24, \fivedots, \fivedots, 42, \fivedots, \fivedots, 60 \]
% All missing
\[ 6, \fivedots, \fivedots, \fivedots, \fivedots, \fivedots, \fivedots, \fivedots, \fivedots, \fivedots \]

Write in the missing values in each multiplication equation.
% 6 Times Table with variations
\begin{center}
\begin{tabular}{|c|c|c|c|c|}
\hline
$6 \times 1 = 6$ & $6 \times 1 = \fivedots$ & $6 \times \fivedots = \fivedots$ & $\fivedots \times \fivedots = \fivedots$ \\
$6 \times 2 = 12$ & $6 \times 2 = \fivedots$ & $6 \times \fivedots = \fivedots$ & $\fivedots \times \fivedots = \fivedots$ \\
$6 \times 3 = 18$ & $6 \times 3 = \fivedots$ & $6 \times \fivedots = \fivedots$ & $\fivedots \times \fivedots = \fivedots$ \\
$6 \times 4 = 24$ & $6 \times 4 = \fivedots$ & $6 \times \fivedots = \fivedots$ & $\fivedots \times \fivedots = \fivedots$ \\
$6 \times 5 = 30$ & $6 \times 5 = \fivedots$ & $6 \times \fivedots = \fivedots$ & $\fivedots \times \fivedots = \fivedots$ \\
$6 \times 6 = 36$ & $6 \times 6 = \fivedots$ & $6 \times \fivedots = \fivedots$ & $\fivedots \times \fivedots = \fivedots$ \\
$6 \times 7 = 42$ & $6 \times 7 = \fivedots$ & $6 \times \fivedots = \fivedots$ & $\fivedots \times \fivedots = \fivedots$ \\
$6 \times 8 = 48$ & $6 \times 8 = \fivedots$ & $6 \times \fivedots = \fivedots$ & $\fivedots \times \fivedots = \fivedots$ \\
$6 \times 9 = 54$ & $6 \times 9 = \fivedots$ & $6 \times \fivedots = \fivedots$ & $\fivedots \times \fivedots = \fivedots$ \\
$6 \times 10 = 60$ & $6 \times 10 = \fivedots$ & $6 \times \fivedots = \fivedots$ & $\fivedots \times \fivedots = \fivedots$ \\
\hline
\end{tabular}
\end{center}

Write in the missing values in each multiplication equation.
% New table with 4 columns, each containing 10 randomly sorted equations
\begin{center}
\begin{tabular}{|c|c|c|c|}
\hline
$6 \times 7 = \fivedots$ & $6 \times 3 = \fivedots$ & $6 \times 9 = \fivedots$ & $6 \times 5 = \fivedots$ \\
$6 \times 2 = \fivedots$ & $6 \times 10 = \fivedots$ & $6 \times 6 = \fivedots$ & $6 \times 4 = \fivedots$ \\
$6 \times 1 = \fivedots$ & $6 \times 8 = \fivedots$ & $6 \times 7 = \fivedots$ & $6 \times 3 = \fivedots$ \\
$6 \times 9 = \fivedots$ & $6 \times 5 = \fivedots$ & $6 \times 2 = \fivedots$ & $6 \times 10 = \fivedots$ \\
$6 \times 6 = \fivedots$ & $6 \times 4 = \fivedots$ & $6 \times 1 = \fivedots$ & $6 \times 8 = \fivedots$ \\
$6 \times 7 = \fivedots$ & $6 \times 3 = \fivedots$ & $6 \times 9 = \fivedots$ & $6 \times 5 = \fivedots$ \\
$6 \times 2 = \fivedots$ & $6 \times 10 = \fivedots$ & $6 \times 6 = \fivedots$ & $6 \times 4 = \fivedots$ \\
$6 \times 1 = \fivedots$ & $6 \times 8 = \fivedots$ & $6 \times 7 = \fivedots$ & $6 \times 3 = \fivedots$ \\
$6 \times 9 = \fivedots$ & $6 \times 5 = \fivedots$ & $6 \times 2 = \fivedots$ & $6 \times 10 = \fivedots$ \\
$6 \times 6 = \fivedots$ & $6 \times 4 = \fivedots$ & $6 \times 1 = \fivedots$ & $6 \times 8 = \fivedots$ \\
\hline
\end{tabular}
\end{center}

\end{document}
