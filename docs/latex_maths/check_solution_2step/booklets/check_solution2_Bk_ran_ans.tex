\documentclass[12pt]{article}
\usepackage{tikz}
\usepackage{amssymb}  % For \therefore symbol
\usepackage{amsmath}
% Underlining package
\usepackage[normalem]{ulem} % [normalem] prevents the package from changing the default behavior of `\\emph` to underline.
\usepackage[a4paper, portrait, margin=1cm]{geometry}
\usepackage{multicol}
\usepackage{fancyhdr}
\usepackage{ragged2e}
\usepackage[none]{hyphenat}

% \hyphenpenalty=10000
% \exhyphenpenalty=10000
% \hyphenchar\font=-1

\def \HeadingAnswers {\section*{\Large Name: \underline{\hspace{8cm}} \hfill Date: \underline{\hspace{3cm}}} \vspace{-3mm}
{ran Check Solution: Answers} \vspace{1pt}\hrule}

% \linespread{1.5} % Adjust line spacing factor
\raggedbottom

% raise footer with page number; no header
\fancypagestyle{myfancypagestyle}{
  \fancyhf{}% clear all header and footer fields
  \renewcommand{\headrulewidth}{0pt} % no rule under header
  \fancyfoot[C] {\thepage} \setlength{\footskip}{14.5pt} % raise page number allowed min 14.5pt
}
\pagestyle{myfancypagestyle}  % apply myfancypagestyle
\newcounter{minipagecount}
\begin{document}
\HeadingAnswers
\vspace{1pt}
\begin{multicols}{2}
  %  \begin{enumerate}
\refstepcounter{minipagecount} % increments the counter minipagecount by one.
\noindent{(\theminipagecount)}\hspace{0.1mm} % By default, LaTeX indents the first line of a new paragraph, but \noindent overrides this
% and inserts the current value of the minipagecount counter, enclosed in parentheses
\begin{minipage}[t]{0.40\textwidth} % The [t] option aligns the top of the minipage with the baseline of the surrounding text.
    % \RaggedRight % From ragged2e to ensure no hyphenation

    \noindent Determine whether \(x = -11\) is a solution to the equation \(x + 10 + 6 = 3\):
    \vspace{4pt}  % Ensure spacing between problem statement and solution

    \noindent

    \renewcommand{\arraystretch}{1.3} % Adjust line spacing in the aligned environment
    \begin{tabular}{@{}p{0.60\linewidth}@{}p{0.40\linewidth}@{}}
        \(\begin{aligned}
            \text{LHS} &= x + 10 + 6 \\
                    &= -11 + 10 + 6 \\
                    &= 5
        \end{aligned}\) &
        \(\begin{aligned}
            \text{RHS} &= 3\\
                    & \\
                    &
        \end{aligned}\)
    \end{tabular}
    \renewcommand{\arraystretch}{1.0} % Adjust line spacing in the aligned environment
    \vspace{2pt}  % Optional spacing

    \noindent \(\therefore\) Since \(\text{LHS} \neq \text{RHS}\), \(x = -11\) is not  a solution to the equation.

\end{minipage}

 \vspace*{16pt}
%  \begin{enumerate}
\refstepcounter{minipagecount} % increments the counter minipagecount by one.
\noindent{(\theminipagecount)}\hspace{0.1mm} % By default, LaTeX indents the first line of a new paragraph, but \noindent overrides this
% and inserts the current value of the minipagecount counter, enclosed in parentheses
\begin{minipage}[t]{0.40\textwidth} % The [t] option aligns the top of the minipage with the baseline of the surrounding text.
    % \RaggedRight % From ragged2e to ensure no hyphenation

    \noindent Determine whether \(x = 36\) is a solution to the equation \(\frac{x}{9} + 2 = 3\):
    \vspace{4pt}  % Ensure spacing between problem statement and solution

    \noindent

    \renewcommand{\arraystretch}{1.3} % Adjust line spacing in the aligned environment
    \begin{tabular}{@{}p{0.60\linewidth}@{}p{0.40\linewidth}@{}}
        \(\begin{aligned}
            \text{LHS} &= \frac{x}{9} + 2 \\
                    &= \frac{36}{9} + 2 \\
                    &= 6
        \end{aligned}\) &
        \(\begin{aligned}
            \text{RHS} &= 3\\
                    & \\
                    &
        \end{aligned}\)
    \end{tabular}
    \renewcommand{\arraystretch}{1.0} % Adjust line spacing in the aligned environment
    \vspace{2pt}  % Optional spacing

    \noindent \(\therefore\) Since \(\text{LHS} \neq \text{RHS}\), \(x = 36\) is not  a solution to the equation.

\end{minipage}

 \vspace*{16pt}
%  \begin{enumerate}
\refstepcounter{minipagecount} % increments the counter minipagecount by one.
\noindent{(\theminipagecount)}\hspace{0.1mm} % By default, LaTeX indents the first line of a new paragraph, but \noindent overrides this
% and inserts the current value of the minipagecount counter, enclosed in parentheses
\begin{minipage}[t]{0.40\textwidth} % The [t] option aligns the top of the minipage with the baseline of the surrounding text.
    % \RaggedRight % From ragged2e to ensure no hyphenation

    \noindent Determine whether \(x = 5\) is a solution to the equation \(6(x - 6) = -24\):
    \vspace{4pt}  % Ensure spacing between problem statement and solution

    \noindent

    \renewcommand{\arraystretch}{1.3} % Adjust line spacing in the aligned environment
    \begin{tabular}{@{}p{0.60\linewidth}@{}p{0.40\linewidth}@{}}
        \(\begin{aligned}
            \text{LHS} &= 6(x - 6) \\
                    &= 6 \times(5 - 6) \\
                    &= -6
        \end{aligned}\) &
        \(\begin{aligned}
            \text{RHS} &= -24\\
                    & \\
                    &
        \end{aligned}\)
    \end{tabular}
    \renewcommand{\arraystretch}{1.0} % Adjust line spacing in the aligned environment
    \vspace{2pt}  % Optional spacing

    \noindent \(\therefore\) Since \(\text{LHS} \neq \text{RHS}\), \(x = 5\) is not  a solution to the equation.

\end{minipage}

 \vspace*{16pt}
%  \begin{enumerate}
\refstepcounter{minipagecount} % increments the counter minipagecount by one.
\noindent{(\theminipagecount)}\hspace{0.1mm} % By default, LaTeX indents the first line of a new paragraph, but \noindent overrides this
% and inserts the current value of the minipagecount counter, enclosed in parentheses
\begin{minipage}[t]{0.40\textwidth} % The [t] option aligns the top of the minipage with the baseline of the surrounding text.
    % \RaggedRight % From ragged2e to ensure no hyphenation

    \noindent Determine whether \(x = 21\) is a solution to the equation \(\frac{x + 9}{6} = 6\):
    \vspace{4pt}  % Ensure spacing between problem statement and solution

    \noindent

    \renewcommand{\arraystretch}{1.3} % Adjust line spacing in the aligned environment
    \begin{tabular}{@{}p{0.60\linewidth}@{}p{0.40\linewidth}@{}}
        \(\begin{aligned}
            \text{LHS} &= \frac{x + 9}{6} \\
                    &= \frac{21 + 9}{6} \\
                    &= 5.0
        \end{aligned}\) &
        \(\begin{aligned}
            \text{RHS} &= 6\\
                    & \\
                    &
        \end{aligned}\)
    \end{tabular}
    \renewcommand{\arraystretch}{1.0} % Adjust line spacing in the aligned environment
    \vspace{2pt}  % Optional spacing

    \noindent \(\therefore\) Since \(\text{LHS} \neq \text{RHS}\), \(x = 21\) is not  a solution to the equation.

\end{minipage}

 \vspace*{16pt}
%  \begin{enumerate}
\refstepcounter{minipagecount} % increments the counter minipagecount by one.
\noindent{(\theminipagecount)}\hspace{0.1mm} % By default, LaTeX indents the first line of a new paragraph, but \noindent overrides this
% and inserts the current value of the minipagecount counter, enclosed in parentheses
\begin{minipage}[t]{0.40\textwidth} % The [t] option aligns the top of the minipage with the baseline of the surrounding text.
    % \RaggedRight % From ragged2e to ensure no hyphenation

    \noindent Determine whether \(x = 4\) is a solution to the equation \(x + 1 - 4 = 1\):
    \vspace{4pt}  % Ensure spacing between problem statement and solution

    \noindent

    \renewcommand{\arraystretch}{1.3} % Adjust line spacing in the aligned environment
    \begin{tabular}{@{}p{0.60\linewidth}@{}p{0.40\linewidth}@{}}
        \(\begin{aligned}
            \text{LHS} &= x + 1 - 4 \\
                    &= 4 + 1 - 4 \\
                    &= 1
        \end{aligned}\) &
        \(\begin{aligned}
            \text{RHS} &= 1\\
                    & \\
                    &
        \end{aligned}\)
    \end{tabular}
    \renewcommand{\arraystretch}{1.0} % Adjust line spacing in the aligned environment
    \vspace{2pt}  % Optional spacing

    \noindent \(\therefore\) Since \(\text{LHS} = \text{RHS}\), \(x = 4\) is  a solution to the equation.

\end{minipage}

 \vspace*{16pt}
\columnbreak
    %  \begin{enumerate}
\refstepcounter{minipagecount} % increments the counter minipagecount by one.
\noindent{(\theminipagecount)}\hspace{0.1mm} % By default, LaTeX indents the first line of a new paragraph, but \noindent overrides this
% and inserts the current value of the minipagecount counter, enclosed in parentheses
\begin{minipage}[t]{0.40\textwidth} % The [t] option aligns the top of the minipage with the baseline of the surrounding text.
    % \RaggedRight % From ragged2e to ensure no hyphenation

    \noindent Determine whether \(x = 32\) is a solution to the equation \(\frac{x}{4} + 2 = 10\):
    \vspace{4pt}  % Ensure spacing between problem statement and solution

    \noindent

    \renewcommand{\arraystretch}{1.3} % Adjust line spacing in the aligned environment
    \begin{tabular}{@{}p{0.60\linewidth}@{}p{0.40\linewidth}@{}}
        \(\begin{aligned}
            \text{LHS} &= \frac{x}{4} + 2 \\
                    &= \frac{32}{4} + 2 \\
                    &= 10
        \end{aligned}\) &
        \(\begin{aligned}
            \text{RHS} &= 10\\
                    & \\
                    &
        \end{aligned}\)
    \end{tabular}
    \renewcommand{\arraystretch}{1.0} % Adjust line spacing in the aligned environment
    \vspace{2pt}  % Optional spacing

    \noindent \(\therefore\) Since \(\text{LHS} = \text{RHS}\), \(x = 32\) is  a solution to the equation.

\end{minipage}

 \vspace*{16pt}
%  \begin{enumerate}
\refstepcounter{minipagecount} % increments the counter minipagecount by one.
\noindent{(\theminipagecount)}\hspace{0.1mm} % By default, LaTeX indents the first line of a new paragraph, but \noindent overrides this
% and inserts the current value of the minipagecount counter, enclosed in parentheses
\begin{minipage}[t]{0.40\textwidth} % The [t] option aligns the top of the minipage with the baseline of the surrounding text.
    % \RaggedRight % From ragged2e to ensure no hyphenation

    \noindent Determine whether \(x = 24\) is a solution to the equation \(x - 4 - 10 = 10\):
    \vspace{4pt}  % Ensure spacing between problem statement and solution

    \noindent

    \renewcommand{\arraystretch}{1.3} % Adjust line spacing in the aligned environment
    \begin{tabular}{@{}p{0.60\linewidth}@{}p{0.40\linewidth}@{}}
        \(\begin{aligned}
            \text{LHS} &= x + 4 - 10 \\
                    &= 24 - 4 - 10 \\
                    &= 10
        \end{aligned}\) &
        \(\begin{aligned}
            \text{RHS} &= 10\\
                    & \\
                    &
        \end{aligned}\)
    \end{tabular}
    \renewcommand{\arraystretch}{1.0} % Adjust line spacing in the aligned environment
    \vspace{2pt}  % Optional spacing

    \noindent \(\therefore\) Since \(\text{LHS} = \text{RHS}\), \(x = 24\) is  a solution to the equation.

\end{minipage}

 \vspace*{16pt}
%  \begin{enumerate}
\refstepcounter{minipagecount} % increments the counter minipagecount by one.
\noindent{(\theminipagecount)}\hspace{0.1mm} % By default, LaTeX indents the first line of a new paragraph, but \noindent overrides this
% and inserts the current value of the minipagecount counter, enclosed in parentheses
\begin{minipage}[t]{0.40\textwidth} % The [t] option aligns the top of the minipage with the baseline of the surrounding text.
    % \RaggedRight % From ragged2e to ensure no hyphenation

    \noindent Determine whether \(x = 9\) is a solution to the equation \(x + 1 - 7 = 3\):
    \vspace{4pt}  % Ensure spacing between problem statement and solution

    \noindent

    \renewcommand{\arraystretch}{1.3} % Adjust line spacing in the aligned environment
    \begin{tabular}{@{}p{0.60\linewidth}@{}p{0.40\linewidth}@{}}
        \(\begin{aligned}
            \text{LHS} &= x + 1 - 7 \\
                    &= 9 + 1 - 7 \\
                    &= 3
        \end{aligned}\) &
        \(\begin{aligned}
            \text{RHS} &= 3\\
                    & \\
                    &
        \end{aligned}\)
    \end{tabular}
    \renewcommand{\arraystretch}{1.0} % Adjust line spacing in the aligned environment
    \vspace{2pt}  % Optional spacing

    \noindent \(\therefore\) Since \(\text{LHS} = \text{RHS}\), \(x = 9\) is  a solution to the equation.

\end{minipage}

 \vspace*{16pt}
%  \begin{enumerate}
\refstepcounter{minipagecount} % increments the counter minipagecount by one.
\noindent{(\theminipagecount)}\hspace{0.1mm} % By default, LaTeX indents the first line of a new paragraph, but \noindent overrides this
% and inserts the current value of the minipagecount counter, enclosed in parentheses
\begin{minipage}[t]{0.40\textwidth} % The [t] option aligns the top of the minipage with the baseline of the surrounding text.
    % \RaggedRight % From ragged2e to ensure no hyphenation

    \noindent Determine whether \(x = 15\) is a solution to the equation \(x - 7 - 7 = 4\):
    \vspace{4pt}  % Ensure spacing between problem statement and solution

    \noindent

    \renewcommand{\arraystretch}{1.3} % Adjust line spacing in the aligned environment
    \begin{tabular}{@{}p{0.60\linewidth}@{}p{0.40\linewidth}@{}}
        \(\begin{aligned}
            \text{LHS} &= x + 7 - 7 \\
                    &= 15 - 7 - 7 \\
                    &= 1
        \end{aligned}\) &
        \(\begin{aligned}
            \text{RHS} &= 4\\
                    & \\
                    &
        \end{aligned}\)
    \end{tabular}
    \renewcommand{\arraystretch}{1.0} % Adjust line spacing in the aligned environment
    \vspace{2pt}  % Optional spacing

    \noindent \(\therefore\) Since \(\text{LHS} \neq \text{RHS}\), \(x = 15\) is not  a solution to the equation.

\end{minipage}

 \vspace*{16pt}
%  \begin{enumerate}
\refstepcounter{minipagecount} % increments the counter minipagecount by one.
\noindent{(\theminipagecount)}\hspace{0.1mm} % By default, LaTeX indents the first line of a new paragraph, but \noindent overrides this
% and inserts the current value of the minipagecount counter, enclosed in parentheses
\begin{minipage}[t]{0.40\textwidth} % The [t] option aligns the top of the minipage with the baseline of the surrounding text.
    % \RaggedRight % From ragged2e to ensure no hyphenation

    \noindent Determine whether \(x = 3\) is a solution to the equation \(2x \times 6 = 60\):
    \vspace{4pt}  % Ensure spacing between problem statement and solution

    \noindent

    \renewcommand{\arraystretch}{1.3} % Adjust line spacing in the aligned environment
    \begin{tabular}{@{}p{0.60\linewidth}@{}p{0.40\linewidth}@{}}
        \(\begin{aligned}
            \text{LHS} &= 2x \times 6 \\
                    &= 2 \times3 \times 6 \\
                    &= 36
        \end{aligned}\) &
        \(\begin{aligned}
            \text{RHS} &= 60\\
                    & \\
                    &
        \end{aligned}\)
    \end{tabular}
    \renewcommand{\arraystretch}{1.0} % Adjust line spacing in the aligned environment
    \vspace{2pt}  % Optional spacing

    \noindent \(\therefore\) Since \(\text{LHS} \neq \text{RHS}\), \(x = 3\) is not  a solution to the equation.

\end{minipage}

 \vspace*{16pt}
\newpage
    %  \begin{enumerate}
\refstepcounter{minipagecount} % increments the counter minipagecount by one.
\noindent{(\theminipagecount)}\hspace{0.1mm} % By default, LaTeX indents the first line of a new paragraph, but \noindent overrides this
% and inserts the current value of the minipagecount counter, enclosed in parentheses
\begin{minipage}[t]{0.40\textwidth} % The [t] option aligns the top of the minipage with the baseline of the surrounding text.
    % \RaggedRight % From ragged2e to ensure no hyphenation

    \noindent Determine whether \(x = 6\) is a solution to the equation \(4(x - 1) = 32\):
    \vspace{4pt}  % Ensure spacing between problem statement and solution

    \noindent

    \renewcommand{\arraystretch}{1.3} % Adjust line spacing in the aligned environment
    \begin{tabular}{@{}p{0.60\linewidth}@{}p{0.40\linewidth}@{}}
        \(\begin{aligned}
            \text{LHS} &= 4(x - 1) \\
                    &= 4 \times(6 - 1) \\
                    &= 20
        \end{aligned}\) &
        \(\begin{aligned}
            \text{RHS} &= 32\\
                    & \\
                    &
        \end{aligned}\)
    \end{tabular}
    \renewcommand{\arraystretch}{1.0} % Adjust line spacing in the aligned environment
    \vspace{2pt}  % Optional spacing

    \noindent \(\therefore\) Since \(\text{LHS} \neq \text{RHS}\), \(x = 6\) is not  a solution to the equation.

\end{minipage}

 \vspace*{16pt}
%  \begin{enumerate}
\refstepcounter{minipagecount} % increments the counter minipagecount by one.
\noindent{(\theminipagecount)}\hspace{0.1mm} % By default, LaTeX indents the first line of a new paragraph, but \noindent overrides this
% and inserts the current value of the minipagecount counter, enclosed in parentheses
\begin{minipage}[t]{0.40\textwidth} % The [t] option aligns the top of the minipage with the baseline of the surrounding text.
    % \RaggedRight % From ragged2e to ensure no hyphenation

    \noindent Determine whether \(x = 19\) is a solution to the equation \(x - 6 - 1 = 9\):
    \vspace{4pt}  % Ensure spacing between problem statement and solution

    \noindent

    \renewcommand{\arraystretch}{1.3} % Adjust line spacing in the aligned environment
    \begin{tabular}{@{}p{0.60\linewidth}@{}p{0.40\linewidth}@{}}
        \(\begin{aligned}
            \text{LHS} &= x + 6 - 1 \\
                    &= 19 - 6 - 1 \\
                    &= 12
        \end{aligned}\) &
        \(\begin{aligned}
            \text{RHS} &= 9\\
                    & \\
                    &
        \end{aligned}\)
    \end{tabular}
    \renewcommand{\arraystretch}{1.0} % Adjust line spacing in the aligned environment
    \vspace{2pt}  % Optional spacing

    \noindent \(\therefore\) Since \(\text{LHS} \neq \text{RHS}\), \(x = 19\) is not  a solution to the equation.

\end{minipage}

 \vspace*{16pt}
%  \begin{enumerate}
\refstepcounter{minipagecount} % increments the counter minipagecount by one.
\noindent{(\theminipagecount)}\hspace{0.1mm} % By default, LaTeX indents the first line of a new paragraph, but \noindent overrides this
% and inserts the current value of the minipagecount counter, enclosed in parentheses
\begin{minipage}[t]{0.40\textwidth} % The [t] option aligns the top of the minipage with the baseline of the surrounding text.
    % \RaggedRight % From ragged2e to ensure no hyphenation

    \noindent Determine whether \(x = 2\) is a solution to the equation \(2(x + 1) = 6\):
    \vspace{4pt}  % Ensure spacing between problem statement and solution

    \noindent

    \renewcommand{\arraystretch}{1.3} % Adjust line spacing in the aligned environment
    \begin{tabular}{@{}p{0.60\linewidth}@{}p{0.40\linewidth}@{}}
        \(\begin{aligned}
            \text{LHS} &= 2(x + 1) \\
                    &= 2 \times(2 + 1) \\
                    &= 6
        \end{aligned}\) &
        \(\begin{aligned}
            \text{RHS} &= 6\\
                    & \\
                    &
        \end{aligned}\)
    \end{tabular}
    \renewcommand{\arraystretch}{1.0} % Adjust line spacing in the aligned environment
    \vspace{2pt}  % Optional spacing

    \noindent \(\therefore\) Since \(\text{LHS} = \text{RHS}\), \(x = 2\) is  a solution to the equation.

\end{minipage}

 \vspace*{16pt}
%  \begin{enumerate}
\refstepcounter{minipagecount} % increments the counter minipagecount by one.
\noindent{(\theminipagecount)}\hspace{0.1mm} % By default, LaTeX indents the first line of a new paragraph, but \noindent overrides this
% and inserts the current value of the minipagecount counter, enclosed in parentheses
\begin{minipage}[t]{0.40\textwidth} % The [t] option aligns the top of the minipage with the baseline of the surrounding text.
    % \RaggedRight % From ragged2e to ensure no hyphenation

    \noindent Determine whether \(x = 6\) is a solution to the equation \(\frac{5x}{3} = 10\):
    \vspace{4pt}  % Ensure spacing between problem statement and solution

    \noindent

    \renewcommand{\arraystretch}{1.3} % Adjust line spacing in the aligned environment
    \begin{tabular}{@{}p{0.60\linewidth}@{}p{0.40\linewidth}@{}}
        \(\begin{aligned}
            \text{LHS} &= \frac{5x}{3} \\
                    &= \frac{5 \times6}{3} \\
                    &= 10.0
        \end{aligned}\) &
        \(\begin{aligned}
            \text{RHS} &= 10\\
                    & \\
                    &
        \end{aligned}\)
    \end{tabular}
    \renewcommand{\arraystretch}{1.0} % Adjust line spacing in the aligned environment
    \vspace{2pt}  % Optional spacing

    \noindent \(\therefore\) Since \(\text{LHS} = \text{RHS}\), \(x = 6\) is  a solution to the equation.

\end{minipage}

 \vspace*{16pt}
%  \begin{enumerate}
\refstepcounter{minipagecount} % increments the counter minipagecount by one.
\noindent{(\theminipagecount)}\hspace{0.1mm} % By default, LaTeX indents the first line of a new paragraph, but \noindent overrides this
% and inserts the current value of the minipagecount counter, enclosed in parentheses
\begin{minipage}[t]{0.40\textwidth} % The [t] option aligns the top of the minipage with the baseline of the surrounding text.
    % \RaggedRight % From ragged2e to ensure no hyphenation

    \noindent Determine whether \(x = 24\) is a solution to the equation \(\frac{x}{3} + 4 = 10\):
    \vspace{4pt}  % Ensure spacing between problem statement and solution

    \noindent

    \renewcommand{\arraystretch}{1.3} % Adjust line spacing in the aligned environment
    \begin{tabular}{@{}p{0.60\linewidth}@{}p{0.40\linewidth}@{}}
        \(\begin{aligned}
            \text{LHS} &= \frac{x}{3} + 4 \\
                    &= \frac{24}{3} + 4 \\
                    &= 12
        \end{aligned}\) &
        \(\begin{aligned}
            \text{RHS} &= 10\\
                    & \\
                    &
        \end{aligned}\)
    \end{tabular}
    \renewcommand{\arraystretch}{1.0} % Adjust line spacing in the aligned environment
    \vspace{2pt}  % Optional spacing

    \noindent \(\therefore\) Since \(\text{LHS} \neq \text{RHS}\), \(x = 24\) is not  a solution to the equation.

\end{minipage}

 \vspace*{16pt}
\columnbreak
    %  \begin{enumerate}
\refstepcounter{minipagecount} % increments the counter minipagecount by one.
\noindent{(\theminipagecount)}\hspace{0.1mm} % By default, LaTeX indents the first line of a new paragraph, but \noindent overrides this
% and inserts the current value of the minipagecount counter, enclosed in parentheses
\begin{minipage}[t]{0.40\textwidth} % The [t] option aligns the top of the minipage with the baseline of the surrounding text.
    % \RaggedRight % From ragged2e to ensure no hyphenation

    \noindent Determine whether \(x = 80\) is a solution to the equation \(\frac{x}{8} - 2 = 6\):
    \vspace{4pt}  % Ensure spacing between problem statement and solution

    \noindent

    \renewcommand{\arraystretch}{1.3} % Adjust line spacing in the aligned environment
    \begin{tabular}{@{}p{0.60\linewidth}@{}p{0.40\linewidth}@{}}
        \(\begin{aligned}
            \text{LHS} &= \frac{x}{8} - 2 \\
                    &= \frac{80}{8} - 2 \\
                    &= 8
        \end{aligned}\) &
        \(\begin{aligned}
            \text{RHS} &= 6\\
                    & \\
                    &
        \end{aligned}\)
    \end{tabular}
    \renewcommand{\arraystretch}{1.0} % Adjust line spacing in the aligned environment
    \vspace{2pt}  % Optional spacing

    \noindent \(\therefore\) Since \(\text{LHS} \neq \text{RHS}\), \(x = 80\) is not  a solution to the equation.

\end{minipage}

 \vspace*{16pt}
%  \begin{enumerate}
\refstepcounter{minipagecount} % increments the counter minipagecount by one.
\noindent{(\theminipagecount)}\hspace{0.1mm} % By default, LaTeX indents the first line of a new paragraph, but \noindent overrides this
% and inserts the current value of the minipagecount counter, enclosed in parentheses
\begin{minipage}[t]{0.40\textwidth} % The [t] option aligns the top of the minipage with the baseline of the surrounding text.
    % \RaggedRight % From ragged2e to ensure no hyphenation

    \noindent Determine whether \(x = 126\) is a solution to the equation \(\frac{x}{6} \times \frac{1}{7} = 3\):
    \vspace{4pt}  % Ensure spacing between problem statement and solution

    \noindent

    \renewcommand{\arraystretch}{1.3} % Adjust line spacing in the aligned environment
    \begin{tabular}{@{}p{0.60\linewidth}@{}p{0.40\linewidth}@{}}
        \(\begin{aligned}
            \text{LHS} &= \frac{x}{6} \times \frac{1}{7} \\
                    &= \frac{126}{6} \times \frac{1}{7} \\
                    &= 3
        \end{aligned}\) &
        \(\begin{aligned}
            \text{RHS} &= 3\\
                    & \\
                    &
        \end{aligned}\)
    \end{tabular}
    \renewcommand{\arraystretch}{1.0} % Adjust line spacing in the aligned environment
    \vspace{2pt}  % Optional spacing

    \noindent \(\therefore\) Since \(\text{LHS} = \text{RHS}\), \(x = 126\) is  a solution to the equation.

\end{minipage}

 \vspace*{16pt}
%  \begin{enumerate}
\refstepcounter{minipagecount} % increments the counter minipagecount by one.
\noindent{(\theminipagecount)}\hspace{0.1mm} % By default, LaTeX indents the first line of a new paragraph, but \noindent overrides this
% and inserts the current value of the minipagecount counter, enclosed in parentheses
\begin{minipage}[t]{0.40\textwidth} % The [t] option aligns the top of the minipage with the baseline of the surrounding text.
    % \RaggedRight % From ragged2e to ensure no hyphenation

    \noindent Determine whether \(x = 54\) is a solution to the equation \(\frac{x}{6} \times \frac{1}{3} = 3\):
    \vspace{4pt}  % Ensure spacing between problem statement and solution

    \noindent

    \renewcommand{\arraystretch}{1.3} % Adjust line spacing in the aligned environment
    \begin{tabular}{@{}p{0.60\linewidth}@{}p{0.40\linewidth}@{}}
        \(\begin{aligned}
            \text{LHS} &= \frac{x}{6} \times \frac{1}{3} \\
                    &= \frac{54}{6} \times \frac{1}{3} \\
                    &= 3
        \end{aligned}\) &
        \(\begin{aligned}
            \text{RHS} &= 3\\
                    & \\
                    &
        \end{aligned}\)
    \end{tabular}
    \renewcommand{\arraystretch}{1.0} % Adjust line spacing in the aligned environment
    \vspace{2pt}  % Optional spacing

    \noindent \(\therefore\) Since \(\text{LHS} = \text{RHS}\), \(x = 54\) is  a solution to the equation.

\end{minipage}

 \vspace*{16pt}
%  \begin{enumerate}
\refstepcounter{minipagecount} % increments the counter minipagecount by one.
\noindent{(\theminipagecount)}\hspace{0.1mm} % By default, LaTeX indents the first line of a new paragraph, but \noindent overrides this
% and inserts the current value of the minipagecount counter, enclosed in parentheses
\begin{minipage}[t]{0.40\textwidth} % The [t] option aligns the top of the minipage with the baseline of the surrounding text.
    % \RaggedRight % From ragged2e to ensure no hyphenation

    \noindent Determine whether \(x = 7\) is a solution to the equation \(8x + 10 = 66\):
    \vspace{4pt}  % Ensure spacing between problem statement and solution

    \noindent

    \renewcommand{\arraystretch}{1.3} % Adjust line spacing in the aligned environment
    \begin{tabular}{@{}p{0.60\linewidth}@{}p{0.40\linewidth}@{}}
        \(\begin{aligned}
            \text{LHS} &= 8x + 10 \\
                    &= 8 \times7 + 10 \\
                    &= 66
        \end{aligned}\) &
        \(\begin{aligned}
            \text{RHS} &= 66\\
                    & \\
                    &
        \end{aligned}\)
    \end{tabular}
    \renewcommand{\arraystretch}{1.0} % Adjust line spacing in the aligned environment
    \vspace{2pt}  % Optional spacing

    \noindent \(\therefore\) Since \(\text{LHS} = \text{RHS}\), \(x = 7\) is  a solution to the equation.

\end{minipage}

 \vspace*{16pt}
%  \begin{enumerate}
\refstepcounter{minipagecount} % increments the counter minipagecount by one.
\noindent{(\theminipagecount)}\hspace{0.1mm} % By default, LaTeX indents the first line of a new paragraph, but \noindent overrides this
% and inserts the current value of the minipagecount counter, enclosed in parentheses
\begin{minipage}[t]{0.40\textwidth} % The [t] option aligns the top of the minipage with the baseline of the surrounding text.
    % \RaggedRight % From ragged2e to ensure no hyphenation

    \noindent Determine whether \(x = 43\) is a solution to the equation \(\frac{x + 5}{8} = 4\):
    \vspace{4pt}  % Ensure spacing between problem statement and solution

    \noindent

    \renewcommand{\arraystretch}{1.3} % Adjust line spacing in the aligned environment
    \begin{tabular}{@{}p{0.60\linewidth}@{}p{0.40\linewidth}@{}}
        \(\begin{aligned}
            \text{LHS} &= \frac{x + 5}{8} \\
                    &= \frac{43 + 5}{8} \\
                    &= 6.0
        \end{aligned}\) &
        \(\begin{aligned}
            \text{RHS} &= 4\\
                    & \\
                    &
        \end{aligned}\)
    \end{tabular}
    \renewcommand{\arraystretch}{1.0} % Adjust line spacing in the aligned environment
    \vspace{2pt}  % Optional spacing

    \noindent \(\therefore\) Since \(\text{LHS} \neq \text{RHS}\), \(x = 43\) is not  a solution to the equation.

\end{minipage}

 \vspace*{16pt}
\newpage

\end{multicols}
\end{document}
